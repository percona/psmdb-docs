%% Generated by Sphinx.
\def\sphinxdocclass{report}
\documentclass[letterpaper,10pt,english]{sphinxmanual}
\ifdefined\pdfpxdimen
   \let\sphinxpxdimen\pdfpxdimen\else\newdimen\sphinxpxdimen
\fi \sphinxpxdimen=.75bp\relax
\ifdefined\pdfimageresolution
    \pdfimageresolution= \numexpr \dimexpr1in\relax/\sphinxpxdimen\relax
\fi
%% let collapsible pdf bookmarks panel have high depth per default
\PassOptionsToPackage{bookmarksdepth=5}{hyperref}

\PassOptionsToPackage{warn}{textcomp}
\usepackage[utf8]{inputenc}
\ifdefined\DeclareUnicodeCharacter
% support both utf8 and utf8x syntaxes
  \ifdefined\DeclareUnicodeCharacterAsOptional
    \def\sphinxDUC#1{\DeclareUnicodeCharacter{"#1}}
  \else
    \let\sphinxDUC\DeclareUnicodeCharacter
  \fi
  \sphinxDUC{00A0}{\nobreakspace}
  \sphinxDUC{2500}{\sphinxunichar{2500}}
  \sphinxDUC{2502}{\sphinxunichar{2502}}
  \sphinxDUC{2514}{\sphinxunichar{2514}}
  \sphinxDUC{251C}{\sphinxunichar{251C}}
  \sphinxDUC{2572}{\textbackslash}
\fi
\usepackage{cmap}
\usepackage[T1]{fontenc}
\usepackage{amsmath,amssymb,amstext}
\usepackage{babel}



\usepackage{tgtermes}
\usepackage{tgheros}
\renewcommand{\ttdefault}{txtt}



\usepackage[Bjarne]{fncychap}
\usepackage{sphinx}

\fvset{fontsize=auto}
\usepackage{geometry}


% Include hyperref last.
\usepackage{hyperref}
% Fix anchor placement for figures with captions.
\usepackage{hypcap}% it must be loaded after hyperref.
% Set up styles of URL: it should be placed after hyperref.
\urlstyle{same}


\usepackage{sphinxmessages}
\setcounter{tocdepth}{-1}



\title{Percona Server for MongoDB 4.4 Documentation}
\date{Nov 10, 2022}
\release{4.4.17\sphinxhyphen{}17}
\author{Percona LLC and/or its affiliates 2015-2022}
\newcommand{\sphinxlogo}{\sphinxincludegraphics{Percona_Logo_Color.png}\par}
\renewcommand{\releasename}{Release}
\makeindex
\begin{document}

\pagestyle{empty}
\sphinxmaketitle
\pagestyle{plain}
\sphinxtableofcontents
\pagestyle{normal}
\phantomsection\label{\detokenize{index::doc}}


\sphinxAtStartPar
\sphinxstyleemphasis{Percona Server for MongoDB} is a free, enhanced, fully compatible, source available, drop\sphinxhyphen{}in replacement
for MongoDB 4.4 Community Edition with enterprise\sphinxhyphen{}grade features.
It requires no changes to MongoDB applications or code.

\begin{sphinxadmonition}{hint}{Hint:}
\sphinxAtStartPar
To see which version of \sphinxstyleemphasis{Percona Server for MongoDB} you are using check the value of the
\sphinxcode{\sphinxupquote{psmdbVersion}} key in the output of the \sphinxhref{https://docs.mongodb.com/manual/reference/command/buildInfo/\#dbcmd.buildInfo}{buildInfo} database command. If
this key does not exist, \sphinxstyleemphasis{Percona Server for MongoDB} is not installed on the server.
\end{sphinxadmonition}

\sphinxAtStartPar
\sphinxstyleemphasis{Percona Server for MongoDB} provides the following features:
\begin{itemize}
\item {} 
\sphinxAtStartPar
MongoDB’s default \sphinxhref{https://docs.mongodb.org/manual/core/wiredtiger/}{WiredTiger} engine

\item {} 
\sphinxAtStartPar
{\hyperref[\detokenize{inmemory:inmemory}]{\sphinxcrossref{\DUrole{std,std-ref}{Percona Memory Engine}}}} storage engine

\item {} 
\sphinxAtStartPar
{\hyperref[\detokenize{data-at-rest-encryption:psmdb-data-at-rest-encryption}]{\sphinxcrossref{\DUrole{std,std-ref}{Data at Rest Encryption}}}}

\item {} 
\sphinxAtStartPar
{\hyperref[\detokenize{authentication:ext-auth}]{\sphinxcrossref{\DUrole{std,std-ref}{External authentication}}}}
using OpenLDAP or Active Directory

\item {} 
\sphinxAtStartPar
{\hyperref[\detokenize{audit-logging:audit-log}]{\sphinxcrossref{\DUrole{std,std-ref}{Audit logging}}}}
to track and query database interactions of users or applications

\item {} 
\sphinxAtStartPar
{\hyperref[\detokenize{hot-backup:hot-backup}]{\sphinxcrossref{\DUrole{std,std-ref}{Hot Backup}}}} for the default \sphinxhref{https://docs.mongodb.org/manual/core/wiredtiger/}{WiredTiger}

\item {} 
\sphinxAtStartPar
{\hyperref[\detokenize{rate-limit:rate-limit}]{\sphinxcrossref{\DUrole{std,std-ref}{Profiling Rate Limit}}}} to decrease the impact of the profiler on performance

\end{itemize}

\sphinxAtStartPar
To learn more about the features, available in \sphinxstyleemphasis{Percona Server for MongoDB}, see {\hyperref[\detokenize{comparison:compare}]{\sphinxcrossref{\DUrole{std,std-ref}{Percona Server for MongoDB Feature Comparison}}}}


\bigskip\hrule\bigskip



\part{About Percona Server for MongoDB}
\label{\detokenize{index:about-percona-server-for-mongodb}}

\chapter{Percona Server for MongoDB Feature Comparison}
\label{\detokenize{comparison:percona-server-for-mongodb-feature-comparison}}\label{\detokenize{comparison:compare}}\label{\detokenize{comparison::doc}}
\sphinxAtStartPar
\sphinxstyleemphasis{Percona Server for MongoDB} 4.4 is based on \sphinxhref{https://docs.mongodb.com/manual/introduction/}{MongoDB 4.4}. \sphinxstyleemphasis{Percona Server for MongoDB} extends MongoDB
Community Edition to include the functionality that is otherwise only available
in MongoDB Enterprise Edition.


\begin{savenotes}\sphinxattablestart
\centering
\begin{tabular}[t]{|*{3}{\X{1}{3}|}}
\hline
\sphinxstyletheadfamily &\sphinxstyletheadfamily 
\sphinxAtStartPar
PSMDB
&\sphinxstyletheadfamily 
\sphinxAtStartPar
MongoDB
\\
\hline\sphinxstyletheadfamily 
\sphinxAtStartPar
Storage Engines
&\begin{itemize}
\item {} 
\sphinxAtStartPar
\sphinxhref{https://docs.mongodb.org/manual/core/wiredtiger/}{WiredTiger} (default)

\item {} 
\sphinxAtStartPar
{\hyperref[\detokenize{inmemory:inmemory}]{\sphinxcrossref{\DUrole{std,std-ref}{Percona Memory Engine}}}}

\end{itemize}
&\begin{itemize}
\item {} 
\sphinxAtStartPar
\sphinxhref{https://docs.mongodb.org/manual/core/wiredtiger/}{WiredTiger} (default)

\item {} 
\sphinxAtStartPar
\sphinxhref{https://docs.mongodb.com/v4.4/core/inmemory/}{In\sphinxhyphen{}Memory} (Enterprise only)

\end{itemize}
\\
\hline\sphinxstyletheadfamily 
\sphinxAtStartPar
Encryption\sphinxhyphen{}at\sphinxhyphen{}Rest
&\begin{itemize}
\item {} 
\sphinxAtStartPar
Key servers = {\hyperref[\detokenize{vault:vault}]{\sphinxcrossref{\DUrole{std,std-ref}{Hashicorp Vault}}}}, {\hyperref[\detokenize{kmip:kmip}]{\sphinxcrossref{\DUrole{std,std-ref}{KMIP}}}}

\item {} 
\sphinxAtStartPar
Fully open source

\end{itemize}
&\begin{itemize}
\item {} 
\sphinxAtStartPar
Key server = KMIP

\item {} 
\sphinxAtStartPar
Enterprise only

\end{itemize}
\\
\hline\sphinxstyletheadfamily 
\sphinxAtStartPar
Hot Backup
&
\sphinxAtStartPar
{\hyperref[\detokenize{hot-backup:hot-backup}]{\sphinxcrossref{\DUrole{std,std-ref}{YES}}}} (replica set)
&
\sphinxAtStartPar
NO
\\
\hline\sphinxstyletheadfamily 
\sphinxAtStartPar
LDAP Authentication
&\begin{itemize}
\item {} 
\sphinxAtStartPar
(legacy) {\hyperref[\detokenize{authentication:ldap-authentication-sasl}]{\sphinxcrossref{\DUrole{std,std-ref}{LDAP authentication with SASL}}}}

\end{itemize}
&\begin{itemize}
\item {} 
\sphinxAtStartPar
Enterprise only

\end{itemize}
\\
\hline\sphinxstyletheadfamily 
\sphinxAtStartPar
LDAP Authorization
&
\sphinxAtStartPar
{\hyperref[\detokenize{authorization:ldap-authorization}]{\sphinxcrossref{\DUrole{std,std-ref}{YES}}}}
&
\sphinxAtStartPar
Enterprise only
\\
\hline\sphinxstyletheadfamily 
\sphinxAtStartPar
Kerberos Authentication
&
\sphinxAtStartPar
{\hyperref[\detokenize{authentication:kerberos-authentication}]{\sphinxcrossref{\DUrole{std,std-ref}{YES}}}}
&
\sphinxAtStartPar
Enterprise only
\\
\hline\sphinxstyletheadfamily 
\sphinxAtStartPar
Audit Logging
&
\sphinxAtStartPar
{\hyperref[\detokenize{audit-logging:audit-log}]{\sphinxcrossref{\DUrole{std,std-ref}{YES}}}}
&
\sphinxAtStartPar
Enterprise only
\\
\hline\sphinxstyletheadfamily 
\sphinxAtStartPar
Log redaction
&
\sphinxAtStartPar
{\hyperref[\detokenize{log-redaction:log-redaction}]{\sphinxcrossref{\DUrole{std,std-ref}{YES}}}}
&
\sphinxAtStartPar
Enterprise only
\\
\hline\sphinxstyletheadfamily 
\sphinxAtStartPar
SNMP Monitoring
&
\sphinxAtStartPar
NO
&
\sphinxAtStartPar
Enterprise only
\\
\hline
\end{tabular}
\par
\sphinxattableend\end{savenotes}


\section{Profiling Rate Limiting}
\label{\detokenize{comparison:profiling-rate-limiting}}
\sphinxAtStartPar
Profiling Rate Limiting was added to \sphinxstyleemphasis{Percona Server for MongoDB} in v3.4 with the \sphinxcode{\sphinxupquote{\sphinxhyphen{}\sphinxhyphen{}rateLimit}} argument. Since v3.6, MongoDB Community (and Enterprise) Edition includes a similar option \sphinxhref{https://docs.mongodb.com/manual/reference/program/mongod/index.html\#cmdoption-mongod-slowopsamplerate}{slowOpSampleRate}. Please see {\hyperref[\detokenize{rate-limit:rate-limit}]{\sphinxcrossref{\DUrole{std,std-ref}{Profiling Rate Limit}}}} for more information.


\part{Installation}
\label{\detokenize{index:installation}}

\chapter{Installing Percona Server for MongoDB}
\label{\detokenize{install/index:installing-percona-server-for-mongodb}}\label{\detokenize{install/index:install}}\label{\detokenize{install/index::doc}}
\sphinxAtStartPar
Percona provides installation packages of \sphinxstyleemphasis{Percona Server for MongoDB} for the most 64\sphinxhyphen{}bit Linux distributions. Find the full list of supported platforms on the \sphinxhref{https://www.percona.com/services/policies/percona-software-platform-lifecycle\#mongodb}{Percona Software and Platform Lifecycle} page.

\sphinxAtStartPar
The recommended installation method is from \sphinxstyleemphasis{Percona} repositories. Follow the links below for the installation instructions for your operating system.
\begin{itemize}
\item {} 
\sphinxAtStartPar
{\hyperref[\detokenize{install/apt:apt}]{\sphinxcrossref{\DUrole{std,std-ref}{Install on Debian or Ubuntu}}}}

\item {} 
\sphinxAtStartPar
{\hyperref[\detokenize{install/yum:yum}]{\sphinxcrossref{\DUrole{std,std-ref}{Install on RHEL or CentOS}}}}

\end{itemize}


\section{Alternative Install Instructions}
\label{\detokenize{install/index:alternative-install-instructions}}
\sphinxAtStartPar
You can also download packages from the \sphinxhref{https://www.percona.com/downloads/percona-server-mongodb-4.4/}{Percona website} and install them
manually using \sphinxstyleliteralstrong{\sphinxupquote{dpkg}} or \sphinxstyleliteralstrong{\sphinxupquote{rpm}}.

\begin{sphinxadmonition}{note}{Note:}
\sphinxAtStartPar
In this case, you will have to make sure that all dependencies are satisfied.
\end{sphinxadmonition}

\sphinxAtStartPar
If you want more control over the installation, you can {\hyperref[\detokenize{install/tarball:tarball}]{\sphinxcrossref{\DUrole{std,std-ref}{install Percona
Server for MongoDB from binary tarballs}}}}.

\begin{sphinxadmonition}{note}{Note:}
\sphinxAtStartPar
This method is for advanced users with specific needs that are not addressed
by DEB and RPM packages.
\end{sphinxadmonition}

\sphinxAtStartPar
If you want to run \sphinxstyleemphasis{Percona Server for MongoDB} in a Docker container, see {\hyperref[\detokenize{install/docker:docker}]{\sphinxcrossref{\DUrole{std,std-ref}{Running Percona Server for MongoDB in a Docker Container}}}}.


\section{Upgrade Instructions}
\label{\detokenize{install/index:upgrade-instructions}}
\sphinxAtStartPar
If you are currently using MongoDB, see {\hyperref[\detokenize{install/upgrade-from-mongodb:upgrade-from-mongodb}]{\sphinxcrossref{\DUrole{std,std-ref}{Upgrading from MongoDB}}}}.

\sphinxAtStartPar
If you are running an earlier version of \sphinxstyleemphasis{Percona Server for MongoDB}, see {\hyperref[\detokenize{install/upgrade-from-42:upgrade-from-42}]{\sphinxcrossref{\DUrole{std,std-ref}{Upgrading from Version 4.2}}}}.


\section{Uninstall Instructions}
\label{\detokenize{install/index:uninstall-instructions}}
\sphinxAtStartPar
To uninstall \sphinxstyleemphasis{Percona Server for MongoDB}, see {\hyperref[\detokenize{install/uninstall:uninstall}]{\sphinxcrossref{\DUrole{std,std-ref}{Uninstalling Percona Server for MongoDB}}}}.


\subsection{Installing Percona Server for MongoDB on Debian and Ubuntu}
\label{\detokenize{install/apt:installing-percona-server-for-mongodb-on-debian-and-ubuntu}}\label{\detokenize{install/apt:apt}}\label{\detokenize{install/apt::doc}}
\sphinxAtStartPar
Use this document to install \sphinxstyleemphasis{Percona Server for MongoDB} from Percona repositories on DEB\sphinxhyphen{}based distributions.

\begin{sphinxadmonition}{note}{Note:}
\sphinxAtStartPar
\sphinxstyleemphasis{Percona Server for MongoDB} should work on other DEB\sphinxhyphen{}based distributions,
but it is tested only on platforms listed on the \sphinxhref{https://www.percona.com/services/policies/percona-software-platform-lifecycle\#mongodb}{Percona Software and Platform Lifecycle} page.
\end{sphinxadmonition}

\begin{sphinxShadowBox}
\begin{itemize}
\item {} 
\sphinxAtStartPar
\phantomsection\label{\detokenize{install/apt:id1}}{\hyperref[\detokenize{install/apt:package-contents}]{\sphinxcrossref{Package Contents}}}

\item {} 
\sphinxAtStartPar
\phantomsection\label{\detokenize{install/apt:id2}}{\hyperref[\detokenize{install/apt:installing-from-percona-repositories}]{\sphinxcrossref{Installing from Percona repositories}}}
\begin{itemize}
\item {} 
\sphinxAtStartPar
\phantomsection\label{\detokenize{install/apt:id3}}{\hyperref[\detokenize{install/apt:configure-percona-repository}]{\sphinxcrossref{Configure Percona repository}}}

\item {} 
\sphinxAtStartPar
\phantomsection\label{\detokenize{install/apt:id4}}{\hyperref[\detokenize{install/apt:install-the-latest-version}]{\sphinxcrossref{Install the latest version}}}

\item {} 
\sphinxAtStartPar
\phantomsection\label{\detokenize{install/apt:id5}}{\hyperref[\detokenize{install/apt:install-a-specific-version}]{\sphinxcrossref{Install a specific version}}}

\end{itemize}

\item {} 
\sphinxAtStartPar
\phantomsection\label{\detokenize{install/apt:id6}}{\hyperref[\detokenize{install/apt:running-percona-server-for-mongodb}]{\sphinxcrossref{Running Percona Server for MongoDB}}}

\end{itemize}
\end{sphinxShadowBox}


\subsubsection{Package Contents}
\label{\detokenize{install/apt:package-contents}}

\begin{savenotes}\sphinxattablestart
\centering
\begin{tabular}[t]{|\X{25}{100}|\X{75}{100}|}
\hline

\sphinxAtStartPar
Package
&
\sphinxAtStartPar
Contains
\\
\hline
\sphinxAtStartPar
percona\sphinxhyphen{}server\sphinxhyphen{}mongodb
&
\sphinxAtStartPar
The \sphinxcode{\sphinxupquote{mongo}} shell, import/export tools, other client
utilities, server software, default configuration, and init.d scripts.
\\
\hline
\sphinxAtStartPar
percona\sphinxhyphen{}server\sphinxhyphen{}mongodb\sphinxhyphen{}server
&
\sphinxAtStartPar
The \sphinxstyleliteralstrong{\sphinxupquote{mongod}} server, default configuration files, and \sphinxcode{\sphinxupquote{init.d}}
scripts
\\
\hline
\sphinxAtStartPar
percona\sphinxhyphen{}server\sphinxhyphen{}mongodb\sphinxhyphen{}shell
&
\sphinxAtStartPar
The \sphinxcode{\sphinxupquote{mongo}} shell
\\
\hline
\sphinxAtStartPar
percona\sphinxhyphen{}server\sphinxhyphen{}mongodb\sphinxhyphen{}mongos
&
\sphinxAtStartPar
The \sphinxcode{\sphinxupquote{mongos}} sharded cluster query router
\\
\hline
\sphinxAtStartPar
percona\sphinxhyphen{}server\sphinxhyphen{}mongodb\sphinxhyphen{}tools
&
\sphinxAtStartPar
Mongo tools for high\sphinxhyphen{}performance MongoDB fork from Percona
\\
\hline
\sphinxAtStartPar
percona\sphinxhyphen{}server\sphinxhyphen{}mongodb\sphinxhyphen{}dbg
&
\sphinxAtStartPar
Debug symbols for the server
\\
\hline
\end{tabular}
\par
\sphinxattableend\end{savenotes}


\subsubsection{Installing from Percona repositories}
\label{\detokenize{install/apt:installing-from-percona-repositories}}
\sphinxAtStartPar
It is recommended to install \sphinxstyleemphasis{Percona Server for MongoDB} from official Percona repositories using
the \sphinxstyleliteralstrong{\sphinxupquote{percona\sphinxhyphen{}release}} utility.


\paragraph{Configure Percona repository}
\label{\detokenize{install/apt:configure-percona-repository}}\begin{enumerate}
\sphinxsetlistlabels{\arabic}{enumi}{enumii}{}{.}%
\item {} 
\sphinxAtStartPar
Fetch \sphinxstyleliteralstrong{\sphinxupquote{percona\sphinxhyphen{}release}} packages from Percona web:

\begin{sphinxVerbatim}[commandchars=\\\{\}]
\PYGZdl{} wget https://repo.percona.com/apt/percona\PYGZhy{}release\PYGZus{}latest.\PYG{k}{\PYGZdl{}(}lsb\PYGZus{}release \PYGZhy{}sc\PYG{k}{)}\PYGZus{}all.deb
\end{sphinxVerbatim}

\item {} 
\sphinxAtStartPar
Install the downloaded package with \sphinxstyleliteralstrong{\sphinxupquote{dpkg}}:

\begin{sphinxVerbatim}[commandchars=\\\{\}]
\PYGZdl{} sudo dpkg \PYGZhy{}i percona\PYGZhy{}release\PYGZus{}latest.\PYG{k}{\PYGZdl{}(}lsb\PYGZus{}release \PYGZhy{}sc\PYG{k}{)}\PYGZus{}all.deb
\end{sphinxVerbatim}

\sphinxAtStartPar
Once you install this package the \sphinxstyleemphasis{Percona} repositories should be added. You
can check the repository setup in the
\sphinxcode{\sphinxupquote{/etc/apt/sources.list.d/percona\sphinxhyphen{}release.list}} file.

\item {} 
\sphinxAtStartPar
Enable the repository:

\begin{sphinxVerbatim}[commandchars=\\\{\}]
\PYGZdl{} sudo percona\PYGZhy{}release \PYG{n+nb}{enable} psmdb\PYGZhy{}44 release
\end{sphinxVerbatim}

\item {} 
\sphinxAtStartPar
Remember to update the local cache:

\begin{sphinxVerbatim}[commandchars=\\\{\}]
\PYGZdl{} sudo apt update
\end{sphinxVerbatim}

\end{enumerate}


\paragraph{Install the latest version}
\label{\detokenize{install/apt:install-the-latest-version}}
\sphinxAtStartPar
Run the following command to install the latest version of \sphinxstyleemphasis{Percona Server for MongoDB}:

\begin{sphinxVerbatim}[commandchars=\\\{\}]
\PYGZdl{} sudo apt install percona\PYGZhy{}server\PYGZhy{}mongodb
\end{sphinxVerbatim}


\paragraph{Install a specific version}
\label{\detokenize{install/apt:install-a-specific-version}}
\sphinxAtStartPar
To install a specific version of \sphinxstyleemphasis{Percona Server for MongoDB}, do the following:
\begin{enumerate}
\sphinxsetlistlabels{\arabic}{enumi}{enumii}{}{.}%
\item {} 
\sphinxAtStartPar
List available versions:

\begin{sphinxVerbatim}[commandchars=\\\{\}]
\PYGZdl{} sudo apt\PYGZhy{}cache madison percona\PYGZhy{}server\PYGZhy{}mongodb
\end{sphinxVerbatim}

\begin{sphinxadmonition}{note}{Sample Output}

\begin{sphinxVerbatim}[commandchars=\\\{\}]
percona\PYGZhy{}server\PYGZhy{}mongodb | 4.4.0\PYGZhy{}1.buster | http://repo.percona.com/psmdb\PYGZhy{}44/apt buster/main amd64 Packages
\end{sphinxVerbatim}
\end{sphinxadmonition}

\item {} 
\sphinxAtStartPar
Install a specific version packages. You must specify each package with the version number. For example, to install \sphinxstyleemphasis{Percona Server for MongoDB} 4.4.0\sphinxhyphen{}1, run the following command:

\begin{sphinxVerbatim}[commandchars=\\\{\}]
\PYGZdl{} sudo apt install percona\PYGZhy{}server\PYGZhy{}mongodb\PYG{o}{=}\PYG{l+m}{4}.4.0\PYGZhy{}1.buster percona\PYGZhy{}server\PYGZhy{}mongodb\PYGZhy{}mongos\PYG{o}{=}\PYG{l+m}{4}.4.0\PYGZhy{}1.buster percona\PYGZhy{}server\PYGZhy{}mongodb\PYGZhy{}shell\PYG{o}{=}\PYG{l+m}{4}.4.0\PYGZhy{}1.buster percona\PYGZhy{}server\PYGZhy{}mongodb\PYGZhy{}server\PYG{o}{=}\PYG{l+m}{4}.4.0\PYGZhy{}1.buster percona\PYGZhy{}server\PYGZhy{}mongodb\PYGZhy{}tools\PYG{o}{=}\PYG{l+m}{4}.4.0\PYGZhy{}1.buster
\end{sphinxVerbatim}

\end{enumerate}


\subsubsection{Running Percona Server for MongoDB}
\label{\detokenize{install/apt:running-percona-server-for-mongodb}}
\sphinxAtStartPar
By default, \sphinxstyleemphasis{Percona Server for MongoDB} stores data files in \sphinxcode{\sphinxupquote{/var/lib/mongodb/}}
and configuration parameters in \sphinxcode{\sphinxupquote{/etc/mongod.conf}}.

\sphinxAtStartPar
\sphinxstylestrong{Starting the service}

\sphinxAtStartPar
\sphinxstyleemphasis{Percona Server for MongoDB} is started automatically after installation unless it encounters errors during the installation process.

\sphinxAtStartPar
You can also manually start it using the following command:

\begin{sphinxVerbatim}[commandchars=\\\{\}]
\PYGZdl{} sudo systemctl start mongod
\end{sphinxVerbatim}

\sphinxAtStartPar
\sphinxstylestrong{Confirming that the service is running}

\sphinxAtStartPar
Check the service status using the following command:

\begin{sphinxVerbatim}[commandchars=\\\{\}]
\PYGZdl{} sudo systemctl status mongod
\end{sphinxVerbatim}

\sphinxAtStartPar
\sphinxstylestrong{Stopping the service}

\sphinxAtStartPar
Stop the service using the following command:

\begin{sphinxVerbatim}[commandchars=\\\{\}]
\PYGZdl{} sudo systemctl stop mongod
\end{sphinxVerbatim}

\sphinxAtStartPar
\sphinxstylestrong{Restarting the service}

\sphinxAtStartPar
Restart the service using the following command:

\begin{sphinxVerbatim}[commandchars=\\\{\}]
\PYGZdl{} sudo systemctl restart mongod
\end{sphinxVerbatim}


\subsection{Installing Percona Server for MongoDB on Red Hat Enterprise Linux and CentOS}
\label{\detokenize{install/yum:installing-percona-server-for-mongodb-on-red-hat-enterprise-linux-and-centos}}\label{\detokenize{install/yum:yum}}\label{\detokenize{install/yum::doc}}
\sphinxAtStartPar
Use this document to install Percona Server for MongoDB on RPM\sphinxhyphen{}based distributions from Percona repositories.

\begin{sphinxadmonition}{note}{Note:}
\sphinxAtStartPar
\sphinxstyleemphasis{Percona Server for MongoDB} should work on other RPM\sphinxhyphen{}based distributions
(for example, Amazon Linux AMI and Oracle Linux),
but it is tested only on platforms listed on the \sphinxhref{https://www.percona.com/services/policies/percona-software-platform-lifecycle\#mongodb}{Percona Software and Platform Lifecycle} page. %
\begin{footnote}[1]\sphinxAtStartFootnote
We support only the current stable RHEL 6 and CentOS 6 releases,
because there is no official (i.e. RedHat provided) method to support
or download the latest OpenSSL on RHEL and CentOS versions prior to 6.5.
Similarly, and also as a result thereof,
there is no official Percona way to support the latest Percona Server builds
on RHEL and CentOS versions prior to 6.5.
Additionally, many users will need to upgrade to OpenSSL 1.0.1g or later
(due to the \sphinxhref{http://www.percona.com/resources/ceo-customer-advisory-heartbleed}{Heartbleed vulnerability}),
and this OpenSSL version is not available for download
from any official RHEL and CentOS repositories for versions 6.4 and prior.
For any officially unsupported system, \sphinxcode{\sphinxupquote{src.rpm}} packages can be used
to rebuild Percona Server for any environment.
Please contact our \sphinxhref{http://www.percona.com/products/mysql-support}{support service}
if you require further information on this.
%
\end{footnote}
\end{sphinxadmonition}

\begin{sphinxShadowBox}
\begin{itemize}
\item {} 
\sphinxAtStartPar
\phantomsection\label{\detokenize{install/yum:id2}}{\hyperref[\detokenize{install/yum:package-contents}]{\sphinxcrossref{Package Contents}}}

\item {} 
\sphinxAtStartPar
\phantomsection\label{\detokenize{install/yum:id3}}{\hyperref[\detokenize{install/yum:installing-from-percona-repositories}]{\sphinxcrossref{Installing from Percona repositories}}}
\begin{itemize}
\item {} 
\sphinxAtStartPar
\phantomsection\label{\detokenize{install/yum:id4}}{\hyperref[\detokenize{install/yum:configure-percona-repository}]{\sphinxcrossref{Configure Percona repository}}}

\item {} 
\sphinxAtStartPar
\phantomsection\label{\detokenize{install/yum:id5}}{\hyperref[\detokenize{install/yum:install-the-latest-version}]{\sphinxcrossref{Install the latest version}}}

\item {} 
\sphinxAtStartPar
\phantomsection\label{\detokenize{install/yum:id6}}{\hyperref[\detokenize{install/yum:install-a-specific-version}]{\sphinxcrossref{Install a specific version}}}

\end{itemize}

\item {} 
\sphinxAtStartPar
\phantomsection\label{\detokenize{install/yum:id7}}{\hyperref[\detokenize{install/yum:running-percona-server-for-mongodb}]{\sphinxcrossref{Running Percona Server for MongoDB}}}
\begin{itemize}
\item {} 
\sphinxAtStartPar
\phantomsection\label{\detokenize{install/yum:id8}}{\hyperref[\detokenize{install/yum:running-after-reboot}]{\sphinxcrossref{Running after reboot}}}

\end{itemize}

\end{itemize}
\end{sphinxShadowBox}


\subsubsection{Package Contents}
\label{\detokenize{install/yum:package-contents}}

\begin{savenotes}\sphinxattablestart
\centering
\begin{tabular}[t]{|\X{25}{100}|\X{75}{100}|}
\hline

\sphinxAtStartPar
Package
&
\sphinxAtStartPar
Contains
\\
\hline
\sphinxAtStartPar
percona\sphinxhyphen{}server\sphinxhyphen{}mongodb
&
\sphinxAtStartPar
The \sphinxcode{\sphinxupquote{mongo}} shell, import/export tools, other client
utilities, server software, default configuration, and init.d scripts.
\\
\hline
\sphinxAtStartPar
percona\sphinxhyphen{}server\sphinxhyphen{}mongodb\sphinxhyphen{}server
&
\sphinxAtStartPar
The \sphinxstyleliteralstrong{\sphinxupquote{mongod}} server, default configuration files, and \sphinxcode{\sphinxupquote{init.d}}
scripts
\\
\hline
\sphinxAtStartPar
percona\sphinxhyphen{}server\sphinxhyphen{}mongodb\sphinxhyphen{}shell
&
\sphinxAtStartPar
The \sphinxcode{\sphinxupquote{mongo}} shell
\\
\hline
\sphinxAtStartPar
percona\sphinxhyphen{}server\sphinxhyphen{}mongodb\sphinxhyphen{}mongos
&
\sphinxAtStartPar
The \sphinxcode{\sphinxupquote{mongos}} sharded cluster query router
\\
\hline
\sphinxAtStartPar
percona\sphinxhyphen{}server\sphinxhyphen{}mongodb\sphinxhyphen{}tools
&
\sphinxAtStartPar
Mongo tools for high\sphinxhyphen{}performance MongoDB fork from Percona
\\
\hline
\sphinxAtStartPar
percona\sphinxhyphen{}server\sphinxhyphen{}mongodb\sphinxhyphen{}dbg
&
\sphinxAtStartPar
Debug symbols for the server
\\
\hline
\end{tabular}
\par
\sphinxattableend\end{savenotes}


\subsubsection{Installing from Percona repositories}
\label{\detokenize{install/yum:installing-from-percona-repositories}}
\sphinxAtStartPar
The preferable way to install \sphinxstyleemphasis{Percona Server for MongoDB} is from Percona repositories. Percona repositories are managed using the \sphinxstyleliteralstrong{\sphinxupquote{percona\sphinxhyphen{}release}} tool.


\paragraph{Configure Percona repository}
\label{\detokenize{install/yum:configure-percona-repository}}\begin{enumerate}
\sphinxsetlistlabels{\arabic}{enumi}{enumii}{}{.}%
\item {} 
\sphinxAtStartPar
Install \sphinxstyleliteralstrong{\sphinxupquote{percona\sphinxhyphen{}release}}:

\begin{sphinxVerbatim}[commandchars=\\\{\}]
\PYGZdl{} sudo yum install https://repo.percona.com/yum/percona\PYGZhy{}release\PYGZhy{}latest.noarch.rpm
\end{sphinxVerbatim}

\begin{sphinxadmonition}{note}{Example of Output}

\begin{sphinxVerbatim}[commandchars=\\\{\}]
Retrieving https://repo.percona.com/yum/percona\PYGZhy{}release\PYGZhy{}latest.noarch.rpm
Preparing...                \PYG{c+c1}{\PYGZsh{}\PYGZsh{}\PYGZsh{}\PYGZsh{}\PYGZsh{}\PYGZsh{}\PYGZsh{}\PYGZsh{}\PYGZsh{}\PYGZsh{}\PYGZsh{}\PYGZsh{}\PYGZsh{}\PYGZsh{}\PYGZsh{}\PYGZsh{}\PYGZsh{}\PYGZsh{}\PYGZsh{}\PYGZsh{}\PYGZsh{}\PYGZsh{}\PYGZsh{}\PYGZsh{}\PYGZsh{}\PYGZsh{}\PYGZsh{}\PYGZsh{}\PYGZsh{}\PYGZsh{}\PYGZsh{}\PYGZsh{}\PYGZsh{}\PYGZsh{}\PYGZsh{}\PYGZsh{}\PYGZsh{}\PYGZsh{}\PYGZsh{}\PYGZsh{}\PYGZsh{}\PYGZsh{}\PYGZsh{} [100\PYGZpc{}]}
\PYG{l+m}{1}:percona\PYGZhy{}release        \PYG{c+c1}{\PYGZsh{}\PYGZsh{}\PYGZsh{}\PYGZsh{}\PYGZsh{}\PYGZsh{}\PYGZsh{}\PYGZsh{}\PYGZsh{}\PYGZsh{}\PYGZsh{}\PYGZsh{}\PYGZsh{}\PYGZsh{}\PYGZsh{}\PYGZsh{}\PYGZsh{}\PYGZsh{}\PYGZsh{}\PYGZsh{}\PYGZsh{}\PYGZsh{}\PYGZsh{}\PYGZsh{}\PYGZsh{}\PYGZsh{}\PYGZsh{}\PYGZsh{}\PYGZsh{}\PYGZsh{}\PYGZsh{}\PYGZsh{}\PYGZsh{}\PYGZsh{}\PYGZsh{}\PYGZsh{}\PYGZsh{}\PYGZsh{}\PYGZsh{}\PYGZsh{}\PYGZsh{}\PYGZsh{}\PYGZsh{} [100\PYGZpc{}]}
\end{sphinxVerbatim}
\end{sphinxadmonition}

\item {} 
\sphinxAtStartPar
Enable the repository: \sphinxcode{\sphinxupquote{percona\sphinxhyphen{}release \DUrole{name,builtin}{enable} psmdb\sphinxhyphen{}44 release}}

\end{enumerate}


\sphinxstrong{See also:}
\nopagebreak

\begin{description}
\item[{More information about how to use the \sphinxcode{\sphinxupquote{percona\sphinxhyphen{}release}} tool}] \leavevmode
\sphinxAtStartPar
\sphinxurl{https://www.percona.com/doc/percona-repo-config/index.html}

\end{description}




\paragraph{Install the latest version}
\label{\detokenize{install/yum:install-the-latest-version}}
\sphinxAtStartPar
To install the latest version of \sphinxstyleemphasis{Percona Server for MongoDB}, use the following command:

\begin{sphinxVerbatim}[commandchars=\\\{\}]
\PYGZdl{} sudo yum install percona\PYGZhy{}server\PYGZhy{}mongodb
\end{sphinxVerbatim}


\paragraph{Install a specific version}
\label{\detokenize{install/yum:install-a-specific-version}}
\sphinxAtStartPar
To install a specific version of \sphinxstyleemphasis{Percona Server for MongoDB}, do the following:
\begin{enumerate}
\sphinxsetlistlabels{\arabic}{enumi}{enumii}{}{.}%
\item {} 
\sphinxAtStartPar
List available versions:

\begin{sphinxVerbatim}[commandchars=\\\{\}]
\PYGZdl{} sudo yum list percona\PYGZhy{}server\PYGZhy{}mongodb \PYGZhy{}\PYGZhy{}showduplicates
\end{sphinxVerbatim}

\begin{sphinxadmonition}{note}{Sample Output}

\begin{sphinxVerbatim}[commandchars=\\\{\}]
    Available Packages
percona\PYGZhy{}server\PYGZhy{}mongodb.x86\PYGZus{}64    4.4.0\PYGZhy{}1.el8       psmdb\PYGZhy{}44\PYGZhy{}release\PYGZhy{}x86\PYGZus{}64
\end{sphinxVerbatim}
\end{sphinxadmonition}

\item {} 
\sphinxAtStartPar
Install a specific version packages. For example, to install \sphinxstyleemphasis{Percona Server for MongoDB} 4.4.0\sphinxhyphen{}1, run the following command:

\begin{sphinxVerbatim}[commandchars=\\\{\}]
\PYGZdl{} sudo yum install percona\PYGZhy{}server\PYGZhy{}mongodb\PYGZhy{}4.4.0\PYGZhy{}1.el8
\end{sphinxVerbatim}

\end{enumerate}


\subsubsection{Running Percona Server for MongoDB}
\label{\detokenize{install/yum:running-percona-server-for-mongodb}}
\begin{sphinxadmonition}{note}{Note:}
\sphinxAtStartPar
If you are using SELinux in enforcing mode, you must customize your SELinux user policies to allow access to certain \sphinxcode{\sphinxupquote{/sys}} and \sphinxcode{\sphinxupquote{/proc}} files for OS\sphinxhyphen{}level statistics. Also, you must customize directory and port access policies if you are using non\sphinxhyphen{}default locations.

\sphinxAtStartPar
Please refer to \sphinxhref{https://docs.mongodb.com/v4.4/tutorial/install-mongodb-on-red-hat/\#configure-selinux}{Configure SELinux} section of MongoDB Documentation for policy configuration guidelines.
\end{sphinxadmonition}

\sphinxAtStartPar
\sphinxstyleemphasis{Percona Server for MongoDB} stores data files in \sphinxcode{\sphinxupquote{/var/lib/mongodb/}} by default.
The configuration file is \sphinxcode{\sphinxupquote{/etc/mongod.conf}}.

\sphinxAtStartPar
\sphinxstylestrong{Starting the service}

\sphinxAtStartPar
\sphinxstyleemphasis{Percona Server for MongoDB} is not started automatically after installation. Start it manually using the following command:

\begin{sphinxVerbatim}[commandchars=\\\{\}]
\PYGZdl{} sudo systemctl start mongod
\end{sphinxVerbatim}

\sphinxAtStartPar
\sphinxstylestrong{Confirming that service is running}

\sphinxAtStartPar
Check the service status using the following command: \sphinxcode{\sphinxupquote{systemctl status mongod}}

\begin{sphinxVerbatim}[commandchars=\\\{\}]
\PYGZdl{} sudo systemctl status mongod
\end{sphinxVerbatim}

\sphinxAtStartPar
\sphinxstylestrong{Stopping the service}

\sphinxAtStartPar
Stop the service using the following command: \sphinxcode{\sphinxupquote{systemctl stop mongod}}

\begin{sphinxVerbatim}[commandchars=\\\{\}]
\PYGZdl{} sudo systemctl stop mongod
\end{sphinxVerbatim}

\sphinxAtStartPar
\sphinxstylestrong{Restarting the service}

\sphinxAtStartPar
Restart the service using the following command: \sphinxcode{\sphinxupquote{systemctl restart mongod}}

\begin{sphinxVerbatim}[commandchars=\\\{\}]
\PYGZdl{} sudo systemctl restart mongod
\end{sphinxVerbatim}


\paragraph{Running after reboot}
\label{\detokenize{install/yum:running-after-reboot}}
\sphinxAtStartPar
The \sphinxcode{\sphinxupquote{mongod}} service is not automatically started
after you reboot the system.

\sphinxAtStartPar
For RHEL or CentOS versions 5 and 6, you can use the \sphinxcode{\sphinxupquote{chkconfig}} utility
to enable auto\sphinxhyphen{}start as follows:

\begin{sphinxVerbatim}[commandchars=\\\{\}]
\PYGZdl{} sudo chkconfig \PYGZhy{}\PYGZhy{}add mongod
\end{sphinxVerbatim}

\sphinxAtStartPar
For RHEL or CentOS version 7, you can use the \sphinxcode{\sphinxupquote{systemctl}} utility:

\begin{sphinxVerbatim}[commandchars=\\\{\}]
\PYGZdl{} sudo systemctl \PYG{n+nb}{enable} mongod
\end{sphinxVerbatim}


\subsection{Installing Percona Server for MongoDB from Binary Tarball}
\label{\detokenize{install/tarball:installing-percona-server-for-mongodb-from-binary-tarball}}\label{\detokenize{install/tarball:tarball}}\label{\detokenize{install/tarball::doc}}
\sphinxAtStartPar
You can find links to the binary tarballs under the \sphinxstyleemphasis{Generic Linux} menu item on the \sphinxhref{https://www.percona.com/downloads/percona-server-mongodb-4.4/}{Percona website}

\sphinxAtStartPar
There are two tarballs available:
\begin{itemize}
\item {} 
\sphinxAtStartPar
\sphinxcode{\sphinxupquote{percona\sphinxhyphen{}server\sphinxhyphen{}mongodb\sphinxhyphen{}\textless{}version\textgreater{}\sphinxhyphen{}x86\_64.glibc2.17.tar.gz}} is the general tarball, compatible with any \sphinxhref{https://www.percona.com/services/policies/percona-software-support-lifecycle\#mongodb}{supported operating system} except Ubuntu 22.04.

\item {} 
\sphinxAtStartPar
\sphinxcode{\sphinxupquote{percona\sphinxhyphen{}server\sphinxhyphen{}mongodb\sphinxhyphen{}\textless{}version\textgreater{}\sphinxhyphen{}x86\_64.glibc2.35.tar.gz}} is the tarball for Ubuntu 22.04.

\end{itemize}


\subsubsection{Preconditions}
\label{\detokenize{install/tarball:preconditions}}
\sphinxAtStartPar
The following packages are required for the installation.
\begin{quote}

\begin{sphinxadmonition}{note}{On Debian / Ubuntu}
\begin{itemize}
\item {} 
\sphinxAtStartPar
\sphinxcode{\sphinxupquote{libcurl4}}

\item {} 
\sphinxAtStartPar
\sphinxcode{\sphinxupquote{libsasl2\sphinxhyphen{}modules}}

\item {} 
\sphinxAtStartPar
\sphinxcode{\sphinxupquote{libsasl2\sphinxhyphen{}modules\sphinxhyphen{}gssapi\sphinxhyphen{}mit}}

\end{itemize}
\end{sphinxadmonition}

\begin{sphinxadmonition}{note}{On Red Hat Enterprise Linux and derivatives}
\begin{itemize}
\item {} 
\sphinxAtStartPar
\sphinxcode{\sphinxupquote{libcurl}}

\item {} 
\sphinxAtStartPar
\sphinxcode{\sphinxupquote{cyrus\sphinxhyphen{}sasl\sphinxhyphen{}gssapi}}

\item {} 
\sphinxAtStartPar
\sphinxcode{\sphinxupquote{cyrus\sphinxhyphen{}sasl\sphinxhyphen{}plain}}

\end{itemize}
\end{sphinxadmonition}
\end{quote}

\sphinxAtStartPar
Check that they are installed in your operating system. Otherwise install them.


\subsubsection{Procedure}
\label{\detokenize{install/tarball:procedure}}\begin{enumerate}
\sphinxsetlistlabels{\arabic}{enumi}{enumii}{}{.}%
\item {} 
\sphinxAtStartPar
Fetch and extract the correct binary tarball. For example, if you
are running Debian 10 (“buster”):

\begin{sphinxVerbatim}[commandchars=\\\{\}]
\PYGZdl{} wget https://downloads.percona.com/downloads/percona\PYGZhy{}server\PYGZhy{}mongodb\PYGZhy{}4.4/percona\PYGZhy{}server\PYGZhy{}mongodb\PYGZhy{}4.4.15\PYGZhy{}15/binary/tarball/percona\PYGZhy{}server\PYGZhy{}mongodb\PYGZhy{}4.4.15\PYGZhy{}15\PYGZhy{}x86\PYGZus{}64.glibc2.17.tar.gz
\PYGZdl{} tar \PYGZhy{}xf percona\PYGZhy{}server\PYGZhy{}mongodb\PYGZhy{}4.4.15\PYGZhy{}15\PYGZhy{}x86\PYGZus{}64.glibc2.17.tar.gz
\end{sphinxVerbatim}

\item {} 
\sphinxAtStartPar
Add the location of the binaries to the \sphinxcode{\sphinxupquote{PATH}} variable:

\begin{sphinxVerbatim}[commandchars=\\\{\}]
\PYGZdl{} \PYG{n+nb}{export} \PYG{n+nv}{PATH}\PYG{o}{=}\PYGZti{}/percona\PYGZhy{}server\PYGZhy{}mongodb\PYGZhy{}4.4.15\PYGZhy{}15/bin/:\PYG{n+nv}{\PYGZdl{}PATH}
\end{sphinxVerbatim}

\item {} 
\sphinxAtStartPar
Create the default data directory:

\begin{sphinxVerbatim}[commandchars=\\\{\}]
\PYGZdl{} mkdir \PYGZhy{}p /data/db
\end{sphinxVerbatim}

\item {} 
\sphinxAtStartPar
Make sure that you have read and write permissions for the data
directory and run \sphinxstyleliteralstrong{\sphinxupquote{mongod}}.

\end{enumerate}


\subsection{Running \sphinxstyleemphasis{Percona Server for MongoDB} in a Docker Container}
\label{\detokenize{install/docker:running-psmdb-in-a-docker-container}}\label{\detokenize{install/docker:docker}}\label{\detokenize{install/docker::doc}}
\sphinxAtStartPar
Docker images of \sphinxstyleemphasis{Percona Server for MongoDB} are hosted publicly on Docker Hub at
\sphinxurl{https://hub.docker.com/r/percona/percona-server-mongodb/}.

\sphinxAtStartPar
For more information about using Docker, see the \sphinxhref{https://docs.docker.com/}{Docker Docs}.

\begin{sphinxadmonition}{note}{Note:}
\sphinxAtStartPar
Make sure that you are using the latest version of Docker.  The ones provided
via \sphinxcode{\sphinxupquote{apt}} and \sphinxcode{\sphinxupquote{yum}} may be outdated and cause errors.
\end{sphinxadmonition}

\begin{sphinxadmonition}{note}{Note:}
\sphinxAtStartPar
By default, Docker will pull the image from Docker Hub
if it is not available locally.
\end{sphinxadmonition}

\sphinxAtStartPar
To run the latest \sphinxstyleemphasis{Percona Server for MongoDB} 4.4 in a Docker container, use the following command:

\sphinxAtStartPar
Run this command as root or by using the \sphinxstyleliteralstrong{\sphinxupquote{sudo}} command

\begin{sphinxVerbatim}[commandchars=\\\{\}]
\PYGZdl{} docker run \PYGZhy{}d \PYGZhy{}\PYGZhy{}name psmdb \PYGZhy{}\PYGZhy{}restart always \PYG{l+s+se}{\PYGZbs{}}
percona/percona\PYGZhy{}server\PYGZhy{}mongodb:4.4
\end{sphinxVerbatim}

\sphinxAtStartPar
The previous command does the following:
\begin{itemize}
\item {} 
\sphinxAtStartPar
The \sphinxcode{\sphinxupquote{docker run}} command instructs the \sphinxcode{\sphinxupquote{docker}} daemon
to run a container from an image.

\item {} 
\sphinxAtStartPar
The \sphinxcode{\sphinxupquote{\sphinxhyphen{}d}} option starts the container in detached mode
(that is, in the background).

\item {} 
\sphinxAtStartPar
The \sphinxcode{\sphinxupquote{\sphinxhyphen{}\sphinxhyphen{}name}} option assigns a custom name for the container
that you can use to reference the container within a Docker network.
In this case: \sphinxcode{\sphinxupquote{psmdb}}.

\item {} 
\sphinxAtStartPar
The \sphinxcode{\sphinxupquote{\sphinxhyphen{}\sphinxhyphen{}restart}} option defines the container’s restart policy.
Setting it to \sphinxcode{\sphinxupquote{always}} ensures that the Docker daemon
will start the container on startup
and restart it if the container exits.

\item {} 
\sphinxAtStartPar
\sphinxcode{\sphinxupquote{percona/percona\sphinxhyphen{}server\sphinxhyphen{}mongodb:4.4}} is the name and version tag
of the image to derive the container from.

\end{itemize}


\sphinxstrong{See also:}
\nopagebreak

\begin{description}
\item[{Docker Documentation: the full list of tags}] \leavevmode
\sphinxAtStartPar
\sphinxurl{https://hub.docker.com/r/percona/percona-server-mongodb/tags/}

\end{description}




\subsubsection{Connecting from Another Docker Container}
\label{\detokenize{install/docker:connecting-from-another-docker-container}}
\sphinxAtStartPar
The \sphinxstyleemphasis{Percona Server for MongoDB} container exposes standard MongoDB port (27017),
which can be used for connection from an application
running in another container.
To link the application container to the \sphinxcode{\sphinxupquote{psmdb}} container,
use the \sphinxcode{\sphinxupquote{\sphinxhyphen{}\sphinxhyphen{}link psmdb}} option when running the container with your app.


\subsubsection{Connecting with the Mongo Shell}
\label{\detokenize{install/docker:connecting-with-the-mongo-shell}}
\sphinxAtStartPar
To start another container with the \sphinxcode{\sphinxupquote{mongo}} shell
that connects to your \sphinxstyleemphasis{Percona Server for MongoDB} container,
run the following comand: \sphinxcode{\sphinxupquote{docker run \sphinxhyphen{}it \sphinxhyphen{}\sphinxhyphen{}link psmdb \sphinxhyphen{}\sphinxhyphen{}rm percona/percona\sphinxhyphen{}server\sphinxhyphen{}mongodb:mongo mongo \sphinxhyphen{}h psmdb}}


\part{Features}
\label{\detokenize{index:features}}

\chapter{Percona Memory Engine}
\label{\detokenize{inmemory:percona-memory-engine}}\label{\detokenize{inmemory:inmemory}}\label{\detokenize{inmemory::doc}}
\sphinxAtStartPar
\sphinxstyleemphasis{Percona Memory Engine} is a special configuration of \sphinxhref{https://docs.mongodb.org/manual/core/wiredtiger/}{WiredTiger} that does
not store user data on disk.  Data fully resides in the main memory, making
processing much faster and smoother.  Keep in mind that you need to have enough
memory to hold the data set, and ensure that the server does not shut down.

\begin{sphinxShadowBox}
\begin{itemize}
\item {} 
\sphinxAtStartPar
\phantomsection\label{\detokenize{inmemory:id3}}{\hyperref[\detokenize{inmemory:using-percona-memory-engine}]{\sphinxcrossref{Using Percona Memory Engine}}}

\item {} 
\sphinxAtStartPar
\phantomsection\label{\detokenize{inmemory:id4}}{\hyperref[\detokenize{inmemory:switching-storage-engines}]{\sphinxcrossref{Switching storage engines}}}

\item {} 
\sphinxAtStartPar
\phantomsection\label{\detokenize{inmemory:id5}}{\hyperref[\detokenize{inmemory:configuring-percona-memory-engine}]{\sphinxcrossref{Configuring Percona Memory Engine}}}

\end{itemize}
\end{sphinxShadowBox}

\sphinxAtStartPar
The \sphinxstyleemphasis{Percona Memory Engine} is available in \sphinxstyleemphasis{Percona Server for MongoDB} along with the default
MongoDB engine \sphinxhref{https://docs.mongodb.org/manual/core/wiredtiger/}{WiredTiger}.


\section{Using Percona Memory Engine}
\label{\detokenize{inmemory:using-percona-memory-engine}}
\sphinxAtStartPar
As of version 3.2, \sphinxstyleemphasis{Percona Server for MongoDB} runs with \sphinxhref{https://docs.mongodb.org/manual/core/wiredtiger/}{WiredTiger} by default.  You can select a
storage engine using the \sphinxcode{\sphinxupquote{\sphinxhyphen{}\sphinxhyphen{}storageEngine}} command\sphinxhyphen{}line option when you start
\sphinxcode{\sphinxupquote{mongod}}.  Alternatively, you can set the \sphinxcode{\sphinxupquote{storage.engine}} variable in the
configuration file (by default, \sphinxcode{\sphinxupquote{/etc/mongod.conf}}):

\begin{sphinxVerbatim}[commandchars=\\\{\}]
\PYG{n+nt}{storage}\PYG{p}{:}
\PYG{+w}{  }\PYG{n+nt}{dbPath}\PYG{p}{:}\PYG{+w}{ }\PYG{l+lScalar+lScalarPlain}{\PYGZlt{}dataDir\PYGZgt{}}
\PYG{+w}{  }\PYG{n+nt}{engine}\PYG{p}{:}\PYG{+w}{ }\PYG{l+lScalar+lScalarPlain}{inMemory}
\end{sphinxVerbatim}


\sphinxstrong{See also:}
\nopagebreak

\begin{description}
\item[{MongoDB Documentation: Configuration File Options}] \leavevmode\begin{itemize}
\item {} 
\sphinxAtStartPar
\sphinxhref{https://docs.mongodb.com/manual/reference/configuration-options/\#storage.engine}{storage.engine Options}

\item {} 
\sphinxAtStartPar
\sphinxhref{https://docs.mongodb.com/manual/reference/configuration-options/\#storage-wiredtiger-options}{storage.wiredTiger Options}

\item {} 
\sphinxAtStartPar
\sphinxhref{https://docs.mongodb.com/manual/reference/configuration-options/\#storage-inmemory-options}{storage.inmemory Options}

\end{itemize}

\end{description}




\section{Switching storage engines}
\label{\detokenize{inmemory:switching-storage-engines}}\label{\detokenize{inmemory:switch-storage-engines}}

\subsection{Considerations}
\label{\detokenize{inmemory:considerations}}
\sphinxAtStartPar
If you have data files in your database and want to change to Percona Memory Engine, consider the following:
\begin{itemize}
\item {} 
\sphinxAtStartPar
Data files created by one storage engine are not compatible with other engines, because each one has its own data model.

\item {} 
\sphinxAtStartPar
When changing the storage engine, the \sphinxstyleliteralstrong{\sphinxupquote{mongod}} node requires an empty \sphinxcode{\sphinxupquote{dbPath}} data directory when it is restarted. Though Percona Memory Engine stores all data in memory, some metadata files, diagnostics logs and statistics metrics are still written to disk. This is controlled with the {\hyperref[\detokenize{inmemory:cmdoption-inMemoryStatisticsLogDelaySecs}]{\sphinxcrossref{\sphinxcode{\sphinxupquote{\sphinxhyphen{}\sphinxhyphen{}inMemoryStatisticsLogDelaySecs}}}}} option.

\sphinxAtStartPar
Creating a new \sphinxcode{\sphinxupquote{dbPath}} data directory for a different storage engine is the simplest solution. Yet when you switch between disk\sphinxhyphen{}using storage engines (e.g. from \sphinxhref{https://docs.mongodb.org/manual/core/wiredtiger/}{WiredTiger} to {\hyperref[\detokenize{inmemory:inmemory}]{\sphinxcrossref{\DUrole{std,std-ref}{Percona Memory Engine}}}}), you may have to delete the old data if there is not enough disk space for both. Double\sphinxhyphen{}check that your backups are solid and/or the replica set nodes are healthy before you switch to the new storage engine.

\end{itemize}


\subsection{Procedure}
\label{\detokenize{inmemory:procedure}}
\sphinxAtStartPar
To change a storage engine, you have the following options:
\begin{itemize}
\item {} 
\sphinxAtStartPar
If you simply want to temporarily test Percona Memory Engine, set a different
data directory for the \sphinxcode{\sphinxupquote{dbPath}} variable in the configuration file.
Make sure that the user running \sphinxstyleliteralstrong{\sphinxupquote{mongod}} has read and write
permissions for the new data directory.

\begin{sphinxVerbatim}[commandchars=\\\{\}]
\PYGZdl{} service mongod stop
\PYGZdl{} \PYG{c+c1}{\PYGZsh{} In the configuration file, set the inmemory}
\PYGZdl{} \PYG{c+c1}{\PYGZsh{} value for the storage.engine variable}
\PYGZdl{} \PYG{c+c1}{\PYGZsh{} Set the \PYGZlt{}newDataDir\PYGZgt{} for the dbPath variable}
\PYGZdl{} service mongod start
\end{sphinxVerbatim}

\item {} 
\sphinxAtStartPar
If you want to permanently switch to Percona Memory Engine and do not have any
valuable data in your database, clean out the \sphinxcode{\sphinxupquote{dbPath}} data directory
(by default, \sphinxcode{\sphinxupquote{/var/lib/mongodb}}) and edit the configuration file:

\begin{sphinxVerbatim}[commandchars=\\\{\}]
\PYGZdl{} service mongod stop
\PYGZdl{} rm \PYGZhy{}rf \PYGZlt{}dbpathDataDir\PYGZgt{}
\PYGZdl{} \PYG{c+c1}{\PYGZsh{} Update the configuration file by setting the new}
\PYGZdl{} \PYG{c+c1}{\PYGZsh{} value for the storage.engine variable}
\PYGZdl{} \PYG{c+c1}{\PYGZsh{} set the engine\PYGZhy{}specific settings such as}
\PYGZdl{} \PYG{c+c1}{\PYGZsh{} storage.inMemory.engineConfig.inMemorySizeGB}
\PYGZdl{} service mongod start
\end{sphinxVerbatim}

\item {} 
\sphinxAtStartPar
If there is data that you want to migrate
and make compatible with Percona Memory Engine,
use the following methods:
\begin{itemize}
\item {} 
\sphinxAtStartPar
for replica sets, use the “rolling restart” process.
Switch to the Percona Memory Engine on the secondary node. Clean out the \sphinxcode{\sphinxupquote{dbPath}} data directory and edit the configuration file:

\begin{sphinxVerbatim}[commandchars=\\\{\}]
\PYGZdl{} service mongod stop
\PYGZdl{} rm \PYGZhy{}rf \PYGZlt{}dbpathDataDir\PYGZgt{}
\PYGZdl{} \PYG{c+c1}{\PYGZsh{} Update the configuration file by setting the new}
\PYGZdl{} \PYG{c+c1}{\PYGZsh{} value for the storage.engine variable}
\PYGZdl{} \PYG{c+c1}{\PYGZsh{} set the engine\PYGZhy{}specific settings such as}
\PYGZdl{} \PYG{c+c1}{\PYGZsh{} storage.inMemory.engineConfig.inMemorySizeGB}
\PYGZdl{} service mongod start
\end{sphinxVerbatim}

\sphinxAtStartPar
Wait for the node to rejoin with the other nodes and report the SECONDARY status.

\sphinxAtStartPar
Repeat the procedure to switch the remaining nodes to Percona Memory Engine.

\item {} 
\sphinxAtStartPar
for a standalone instance or a single\sphinxhyphen{}node replica set, use the \sphinxcode{\sphinxupquote{mongodump}} and \sphinxcode{\sphinxupquote{mongorestore}} utilities:

\begin{sphinxVerbatim}[commandchars=\\\{\}]
\PYGZdl{} mongodump \PYGZhy{}\PYGZhy{}out \PYGZlt{}dumpDir\PYGZgt{}
\PYGZdl{} service mongod stop
\PYGZdl{} rm \PYGZhy{}rf \PYGZlt{}dbpathDataDir\PYGZgt{}
\PYGZdl{} \PYG{c+c1}{\PYGZsh{} Update the configuration file by setting the new}
\PYGZdl{} \PYG{c+c1}{\PYGZsh{} value for the storage.engine variable}
\PYGZdl{} \PYG{c+c1}{\PYGZsh{} set the engine\PYGZhy{}specific settings such as}
\PYGZdl{} \PYG{c+c1}{\PYGZsh{} storage.inMemory.engineConfig.inMemorySizeGB}
\PYGZdl{} service mongod start
\PYGZdl{} mongorestore \PYGZlt{}dumpDir\PYGZgt{}
\end{sphinxVerbatim}

\end{itemize}

\end{itemize}


\subsection{Switching engines with encrypted data}
\label{\detokenize{inmemory:switching-engines-with-encrypted-data}}
\sphinxAtStartPar
Using {\hyperref[\detokenize{data-at-rest-encryption:psmdb-data-at-rest-encryption}]{\sphinxcrossref{\DUrole{std,std-ref}{Data at Rest Encryption}}}} means using the same \sphinxcode{\sphinxupquote{storage.*}}
configuration options as for \sphinxhref{https://docs.mongodb.org/manual/core/wiredtiger/}{WiredTiger}. To change from normal to {\hyperref[\detokenize{data-at-rest-encryption:psmdb-data-at-rest-encryption}]{\sphinxcrossref{\DUrole{std,std-ref}{Data at Rest Encryption}}}}
mode or backward, you must clean up the \sphinxcode{\sphinxupquote{dbPath}} data directory,
just as if you change the storage engine. This is because
\sphinxstyleliteralstrong{\sphinxupquote{mongod}} cannot convert the data files to an encrypted format ‘in place’. It
must get the document data again either via the initial sync from another
replica set member, or from imported backup dump.


\section{Configuring Percona Memory Engine}
\label{\detokenize{inmemory:configuring-percona-memory-engine}}
\sphinxAtStartPar
You can configure the Percona Memory Engine using either command\sphinxhyphen{}line options or
corresponding parameters in the \sphinxcode{\sphinxupquote{/etc/mongod.conf}} file.  The
configuration file is formatted in YAML. For example:

\begin{sphinxVerbatim}[commandchars=\\\{\}]
\PYG{n+nt}{storage}\PYG{p}{:}
\PYG{+w}{  }\PYG{n+nt}{engine}\PYG{p}{:}\PYG{+w}{ }\PYG{l+lScalar+lScalarPlain}{inMemory}
\PYG{+w}{  }\PYG{n+nt}{inMemory}\PYG{p}{:}
\PYG{+w}{    }\PYG{n+nt}{engineConfig}\PYG{p}{:}
\PYG{+w}{      }\PYG{n+nt}{inMemorySizeGB}\PYG{p}{:}\PYG{+w}{ }\PYG{l+lScalar+lScalarPlain}{140}
\PYG{+w}{      }\PYG{n+nt}{statisticsLogDelaySecs}\PYG{p}{:}\PYG{+w}{ }\PYG{l+lScalar+lScalarPlain}{0}
\end{sphinxVerbatim}

\sphinxAtStartPar
Setting parameters in the previous example configuration file is the same as
starting the \sphinxcode{\sphinxupquote{mongod}} daemon with the following options:

\begin{sphinxVerbatim}[commandchars=\\\{\}]
\PYGZdl{} mongod \PYGZhy{}\PYGZhy{}storageEngine\PYG{o}{=}inMemory \PYG{l+s+se}{\PYGZbs{}}
\PYGZhy{}\PYGZhy{}inMemorySizeGB\PYG{o}{=}\PYG{l+m}{140} \PYG{l+s+se}{\PYGZbs{}}
\PYGZhy{}\PYGZhy{}inMemoryStatisticsLogDelaySecs\PYG{o}{=}\PYG{l+m}{0}
\end{sphinxVerbatim}

\sphinxAtStartPar
The following options are available (with corresponding YAML configuration file
parameters):
\index{command line option@\spxentry{command line option}!\sphinxhyphen{}\sphinxhyphen{}inMemorySizeGB@\spxentry{\sphinxhyphen{}\sphinxhyphen{}inMemorySizeGB}}\index{\sphinxhyphen{}\sphinxhyphen{}inMemorySizeGB@\spxentry{\sphinxhyphen{}\sphinxhyphen{}inMemorySizeGB}!command line option@\spxentry{command line option}}

\begin{fulllineitems}
\phantomsection\label{\detokenize{inmemory:cmdoption-inMemorySizeGB}}\pysigline{\sphinxbfcode{\sphinxupquote{\sphinxhyphen{}\sphinxhyphen{}inMemorySizeGB}}\sphinxcode{\sphinxupquote{}}}~\begin{quote}\begin{description}
\item[{Config}] \leavevmode
\sphinxAtStartPar
\sphinxcode{\sphinxupquote{storage.inMemory.engineConfig.inMemorySizeGB}}

\item[{Default}] \leavevmode
\sphinxAtStartPar
50\% of total memory minus 1024 MB, but not less than 256 MB

\end{description}\end{quote}

\sphinxAtStartPar
Specifies the maximum memory in gigabytes to use for data.

\end{fulllineitems}

\index{command line option@\spxentry{command line option}!\sphinxhyphen{}\sphinxhyphen{}inMemoryStatisticsLogDelaySecs@\spxentry{\sphinxhyphen{}\sphinxhyphen{}inMemoryStatisticsLogDelaySecs}}\index{\sphinxhyphen{}\sphinxhyphen{}inMemoryStatisticsLogDelaySecs@\spxentry{\sphinxhyphen{}\sphinxhyphen{}inMemoryStatisticsLogDelaySecs}!command line option@\spxentry{command line option}}

\begin{fulllineitems}
\phantomsection\label{\detokenize{inmemory:cmdoption-inMemoryStatisticsLogDelaySecs}}\pysigline{\sphinxbfcode{\sphinxupquote{\sphinxhyphen{}\sphinxhyphen{}inMemoryStatisticsLogDelaySecs}}\sphinxcode{\sphinxupquote{}}}~\begin{quote}\begin{description}
\item[{Config}] \leavevmode
\sphinxAtStartPar
\sphinxcode{\sphinxupquote{storage.inMemory.engineConfig.statisticsLogDelaySecs}}

\item[{Default}] \leavevmode
\sphinxAtStartPar
0

\end{description}\end{quote}

\sphinxAtStartPar
Specifies the number of seconds between writes to statistics log.  If 0 is
specified then statistics are not logged.

\end{fulllineitems}



\chapter{Hot Backup}
\label{\detokenize{hot-backup:hot-backup}}\label{\detokenize{hot-backup:id1}}\label{\detokenize{hot-backup::doc}}
\sphinxAtStartPar
\sphinxstyleemphasis{Percona Server for MongoDB} includes an integrated open source hot backup system for the default
\sphinxhref{https://docs.mongodb.org/manual/core/wiredtiger/}{WiredTiger} storage engine.  It creates a physical data backup on a running
server without notable performance and operating degradation.

\begin{sphinxadmonition}{note}{Note:}
\sphinxAtStartPar
Hot backups are done on \sphinxcode{\sphinxupquote{mongod}} servers independently, without synchronizing them across replica set members and shards in a cluster. To ensure data consistency during backups and restores, we recommend using \sphinxhref{https://docs.percona.com/percona-backup-mongodb/index.html}{Percona Backup for MongoDB}.
\end{sphinxadmonition}

\begin{sphinxShadowBox}
\begin{itemize}
\item {} 
\sphinxAtStartPar
\phantomsection\label{\detokenize{hot-backup:id3}}{\hyperref[\detokenize{hot-backup:making-a-backup}]{\sphinxcrossref{Making a backup}}}

\item {} 
\sphinxAtStartPar
\phantomsection\label{\detokenize{hot-backup:id4}}{\hyperref[\detokenize{hot-backup:saving-a-backup-to-a-tar-archive}]{\sphinxcrossref{Saving a backup to a TAR archive}}}

\item {} 
\sphinxAtStartPar
\phantomsection\label{\detokenize{hot-backup:id5}}{\hyperref[\detokenize{hot-backup:view-backup-status}]{\sphinxcrossref{View backup status}}}

\item {} 
\sphinxAtStartPar
\phantomsection\label{\detokenize{hot-backup:id6}}{\hyperref[\detokenize{hot-backup:streaming-hot-backups-to-a-remote-destination}]{\sphinxcrossref{Streaming hot backups to a remote destination}}}
\begin{itemize}
\item {} 
\sphinxAtStartPar
\phantomsection\label{\detokenize{hot-backup:id7}}{\hyperref[\detokenize{hot-backup:credentials}]{\sphinxcrossref{Credentials}}}

\item {} 
\sphinxAtStartPar
\phantomsection\label{\detokenize{hot-backup:id8}}{\hyperref[\detokenize{hot-backup:examples}]{\sphinxcrossref{Examples}}}

\end{itemize}

\item {} 
\sphinxAtStartPar
\phantomsection\label{\detokenize{hot-backup:id9}}{\hyperref[\detokenize{hot-backup:restoring-data-from-backup}]{\sphinxcrossref{Restoring data from backup}}}

\end{itemize}
\end{sphinxShadowBox}


\section{Making a backup}
\label{\detokenize{hot-backup:making-a-backup}}
\sphinxAtStartPar
To take a hot backup of the database in your current \sphinxcode{\sphinxupquote{dbpath}}, do the following:
\begin{itemize}
\item {} 
\sphinxAtStartPar
Make sure to provide access to the backup directory for the \sphinxcode{\sphinxupquote{mongod}} user

\begin{sphinxVerbatim}[commandchars=\\\{\}]
chown mongod:mongod \PYGZlt{}backupDir\PYGZgt{}
\end{sphinxVerbatim}

\item {} 
\sphinxAtStartPar
Run the \sphinxcode{\sphinxupquote{createBackup}} command as administrator on the \sphinxcode{\sphinxupquote{admin}} database and specify the backup directory.

\begin{sphinxVerbatim}[commandchars=\\\{\}]
\PYG{o}{\PYGZgt{}}\PYG{+w}{ }\PYG{n+nx}{use}\PYG{+w}{ }\PYG{n+nx}{admin}
\PYG{n+nx}{switched}\PYG{+w}{ }\PYG{n+nx}{to}\PYG{+w}{ }\PYG{n+nx}{db}\PYG{+w}{ }\PYG{n+nx}{admin}
\PYG{o}{\PYGZgt{}}\PYG{+w}{ }\PYG{n+nx}{db}\PYG{p}{.}\PYG{n+nx}{runCommand}\PYG{p}{(}\PYG{p}{\PYGZob{}}\PYG{n+nx}{createBackup}\PYG{o}{:}\PYG{+w}{ }\PYG{l+m+mf}{1}\PYG{p}{,}\PYG{+w}{ }\PYG{n+nx}{backupDir}\PYG{o}{:}\PYG{+w}{ }\PYG{l+s+s2}{\PYGZdq{}\PYGZlt{}backup\PYGZus{}data\PYGZus{}path\PYGZgt{}\PYGZdq{}}\PYG{p}{\PYGZcb{}}\PYG{p}{)}
\PYG{p}{\PYGZob{}}\PYG{+w}{ }\PYG{l+s+s2}{\PYGZdq{}ok\PYGZdq{}}\PYG{+w}{ }\PYG{o}{:}\PYG{+w}{ }\PYG{l+m+mf}{1}\PYG{+w}{ }\PYG{p}{\PYGZcb{}}
\end{sphinxVerbatim}

\end{itemize}

\sphinxAtStartPar
The backup taken is the snapshot of the \sphinxcode{\sphinxupquote{mongod}} server’s \sphinxcode{\sphinxupquote{dataDir}} at the moment of the \sphinxcode{\sphinxupquote{createBackup}} command start.

\sphinxAtStartPar
If the backup was successful, you should receive an \sphinxcode{\sphinxupquote{\{ "ok" : 1 \}}} object.
If there was an error, you will receive a failing \sphinxcode{\sphinxupquote{ok}} status
with the error message, for example:

\begin{sphinxVerbatim}[commandchars=\\\{\}]
\PYG{o}{\PYGZgt{}}\PYG{+w}{ }\PYG{n+nx}{db}\PYG{p}{.}\PYG{n+nx}{runCommand}\PYG{p}{(}\PYG{p}{\PYGZob{}}\PYG{n+nx}{createBackup}\PYG{o}{:}\PYG{+w}{ }\PYG{l+m+mf}{1}\PYG{p}{,}\PYG{+w}{ }\PYG{n+nx}{backupDir}\PYG{o}{:}\PYG{+w}{ }\PYG{l+s+s2}{\PYGZdq{}\PYGZdq{}}\PYG{p}{\PYGZcb{}}\PYG{p}{)}
\PYG{p}{\PYGZob{}}\PYG{+w}{ }\PYG{l+s+s2}{\PYGZdq{}ok\PYGZdq{}}\PYG{+w}{ }\PYG{o}{:}\PYG{+w}{ }\PYG{l+m+mf}{0}\PYG{p}{,}\PYG{+w}{ }\PYG{l+s+s2}{\PYGZdq{}errmsg\PYGZdq{}}\PYG{+w}{ }\PYG{o}{:}\PYG{+w}{ }\PYG{l+s+s2}{\PYGZdq{}Destination path must be absolute\PYGZdq{}}\PYG{+w}{ }\PYG{p}{\PYGZcb{}}
\end{sphinxVerbatim}


\section{Saving a backup to a TAR archive}
\label{\detokenize{hot-backup:saving-a-backup-to-a-tar-archive}}
\begin{sphinxadmonition}{note}{Implementation details}

\sphinxAtStartPar
This feature was implemented in \sphinxstyleemphasis{Percona Server for MongoDB} 4.2.1\sphinxhyphen{}1.
\end{sphinxadmonition}

\sphinxAtStartPar
To save a backup as a \sphinxstyleemphasis{tar} archive, use the \sphinxstyleemphasis{archive} field to
specify the destination path:

\begin{sphinxVerbatim}[commandchars=\\\{\}]
\PYG{o}{\PYGZgt{}}\PYG{+w}{ }\PYG{n+nx}{use}\PYG{+w}{ }\PYG{n+nx}{admin}
\PYG{p}{...}
\PYG{o}{\PYGZgt{}}\PYG{+w}{ }\PYG{n+nx}{db}\PYG{p}{.}\PYG{n+nx}{runCommand}\PYG{p}{(}\PYG{p}{\PYGZob{}}\PYG{n+nx}{createBackup}\PYG{o}{:}\PYG{+w}{ }\PYG{l+m+mf}{1}\PYG{p}{,}\PYG{+w}{ }\PYG{n+nx}{archive}\PYG{o}{:}\PYG{+w}{ }\PYG{o}{\PYGZlt{}}\PYG{n+nx}{path\PYGZus{}to\PYGZus{}archive}\PYG{o}{\PYGZgt{}}\PYG{p}{.}\PYG{n+nx}{tar}\PYG{+w}{ }\PYG{p}{\PYGZcb{}}\PYG{p}{)}
\end{sphinxVerbatim}


\section{View backup status}
\label{\detokenize{hot-backup:view-backup-status}}
\sphinxAtStartPar
As of version 4.4.8\sphinxhyphen{}9, you can view the backup status using the \sphinxhref{https://docs.mongodb.com/manual/reference/operator/aggregation/currentOp/}{\$currentOp} aggregation stage. You can also use the \sphinxhref{https://docs.mongodb.com/manual/reference/command/currentOp/\#mongodb-dbcommand-dbcmd.currentOp}{currentOp} command though it is considered legacy.

\sphinxAtStartPar
Run the \sphinxcode{\sphinxupquote{\$currentOp}} against the \sphinxcode{\sphinxupquote{admin}} database:

\begin{sphinxVerbatim}[commandchars=\\\{\}]
\PYG{n+nx}{db}\PYG{p}{.}\PYG{n+nx}{getSiblingDB}\PYG{p}{(}\PYG{l+s+s2}{\PYGZdq{}admin\PYGZdq{}}\PYG{p}{)}\PYG{p}{.}\PYG{n+nx}{aggregate}\PYG{p}{(}\PYG{p}{[}
\PYG{+w}{   }\PYG{p}{\PYGZob{}}\PYG{+w}{ }\PYG{n+nx}{\PYGZdl{}currentOp}\PYG{o}{:}\PYG{+w}{ }\PYG{p}{\PYGZob{}}\PYG{p}{\PYGZcb{}}\PYG{+w}{ }\PYG{p}{\PYGZcb{}}\PYG{p}{,}
\PYG{+w}{   }\PYG{p}{\PYGZob{}}\PYG{+w}{ }\PYG{n+nx}{\PYGZdl{}match}\PYG{o}{:}\PYG{+w}{ }\PYG{p}{\PYGZob{}}\PYG{l+s+s2}{\PYGZdq{}command.createBackup\PYGZdq{}}\PYG{o}{:}\PYG{+w}{ }\PYG{p}{\PYGZob{}}\PYG{n+nx}{\PYGZdl{}exists}\PYG{o}{:}\PYG{+w}{ }\PYG{k+kc}{true}\PYG{p}{\PYGZcb{}}\PYG{p}{\PYGZcb{}}\PYG{+w}{ }\PYG{p}{\PYGZcb{}}
\PYG{p}{]}\PYG{p}{)}
\end{sphinxVerbatim}

\begin{sphinxadmonition}{note}{Sample output}

\begin{sphinxVerbatim}[commandchars=\\\{\}]
\PYG{p}{\PYGZob{}}
\PYG{+w}{  }\PYG{l+s+s2}{\PYGZdq{}type\PYGZdq{}}\PYG{+w}{ }\PYG{o}{:}\PYG{+w}{ }\PYG{l+s+s2}{\PYGZdq{}op\PYGZdq{}}\PYG{p}{,}
\PYG{+w}{  }\PYG{l+s+s2}{\PYGZdq{}host\PYGZdq{}}\PYG{+w}{ }\PYG{o}{:}\PYG{+w}{ }\PYG{l+s+s2}{\PYGZdq{}bionic:27017\PYGZdq{}}\PYG{p}{,}
\PYG{+w}{  }\PYG{l+s+s2}{\PYGZdq{}desc\PYGZdq{}}\PYG{+w}{ }\PYG{o}{:}\PYG{+w}{ }\PYG{l+s+s2}{\PYGZdq{}conn1251\PYGZdq{}}\PYG{p}{,}
\PYG{+w}{  }\PYG{l+s+s2}{\PYGZdq{}connectionId\PYGZdq{}}\PYG{+w}{ }\PYG{o}{:}\PYG{+w}{ }\PYG{l+m+mf}{1251}\PYG{p}{,}
\PYG{+w}{  }\PYG{l+s+s2}{\PYGZdq{}client\PYGZdq{}}\PYG{+w}{ }\PYG{o}{:}\PYG{+w}{ }\PYG{l+s+s2}{\PYGZdq{}127.0.0.1:52898\PYGZdq{}}\PYG{p}{,}
\PYG{+w}{  }\PYG{l+s+s2}{\PYGZdq{}appName\PYGZdq{}}\PYG{+w}{ }\PYG{o}{:}\PYG{+w}{ }\PYG{l+s+s2}{\PYGZdq{}MongoDB Shell\PYGZdq{}}\PYG{p}{,}
\PYG{+w}{  }\PYG{l+s+s2}{\PYGZdq{}clientMetadata\PYGZdq{}}\PYG{+w}{ }\PYG{o}{:}\PYG{+w}{ }\PYG{p}{\PYGZob{}}
\PYG{+w}{      }\PYG{l+s+s2}{\PYGZdq{}application\PYGZdq{}}\PYG{+w}{ }\PYG{o}{:}\PYG{+w}{ }\PYG{p}{\PYGZob{}}
\PYG{+w}{          }\PYG{l+s+s2}{\PYGZdq{}name\PYGZdq{}}\PYG{+w}{ }\PYG{o}{:}\PYG{+w}{ }\PYG{l+s+s2}{\PYGZdq{}MongoDB Shell\PYGZdq{}}
\PYG{+w}{      }\PYG{p}{\PYGZcb{}}\PYG{p}{,}
\PYG{+w}{      }\PYG{l+s+s2}{\PYGZdq{}driver\PYGZdq{}}\PYG{+w}{ }\PYG{o}{:}\PYG{+w}{ }\PYG{p}{\PYGZob{}}
\PYG{+w}{          }\PYG{l+s+s2}{\PYGZdq{}name\PYGZdq{}}\PYG{+w}{ }\PYG{o}{:}\PYG{+w}{ }\PYG{l+s+s2}{\PYGZdq{}MongoDB Internal Client\PYGZdq{}}\PYG{p}{,}
\PYG{+w}{          }\PYG{l+s+s2}{\PYGZdq{}version\PYGZdq{}}\PYG{+w}{ }\PYG{o}{:}\PYG{+w}{ }\PYG{l+s+s2}{\PYGZdq{}4.4.8\PYGZhy{}9\PYGZdq{}}
\PYG{+w}{      }\PYG{p}{\PYGZcb{}}\PYG{p}{,}
\PYG{+w}{      }\PYG{l+s+s2}{\PYGZdq{}os\PYGZdq{}}\PYG{+w}{ }\PYG{o}{:}\PYG{+w}{ }\PYG{p}{\PYGZob{}}
\PYG{+w}{          }\PYG{l+s+s2}{\PYGZdq{}type\PYGZdq{}}\PYG{+w}{ }\PYG{o}{:}\PYG{+w}{ }\PYG{l+s+s2}{\PYGZdq{}Linux\PYGZdq{}}\PYG{p}{,}
\PYG{+w}{          }\PYG{l+s+s2}{\PYGZdq{}name\PYGZdq{}}\PYG{+w}{ }\PYG{o}{:}\PYG{+w}{ }\PYG{l+s+s2}{\PYGZdq{}Ubuntu\PYGZdq{}}\PYG{p}{,}
\PYG{+w}{          }\PYG{l+s+s2}{\PYGZdq{}architecture\PYGZdq{}}\PYG{+w}{ }\PYG{o}{:}\PYG{+w}{ }\PYG{l+s+s2}{\PYGZdq{}x86\PYGZus{}64\PYGZdq{}}\PYG{p}{,}
\PYG{+w}{          }\PYG{l+s+s2}{\PYGZdq{}version\PYGZdq{}}\PYG{+w}{ }\PYG{o}{:}\PYG{+w}{ }\PYG{l+s+s2}{\PYGZdq{}18.04\PYGZdq{}}
\PYG{+w}{      }\PYG{p}{\PYGZcb{}}
\PYG{+w}{  }\PYG{p}{\PYGZcb{}}\PYG{p}{,}
\PYG{+w}{  }\PYG{l+s+s2}{\PYGZdq{}active\PYGZdq{}}\PYG{+w}{ }\PYG{o}{:}\PYG{+w}{ }\PYG{k+kc}{true}\PYG{p}{,}
\PYG{+w}{  }\PYG{l+s+s2}{\PYGZdq{}currentOpTime\PYGZdq{}}\PYG{+w}{ }\PYG{o}{:}\PYG{+w}{ }\PYG{l+s+s2}{\PYGZdq{}2021\PYGZhy{}08\PYGZhy{}12T11:39:57.675+0000\PYGZdq{}}\PYG{p}{,}
\PYG{+w}{  }\PYG{l+s+s2}{\PYGZdq{}opid\PYGZdq{}}\PYG{+w}{ }\PYG{o}{:}\PYG{+w}{ }\PYG{l+m+mf}{222817}\PYG{p}{,}
\PYG{+w}{  }\PYG{l+s+s2}{\PYGZdq{}lsid\PYGZdq{}}\PYG{+w}{ }\PYG{o}{:}\PYG{+w}{ }\PYG{p}{\PYGZob{}}
\PYG{+w}{      }\PYG{l+s+s2}{\PYGZdq{}id\PYGZdq{}}\PYG{+w}{ }\PYG{o}{:}\PYG{+w}{ }\PYG{n+nx}{UUID}\PYG{p}{(}\PYG{l+s+s2}{\PYGZdq{}6f8d06fc\PYGZhy{}842b\PYGZhy{}420c\PYGZhy{}a43f\PYGZhy{}495db7bd6d88\PYGZdq{}}\PYG{p}{)}\PYG{p}{,}
\PYG{+w}{      }\PYG{l+s+s2}{\PYGZdq{}uid\PYGZdq{}}\PYG{+w}{ }\PYG{o}{:}\PYG{+w}{ }\PYG{n+nx}{BinData}\PYG{p}{(}\PYG{l+m+mf}{0}\PYG{p}{,}\PYG{l+s+s2}{\PYGZdq{}47DEQpj8HBSa+/TImW+5JCeuQeRkm5NMpJWZG3hSuFU=\PYGZdq{}}\PYG{p}{)}
\PYG{+w}{  }\PYG{p}{\PYGZcb{}}\PYG{p}{,}
\PYG{+w}{  }\PYG{l+s+s2}{\PYGZdq{}secs\PYGZus{}running\PYGZdq{}}\PYG{+w}{ }\PYG{o}{:}\PYG{+w}{ }\PYG{n+nx}{NumberLong}\PYG{p}{(}\PYG{l+m+mf}{1}\PYG{p}{)}\PYG{p}{,}
\PYG{+w}{  }\PYG{l+s+s2}{\PYGZdq{}microsecs\PYGZus{}running\PYGZdq{}}\PYG{+w}{ }\PYG{o}{:}\PYG{+w}{ }\PYG{n+nx}{NumberLong}\PYG{p}{(}\PYG{l+m+mf}{1516769}\PYG{p}{)}\PYG{p}{,}
\PYG{+w}{  }\PYG{l+s+s2}{\PYGZdq{}op\PYGZdq{}}\PYG{+w}{ }\PYG{o}{:}\PYG{+w}{ }\PYG{l+s+s2}{\PYGZdq{}command\PYGZdq{}}\PYG{p}{,}
\PYG{+w}{  }\PYG{l+s+s2}{\PYGZdq{}ns\PYGZdq{}}\PYG{+w}{ }\PYG{o}{:}\PYG{+w}{ }\PYG{l+s+s2}{\PYGZdq{}admin.\PYGZdl{}cmd\PYGZdq{}}\PYG{p}{,}
\PYG{+w}{  }\PYG{l+s+s2}{\PYGZdq{}command\PYGZdq{}}\PYG{+w}{ }\PYG{o}{:}\PYG{+w}{ }\PYG{p}{\PYGZob{}}
\PYG{+w}{      }\PYG{l+s+s2}{\PYGZdq{}createBackup\PYGZdq{}}\PYG{+w}{ }\PYG{o}{:}\PYG{+w}{ }\PYG{l+m+mf}{1}\PYG{p}{,}
\PYG{+w}{      }\PYG{l+s+s2}{\PYGZdq{}backupDir\PYGZdq{}}\PYG{+w}{ }\PYG{o}{:}\PYG{+w}{ }\PYG{l+s+s2}{\PYGZdq{}/tmp/mongo\PYGZdq{}}\PYG{p}{,}
\PYG{+w}{      }\PYG{l+s+s2}{\PYGZdq{}lsid\PYGZdq{}}\PYG{+w}{ }\PYG{o}{:}\PYG{+w}{ }\PYG{p}{\PYGZob{}}
\PYG{+w}{          }\PYG{l+s+s2}{\PYGZdq{}id\PYGZdq{}}\PYG{+w}{ }\PYG{o}{:}\PYG{+w}{ }\PYG{n+nx}{UUID}\PYG{p}{(}\PYG{l+s+s2}{\PYGZdq{}6f8d06fc\PYGZhy{}842b\PYGZhy{}420c\PYGZhy{}a43f\PYGZhy{}495db7bd6d88\PYGZdq{}}\PYG{p}{)}
\PYG{+w}{      }\PYG{p}{\PYGZcb{}}\PYG{p}{,}
\PYG{+w}{      }\PYG{l+s+s2}{\PYGZdq{}\PYGZdl{}db\PYGZdq{}}\PYG{+w}{ }\PYG{o}{:}\PYG{+w}{ }\PYG{l+s+s2}{\PYGZdq{}admin\PYGZdq{}}
\PYG{+w}{  }\PYG{p}{\PYGZcb{}}\PYG{p}{,}
\PYG{+w}{  }\PYG{l+s+s2}{\PYGZdq{}msg\PYGZdq{}}\PYG{+w}{ }\PYG{o}{:}\PYG{+w}{ }\PYG{l+s+s2}{\PYGZdq{}Hot Backup: copying data bytes Hot Backup: copying data bytes:}
\PYG{l+s+s2}{  971530240/1147213741 84\PYGZpc{}\PYGZdq{}}\PYG{p}{,}
\PYG{+w}{  }\PYG{l+s+s2}{\PYGZdq{}progress\PYGZdq{}}\PYG{+w}{ }\PYG{o}{:}\PYG{+w}{ }\PYG{p}{\PYGZob{}}
\PYG{+w}{      }\PYG{l+s+s2}{\PYGZdq{}done\PYGZdq{}}\PYG{+w}{ }\PYG{o}{:}\PYG{+w}{ }\PYG{l+m+mf}{971530240}\PYG{p}{,}
\PYG{+w}{      }\PYG{l+s+s2}{\PYGZdq{}total\PYGZdq{}}\PYG{+w}{ }\PYG{o}{:}\PYG{+w}{ }\PYG{l+m+mf}{1147213741}
\PYG{+w}{  }\PYG{p}{\PYGZcb{}}\PYG{p}{,}
\PYG{+w}{  }\PYG{l+s+s2}{\PYGZdq{}numYields\PYGZdq{}}\PYG{+w}{ }\PYG{o}{:}\PYG{+w}{ }\PYG{l+m+mf}{0}\PYG{p}{,}
\PYG{+w}{  }\PYG{l+s+s2}{\PYGZdq{}locks\PYGZdq{}}\PYG{+w}{ }\PYG{o}{:}\PYG{+w}{ }\PYG{p}{\PYGZob{}}

\PYG{+w}{  }\PYG{p}{\PYGZcb{}}\PYG{p}{,}
\PYG{+w}{  }\PYG{l+s+s2}{\PYGZdq{}waitingForLock\PYGZdq{}}\PYG{+w}{ }\PYG{o}{:}\PYG{+w}{ }\PYG{k+kc}{false}\PYG{p}{,}
\PYG{+w}{  }\PYG{l+s+s2}{\PYGZdq{}lockStats\PYGZdq{}}\PYG{+w}{ }\PYG{o}{:}\PYG{+w}{ }\PYG{p}{\PYGZob{}}

\PYG{+w}{  }\PYG{p}{\PYGZcb{}}\PYG{p}{,}
\PYG{+w}{  }\PYG{l+s+s2}{\PYGZdq{}waitingForFlowControl\PYGZdq{}}\PYG{+w}{ }\PYG{o}{:}\PYG{+w}{ }\PYG{k+kc}{false}\PYG{p}{,}
\PYG{+w}{  }\PYG{l+s+s2}{\PYGZdq{}flowControlStats\PYGZdq{}}\PYG{+w}{ }\PYG{o}{:}\PYG{+w}{ }\PYG{p}{\PYGZob{}}

\PYG{+w}{  }\PYG{p}{\PYGZcb{}}
\PYG{p}{\PYGZcb{}}
\end{sphinxVerbatim}
\end{sphinxadmonition}


\section{Streaming hot backups to a remote destination}
\label{\detokenize{hot-backup:streaming-hot-backups-to-a-remote-destination}}\label{\detokenize{hot-backup:psmdb-hot-backup-remote-destination}}
\sphinxAtStartPar
\sphinxstyleemphasis{Percona Server for MongoDB} enables uploading hot backups to an \sphinxhref{https://aws.amazon.com/s3/}{Amazon S3} or a compatible storage service, such
as \sphinxhref{https://min.io/}{MinIO}.

\sphinxAtStartPar
This method requires that you provide the \sphinxstyleemphasis{bucket} field in the \sphinxstyleemphasis{s3} object:

\begin{sphinxVerbatim}[commandchars=\\\{\}]
\PYGZgt{} use admin
...
\PYGZgt{} db.runCommand(\PYGZob{}createBackup: 1, s3: \PYGZob{}bucket: \PYGZdq{}backup20190510\PYGZdq{}, path: \PYGZlt{}some\PYGZus{}optional\PYGZus{}path\PYGZgt{}\PYGZcb{} \PYGZcb{})
\end{sphinxVerbatim}

\sphinxAtStartPar
In addition to the mandatory \sphinxstyleemphasis{bucket} field, the \sphinxstyleemphasis{s3} object may contain the following fields:


\begin{savenotes}\sphinxattablestart
\centering
\begin{tabular}[t]{|\X{30}{100}|\X{15}{100}|\X{55}{100}|}
\hline
\sphinxstyletheadfamily 
\sphinxAtStartPar
Field
&\sphinxstyletheadfamily 
\sphinxAtStartPar
Type
&\sphinxstyletheadfamily 
\sphinxAtStartPar
Description
\\
\hline
\sphinxAtStartPar
bucket
&
\sphinxAtStartPar
string
&
\sphinxAtStartPar
The only mandatory field. Names are subject to restrictions described in
the \sphinxhref{https://docs.aws.amazon.com/AmazonS3/latest/dev/BucketRestrictions.html}{Bucket Restrictions and Limitations section of Amazon S3 documentation}
\\
\hline
\sphinxAtStartPar
path
&
\sphinxAtStartPar
string
&
\sphinxAtStartPar
The virtual path inside the specified bucket where the backup will be
created. If the \sphinxstyleemphasis{path} is not specified then the backup is created in the root
of the bucket. If there are any objects under the specified path, the backup
will not be created and an error will be reported.
\\
\hline
\sphinxAtStartPar
endpoint
&
\sphinxAtStartPar
string
&
\sphinxAtStartPar
The endpoint address and port \sphinxhyphen{} mainly for AWS S3 compatible servers such
as the \sphinxstyleemphasis{MinIO} server. For a local MinIO server, this can be
“127.0.0.1:9000”. For AWS S3 this field can be omitted.
\\
\hline
\sphinxAtStartPar
scheme
&
\sphinxAtStartPar
string
&
\sphinxAtStartPar
“HTTP” or “HTTPS” (default). For a local MinIO server started
with the \sphinxstyleemphasis{minio server} command this should field should contain \sphinxstyleemphasis{HTTP}.
\\
\hline
\sphinxAtStartPar
useVirtualAddressing
&
\sphinxAtStartPar
bool
&
\sphinxAtStartPar
The style of addressing buckets in the URL. By default ‘true’. For MinIO
servers, set this field to \sphinxstylestrong{false}. For more information, see \sphinxhref{https://docs.aws.amazon.com/AmazonS3/latest/dev/VirtualHosting.html}{Virtual
Hosting of Buckets}
in the Amazon S3 documentation.
\\
\hline
\sphinxAtStartPar
region
&
\sphinxAtStartPar
string
&
\sphinxAtStartPar
The name of an AWS region. The default region is \sphinxstylestrong{US\_EAST\_1}. For more
information see \sphinxhref{https://docs.aws.amazon.com/general/latest/gr/rande.html}{AWS Service Endpoints} in the
Amazon S3 documentation.
\\
\hline
\sphinxAtStartPar
profile
&
\sphinxAtStartPar
string
&
\sphinxAtStartPar
The name of a credentials profile in the \sphinxstyleemphasis{credentials} configuration file. If
not specified, the profile named \sphinxstylestrong{default} is used.
\\
\hline
\sphinxAtStartPar
accessKeyId
&
\sphinxAtStartPar
string
&
\sphinxAtStartPar
The access key id
\\
\hline
\sphinxAtStartPar
secretAccessKey
&
\sphinxAtStartPar
string
&
\sphinxAtStartPar
The secret access key
\\
\hline
\end{tabular}
\par
\sphinxattableend\end{savenotes}


\subsection{Credentials}
\label{\detokenize{hot-backup:credentials}}
\sphinxAtStartPar
If the user provides the \sphinxstyleemphasis{access key id} and the \sphinxstyleemphasis{secret access key} parameters,
these are used as credentials.

\sphinxAtStartPar
If the \sphinxstyleemphasis{access key id} parameter is not specified then the credentials are loaded from
the credentials configuration file. By default, it is \sphinxcode{\sphinxupquote{\textasciitilde{}/.aws/credentials}}.

\begin{sphinxadmonition}{note}{An example of the credentials file}

\begin{sphinxVerbatim}[commandchars=\\\{\}]
[default]
aws\PYGZus{}access\PYGZus{}key\PYGZus{}id = ABC123XYZ456QQQAAAFFF
aws\PYGZus{}secret\PYGZus{}access\PYGZus{}key = zuf+secretkey0secretkey1secretkey2
[localminio]
aws\PYGZus{}access\PYGZus{}key\PYGZus{}id = ABCABCABCABC55566678
aws\PYGZus{}secret\PYGZus{}access\PYGZus{}key = secretaccesskey1secretaccesskey2secretaccesskey3
\end{sphinxVerbatim}
\end{sphinxadmonition}


\subsection{Examples}
\label{\detokenize{hot-backup:examples}}
\sphinxAtStartPar
\sphinxstylestrong{Backup in root of bucket on local instance of MinIO server}

\begin{sphinxVerbatim}[commandchars=\\\{\}]
\PYGZgt{} db.runCommand(\PYGZob{}createBackup: 1,  s3: \PYGZob{}bucket: \PYGZdq{}backup20190901500\PYGZdq{},
scheme: \PYGZdq{}HTTP\PYGZdq{},
endpoint: \PYGZdq{}127.0.0.1:9000\PYGZdq{},
useVirtualAddressing: false,
profile: \PYGZdq{}localminio\PYGZdq{}\PYGZcb{}\PYGZcb{})
\end{sphinxVerbatim}

\sphinxAtStartPar
\sphinxstylestrong{Backup on MinIO testing server with the default credentials profile}

\sphinxAtStartPar
The following command creates a backup under the virtual path  “year2019/day42” in the \sphinxstyleemphasis{backup} bucket:

\begin{sphinxVerbatim}[commandchars=\\\{\}]
\PYGZgt{} db.runCommand(\PYGZob{}createBackup: 1,  s3: \PYGZob{}bucket: \PYGZdq{}backup\PYGZdq{},
path: \PYGZdq{}year2019/day42\PYGZdq{},
endpoint: \PYGZdq{}sandbox.min.io:9000\PYGZdq{},
useVirtualAddressing: false\PYGZcb{}\PYGZcb{})
\end{sphinxVerbatim}

\sphinxAtStartPar
\sphinxstylestrong{Backup on AWS S3 service using default settings}

\begin{sphinxVerbatim}[commandchars=\\\{\}]
\PYGZgt{} db.runCommand(\PYGZob{}createBackup: 1,  s3: \PYGZob{}bucket: \PYGZdq{}backup\PYGZdq{}, path: \PYGZdq{}year2019/day42\PYGZdq{}\PYGZcb{}\PYGZcb{})
\end{sphinxVerbatim}


\sphinxstrong{See also:}
\nopagebreak

\begin{description}
\item[{AWS Documentation: Providing AWS Credentials}] \leavevmode
\sphinxAtStartPar
\sphinxurl{https://docs.aws.amazon.com/sdk-for-cpp/v1/developer-guide/credentials.html}

\end{description}




\section{Restoring data from backup}
\label{\detokenize{hot-backup:restoring-data-from-backup}}\label{\detokenize{hot-backup:hot-backup-restore}}\subsubsection*{Restoring from backup on a standalone server}

\sphinxAtStartPar
To restore your database on a standalone server, stop the \sphinxcode{\sphinxupquote{mongod}} service, clean out the data directory and copy files from the backup directory to the data directory. The \sphinxcode{\sphinxupquote{mongod}} user requires access to those files to start the service. Therefore, make the \sphinxcode{\sphinxupquote{mongod}} user the owner of the data directory and all files and subdirectories under it, and restart the \sphinxcode{\sphinxupquote{mongod}} service.

\sphinxAtStartPar
Run the following commands as root or by using the \sphinxstyleliteralstrong{\sphinxupquote{sudo}} command

\begin{sphinxVerbatim}[commandchars=\\\{\}]
\PYG{c+c1}{\PYGZsh{} Stop the mongod service}
\PYGZdl{} systemctl stop mongod
\PYG{c+c1}{\PYGZsh{} Clean out the data directory}
\PYGZdl{} rm \PYGZhy{}rf /var/lib/mongodb/*
\PYG{c+c1}{\PYGZsh{} Copy backup files}
\PYGZdl{} cp \PYGZhy{}RT \PYGZlt{}backup\PYGZus{}data\PYGZus{}path\PYGZgt{} /var/lib/mongodb/
\PYG{c+c1}{\PYGZsh{} Grant permissions to data files for the mongod user}
\PYGZdl{} chown \PYGZhy{}R mongod:mongod /var/lib/mongodb/
\PYG{c+c1}{\PYGZsh{} Start the mongod service}
\PYGZdl{} systemctl start mongod
\end{sphinxVerbatim}
\subsubsection*{Restoring from backup in a replica set}

\sphinxAtStartPar
The recommended way to restore the replica set from a backup is to restore it into a standalone node and then initiate it as the first member of a new replica set.

\begin{sphinxadmonition}{note}{Note:}
\sphinxAtStartPar
If you try to restore the node into the existing replica set and there is more recent data, the restored node detects that it is out of date with the other replica set members, deletes the data and makes an initial sync.
\end{sphinxadmonition}

\sphinxAtStartPar
Run the following commands as root or by using the \sphinxstyleliteralstrong{\sphinxupquote{sudo}} command

\sphinxAtStartPar
The restore steps are the following:
\begin{enumerate}
\sphinxsetlistlabels{\arabic}{enumi}{enumii}{}{.}%
\item {} 
\sphinxAtStartPar
Stop the \sphinxcode{\sphinxupquote{mongod}} service:

\begin{sphinxVerbatim}[commandchars=\\\{\}]
\PYGZdl{} systemctl stop mongod
\end{sphinxVerbatim}

\item {} 
\sphinxAtStartPar
Clean the data directory and then copy the files from the backup directory to your data directory. Assuming that the data directory is \sphinxcode{\sphinxupquote{/var/lib/mongodb/}}, use the following commands:

\begin{sphinxVerbatim}[commandchars=\\\{\}]
\PYGZdl{} rm \PYGZhy{}rf /var/lib/mongodb/*
\PYGZdl{} cp \PYGZhy{}RT \PYGZlt{}backup\PYGZus{}data\PYGZus{}path\PYGZgt{} /var/lib/mongodb/
\end{sphinxVerbatim}

\item {} 
\sphinxAtStartPar
Grant permissions to the data files for the \sphinxcode{\sphinxupquote{mongod}} user

\begin{sphinxVerbatim}[commandchars=\\\{\}]
\PYGZdl{} chown \PYGZhy{}R mongod:mongod /var/lib/mongodb/
\end{sphinxVerbatim}

\item {} 
\sphinxAtStartPar
Make sure the replication is disabled in the config file and start the \sphinxcode{\sphinxupquote{mongod}} service.

\begin{sphinxVerbatim}[commandchars=\\\{\}]
\PYGZdl{} systemctl start mongod
\end{sphinxVerbatim}

\item {} 
\sphinxAtStartPar
Connect to your standalone node via the \sphinxcode{\sphinxupquote{mongo}} shell and drop the local database

\begin{sphinxVerbatim}[commandchars=\\\{\}]
\PYGZgt{} mongo
\PYGZgt{} use local
\PYGZgt{} db.dropDatabase()
\end{sphinxVerbatim}

\item {} 
\sphinxAtStartPar
Restart the node with the replication enabled
\begin{itemize}
\item {} 
\sphinxAtStartPar
Shut down the node.

\begin{sphinxVerbatim}[commandchars=\\\{\}]
\PYGZdl{} systemctl stop mongod
\end{sphinxVerbatim}

\item {} 
\sphinxAtStartPar
Edit the configuration file and specify the \sphinxcode{\sphinxupquote{replication.replSetname}} option

\item {} 
\sphinxAtStartPar
Start the \sphinxcode{\sphinxupquote{mongod}} node:

\begin{sphinxVerbatim}[commandchars=\\\{\}]
\PYGZdl{} systemctl start mongod
\end{sphinxVerbatim}

\end{itemize}

\item {} 
\sphinxAtStartPar
Initiate a new replica set

\begin{sphinxVerbatim}[commandchars=\\\{\}]
\PYGZsh{} Start the mongo shell
\PYGZgt{} mongo
\PYGZsh{} Initiate a new replica set
\PYGZgt{} rs.initiate()
\end{sphinxVerbatim}

\end{enumerate}


\chapter{Profiling Rate Limit}
\label{\detokenize{rate-limit:profiling-rate-limit}}\label{\detokenize{rate-limit:rate-limit}}\label{\detokenize{rate-limit::doc}}
\sphinxAtStartPar
\sphinxstyleemphasis{Percona Server for MongoDB} can limit the number of queries collected by the database profiler
to decrease its impact on performance.
Rate limit is an integer between 1 and 1000
and represents the fraction of queries to be profiled.
For example, if you set it to 20, then every 20th query will be logged.
For compatibility reasons, rate limit of 0 is the same as setting it to 1,
and will effectively disable the feature
meaning that every query will be profiled.

\sphinxAtStartPar
The MongoDB database profiler can operate in one of three modes:
\begin{itemize}
\item {} 
\sphinxAtStartPar
\sphinxcode{\sphinxupquote{0}}: Profiling is disabled. This is the default setting.

\item {} 
\sphinxAtStartPar
\sphinxcode{\sphinxupquote{1}}: The profiler collects data only for \sphinxstyleemphasis{slow} queries.
By default, queries that take more than 100 milliseconds to execute
are considered \sphinxstyleemphasis{slow}.

\item {} 
\sphinxAtStartPar
\sphinxcode{\sphinxupquote{2}}: Collects profiling data for all database operations.

\end{itemize}

\sphinxAtStartPar
Mode \sphinxcode{\sphinxupquote{1}} ignores all \sphinxstyleemphasis{fast} queries,
which may be the cause of problems that you are trying to find.
Mode \sphinxcode{\sphinxupquote{2}} provides a comprehensive picture of database performance,
but may introduce unnecessary overhead.

\sphinxAtStartPar
With rate limiting you can collect profiling data for all database operations
and reduce overhead by sampling queries.
Slow queries ignore rate limiting and are always collected by the profiler.


\section{Comparing to the \sphinxstyleliteralintitle{\sphinxupquote{sampleRate}} option}
\label{\detokenize{rate-limit:comparing-to-the-samplerate-option}}
\sphinxAtStartPar
The \sphinxcode{\sphinxupquote{sampleRate}} option (= \sphinxhref{https://docs.mongodb.com/manual/reference/program/mongod/index.html\#cmdoption-mongod-slowopsamplerate}{slowOpSampleRate} config file option) is a similar
concept to \sphinxcode{\sphinxupquote{rateLimit}}. But it works at different profile level, completely
ignores operations faster than \sphinxcode{\sphinxupquote{slowOpsThresholdMs}} (a.k.a. \sphinxcode{\sphinxupquote{slowMs}}), and affects the
log file printing, too.


\begin{savenotes}\sphinxattablestart
\centering
\begin{tabulary}{\linewidth}[t]{|T|T|T|}
\hline
\sphinxstyletheadfamily &\sphinxstyletheadfamily 
\sphinxAtStartPar
sampleRate
&\sphinxstyletheadfamily 
\sphinxAtStartPar
rateLimit
\\
\hline\sphinxstyletheadfamily 
\sphinxAtStartPar
Affects profiling level 1
&
\sphinxAtStartPar
yes
&
\sphinxAtStartPar
no
\\
\hline\sphinxstyletheadfamily 
\sphinxAtStartPar
Affects profiling level 2
&
\sphinxAtStartPar
no
&
\sphinxAtStartPar
yes
\\
\hline\sphinxstyletheadfamily 
\sphinxAtStartPar
Discards/filters slow ops
&
\sphinxAtStartPar
yes
&
\sphinxAtStartPar
no
\\
\hline\sphinxstyletheadfamily 
\sphinxAtStartPar
Discards/filters fast ops
&
\sphinxAtStartPar
no
&
\sphinxAtStartPar
yes
\\
\hline\sphinxstyletheadfamily 
\sphinxAtStartPar
Affects log file
&
\sphinxAtStartPar
yes
&
\sphinxAtStartPar
no
\\
\hline\sphinxstyletheadfamily 
\sphinxAtStartPar
Example value of option
&
\sphinxAtStartPar
0.02
&
\sphinxAtStartPar
50
\\
\hline
\end{tabulary}
\par
\sphinxattableend\end{savenotes}

\sphinxAtStartPar
\sphinxcode{\sphinxupquote{rateLimit}} is a better way to have continuous profiling for monitoring or live
analysis purposes. \sphinxcode{\sphinxupquote{sampleRate}} requires setting slowOpsThresholdMs to zero if
you want to sample all types of operations. \sphinxcode{\sphinxupquote{sampleRate}} has an effect on the log file
which may either decrease or increase the log volume.


\section{Enabling the Rate Limit}
\label{\detokenize{rate-limit:enabling-the-rate-limit}}
\sphinxAtStartPar
To enable rate limiting, set the profiler mode to \sphinxcode{\sphinxupquote{2}}
and specify the value of the rate limit.
Optionally, you can also change the default threshold for slow queries,
which will not be sampled by rate limiting.

\sphinxAtStartPar
For example, to set the rate limit to \sphinxcode{\sphinxupquote{100}}
(profile every 100th \sphinxstyleemphasis{fast} query)
and the slow query threshold to \sphinxcode{\sphinxupquote{200}}
(profile all queries slower than 200 milliseconds),
run the \sphinxcode{\sphinxupquote{mongod}} instance as follows:

\begin{sphinxVerbatim}[commandchars=\\\{\}]
\PYGZdl{} mongod \PYGZhy{}\PYGZhy{}profile 2 \PYGZhy{}\PYGZhy{}slowms 200 \PYGZhy{}\PYGZhy{}rateLimit 100
\end{sphinxVerbatim}

\sphinxAtStartPar
To do the same at runtime,
use the \sphinxcode{\sphinxupquote{profile}} command.
It returns the \sphinxstyleemphasis{previous} settings
and \sphinxcode{\sphinxupquote{"ok" : 1}} indicates that the operation was successful:

\begin{sphinxVerbatim}[commandchars=\\\{\}]
\PYG{o}{\PYGZgt{}} \PYG{n}{db}\PYG{o}{.}\PYG{n}{runCommand}\PYG{p}{(} \PYG{p}{\PYGZob{}} \PYG{n}{profile}\PYG{p}{:} \PYG{l+m+mi}{2}\PYG{p}{,} \PYG{n}{slowms}\PYG{p}{:} \PYG{l+m+mi}{200}\PYG{p}{,} \PYG{n}{ratelimit}\PYG{p}{:} \PYG{l+m+mi}{100} \PYG{p}{\PYGZcb{}} \PYG{p}{)}\PYG{p}{;}
\PYG{p}{\PYGZob{}} \PYG{l+s+s2}{\PYGZdq{}}\PYG{l+s+s2}{was}\PYG{l+s+s2}{\PYGZdq{}} \PYG{p}{:} \PYG{l+m+mi}{0}\PYG{p}{,} \PYG{l+s+s2}{\PYGZdq{}}\PYG{l+s+s2}{slowms}\PYG{l+s+s2}{\PYGZdq{}} \PYG{p}{:} \PYG{l+m+mi}{100}\PYG{p}{,} \PYG{l+s+s2}{\PYGZdq{}}\PYG{l+s+s2}{ratelimit}\PYG{l+s+s2}{\PYGZdq{}} \PYG{p}{:} \PYG{l+m+mi}{1}\PYG{p}{,} \PYG{l+s+s2}{\PYGZdq{}}\PYG{l+s+s2}{ok}\PYG{l+s+s2}{\PYGZdq{}} \PYG{p}{:} \PYG{l+m+mi}{1} \PYG{p}{\PYGZcb{}}
\end{sphinxVerbatim}

\sphinxAtStartPar
To check the current settings, run \sphinxcode{\sphinxupquote{profile: \sphinxhyphen{}1}}:

\begin{sphinxVerbatim}[commandchars=\\\{\}]
\PYG{o}{\PYGZgt{}} \PYG{n}{db}\PYG{o}{.}\PYG{n}{runCommand}\PYG{p}{(} \PYG{p}{\PYGZob{}} \PYG{n}{profile}\PYG{p}{:} \PYG{o}{\PYGZhy{}}\PYG{l+m+mi}{1} \PYG{p}{\PYGZcb{}} \PYG{p}{)}\PYG{p}{;}
\PYG{p}{\PYGZob{}} \PYG{l+s+s2}{\PYGZdq{}}\PYG{l+s+s2}{was}\PYG{l+s+s2}{\PYGZdq{}} \PYG{p}{:} \PYG{l+m+mi}{2}\PYG{p}{,} \PYG{l+s+s2}{\PYGZdq{}}\PYG{l+s+s2}{slowms}\PYG{l+s+s2}{\PYGZdq{}} \PYG{p}{:} \PYG{l+m+mi}{200}\PYG{p}{,} \PYG{l+s+s2}{\PYGZdq{}}\PYG{l+s+s2}{ratelimit}\PYG{l+s+s2}{\PYGZdq{}} \PYG{p}{:} \PYG{l+m+mi}{100}\PYG{p}{,} \PYG{l+s+s2}{\PYGZdq{}}\PYG{l+s+s2}{ok}\PYG{l+s+s2}{\PYGZdq{}} \PYG{p}{:} \PYG{l+m+mi}{1} \PYG{p}{\PYGZcb{}}
\end{sphinxVerbatim}

\sphinxAtStartPar
If you want to set or get just the rate limit value,
use the \sphinxcode{\sphinxupquote{profilingRateLimit}} parameter on the \sphinxcode{\sphinxupquote{admin}} database:

\begin{sphinxVerbatim}[commandchars=\\\{\}]
\PYG{o}{\PYGZgt{}} \PYG{n}{db}\PYG{o}{.}\PYG{n}{getSiblingDB}\PYG{p}{(}\PYG{l+s+s1}{\PYGZsq{}}\PYG{l+s+s1}{admin}\PYG{l+s+s1}{\PYGZsq{}}\PYG{p}{)}\PYG{o}{.}\PYG{n}{runCommand}\PYG{p}{(} \PYG{p}{\PYGZob{}} \PYG{n}{setParameter}\PYG{p}{:} \PYG{l+m+mi}{1}\PYG{p}{,} \PYG{l+s+s2}{\PYGZdq{}}\PYG{l+s+s2}{profilingRateLimit}\PYG{l+s+s2}{\PYGZdq{}}\PYG{p}{:} \PYG{l+m+mi}{100} \PYG{p}{\PYGZcb{}} \PYG{p}{)}\PYG{p}{;}
\PYG{p}{\PYGZob{}} \PYG{l+s+s2}{\PYGZdq{}}\PYG{l+s+s2}{was}\PYG{l+s+s2}{\PYGZdq{}} \PYG{p}{:} \PYG{l+m+mi}{1}\PYG{p}{,} \PYG{l+s+s2}{\PYGZdq{}}\PYG{l+s+s2}{ok}\PYG{l+s+s2}{\PYGZdq{}} \PYG{p}{:} \PYG{l+m+mi}{1} \PYG{p}{\PYGZcb{}}
\PYG{o}{\PYGZgt{}} \PYG{n}{db}\PYG{o}{.}\PYG{n}{getSiblingDB}\PYG{p}{(}\PYG{l+s+s1}{\PYGZsq{}}\PYG{l+s+s1}{admin}\PYG{l+s+s1}{\PYGZsq{}}\PYG{p}{)}\PYG{o}{.}\PYG{n}{runCommand}\PYG{p}{(} \PYG{p}{\PYGZob{}} \PYG{n}{getParameter}\PYG{p}{:} \PYG{l+m+mi}{1}\PYG{p}{,} \PYG{l+s+s2}{\PYGZdq{}}\PYG{l+s+s2}{profilingRateLimit}\PYG{l+s+s2}{\PYGZdq{}}\PYG{p}{:} \PYG{l+m+mi}{1} \PYG{p}{\PYGZcb{}} \PYG{p}{)}\PYG{p}{;}
\PYG{p}{\PYGZob{}} \PYG{l+s+s2}{\PYGZdq{}}\PYG{l+s+s2}{profilingRateLimit}\PYG{l+s+s2}{\PYGZdq{}} \PYG{p}{:} \PYG{l+m+mi}{100}\PYG{p}{,} \PYG{l+s+s2}{\PYGZdq{}}\PYG{l+s+s2}{ok}\PYG{l+s+s2}{\PYGZdq{}} \PYG{p}{:} \PYG{l+m+mi}{1} \PYG{p}{\PYGZcb{}}
\end{sphinxVerbatim}

\sphinxAtStartPar
If you want rate limiting to persist when you restart \sphinxcode{\sphinxupquote{mongod}},
set the corresponding variables in the MongoDB configuration file
(by default, \sphinxcode{\sphinxupquote{/etc/mongod.conf}}):

\begin{sphinxVerbatim}[commandchars=\\\{\}]
\PYG{n}{operationProfiling}\PYG{p}{:}
  \PYG{n}{mode}\PYG{p}{:} \PYG{n+nb}{all}
  \PYG{n}{slowOpThresholdMs}\PYG{p}{:} \PYG{l+m+mi}{200}
  \PYG{n}{rateLimit}\PYG{p}{:} \PYG{l+m+mi}{100}
\end{sphinxVerbatim}

\begin{sphinxadmonition}{note}{Note:}
\sphinxAtStartPar
The value of the \sphinxcode{\sphinxupquote{operationProfiling.mode}} variable is a string,
which you can set to either \sphinxcode{\sphinxupquote{off}}, \sphinxcode{\sphinxupquote{slowOp}}, or \sphinxcode{\sphinxupquote{all}},
corresponding to profiling modes \sphinxcode{\sphinxupquote{0}}, \sphinxcode{\sphinxupquote{1}}, and \sphinxcode{\sphinxupquote{2}}.
\end{sphinxadmonition}


\section{Profiler Collection Extension}
\label{\detokenize{rate-limit:profiler-collection-extension}}
\sphinxAtStartPar
Each document in the \sphinxcode{\sphinxupquote{system.profile}} collection
includes an additional \sphinxcode{\sphinxupquote{rateLimit}} field.
This field always has the value of \sphinxcode{\sphinxupquote{1}} for \sphinxstyleemphasis{slow} queries
and the current rate limit value for \sphinxstyleemphasis{fast} queries.


\chapter{Authentication}
\label{\detokenize{authentication:authentication}}\label{\detokenize{authentication:ext-auth}}\label{\detokenize{authentication::doc}}
\sphinxAtStartPar
Authentication is the process of verifying a client’s identity. Normally, a client needs to authenticate themselves
against the MongoDB server user database before doing any work
or reading any data from a \sphinxcode{\sphinxupquote{mongod}} or \sphinxcode{\sphinxupquote{mongos}} instance.

\sphinxAtStartPar
By default, \sphinxstyleemphasis{Percona Server for MongoDB} provides a \sphinxstyleabbreviation{SCRAM} (Salted Challenge Response Authentication Mechanism) authentication mechanism where clients authenticate themselves by providing their user credentials.
In addition, you can integrate \sphinxstyleemphasis{Percona Server for MongoDB} with a separate service,
such as OpenLDAP or Active Directory. This enables users to access the database
with the same credentials they use for their emails or workstations.

\sphinxAtStartPar
You can use any of these authentication mechanisms supported in \sphinxstyleemphasis{Percona Server for MongoDB}:
\begin{itemize}
\item {} 
\sphinxAtStartPar
{\hyperref[\detokenize{authentication:scram}]{\sphinxcrossref{\DUrole{std,std-ref}{SCRAM}}}} (default)

\item {} 
\sphinxAtStartPar
{\hyperref[\detokenize{authentication:x509}]{\sphinxcrossref{\DUrole{std,std-ref}{x.509 certificate authentication}}}}

\item {} 
\sphinxAtStartPar
{\hyperref[\detokenize{authentication:ldap-authentication-sasl}]{\sphinxcrossref{\DUrole{std,std-ref}{LDAP authentication with SASL}}}}

\item {} 
\sphinxAtStartPar
{\hyperref[\detokenize{authentication:kerberos-authentication}]{\sphinxcrossref{\DUrole{std,std-ref}{Kerberos Authentication}}}}

\item {} 
\sphinxAtStartPar
{\hyperref[\detokenize{authorization:native-ldap}]{\sphinxcrossref{\DUrole{std,std-ref}{Authentication and authorization with direct binding to LDAP}}}}

\end{itemize}


\section{SCRAM}
\label{\detokenize{authentication:scram}}\label{\detokenize{authentication:id1}}
\sphinxAtStartPar
\sphinxstyleabbreviation{SCRAM} is the default authentication mechanism. \sphinxstyleemphasis{Percona Server for MongoDB} verifies the credentials against the user’s name, password and the database where the user record is created for a client (authentication database). For how to enable this mechanism, see {\hyperref[\detokenize{enable-auth:enable-auth}]{\sphinxcrossref{\DUrole{std,std-ref}{Enabling Authentication}}}}.


\section{x.509 certificate authentication}
\label{\detokenize{authentication:x-509-certificate-authentication}}\label{\detokenize{authentication:x509}}
\sphinxAtStartPar
This authentication mechanism enables a client to authenticate in \sphinxstyleemphasis{Percona Server for MongoDB} by providing an x.509 certificate instead of user credentials. Each certificate contains the \sphinxcode{\sphinxupquote{subject}} field defined in the \sphinxstyleabbreviation{DN} (Distinguished Name) format. In \sphinxstyleemphasis{Percona Server for MongoDB}, each certificate has a corresponding user record in the \sphinxcode{\sphinxupquote{\$external}} database. When a user connects to the database, \sphinxstyleemphasis{Percona Server for MongoDB} matches the \sphinxcode{\sphinxupquote{subject}} value against the usernames defined in the \sphinxcode{\sphinxupquote{\$external}} database.

\sphinxAtStartPar
For production use, we recommend using valid \sphinxstyleabbreviation{CA} (Certified Authority) certificates. For testing purposes, you can generate and use self\sphinxhyphen{}signed certificates.

\sphinxAtStartPar
x.509 authentication is compatible with with {\hyperref[\detokenize{authorization:ldap-authorization}]{\sphinxcrossref{\DUrole{std,std-ref}{LDAP authorization}}}} to enable you to control user access and operations in \sphinxstyleemphasis{Percona Server for MongoDB}. For configuration guidelines, refer to {\hyperref[\detokenize{x509-ldap:ldap-x509}]{\sphinxcrossref{\DUrole{std,std-ref}{Set up x.509 authentication and LDAP authorization}}}}.


\sphinxstrong{See also:}
\nopagebreak


\sphinxAtStartPar
MongoDB Documentation: \sphinxhref{https://docs.mongodb.com/manual/core/security-x.509/}{x.509}

\sphinxAtStartPar
Percona Blog: \sphinxhref{https://www.percona.com/blog/2019/10/28/setting-up-mongodb-with-member-x509-auth-and-ssl-easy-rsa/}{Setting up MongoDB with Member x509 auth and SSL + easy\sphinxhyphen{}rsa}




\section{LDAP authentication with SASL}
\label{\detokenize{authentication:ldap-authentication-with-sasl}}\label{\detokenize{authentication:ldap-authentication-sasl}}

\section{Overview}
\label{\detokenize{authentication:overview}}
\sphinxAtStartPar
LDAP authentication with \sphinxstyleabbreviation{SASL} (Simple Authentication and Security Layer) means that both the client and the server establish a SASL session using the SASL library. Then authentication (bind) requests are sent to the LDAP server through the SASL authentication daemon (\sphinxcode{\sphinxupquote{saslauthd}}) that acts as a remote proxy for the \sphinxcode{\sphinxupquote{mongod}} server.

\sphinxAtStartPar
The following components are necessary for external authentication to work:
\begin{itemize}
\item {} 
\sphinxAtStartPar
\sphinxstylestrong{LDAP Server}: Remotely stores all user credentials
(i.e. user name and associated password).

\item {} 
\sphinxAtStartPar
\sphinxstylestrong{SASL Daemon}: Used as a MongoDB server\sphinxhyphen{}local proxy
for the remote LDAP service.

\item {} 
\sphinxAtStartPar
\sphinxstylestrong{SASL Library}: Used by the MongoDB client and server
to create data necessary for the authentication mechanism.

\end{itemize}

\sphinxAtStartPar
The following image illustrates this architecture:

\noindent{\hspace*{\fill}\sphinxincludegraphics{{psmdb-ext-auth}.png}\hspace*{\fill}}

\sphinxAtStartPar
An authentication session uses the following sequence:
\begin{enumerate}
\sphinxsetlistlabels{\arabic}{enumi}{enumii}{}{.}%
\item {} 
\sphinxAtStartPar
A \sphinxcode{\sphinxupquote{mongo}} client connects to a running \sphinxcode{\sphinxupquote{mongod}} instance.

\item {} 
\sphinxAtStartPar
The client creates a \sphinxcode{\sphinxupquote{PLAIN}} authentication request
using the \sphinxstyleabbreviation{SASL} library.

\item {} 
\sphinxAtStartPar
The client then sends this SASL request to the server
as a special Mongo command.

\item {} 
\sphinxAtStartPar
The \sphinxcode{\sphinxupquote{mongod}} server receives this SASL Message,
with its authentication request payload.

\item {} 
\sphinxAtStartPar
The server then creates a SASL session scoped to this client,
using its own reference to the SASL library.

\item {} 
\sphinxAtStartPar
Then the server passes the authentication payload to the SASL library,
which in turn passes it on to the \sphinxcode{\sphinxupquote{saslauthd}} daemon.

\item {} 
\sphinxAtStartPar
The \sphinxcode{\sphinxupquote{saslauthd}} daemon passes the payload on to the LDAP service
to get a YES or NO authentication response
(in other words, does this user exist and is the password correct).

\item {} 
\sphinxAtStartPar
The YES/NO response moves back from \sphinxcode{\sphinxupquote{saslauthd}},
through the SASL library, to \sphinxcode{\sphinxupquote{mongod}}.

\item {} 
\sphinxAtStartPar
The \sphinxcode{\sphinxupquote{mongod}} server uses this YES/NO response
to authenticate the client or reject the request.

\item {} 
\sphinxAtStartPar
If successful, the client has authenticated and can proceed.

\end{enumerate}

\sphinxAtStartPar
For configuration instructions, refer to {\hyperref[\detokenize{sasl-auth:sasl}]{\sphinxcrossref{\DUrole{std,std-ref}{Setting up LDAP authentication with SASL}}}}.


\section{Kerberos Authentication}
\label{\detokenize{authentication:kerberos-authentication}}\label{\detokenize{authentication:id2}}
\sphinxAtStartPar
\sphinxstyleemphasis{Percona Server for MongoDB} supports Kerberos authentication starting from release 4.2.6\sphinxhyphen{}6.

\sphinxAtStartPar
This authentication mechanism involves the use of a \sphinxtitleref{Key Distribution Center (KDC)} \sphinxhyphen{} a symmetric encryption component which operates with tickets. A \sphinxtitleref{ticket} is a small amount of encrypted data which is used for authentication. It is issued for a user session and has a limited lifetime.

\sphinxAtStartPar
When using Kerberos authentication, you also operate with principals and realms.

\sphinxAtStartPar
A \sphinxtitleref{realm} is the logical network, similar to a domain, for all Kerberos nodes under the same master KDC.

\sphinxAtStartPar
A \sphinxtitleref{principal} is a user or a service which is known to Kerberos. A principal name is used for authentication in Kerberos. A service principal represents the service, e.g. \sphinxcode{\sphinxupquote{mongodb}}. A user principal represents the user. The user principal name corresponds to the username in the \sphinxcode{\sphinxupquote{\$external}} database in \sphinxstyleemphasis{Percona Server for MongoDB}.

\sphinxAtStartPar
The following diagram shows the authentication workflow:

\noindent\sphinxincludegraphics{{Kerberos_auth}.png}

\sphinxAtStartPar
The sequence is the following:
\begin{enumerate}
\sphinxsetlistlabels{\arabic}{enumi}{enumii}{}{.}%
\item {} 
\sphinxAtStartPar
A \sphinxcode{\sphinxupquote{mongo}} client sends the Ticket\sphinxhyphen{}Grantng Ticket (TGT) request to the Key Distribution Center (KDC)

\item {} 
\sphinxAtStartPar
The KDC issues the ticket and sends it to the \sphinxcode{\sphinxupquote{mongo}} client.

\item {} 
\sphinxAtStartPar
The \sphinxcode{\sphinxupquote{mongo}} client sends the authentication request to the \sphinxcode{\sphinxupquote{mongo}} server presenting the ticket.

\item {} 
\sphinxAtStartPar
The \sphinxcode{\sphinxupquote{mongo}} server validates the ticket in the KDC.

\item {} 
\sphinxAtStartPar
Upon successful ticket validation, the authentication request is approved and the user is authenticated.

\end{enumerate}

\sphinxAtStartPar
Kerberos authentication in \sphinxstyleemphasis{Percona Server for MongoDB} is implemented the same way as in MongoDB Enterprise.


\sphinxstrong{See also:}
\nopagebreak


\sphinxAtStartPar
MongoDB Documentation: \sphinxhref{https://docs.mongodb.com/manual/core/kerberos/}{Kerberos Authentication}




\chapter{LDAP authorization}
\label{\detokenize{authorization:ldap-authorization}}\label{\detokenize{authorization:id1}}\label{\detokenize{authorization::doc}}
\sphinxAtStartPar
LDAP authorization allows you to control user access and operations in your database environment using the centralized user management storage \textendash{} an LDAP server. You create and manage user credentials and permission information in the LDAP server. In addition, you create roles in the \sphinxcode{\sphinxupquote{admin}} database with the names that exactly match the LDAP group Distinguished Name. These roles define what privileges the users who belong to the corresponding LDAP group.


\section{Supported authentication mechanisms}
\label{\detokenize{authorization:supported-authentication-mechanisms}}\label{\detokenize{authorization:auth-mech}}
\sphinxAtStartPar
LDAP authorization is compatible with the following authentication mechanisms:
\begin{itemize}
\item {} 
\sphinxAtStartPar
{\hyperref[\detokenize{authentication:x509}]{\sphinxcrossref{\DUrole{std,std-ref}{x.509 certificate authentication}}}}

\item {} 
\sphinxAtStartPar
{\hyperref[\detokenize{authentication:kerberos-authentication}]{\sphinxcrossref{\DUrole{std,std-ref}{Kerberos Authentication}}}}

\item {} 
\sphinxAtStartPar
{\hyperref[\detokenize{authorization:native-ldap}]{\sphinxcrossref{\DUrole{std,std-ref}{Authentication and authorization with direct binding to LDAP}}}}

\end{itemize}


\section{Authentication and authorization with direct binding to LDAP}
\label{\detokenize{authorization:authentication-and-authorization-with-direct-binding-to-ldap}}\label{\detokenize{authorization:native-ldap}}
\sphinxAtStartPar
Starting with release 4.2.5\sphinxhyphen{}5, you can configure \sphinxstyleemphasis{Percona Server for MongoDB} to communicate with the LDAP server directly to authenticate and also authorize users.

\sphinxAtStartPar
The advantage of using this mechanism is that it is easy to setup and does not require pre\sphinxhyphen{}creating users  in the dummy \sphinxcode{\sphinxupquote{\$external}} db. Nevertheless, the \sphinxcode{\sphinxupquote{\sphinxhyphen{}\sphinxhyphen{}authenticationDatabase}} connection argument will still need to be specified as \sphinxcode{\sphinxupquote{\$external}}.

\sphinxAtStartPar
The following example illustrates the connection to \sphinxstyleemphasis{Percona Server for MongoDB} from the \sphinxcode{\sphinxupquote{mongo}} shell:

\begin{sphinxVerbatim}[commandchars=\\\{\}]
mongo \PYGZhy{}u \PYGZdq{}CN=alice,CN=Users,DC=engineering,DC=example,DC=com\PYGZdq{} \PYGZhy{}p \PYGZhy{}\PYGZhy{}authenticationDatabase \PYGZsq{}\PYGZdl{}external\PYGZsq{} \PYGZhy{}\PYGZhy{}authenticationMechanism PLAIN
\end{sphinxVerbatim}

\sphinxAtStartPar
The following diagram illustrates the authentication and authorization flow:

\noindent\sphinxincludegraphics{{NativeLDAP-auth}.png}
\begin{enumerate}
\sphinxsetlistlabels{\arabic}{enumi}{enumii}{}{.}%
\item {} 
\sphinxAtStartPar
A user connects to the db providing their credentials

\item {} 
\sphinxAtStartPar
If required, \sphinxstyleemphasis{Percona Server for MongoDB} {\hyperref[\detokenize{authorization:usertodnmapping}]{\sphinxcrossref{\DUrole{std,std-ref}{transforms the username}}}} to match the \sphinxstyleabbreviation{DN} in the LDAP server according to the mapping rules specified for the \sphinxcode{\sphinxupquote{\sphinxhyphen{}\sphinxhyphen{}ldapUserToDNMapping}} parameter.

\item {} 
\sphinxAtStartPar
\sphinxstyleemphasis{Percona Server for MongoDB} queries the LDAP server for the user identity and /or the LDAP groups this user belongs to.

\item {} 
\sphinxAtStartPar
The LDAP server evaluates the query and if a user exists, returns their LDAP groups.

\item {} 
\sphinxAtStartPar
\sphinxstyleemphasis{Percona Server for MongoDB} authorizes the user by mapping the DN of the returned groups against the roles assigned to the user in the \sphinxcode{\sphinxupquote{admin}} database.  If a user belongs to several groups they receive permissions associated with every group.

\end{enumerate}


\subsection{Username transformation}
\label{\detokenize{authorization:username-transformation}}\label{\detokenize{authorization:usertodnmapping}}
\sphinxAtStartPar
If clients connect to \sphinxstyleemphasis{Percona Server for MongoDB} with usernames that are not LDAP \sphinxstyleabbreviation{DN}, these usernames must be converted to the format acceptable by LDAP.

\sphinxAtStartPar
To achieve this,  the \sphinxcode{\sphinxupquote{\sphinxhyphen{}\sphinxhyphen{}ldapUserToDNMapping}} parameter is available in \sphinxstyleemphasis{Percona Server for MongoDB} configuration.

\sphinxAtStartPar
The \sphinxcode{\sphinxupquote{\sphinxhyphen{}\sphinxhyphen{}ldapUserToDNMapping}} parameter is a JSON string representing an ordered array of rules expressed as JSON documents. Each document provides a regex pattern (\sphinxcode{\sphinxupquote{match}} field) to match against a provided username. If that pattern matches, there are two ways to continue:
\begin{itemize}
\item {} 
\sphinxAtStartPar
If there is the \sphinxcode{\sphinxupquote{substitution}} value, then the matched pattern becomes the username of the user for further processing.

\item {} 
\sphinxAtStartPar
If there is the \sphinxcode{\sphinxupquote{ldapQuery}} value, the matched pattern is sent to the LDAP server and the result of that LDAP query becomes the \sphinxstyleabbreviation{DN} of the user for further processing.

\end{itemize}

\sphinxAtStartPar
Both \sphinxcode{\sphinxupquote{substitution}} and \sphinxcode{\sphinxupquote{ldapQuery}} should contain placeholders to insert parts of the original username \textendash{} those placeholders are replaced with regular expression submatches found on the \sphinxcode{\sphinxupquote{match}} stage.

\sphinxAtStartPar
So having an array of documents, \sphinxstyleemphasis{Percona Server for MongoDB} tries to match each document against the provided name and if it matches, the name is replaced either with the substitution string or with the result of the LDAP query.
\subsubsection*{LDAP referrals}

\sphinxAtStartPar
As of version 4.2.10\sphinxhyphen{}11, \sphinxstyleemphasis{Percona Server for MongoDB} supports LDAP referrals as defined in \sphinxhref{https://www.rfc-editor.org/rfc/rfc4511.txt}{RFC 4511 4.1.10}. For security reasons, referrals are disabled by default. Double\sphinxhyphen{}check that using referrals is safe before enabling them.

\sphinxAtStartPar
To enable LDAP referrals, set the \sphinxcode{\sphinxupquote{ldapFollowReferrals}} server parameter to \sphinxcode{\sphinxupquote{true}} using the {\hyperref[\detokenize{set-parameter:setparameter}]{\sphinxcrossref{\DUrole{std,std-ref}{setParameter}}}} command or by editing the configuration file.

\begin{sphinxVerbatim}[commandchars=\\\{\}]
\PYG{n+nt}{setParameter}\PYG{p}{:}
\PYG{+w}{   }\PYG{n+nt}{ldapFollowReferrals}\PYG{p}{:}\PYG{+w}{ }\PYG{l+lScalar+lScalarPlain}{true}
\end{sphinxVerbatim}
\subsubsection*{Connection pool}

\sphinxAtStartPar
As of version 4.2.10\sphinxhyphen{}11, \sphinxstyleemphasis{Percona Server for MongoDB} always uses a connection pool to LDAP server to process bind requests. The connection pool is enabled by default. The default connection pool size is 2 connections.

\sphinxAtStartPar
You can change the connection pool size either at the server startup or dynamically by specifying the value for the \sphinxcode{\sphinxupquote{ldapConnectionPoolSizePerHost}} server parameter.

\sphinxAtStartPar
For example, to set the number of connections in the pool to 5, use the \sphinxcode{\sphinxupquote{setParameter}} command:
\begin{quote}

\begin{sphinxadmonition}{note}{Command line}

\begin{sphinxVerbatim}[commandchars=\\\{\}]
\PYG{n+nx}{\PYGZdl{}}\PYG{+w}{ }\PYG{n+nx}{db}\PYG{p}{.}\PYG{n+nx}{adminCommand}\PYG{p}{(}\PYG{+w}{ }\PYG{p}{\PYGZob{}}\PYG{+w}{ }\PYG{n+nx}{setParameter}\PYG{o}{:}\PYG{+w}{ }\PYG{l+m+mf}{1}\PYG{p}{,}\PYG{+w}{ }\PYG{n+nx}{ldapConnectionPoolSizePerHost}\PYG{o}{:}\PYG{+w}{ }\PYG{l+m+mf}{5}\PYG{+w}{  }\PYG{p}{\PYGZcb{}}\PYG{+w}{ }\PYG{p}{)}
\end{sphinxVerbatim}
\end{sphinxadmonition}

\begin{sphinxadmonition}{note}{Configuration file:}

\begin{sphinxVerbatim}[commandchars=\\\{\}]
\PYG{n+nt}{setParameter}\PYG{p}{:}
\PYG{+w}{  }\PYG{n+nt}{ldapConnectionPoolSizePerHost}\PYG{p}{:}\PYG{+w}{ }\PYG{l+lScalar+lScalarPlain}{5}
\end{sphinxVerbatim}
\end{sphinxadmonition}
\end{quote}
\subsubsection*{Support for multiple LDAP servers}

\sphinxAtStartPar
As of version 4.2.12\sphinxhyphen{}13, you can specify multiple LDAP servers for failover. \sphinxstyleemphasis{Percona Server for MongoDB} sends bind requests to the first server defined in the list. When this server is down or unavailable, it sends requests to the next server  and so on. Note that \sphinxstyleemphasis{Percona Server for MongoDB} keeps sending requests to this server even after the unavailable server recovers.

\sphinxAtStartPar
Specify the LDAP servers as a comma\sphinxhyphen{}separated list in the format \sphinxcode{\sphinxupquote{\textless{}host\textgreater{}:\textless{}port\textgreater{}}} for the \sphinxhref{https://docs.mongodb.com/manual/reference/program/mongod/index.html\#cmdoption-mongod-ldapservers}{\textendash{}ldapServers} option.

\sphinxAtStartPar
You can define the option value at the server startup by editing the configuration file.

\begin{sphinxVerbatim}[commandchars=\\\{\}]
\PYG{n+nt}{security}\PYG{p}{:}
\PYG{+w}{  }\PYG{n+nt}{authorization}\PYG{p}{:}\PYG{+w}{ }\PYG{l+s}{\PYGZdq{}}\PYG{l+s}{enabled}\PYG{l+s}{\PYGZdq{}}
\PYG{+w}{  }\PYG{n+nt}{ldap}\PYG{p}{:}
\PYG{+w}{    }\PYG{n+nt}{servers}\PYG{p}{:}\PYG{+w}{ }\PYG{l+s}{\PYGZdq{}}\PYG{l+s}{ldap1.example.net,ldap2.example.net}\PYG{l+s}{\PYGZdq{}}
\end{sphinxVerbatim}

\sphinxAtStartPar
You can change \sphinxcode{\sphinxupquote{ldapServers}} dynamically at runtime using the {\hyperref[\detokenize{set-parameter:setparameter}]{\sphinxcrossref{\DUrole{std,std-ref}{setParameter}}}}.

\begin{sphinxVerbatim}[commandchars=\\\{\}]
\PYGZdl{} db.adminCommand( \PYGZob{} setParameter: 1, ldapServers:\PYGZdq{}localhost,ldap1.example.net,ldap2.example.net\PYGZdq{}\PYGZcb{} )
\PYGZob{} \PYGZdq{}was\PYGZdq{} : \PYGZdq{}ldap1.example.net,ldap2.example.net\PYGZdq{}, \PYGZdq{}ok\PYGZdq{} : 1 \PYGZcb{}
\end{sphinxVerbatim}


\sphinxstrong{See also:}
\nopagebreak

\begin{description}
\item[{\sphinxstyleemphasis{MongoDB} Documentation:}] \leavevmode\begin{itemize}
\item {} 
\sphinxAtStartPar
\sphinxhref{https://docs.mongodb.com/manual/tutorial/authenticate-nativeldap-activedirectory/}{Authenticate and Authorize Users Using Active Directory via Native LDAP}

\item {} 
\sphinxAtStartPar
\sphinxhref{https://ldapwiki.com/wiki/LDAP\%20Referral}{LDAP referrals}

\end{itemize}

\end{description}




\section{Configuration}
\label{\detokenize{authorization:configuration}}
\sphinxAtStartPar
For how to configure LDAP authorization with the native LDAP authentication, see {\hyperref[\detokenize{ldap-setup:ldap-setup}]{\sphinxcrossref{\DUrole{std,std-ref}{Setting up LDAP authentication and authorization using NativeLDAP}}}}.


\chapter{Auditing}
\label{\detokenize{audit-logging:auditing}}\label{\detokenize{audit-logging:audit-log}}\label{\detokenize{audit-logging::doc}}
\sphinxAtStartPar
Auditing allows administrators to track
and log user activity on a MongoDB server.
With auditing enabled, the server will generate an audit log file.
This file contains information about different user events
including authentication, authorization failures, and so on.

\sphinxAtStartPar
To enable audit logging, specify where to send audit events
using the {\hyperref[\detokenize{audit-logging:cmdoption-auditDestination}]{\sphinxcrossref{\sphinxcode{\sphinxupquote{\sphinxhyphen{}\sphinxhyphen{}auditDestination}}}}} option on the command line
or the \sphinxcode{\sphinxupquote{auditLog.destination}} variable in the configuration file.

\sphinxAtStartPar
If you want to output events to a file,
also specify the format of the file
using the {\hyperref[\detokenize{audit-logging:cmdoption-auditFormat}]{\sphinxcrossref{\sphinxcode{\sphinxupquote{\sphinxhyphen{}\sphinxhyphen{}auditFormat}}}}} option
or the \sphinxcode{\sphinxupquote{auditLog.format}} variable,
and the path to the file using the {\hyperref[\detokenize{audit-logging:cmdoption-auditPath}]{\sphinxcrossref{\sphinxcode{\sphinxupquote{\sphinxhyphen{}\sphinxhyphen{}auditPath}}}}} option
or the \sphinxcode{\sphinxupquote{auditLog.path}} variable.

\sphinxAtStartPar
To filter recorded events, use the {\hyperref[\detokenize{audit-logging:cmdoption-auditFilter}]{\sphinxcrossref{\sphinxcode{\sphinxupquote{\sphinxhyphen{}\sphinxhyphen{}auditFilter}}}}} option
or the \sphinxcode{\sphinxupquote{auditLog.filter}} variable.

\sphinxAtStartPar
For example, to log only events from a user named \sphinxstyleemphasis{tim}
and write them to a JSON file \sphinxcode{\sphinxupquote{/var/log/psmdb/audit.json}},
start the server with the following parameters:

\begin{sphinxVerbatim}[commandchars=\\\{\}]
mongod \PYG{l+s+se}{\PYGZbs{}}
 \PYGZhy{}\PYGZhy{}dbpath data/db
 \PYGZhy{}\PYGZhy{}auditDestination file \PYG{l+s+se}{\PYGZbs{}}
 \PYGZhy{}\PYGZhy{}auditFormat JSON \PYG{l+s+se}{\PYGZbs{}}
 \PYGZhy{}\PYGZhy{}auditPath /var/log/psmdb/audit.json \PYG{l+s+se}{\PYGZbs{}}
 \PYGZhy{}\PYGZhy{}auditFilter \PYG{l+s+s1}{\PYGZsq{}\PYGZob{} \PYGZdq{}users.user\PYGZdq{} : \PYGZdq{}tim\PYGZdq{} \PYGZcb{}\PYGZsq{}}
\end{sphinxVerbatim}

\sphinxAtStartPar
The options in the previous example can be used as variables
in the MongoDB configuration file:

\begin{sphinxVerbatim}[commandchars=\\\{\}]
\PYG{n}{storage}\PYG{p}{:}
  \PYG{n}{dbPath}\PYG{p}{:} \PYG{n}{data}\PYG{o}{/}\PYG{n}{db}
\PYG{n}{auditLog}\PYG{p}{:}
  \PYG{n}{destination}\PYG{p}{:} \PYG{n}{file}
  \PYG{n+nb}{format}\PYG{p}{:} \PYG{n}{JSON}
  \PYG{n}{path}\PYG{p}{:} \PYG{o}{/}\PYG{n}{var}\PYG{o}{/}\PYG{n}{log}\PYG{o}{/}\PYG{n}{psmdb}\PYG{o}{/}\PYG{n}{audit}\PYG{o}{.}\PYG{n}{json}
  \PYG{n+nb}{filter}\PYG{p}{:} \PYG{l+s+s1}{\PYGZsq{}}\PYG{l+s+s1}{\PYGZob{}}\PYG{l+s+s1}{ }\PYG{l+s+s1}{\PYGZdq{}}\PYG{l+s+s1}{users.user}\PYG{l+s+s1}{\PYGZdq{}}\PYG{l+s+s1}{ : }\PYG{l+s+s1}{\PYGZdq{}}\PYG{l+s+s1}{tim}\PYG{l+s+s1}{\PYGZdq{}}\PYG{l+s+s1}{ \PYGZcb{}}\PYG{l+s+s1}{\PYGZsq{}}
\end{sphinxVerbatim}

\sphinxAtStartPar
This example shows how to send audit events to the
\sphinxcode{\sphinxupquote{syslog}}. Specify the following parameters:

\begin{sphinxVerbatim}[commandchars=\\\{\}]
mongod \PYG{l+s+se}{\PYGZbs{}}
\PYGZhy{}\PYGZhy{}dbpath data/db
\PYGZhy{}\PYGZhy{}auditDestination syslog \PYG{l+s+se}{\PYGZbs{}}
\end{sphinxVerbatim}

\sphinxAtStartPar
Alternatively, you can edit the MongoDB configuration file:

\begin{sphinxVerbatim}[commandchars=\\\{\}]
\PYG{n}{storage}\PYG{p}{:}
 \PYG{n}{dbPath}\PYG{p}{:} \PYG{n}{data}\PYG{o}{/}\PYG{n}{db}
\PYG{n}{auditLog}\PYG{p}{:}
 \PYG{n}{destination}\PYG{p}{:} \PYG{n}{syslog}
\end{sphinxVerbatim}

\begin{sphinxadmonition}{note}{Note:}
\sphinxAtStartPar
If you start the server with auditing enabled,
it cannot be disabled dynamically during runtime.
\end{sphinxadmonition}


\section{Audit Options}
\label{\detokenize{audit-logging:audit-options}}
\sphinxAtStartPar
The following options control audit logging:
\index{command line option@\spxentry{command line option}!\sphinxhyphen{}\sphinxhyphen{}auditDestination@\spxentry{\sphinxhyphen{}\sphinxhyphen{}auditDestination}}\index{\sphinxhyphen{}\sphinxhyphen{}auditDestination@\spxentry{\sphinxhyphen{}\sphinxhyphen{}auditDestination}!command line option@\spxentry{command line option}}

\begin{fulllineitems}
\phantomsection\label{\detokenize{audit-logging:cmdoption-auditDestination}}\pysigline{\sphinxbfcode{\sphinxupquote{\sphinxhyphen{}\sphinxhyphen{}auditDestination}}\sphinxcode{\sphinxupquote{}}}~\begin{quote}\begin{description}
\item[{Variable}] \leavevmode
\sphinxAtStartPar
\sphinxcode{\sphinxupquote{auditLog.destination}}

\item[{Type}] \leavevmode
\sphinxAtStartPar
String

\end{description}\end{quote}

\sphinxAtStartPar
Enables auditing and specifies where to send audit events:
\begin{itemize}
\item {} 
\sphinxAtStartPar
\sphinxcode{\sphinxupquote{console}}: Output audit events to \sphinxcode{\sphinxupquote{stdout}}.

\item {} 
\sphinxAtStartPar
\sphinxcode{\sphinxupquote{file}}: Output audit events to a file
specified by the {\hyperref[\detokenize{audit-logging:cmdoption-auditPath}]{\sphinxcrossref{\sphinxcode{\sphinxupquote{\sphinxhyphen{}\sphinxhyphen{}auditPath}}}}} option
in a format specified by the {\hyperref[\detokenize{audit-logging:cmdoption-auditFormat}]{\sphinxcrossref{\sphinxcode{\sphinxupquote{\sphinxhyphen{}\sphinxhyphen{}auditFormat}}}}} option.

\item {} 
\sphinxAtStartPar
\sphinxcode{\sphinxupquote{syslog}}: Output audit events to \sphinxcode{\sphinxupquote{syslog}}.

\end{itemize}

\end{fulllineitems}

\index{command line option@\spxentry{command line option}!\sphinxhyphen{}\sphinxhyphen{}auditFilter@\spxentry{\sphinxhyphen{}\sphinxhyphen{}auditFilter}}\index{\sphinxhyphen{}\sphinxhyphen{}auditFilter@\spxentry{\sphinxhyphen{}\sphinxhyphen{}auditFilter}!command line option@\spxentry{command line option}}

\begin{fulllineitems}
\phantomsection\label{\detokenize{audit-logging:cmdoption-auditFilter}}\pysigline{\sphinxbfcode{\sphinxupquote{\sphinxhyphen{}\sphinxhyphen{}auditFilter}}\sphinxcode{\sphinxupquote{}}}~\begin{quote}\begin{description}
\item[{Variable}] \leavevmode
\sphinxAtStartPar
\sphinxcode{\sphinxupquote{auditLog.filter}}

\item[{Type}] \leavevmode
\sphinxAtStartPar
String

\end{description}\end{quote}

\sphinxAtStartPar
Specifies a filter to apply to incoming audit events,
enabling the administrator to only capture a subset of them.
The value must be interpreted as a query object with the following syntax:

\begin{sphinxVerbatim}[commandchars=\\\{\}]
\PYG{p}{\PYGZob{}} \PYG{o}{\PYGZlt{}}\PYG{n}{field1}\PYG{o}{\PYGZgt{}}\PYG{p}{:} \PYG{o}{\PYGZlt{}}\PYG{n}{expression1}\PYG{o}{\PYGZgt{}}\PYG{p}{,} \PYG{o}{.}\PYG{o}{.}\PYG{o}{.} \PYG{p}{\PYGZcb{}}
\end{sphinxVerbatim}

\sphinxAtStartPar
Audit log events that match this query will be logged.
Events that do not match this query will be ignored.

\sphinxAtStartPar
For more information, see {\hyperref[\detokenize{audit-logging:audit-filter-examples}]{\sphinxcrossref{\DUrole{std,std-ref}{Audit Filter Examples}}}}.

\end{fulllineitems}

\index{command line option@\spxentry{command line option}!\sphinxhyphen{}\sphinxhyphen{}auditFormat@\spxentry{\sphinxhyphen{}\sphinxhyphen{}auditFormat}}\index{\sphinxhyphen{}\sphinxhyphen{}auditFormat@\spxentry{\sphinxhyphen{}\sphinxhyphen{}auditFormat}!command line option@\spxentry{command line option}}

\begin{fulllineitems}
\phantomsection\label{\detokenize{audit-logging:cmdoption-auditFormat}}\pysigline{\sphinxbfcode{\sphinxupquote{\sphinxhyphen{}\sphinxhyphen{}auditFormat}}\sphinxcode{\sphinxupquote{}}}~\begin{quote}\begin{description}
\item[{Variable}] \leavevmode
\sphinxAtStartPar
\sphinxcode{\sphinxupquote{auditLog.format}}

\item[{Type}] \leavevmode
\sphinxAtStartPar
String

\end{description}\end{quote}

\sphinxAtStartPar
Specifies the format of the audit log file,
if you set the {\hyperref[\detokenize{audit-logging:cmdoption-auditDestination}]{\sphinxcrossref{\sphinxcode{\sphinxupquote{\sphinxhyphen{}\sphinxhyphen{}auditDestination}}}}} option to \sphinxcode{\sphinxupquote{file}}.

\sphinxAtStartPar
The default value is \sphinxcode{\sphinxupquote{JSON}}.
Alternatively, you can set it to \sphinxcode{\sphinxupquote{BSON}}.

\end{fulllineitems}

\index{command line option@\spxentry{command line option}!\sphinxhyphen{}\sphinxhyphen{}auditPath@\spxentry{\sphinxhyphen{}\sphinxhyphen{}auditPath}}\index{\sphinxhyphen{}\sphinxhyphen{}auditPath@\spxentry{\sphinxhyphen{}\sphinxhyphen{}auditPath}!command line option@\spxentry{command line option}}

\begin{fulllineitems}
\phantomsection\label{\detokenize{audit-logging:cmdoption-auditPath}}\pysigline{\sphinxbfcode{\sphinxupquote{\sphinxhyphen{}\sphinxhyphen{}auditPath}}\sphinxcode{\sphinxupquote{}}}~\begin{quote}\begin{description}
\item[{Variable}] \leavevmode
\sphinxAtStartPar
\sphinxcode{\sphinxupquote{auditLog.path}}

\item[{Type}] \leavevmode
\sphinxAtStartPar
String

\end{description}\end{quote}

\sphinxAtStartPar
Specifies the fully qualified path to the file
where audit log events are written,
if you set the {\hyperref[\detokenize{audit-logging:cmdoption-auditDestination}]{\sphinxcrossref{\sphinxcode{\sphinxupquote{\sphinxhyphen{}\sphinxhyphen{}auditDestination}}}}} option to \sphinxcode{\sphinxupquote{file}}.

\sphinxAtStartPar
If this option is not specified,
then the \sphinxcode{\sphinxupquote{auditLog.json}} file is created
in the server’s configured log path.
If log path is not configured on the server,
then the \sphinxcode{\sphinxupquote{auditLog.json}} file is created in the current directory
(from which \sphinxcode{\sphinxupquote{mongod}} was started).

\begin{sphinxadmonition}{note}{Note:}
\sphinxAtStartPar
This file will rotate in the same manner as the system log path,
either on server reboot or using the \sphinxcode{\sphinxupquote{logRotate}} command.
The time of rotation will be added to the old file’s name.
\end{sphinxadmonition}

\end{fulllineitems}



\section{Audit Message Syntax}
\label{\detokenize{audit-logging:audit-message-syntax}}
\sphinxAtStartPar
Audit logging writes messages in JSON format with the following syntax:

\begin{sphinxVerbatim}[commandchars=\\\{\}]
\PYG{p}{\PYGZob{}}
  \PYG{n}{atype}\PYG{p}{:} \PYG{o}{\PYGZlt{}}\PYG{n}{String}\PYG{o}{\PYGZgt{}}\PYG{p}{,}
  \PYG{n}{ts} \PYG{p}{:} \PYG{p}{\PYGZob{}} \PYG{l+s+s2}{\PYGZdq{}}\PYG{l+s+s2}{\PYGZdl{}date}\PYG{l+s+s2}{\PYGZdq{}}\PYG{p}{:} \PYG{o}{\PYGZlt{}}\PYG{n}{timestamp}\PYG{o}{\PYGZgt{}} \PYG{p}{\PYGZcb{}}\PYG{p}{,}
  \PYG{n}{local}\PYG{p}{:} \PYG{p}{\PYGZob{}} \PYG{n}{ip}\PYG{p}{:} \PYG{o}{\PYGZlt{}}\PYG{n}{String}\PYG{o}{\PYGZgt{}}\PYG{p}{,} \PYG{n}{port}\PYG{p}{:} \PYG{o}{\PYGZlt{}}\PYG{n+nb}{int}\PYG{o}{\PYGZgt{}} \PYG{p}{\PYGZcb{}}\PYG{p}{,}
  \PYG{n}{remote}\PYG{p}{:} \PYG{p}{\PYGZob{}} \PYG{n}{ip}\PYG{p}{:} \PYG{o}{\PYGZlt{}}\PYG{n}{String}\PYG{o}{\PYGZgt{}}\PYG{p}{,} \PYG{n}{port}\PYG{p}{:} \PYG{o}{\PYGZlt{}}\PYG{n+nb}{int}\PYG{o}{\PYGZgt{}} \PYG{p}{\PYGZcb{}}\PYG{p}{,}
  \PYG{n}{users} \PYG{p}{:} \PYG{p}{[} \PYG{p}{\PYGZob{}} \PYG{n}{user}\PYG{p}{:} \PYG{o}{\PYGZlt{}}\PYG{n}{String}\PYG{o}{\PYGZgt{}}\PYG{p}{,} \PYG{n}{db}\PYG{p}{:} \PYG{o}{\PYGZlt{}}\PYG{n}{String}\PYG{o}{\PYGZgt{}} \PYG{p}{\PYGZcb{}}\PYG{p}{,} \PYG{o}{.}\PYG{o}{.}\PYG{o}{.} \PYG{p}{]}\PYG{p}{,}
  \PYG{n}{roles}\PYG{p}{:} \PYG{p}{[} \PYG{p}{\PYGZob{}} \PYG{n}{role}\PYG{p}{:} \PYG{o}{\PYGZlt{}}\PYG{n}{String}\PYG{o}{\PYGZgt{}}\PYG{p}{,} \PYG{n}{db}\PYG{p}{:} \PYG{o}{\PYGZlt{}}\PYG{n}{String}\PYG{o}{\PYGZgt{}} \PYG{p}{\PYGZcb{}}\PYG{p}{,} \PYG{o}{.}\PYG{o}{.}\PYG{o}{.} \PYG{p}{]}\PYG{p}{,}
  \PYG{n}{param}\PYG{p}{:} \PYG{o}{\PYGZlt{}}\PYG{n}{document}\PYG{o}{\PYGZgt{}}\PYG{p}{,}
  \PYG{n}{result}\PYG{p}{:} \PYG{o}{\PYGZlt{}}\PYG{n+nb}{int}\PYG{o}{\PYGZgt{}}
\PYG{p}{\PYGZcb{}}
\end{sphinxVerbatim}
\begin{quote}\begin{description}
\item[{atype}] \leavevmode
\sphinxAtStartPar
Event type

\item[{ts}] \leavevmode
\sphinxAtStartPar
Date and UTC time of the event

\item[{local}] \leavevmode
\sphinxAtStartPar
Local IP address and port number of the instance

\item[{remote}] \leavevmode
\sphinxAtStartPar
Remote IP address and port number
of the incoming connection associated with the event

\item[{users}] \leavevmode
\sphinxAtStartPar
Users associated with the event

\item[{roles}] \leavevmode
\sphinxAtStartPar
Roles granted to the user

\item[{param}] \leavevmode
\sphinxAtStartPar
Details of the event associated with the specific type

\item[{result}] \leavevmode
\sphinxAtStartPar
Exit code (\sphinxcode{\sphinxupquote{0}} for success)

\end{description}\end{quote}


\section{Audit Filter Examples}
\label{\detokenize{audit-logging:audit-filter-examples}}\label{\detokenize{audit-logging:id1}}
\sphinxAtStartPar
The following examples demonstrate the flexibility of audit log filters.

\begin{sphinxShadowBox}
\begin{itemize}
\item {} 
\sphinxAtStartPar
\phantomsection\label{\detokenize{audit-logging:id2}}{\hyperref[\detokenize{audit-logging:standard-query-selectors}]{\sphinxcrossref{Standard Query Selectors}}}

\item {} 
\sphinxAtStartPar
\phantomsection\label{\detokenize{audit-logging:id3}}{\hyperref[\detokenize{audit-logging:regular-expressions}]{\sphinxcrossref{Regular Expressions}}}

\item {} 
\sphinxAtStartPar
\phantomsection\label{\detokenize{audit-logging:id4}}{\hyperref[\detokenize{audit-logging:read-and-write-operations}]{\sphinxcrossref{Read and Write Operations}}}

\end{itemize}
\end{sphinxShadowBox}

\begin{sphinxVerbatim}[commandchars=\\\{\}]
auditLog:
   destination: file
      filter: \PYGZsq{}\PYGZob{}atype: \PYGZob{}\PYGZdl{}in: [
         \PYGZdq{}authenticate\PYGZdq{}, \PYGZdq{}authCheck\PYGZdq{},
         \PYGZdq{}renameCollection\PYGZdq{}, \PYGZdq{}dropCollection\PYGZdq{}, \PYGZdq{}dropDatabase\PYGZdq{},
         \PYGZdq{}createUser\PYGZdq{}, \PYGZdq{}dropUser\PYGZdq{}, \PYGZdq{}dropAllUsersFromDatabase\PYGZdq{}, \PYGZdq{}updateuser\PYGZdq{},
         \PYGZdq{}grantRolesToUser\PYGZdq{}, \PYGZdq{}revokeRolesFromUser\PYGZdq{}, \PYGZdq{}createRole\PYGZdq{}, \PYGZdq{}updateRole\PYGZdq{},
         \PYGZdq{}dropRole\PYGZdq{}, \PYGZdq{}dropAllRolesFromDatabase\PYGZdq{}, \PYGZdq{}grantRolesToRole\PYGZdq{}, \PYGZdq{}revokeRolesFromRole\PYGZdq{},
         \PYGZdq{}grantPrivilegesToRole\PYGZdq{}, \PYGZdq{}revokePrivilegesFromRole\PYGZdq{},
         \PYGZdq{}replSetReconfig\PYGZdq{},
         \PYGZdq{}enableSharding\PYGZdq{}, \PYGZdq{}shardCollection\PYGZdq{}, \PYGZdq{}addShard\PYGZdq{}, \PYGZdq{}removeShard\PYGZdq{},
         \PYGZdq{}shutdown\PYGZdq{},
         \PYGZdq{}applicationMessage\PYGZdq{}
      ]\PYGZcb{}\PYGZcb{}\PYGZsq{}
\end{sphinxVerbatim}


\subsection{Standard Query Selectors}
\label{\detokenize{audit-logging:standard-query-selectors}}
\sphinxAtStartPar
You can use query selectors,
such as \sphinxcode{\sphinxupquote{\$eq}}, \sphinxcode{\sphinxupquote{\$in}}, \sphinxcode{\sphinxupquote{\$gt}}, \sphinxcode{\sphinxupquote{\$lt}}, \sphinxcode{\sphinxupquote{\$ne}}, and others
to log multiple event types.

\sphinxAtStartPar
For example, to log only the \sphinxcode{\sphinxupquote{dropCollection}} and \sphinxcode{\sphinxupquote{dropDatabase}} events:
\begin{itemize}
\item {} 
\sphinxAtStartPar
Command line:

\begin{sphinxVerbatim}[commandchars=\\\{\}]
\PYG{o}{\PYGZhy{}}\PYG{o}{\PYGZhy{}}\PYG{n}{auditDestination} \PYG{n}{file} \PYG{o}{\PYGZhy{}}\PYG{o}{\PYGZhy{}}\PYG{n}{auditFilter} \PYG{l+s+s1}{\PYGZsq{}}\PYG{l+s+s1}{\PYGZob{}}\PYG{l+s+s1}{ atype: }\PYG{l+s+s1}{\PYGZob{}}\PYG{l+s+s1}{ \PYGZdl{}in: [ }\PYG{l+s+s1}{\PYGZdq{}}\PYG{l+s+s1}{dropCollection}\PYG{l+s+s1}{\PYGZdq{}}\PYG{l+s+s1}{, }\PYG{l+s+s1}{\PYGZdq{}}\PYG{l+s+s1}{dropDatabase}\PYG{l+s+s1}{\PYGZdq{}}\PYG{l+s+s1}{ ] \PYGZcb{} \PYGZcb{}}\PYG{l+s+s1}{\PYGZsq{}}
\end{sphinxVerbatim}

\item {} 
\sphinxAtStartPar
Config file:

\begin{sphinxVerbatim}[commandchars=\\\{\}]
\PYG{n}{auditLog}\PYG{p}{:}
  \PYG{n}{destination}\PYG{p}{:} \PYG{n}{file}
  \PYG{n+nb}{filter}\PYG{p}{:} \PYG{l+s+s1}{\PYGZsq{}}\PYG{l+s+s1}{\PYGZob{}}\PYG{l+s+s1}{ atype: }\PYG{l+s+s1}{\PYGZob{}}\PYG{l+s+s1}{ \PYGZdl{}in: [ }\PYG{l+s+s1}{\PYGZdq{}}\PYG{l+s+s1}{dropCollection}\PYG{l+s+s1}{\PYGZdq{}}\PYG{l+s+s1}{, }\PYG{l+s+s1}{\PYGZdq{}}\PYG{l+s+s1}{dropDatabase}\PYG{l+s+s1}{\PYGZdq{}}\PYG{l+s+s1}{ ] \PYGZcb{} \PYGZcb{}}\PYG{l+s+s1}{\PYGZsq{}}
\end{sphinxVerbatim}

\end{itemize}


\subsection{Regular Expressions}
\label{\detokenize{audit-logging:regular-expressions}}
\sphinxAtStartPar
Another way to specify multiple event types is using regular expressions.

\sphinxAtStartPar
For example, to filter all \sphinxcode{\sphinxupquote{drop}} operations:
\begin{itemize}
\item {} 
\sphinxAtStartPar
Command line:

\begin{sphinxVerbatim}[commandchars=\\\{\}]
\PYG{o}{\PYGZhy{}}\PYG{o}{\PYGZhy{}}\PYG{n}{auditDestination} \PYG{n}{file} \PYG{o}{\PYGZhy{}}\PYG{o}{\PYGZhy{}}\PYG{n}{auditFilter} \PYG{l+s+s1}{\PYGZsq{}}\PYG{l+s+s1}{\PYGZob{}}\PYG{l+s+s1}{ }\PYG{l+s+s1}{\PYGZdq{}}\PYG{l+s+s1}{atype}\PYG{l+s+s1}{\PYGZdq{}}\PYG{l+s+s1}{ : /\PYGZca{}drop.*/ \PYGZcb{}}\PYG{l+s+s1}{\PYGZsq{}}
\end{sphinxVerbatim}

\item {} 
\sphinxAtStartPar
Config file:

\begin{sphinxVerbatim}[commandchars=\\\{\}]
\PYG{n}{auditLog}\PYG{p}{:}
  \PYG{n}{destination}\PYG{p}{:} \PYG{n}{file}
  \PYG{n+nb}{filter}\PYG{p}{:} \PYG{l+s+s1}{\PYGZsq{}}\PYG{l+s+s1}{\PYGZob{}}\PYG{l+s+s1}{ }\PYG{l+s+s1}{\PYGZdq{}}\PYG{l+s+s1}{atype}\PYG{l+s+s1}{\PYGZdq{}}\PYG{l+s+s1}{ : /\PYGZca{}drop.*/ \PYGZcb{}}\PYG{l+s+s1}{\PYGZsq{}}
\end{sphinxVerbatim}

\end{itemize}


\subsection{Read and Write Operations}
\label{\detokenize{audit-logging:read-and-write-operations}}
\sphinxAtStartPar
By default, operations with successful authorization are not logged,
so for this filter to work, enable \sphinxcode{\sphinxupquote{auditAuthorizationSuccess}} parameter,
as described in {\hyperref[\detokenize{audit-logging:auditauthorizationsuccess}]{\sphinxcrossref{\DUrole{std,std-ref}{Enabling Auditing of Authorization Success}}}}.

\sphinxAtStartPar
For example, to filter read and write operations
on all the collections in the \sphinxcode{\sphinxupquote{test}} database:

\begin{sphinxadmonition}{note}{Note:}
\sphinxAtStartPar
The dot (\sphinxcode{\sphinxupquote{.}}) after the database name in the regular expression
must be escaped with two backslashes (\sphinxcode{\sphinxupquote{\textbackslash{}\textbackslash{}}}).
\end{sphinxadmonition}
\begin{itemize}
\item {} 
\sphinxAtStartPar
Command line:

\begin{sphinxVerbatim}[commandchars=\\\{\}]
\PYG{o}{\PYGZhy{}}\PYG{o}{\PYGZhy{}}\PYG{n}{setParameter} \PYG{n}{auditAuthorizationSuccess}\PYG{o}{=}\PYG{n}{true} \PYG{o}{\PYGZhy{}}\PYG{o}{\PYGZhy{}}\PYG{n}{auditDestination} \PYG{n}{file} \PYG{o}{\PYGZhy{}}\PYG{o}{\PYGZhy{}}\PYG{n}{auditFilter} \PYG{l+s+s1}{\PYGZsq{}}\PYG{l+s+s1}{\PYGZob{}}\PYG{l+s+s1}{ atype: }\PYG{l+s+s1}{\PYGZdq{}}\PYG{l+s+s1}{authCheck}\PYG{l+s+s1}{\PYGZdq{}}\PYG{l+s+s1}{, }\PYG{l+s+s1}{\PYGZdq{}}\PYG{l+s+s1}{param.command}\PYG{l+s+s1}{\PYGZdq{}}\PYG{l+s+s1}{: }\PYG{l+s+s1}{\PYGZob{}}\PYG{l+s+s1}{ \PYGZdl{}in: [ }\PYG{l+s+s1}{\PYGZdq{}}\PYG{l+s+s1}{find}\PYG{l+s+s1}{\PYGZdq{}}\PYG{l+s+s1}{, }\PYG{l+s+s1}{\PYGZdq{}}\PYG{l+s+s1}{insert}\PYG{l+s+s1}{\PYGZdq{}}\PYG{l+s+s1}{, }\PYG{l+s+s1}{\PYGZdq{}}\PYG{l+s+s1}{delete}\PYG{l+s+s1}{\PYGZdq{}}\PYG{l+s+s1}{, }\PYG{l+s+s1}{\PYGZdq{}}\PYG{l+s+s1}{update}\PYG{l+s+s1}{\PYGZdq{}}\PYG{l+s+s1}{, }\PYG{l+s+s1}{\PYGZdq{}}\PYG{l+s+s1}{findandmodify}\PYG{l+s+s1}{\PYGZdq{}}\PYG{l+s+s1}{ ] \PYGZcb{}, }\PYG{l+s+s1}{\PYGZdq{}}\PYG{l+s+s1}{param.ns}\PYG{l+s+s1}{\PYGZdq{}}\PYG{l+s+s1}{: /\PYGZca{}test}\PYG{l+s+se}{\PYGZbs{}\PYGZbs{}}\PYG{l+s+s1}{./ \PYGZcb{} \PYGZcb{}}\PYG{l+s+s1}{\PYGZsq{}}
\end{sphinxVerbatim}

\item {} 
\sphinxAtStartPar
Config file:

\begin{sphinxVerbatim}[commandchars=\\\{\}]
\PYG{n}{auditLog}\PYG{p}{:}
  \PYG{n}{destination}\PYG{p}{:} \PYG{n}{file}
  \PYG{n+nb}{filter}\PYG{p}{:} \PYG{l+s+s1}{\PYGZsq{}}\PYG{l+s+s1}{\PYGZob{}}\PYG{l+s+s1}{ atype: }\PYG{l+s+s1}{\PYGZdq{}}\PYG{l+s+s1}{authCheck}\PYG{l+s+s1}{\PYGZdq{}}\PYG{l+s+s1}{, }\PYG{l+s+s1}{\PYGZdq{}}\PYG{l+s+s1}{param.command}\PYG{l+s+s1}{\PYGZdq{}}\PYG{l+s+s1}{: }\PYG{l+s+s1}{\PYGZob{}}\PYG{l+s+s1}{ \PYGZdl{}in: [ }\PYG{l+s+s1}{\PYGZdq{}}\PYG{l+s+s1}{find}\PYG{l+s+s1}{\PYGZdq{}}\PYG{l+s+s1}{, }\PYG{l+s+s1}{\PYGZdq{}}\PYG{l+s+s1}{insert}\PYG{l+s+s1}{\PYGZdq{}}\PYG{l+s+s1}{, }\PYG{l+s+s1}{\PYGZdq{}}\PYG{l+s+s1}{delete}\PYG{l+s+s1}{\PYGZdq{}}\PYG{l+s+s1}{, }\PYG{l+s+s1}{\PYGZdq{}}\PYG{l+s+s1}{update}\PYG{l+s+s1}{\PYGZdq{}}\PYG{l+s+s1}{, }\PYG{l+s+s1}{\PYGZdq{}}\PYG{l+s+s1}{findandmodify}\PYG{l+s+s1}{\PYGZdq{}}\PYG{l+s+s1}{ ] \PYGZcb{}, }\PYG{l+s+s1}{\PYGZdq{}}\PYG{l+s+s1}{param.ns}\PYG{l+s+s1}{\PYGZdq{}}\PYG{l+s+s1}{: /\PYGZca{}test}\PYG{l+s+se}{\PYGZbs{}\PYGZbs{}}\PYG{l+s+s1}{./ \PYGZcb{} \PYGZcb{}}\PYG{l+s+s1}{\PYGZsq{}}

\PYG{n}{setParameter}\PYG{p}{:} \PYG{p}{\PYGZob{}} \PYG{n}{auditAuthorizationSuccess}\PYG{p}{:} \PYG{n}{true} \PYG{p}{\PYGZcb{}}
\end{sphinxVerbatim}

\end{itemize}


\section{Enabling Auditing of Authorization Success}
\label{\detokenize{audit-logging:enabling-auditing-of-authorization-success}}\label{\detokenize{audit-logging:auditauthorizationsuccess}}
\sphinxAtStartPar
By default, only authorization failures for the \sphinxcode{\sphinxupquote{authCheck}} action
are logged by the audit system. \sphinxcode{\sphinxupquote{authCheck}} is for authorization by
role\sphinxhyphen{}based access control, it does not concern authentication at logins.

\sphinxAtStartPar
To enable logging of authorization successes,
set the \sphinxcode{\sphinxupquote{auditAuthorizationSuccess}} parameter to \sphinxcode{\sphinxupquote{true}}. Audit events
will then be triggered by every command, including CRUD ones.

\begin{sphinxadmonition}{warning}{Warning:}
\sphinxAtStartPar
Enabling the \sphinxcode{\sphinxupquote{auditAuthorizationSuccess}} parameter heavily impacts
the performance compared to logging only authorization failures.
\end{sphinxadmonition}

\sphinxAtStartPar
You can enable it on a running server using the following command:

\begin{sphinxVerbatim}[commandchars=\\\{\}]
\PYG{n}{db}\PYG{o}{.}\PYG{n}{adminCommand}\PYG{p}{(} \PYG{p}{\PYGZob{}} \PYG{n}{setParameter}\PYG{p}{:} \PYG{l+m+mi}{1}\PYG{p}{,} \PYG{n}{auditAuthorizationSuccess}\PYG{p}{:} \PYG{n}{true} \PYG{p}{\PYGZcb{}} \PYG{p}{)}
\end{sphinxVerbatim}

\sphinxAtStartPar
To enable it on the command line, use the following option
when running \sphinxcode{\sphinxupquote{mongod}} or \sphinxcode{\sphinxupquote{mongos}} process:

\begin{sphinxVerbatim}[commandchars=\\\{\}]
\PYG{o}{\PYGZhy{}}\PYG{o}{\PYGZhy{}}\PYG{n}{setParameter} \PYG{n}{auditAuthorizationSuccess}\PYG{o}{=}\PYG{n}{true}
\end{sphinxVerbatim}

\sphinxAtStartPar
You can also add it to the configuration file as follows:

\begin{sphinxVerbatim}[commandchars=\\\{\}]
\PYG{n}{setParameter}\PYG{p}{:}
  \PYG{n}{auditAuthorizationSuccess}\PYG{p}{:} \PYG{n}{true}
\end{sphinxVerbatim}


\chapter{Log Redaction}
\label{\detokenize{log-redaction:log-redaction}}\label{\detokenize{log-redaction:id1}}\label{\detokenize{log-redaction::doc}}
\sphinxAtStartPar
\sphinxstyleemphasis{Percona Server for MongoDB} can prevent writing sensitive data to the diagnostic log
by redacting messages of events before they are logged.
To enable log redaction,
run \sphinxcode{\sphinxupquote{mongod}} with the \sphinxcode{\sphinxupquote{\sphinxhyphen{}\sphinxhyphen{}redactClientLogData}} option.

\begin{sphinxadmonition}{note}{Note:}
\sphinxAtStartPar
Metadata such as error or operation codes, line numbers,
and source file names remain visible in the logs.
\end{sphinxadmonition}

\sphinxAtStartPar
Log redaction is important for complying with security requirements,
but it can make troubleshooting and diagnostics more difficult
due to the lack of data related to the log event.
For this reason, debug messages are not redacted
even when log redaction is enabled.
Keep this in mind when switching between log levels.

\sphinxAtStartPar
You can permanently enable log redaction
by adding the following to the configuration file:

\begin{sphinxVerbatim}[commandchars=\\\{\}]
\PYG{n}{security}\PYG{p}{:}
  \PYG{n}{redactClientLogData}\PYG{p}{:} \PYG{n}{true}
\end{sphinxVerbatim}

\sphinxAtStartPar
To enable log redaction at runtime,
use the \sphinxcode{\sphinxupquote{setParameter}} command as follows:

\begin{sphinxVerbatim}[commandchars=\\\{\}]
\PYG{n}{db}\PYG{o}{.}\PYG{n}{adminCommand}\PYG{p}{(}
  \PYG{p}{\PYGZob{}} \PYG{n}{setParameter}\PYG{p}{:} \PYG{l+m+mi}{1}\PYG{p}{,} \PYG{n}{redactClientLogData} \PYG{p}{:} \PYG{n}{true} \PYG{p}{\PYGZcb{}}
\PYG{p}{)}
\end{sphinxVerbatim}


\chapter{Data at Rest Encryption}
\label{\detokenize{data-at-rest-encryption:data-at-rest-encryption}}\label{\detokenize{data-at-rest-encryption:psmdb-data-at-rest-encryption}}\label{\detokenize{data-at-rest-encryption::doc}}
\begin{sphinxShadowBox}
\begin{itemize}
\item {} 
\sphinxAtStartPar
\phantomsection\label{\detokenize{data-at-rest-encryption:id1}}{\hyperref[\detokenize{data-at-rest-encryption:encrypting-rollback-files}]{\sphinxcrossref{Encrypting Rollback Files}}}

\end{itemize}
\end{sphinxShadowBox}

\sphinxAtStartPar
Data at rest encryption for the WiredTiger storage engine in \sphinxstyleemphasis{MongoDB} was
introduced in MongoDB Enterprise version 3.2 to ensure that encrypted data
files can be decrypted and read by parties with the decryption key.
\subsubsection*{Differences from Upstream}

\sphinxAtStartPar
The data encryption at rest in \sphinxstyleemphasis{Percona Server for MongoDB} is introduced in version 3.6 to be compatible with
data encryption at rest interface in \sphinxstyleemphasis{MongoDB}. In the current release of \sphinxstyleemphasis{Percona Server for MongoDB}, the data encryption at rest does not include support for Amazon AWS key management service. Instead, \sphinxstyleemphasis{Percona Server for MongoDB} is {\hyperref[\detokenize{vault:vault}]{\sphinxcrossref{\DUrole{std,std-ref}{integrated with HashiCorp Vault}}}}. Starting with release 4.4.13\sphinxhyphen{}13, \sphinxstyleemphasis{Percona Server for MongoDB} supports the secure transfer of keys using {\hyperref[\detokenize{kmip:kmip}]{\sphinxcrossref{\DUrole{std,std-ref}{Key Management Interoperability Protocol (KMIP)}}}}. This allows users to store encryption keys in their favorite KMIP\sphinxhyphen{}compatible key manager. when they set up encryption at rest.

\sphinxAtStartPar
Two types of keys are used for data at rest encryption:
\begin{itemize}
\item {} 
\sphinxAtStartPar
Database keys to encrypt data. They are stored internally, near the data that they encrypt.

\item {} 
\sphinxAtStartPar
The master key to encrypt database keys. It is kept separately from the data and database keys and requires external management.

\end{itemize}

\sphinxAtStartPar
To manage the master key, use one of the supported key management options:
\begin{itemize}
\item {} 
\sphinxAtStartPar
Integration with an external key server (recommended). \sphinxstyleemphasis{Percona Server for MongoDB} is {\hyperref[\detokenize{vault:vault}]{\sphinxcrossref{\DUrole{std,std-ref}{integrated with HashiCorp Vault}}}} for this purpose and supports the secure transfer of keys using {\hyperref[\detokenize{kmip:kmip}]{\sphinxcrossref{\DUrole{std,std-ref}{Key Management Interoperability Protocol (KMIP)}}}}.

\item {} 
\sphinxAtStartPar
{\hyperref[\detokenize{keyfile:keyfile}]{\sphinxcrossref{\DUrole{std,std-ref}{Local key management using a keyfile}}}}.

\end{itemize}

\sphinxAtStartPar
Note that you can use only one of the key management options at a time. However, you can switch from one management option to another (e.g. from a keyfile to HashiCorp Vault). Refer to {\hyperref[\detokenize{encryption-mode-switch:psmdb-encryption-mode-switch}]{\sphinxcrossref{\DUrole{std,std-ref}{Migrating from Key File Encryption to HashiCorp Vault Encryption}}}} section for details.

\begin{sphinxadmonition}{important}{Important:}
\sphinxAtStartPar
You can only enable data at rest encryption and provide all encryption settings on an empty database, when you start the \sphinxcode{\sphinxupquote{mongod}} instance for the first time. You cannot enable or disable encryption while the \sphinxstyleemphasis{Percona Server for MongoDB} server is already running and / or has some data. Nor can you change the effective encryption mode by simply restarting the server. Every time you restart the server, the encryption settings must be the same.
\end{sphinxadmonition}


\section{HashiCorp Vault Integration}
\label{\detokenize{vault:vault-integration}}\label{\detokenize{vault:vault}}\label{\detokenize{vault::doc}}
\sphinxAtStartPar
\sphinxstyleemphasis{Percona Server for MongoDB} is integrated with HashiCorp Vault. HashiCorp Vault supports different secrets engines. \sphinxstyleemphasis{Percona Server for MongoDB} only supports the HashiCorp Vault
back end with KV Secrets Engine \sphinxhyphen{} Version 2 (API)
with versioning enabled.


\sphinxstrong{See also:}
\nopagebreak


\sphinxAtStartPar
Percona Blog: \sphinxhref{https://www.percona.com/blog/2020/04/21/using-vault-to-store-the-master-key-for-data-at-rest-encryption-on-percona-server-for-mongodb/}{Using Vault to Store the Master Key for Data at Rest Encryption on  Percona Server for MongoDB}

\sphinxAtStartPar
HashiCorp Vault Documentation: \sphinxhref{https://www.vaultproject.io/api/secret/kv/kv-v2.html}{How to configure the KV Engine}



\begin{sphinxadmonition}{note}{HashiCorp Vault Parameters}


\begin{savenotes}\sphinxattablestart
\centering
\begin{tabular}[t]{|\X{25}{100}|\X{25}{100}|\X{15}{100}|\X{35}{100}|}
\hline
\sphinxstyletheadfamily 
\sphinxAtStartPar
Command line
&\sphinxstyletheadfamily 
\sphinxAtStartPar
Config file
&\sphinxstyletheadfamily 
\sphinxAtStartPar
Type
&\sphinxstyletheadfamily 
\sphinxAtStartPar
Description
\\
\hline
\sphinxAtStartPar
vaultServerName
&
\sphinxAtStartPar
security.vault.serverName
&
\sphinxAtStartPar
string
&
\sphinxAtStartPar
The IP address of the Vault server
\\
\hline
\sphinxAtStartPar
vaultPort
&
\sphinxAtStartPar
security.vault.port
&
\sphinxAtStartPar
int
&
\sphinxAtStartPar
The port on the Vault server
\\
\hline
\sphinxAtStartPar
vaultTokenFile
&
\sphinxAtStartPar
security.vault.tokenFile
&
\sphinxAtStartPar
string
&
\sphinxAtStartPar
The path to the vault token file. The token file is used by \sphinxstyleemphasis{MongoDB} to access HashiCorp Vault. The vault token file consists of the raw vault token and does not include any additional strings or parameters.

\sphinxAtStartPar
Example of a vault token file:

\begin{sphinxVerbatimintable}[commandchars=\\\{\}]
s.uTrHtzsZnEE7KyHeA797CkWA
\end{sphinxVerbatimintable}
\\
\hline
\sphinxAtStartPar
vaultSecret
&
\sphinxAtStartPar
security.vault.secret
&
\sphinxAtStartPar
string
&
\sphinxAtStartPar
The path to the vault secret. Every replica set member must have its own distinct vault secret. It is recommended to use different secret paths for every database node.

\sphinxAtStartPar
Note that vault secrets path format must be:

\begin{sphinxVerbatimintable}[commandchars=\\\{\}]
\PYGZlt{}vault\PYGZus{}secret\PYGZus{}mount\PYGZgt{}/data/\PYGZlt{}custom\PYGZus{}path\PYGZgt{}
\end{sphinxVerbatimintable}

\sphinxAtStartPar
where:
\begin{itemize}
\item {} 
\sphinxAtStartPar
\sphinxcode{\sphinxupquote{\textless{}vault\_secret\_mount\textgreater{}}} is your Vault KV Secrets Engine;

\item {} 
\sphinxAtStartPar
\sphinxcode{\sphinxupquote{data}} is the mandatory path prefix required by Version 2 API;

\item {} 
\sphinxAtStartPar
\sphinxcode{\sphinxupquote{\textless{}custom\_path\textgreater{}}} is your secrets path

\end{itemize}

\sphinxAtStartPar
Example:

\begin{sphinxVerbatimintable}[commandchars=\\\{\}]
secret\PYGZus{}v2/data/psmdb\PYGZhy{}test/rs1\PYGZhy{}27017
\end{sphinxVerbatimintable}
\\
\hline
\sphinxAtStartPar
vaultRotateMasterKey
&
\sphinxAtStartPar
security.vault.rotateMasterKey
&
\sphinxAtStartPar
switch
&
\sphinxAtStartPar
Enables master key rotation
\\
\hline
\sphinxAtStartPar
vaultServerCAFile
&
\sphinxAtStartPar
security.vault.serverCAFile
&
\sphinxAtStartPar
string
&
\sphinxAtStartPar
The path to the TLS certificate file
\\
\hline
\sphinxAtStartPar
vaultDisableTLSForTesting
&
\sphinxAtStartPar
security.vault.disableTLSForTesting
&
\sphinxAtStartPar
switch
&
\sphinxAtStartPar
Disables secure connection to HashiCorp Vault using SSL/TLS client certificates
\\
\hline
\end{tabular}
\par
\sphinxattableend\end{savenotes}
\end{sphinxadmonition}

\begin{sphinxadmonition}{note}{Config file example}

\begin{sphinxVerbatim}[commandchars=\\\{\}]
\PYG{n+nt}{security}\PYG{p}{:}
\PYG{+w}{  }\PYG{n+nt}{enableEncryption}\PYG{p}{:}\PYG{+w}{ }\PYG{l+lScalar+lScalarPlain}{true}
\PYG{+w}{  }\PYG{n+nt}{vault}\PYG{p}{:}
\PYG{+w}{    }\PYG{n+nt}{serverName}\PYG{p}{:}\PYG{+w}{ }\PYG{l+lScalar+lScalarPlain}{127.0.0.1}
\PYG{+w}{    }\PYG{n+nt}{port}\PYG{p}{:}\PYG{+w}{ }\PYG{l+lScalar+lScalarPlain}{8200}
\PYG{+w}{    }\PYG{n+nt}{tokenFile}\PYG{p}{:}\PYG{+w}{ }\PYG{l+lScalar+lScalarPlain}{/home/user/path/token}
\PYG{+w}{    }\PYG{n+nt}{secret}\PYG{p}{:}\PYG{+w}{ }\PYG{l+lScalar+lScalarPlain}{secret/data/hello}
\end{sphinxVerbatim}
\end{sphinxadmonition}

\sphinxAtStartPar
During the first run of the \sphinxstyleemphasis{Percona Server for MongoDB}, the process generates a secure key and writes the key to the vault.

\sphinxAtStartPar
During the subsequent start, the server tries to read the master key from the vault. If the configured secret does not exist, vault responds with HTTP 404 error.


\subsection{Namespaces}
\label{\detokenize{vault:namespaces}}\label{\detokenize{vault:vault-namespaces}}
\sphinxAtStartPar
Namespaces are isolated environments in Vault that allow for separate secret key and policy management.

\sphinxAtStartPar
You can use Vault namespaces with \sphinxstyleemphasis{Percona Server for MongoDB}. Specify the namespace(s) for the \sphinxcode{\sphinxupquote{security.vault.secret}} option value as follows:

\begin{sphinxVerbatim}[commandchars=\\\{\}]
\PYGZlt{}namespace\PYGZgt{}/secret/data/\PYGZlt{}secret\PYGZus{}path\PYGZgt{}
\end{sphinxVerbatim}

\sphinxAtStartPar
For example, the path to secret keys for namespace \sphinxcode{\sphinxupquote{test}} on  the secrets engine \sphinxcode{\sphinxupquote{secret}} will be \sphinxcode{\sphinxupquote{test/secret/\textless{}my\_secret\_path\textgreater{}}}.

\begin{sphinxadmonition}{note}{Note:}
\sphinxAtStartPar
You have the following options of how to target a particular namespace when configuring Vault:
\begin{enumerate}
\sphinxsetlistlabels{\arabic}{enumi}{enumii}{}{.}%
\item {} 
\sphinxAtStartPar
Set the VAULT\_NAMESPACE environment variable so that all subsequent commands are executed against that namespace. Use the following command to set the environment variable for the namespace \sphinxcode{\sphinxupquote{test}}:

\end{enumerate}

\begin{sphinxVerbatim}[commandchars=\\\{\}]
\PYGZdl{} \PYG{n+nb}{export} \PYG{n+nv}{VAULT\PYGZus{}NAMESPACE}\PYG{o}{=}\PYG{n+nb}{test}
\end{sphinxVerbatim}
\begin{enumerate}
\sphinxsetlistlabels{\arabic}{enumi}{enumii}{}{.}%
\setcounter{enumi}{1}
\item {} 
\sphinxAtStartPar
Provide the namespace with the \sphinxcode{\sphinxupquote{\sphinxhyphen{}namespace}} flag in commands

\end{enumerate}
\end{sphinxadmonition}


\sphinxstrong{See also:}
\nopagebreak


\sphinxAtStartPar
HashiCorp Vault Documentation:
\begin{itemize}
\item {} \begin{description}
\item[{Namespaces}] \leavevmode
\sphinxAtStartPar
\sphinxurl{https://www.vaultproject.io/docs/enterprise/namespaces}

\end{description}

\item {} \begin{description}
\item[{Secure Multi\sphinxhyphen{}Tenancy with Namespaces}] \leavevmode
\sphinxAtStartPar
\sphinxurl{https://learn.hashicorp.com/tutorials/vault/namespaces}

\end{description}

\end{itemize}




\subsection{Key Rotation}
\label{\detokenize{vault:key-rotation}}
\sphinxAtStartPar
Key rotation is replacing the old master key with a new one. This process helps to comply with regulatory requirements.

\sphinxAtStartPar
To rotate the keys for a single \sphinxcode{\sphinxupquote{mongod}} instance, do the following:
\begin{enumerate}
\sphinxsetlistlabels{\arabic}{enumi}{enumii}{}{.}%
\item {} 
\sphinxAtStartPar
Stop the \sphinxcode{\sphinxupquote{mongod}} process

\item {} 
\sphinxAtStartPar
Add \sphinxcode{\sphinxupquote{\sphinxhyphen{}\sphinxhyphen{}vaultRotateMasterKey}} option via the command line or \sphinxcode{\sphinxupquote{security.vault.rotateMasterKey}} to the config file.

\item {} 
\sphinxAtStartPar
Run the \sphinxcode{\sphinxupquote{mongod}} process with the selected option, the process will perform the key rotation and exit.

\item {} 
\sphinxAtStartPar
Remove the selected option from the startup command or the config file.

\item {} 
\sphinxAtStartPar
Start \sphinxcode{\sphinxupquote{mongod}} again.

\end{enumerate}

\sphinxAtStartPar
Rotating the master key process also re\sphinxhyphen{}encrypts the keystore using the new master key. The new master key is stored in the vault. The entire dataset is not re\sphinxhyphen{}encrypted.


\subsubsection{Key rotation in replica sets}
\label{\detokenize{vault:key-rotation-in-replica-sets}}
\sphinxAtStartPar
Every \sphinxcode{\sphinxupquote{mongod}} node in a replica set must have its own master key. The key rotation steps are the following:
\begin{enumerate}
\sphinxsetlistlabels{\arabic}{enumi}{enumii}{}{.}%
\item {} 
\sphinxAtStartPar
Rotate the master key for the secondary nodes one by one.

\item {} 
\sphinxAtStartPar
Step down the primary and wait for another primary to be elected.

\item {} 
\sphinxAtStartPar
Rotate the master key for the previous primary node.

\end{enumerate}


\section{Local key management using a keyfile}
\label{\detokenize{keyfile:local-key-management-using-a-keyfile}}\label{\detokenize{keyfile:keyfile}}\label{\detokenize{keyfile::doc}}
\sphinxAtStartPar
The key file must contain a 32 character string encoded in base64. You can generate a random
key and save it to a file by using the \sphinxstyleliteralstrong{\sphinxupquote{openssl}} command:

\begin{sphinxVerbatim}[commandchars=\\\{\}]
\PYGZdl{} openssl rand \PYGZhy{}base64 \PYG{l+m}{32} \PYGZgt{} mongodb\PYGZhy{}keyfile
\end{sphinxVerbatim}

\sphinxAtStartPar
Then, as the owner of the \sphinxcode{\sphinxupquote{mongod}} process, update the file permissions: only
the owner should be able to read and modify this file. The effective permissions
specified with the \sphinxcode{\sphinxupquote{chmod}} command can be:
\begin{itemize}
\item {} 
\sphinxAtStartPar
\sphinxstylestrong{600} \sphinxhyphen{} only the owner may read and modify the file

\item {} 
\sphinxAtStartPar
\sphinxstylestrong{400} \sphinxhyphen{} only the owner may read the file.

\end{itemize}

\begin{sphinxVerbatim}[commandchars=\\\{\}]
\PYGZdl{} chmod \PYG{l+m}{600} mongodb\PYGZhy{}keyfile
\end{sphinxVerbatim}

\sphinxAtStartPar
Enable the data encryption at rest in \sphinxstyleemphasis{Percona Server for MongoDB} by setting these options:
\begin{itemize}
\item {} 
\sphinxAtStartPar
\sphinxcode{\sphinxupquote{\sphinxhyphen{}\sphinxhyphen{}enableEncryption}} to enable data at rest encryption

\item {} 
\sphinxAtStartPar
\sphinxcode{\sphinxupquote{\sphinxhyphen{}\sphinxhyphen{}encryptionKeyFile}} to specify the path to a file that contains the encryption key

\end{itemize}

\begin{sphinxVerbatim}[commandchars=\\\{\}]
\PYGZdl{} mongod ... \PYGZhy{}\PYGZhy{}enableEncryption \PYGZhy{}\PYGZhy{}encryptionKeyFile \PYGZlt{}fileName\PYGZgt{}
\end{sphinxVerbatim}

\sphinxAtStartPar
By default, \sphinxstyleemphasis{Percona Server for MongoDB} uses the \sphinxcode{\sphinxupquote{AES256\sphinxhyphen{}CBC}} cipher mode. If you want to use the \sphinxcode{\sphinxupquote{AES256\sphinxhyphen{}GCM}} cipher mode, then use the \sphinxcode{\sphinxupquote{encryptionCipherMode}} parameter to change it.

\sphinxAtStartPar
If \sphinxcode{\sphinxupquote{mongod}} is started with the \sphinxcode{\sphinxupquote{\sphinxhyphen{}\sphinxhyphen{}relaxPermChecks}} option and the key file
is owned by \sphinxcode{\sphinxupquote{root}}, then \sphinxcode{\sphinxupquote{mongod}} can read the file based on the
group bit set accordingly. The effective key file permissions in this
case are:
\begin{itemize}
\item {} 
\sphinxAtStartPar
\sphinxstylestrong{440} \sphinxhyphen{} both the owner and the group can only read the file, or

\item {} 
\sphinxAtStartPar
\sphinxstylestrong{640} \sphinxhyphen{} only the owner can read and the change the file, the group can only read the file.

\end{itemize}


\sphinxstrong{See also:}
\nopagebreak

\begin{description}
\item[{MongoDB Documentation: Configure Encryption}] \leavevmode
\sphinxAtStartPar
\sphinxurl{https://docs.mongodb.com/manual/tutorial/configure-encryption/\#local-key-management}

\item[{\sphinxstyleemphasis{Percona} Blog: WiredTiger Encryption at Rest with Percona Server for MongoDB}] \leavevmode
\sphinxAtStartPar
\sphinxurl{https://www.percona.com/blog/2018/11/01/wiredtiger-encryption-at-rest-percona-server-for-mongodb/}

\end{description}



\sphinxAtStartPar
All these options can be specified in the configuration file:

\begin{sphinxVerbatim}[commandchars=\\\{\}]
\PYG{n+nt}{security}\PYG{p}{:}
\PYG{+w}{   }\PYG{n+nt}{enableEncryption}\PYG{p}{:}\PYG{+w}{ }\PYG{l+lScalar+lScalarPlain}{\PYGZlt{}boolean\PYGZgt{}}
\PYG{+w}{   }\PYG{n+nt}{encryptionCipherMode}\PYG{p}{:}\PYG{+w}{ }\PYG{l+lScalar+lScalarPlain}{\PYGZlt{}string\PYGZgt{}}
\PYG{+w}{   }\PYG{n+nt}{encryptionKeyFile}\PYG{p}{:}\PYG{+w}{ }\PYG{l+lScalar+lScalarPlain}{\PYGZlt{}string\PYGZgt{}}
\PYG{+w}{   }\PYG{n+nt}{relaxPermChecks}\PYG{p}{:}\PYG{+w}{ }\PYG{l+lScalar+lScalarPlain}{\PYGZlt{}boolean\PYGZgt{}}
\end{sphinxVerbatim}


\sphinxstrong{See also:}
\nopagebreak

\begin{description}
\item[{MongoDB Documentation: How to set options in a configuration file}] \leavevmode
\sphinxAtStartPar
\sphinxurl{https://docs.mongodb.com/manual/reference/configuration-options/index.html\#configuration-file}

\end{description}




\section{Migrating from Key File Encryption to HashiCorp Vault Encryption}
\label{\detokenize{encryption-mode-switch:migrating-from-key-file-encryption-to-vault-encryption}}\label{\detokenize{encryption-mode-switch:psmdb-encryption-mode-switch}}\label{\detokenize{encryption-mode-switch::doc}}
\sphinxAtStartPar
The steps below describe how to migrate from the key file encryption to using  HashiCorp Vault.

\begin{sphinxadmonition}{note}{Note:}
\sphinxAtStartPar
This is a simple guideline and it should be used for testing purposes only. We recommend to use Percona Consulting Services to assist you with migration in production environment.
\end{sphinxadmonition}
\subsubsection*{Assumptions}

\sphinxAtStartPar
We assume that you have installed and configured the vault server and enabled the KV Secrets Engine as the secrets storage for it.
\begin{enumerate}
\sphinxsetlistlabels{\arabic}{enumi}{enumii}{}{.}%
\item {} 
\sphinxAtStartPar
Stop \sphinxcode{\sphinxupquote{mongod}}.

\begin{sphinxVerbatim}[commandchars=\\\{\}]
\PYGZdl{} sudo systemctl stop mongod
\end{sphinxVerbatim}

\item {} 
\sphinxAtStartPar
Insert the key from keyfile into the HashiCorp Vault server to the desired secret
path.

\begin{sphinxVerbatim}[commandchars=\\\{\}]
\PYG{c+c1}{\PYGZsh{} Retrieve the key value from the keyfile}
\PYGZdl{} sudo cat /data/key/mongodb.key
d0JTFcePmvROyLXwCbAH8fmiP/ZRm0nYbeJDMGaI7Zw\PYG{o}{=}
\PYG{c+c1}{\PYGZsh{} Insert the key into vault}
\PYGZdl{} vault kv put secret/dc/psmongodb1 \PYG{n+nv}{value}\PYG{o}{=}d0JTFcePmvROyLXwCbAH8fmiP/ZRm0nYbeJDMGaI7Zw\PYG{o}{=}
\end{sphinxVerbatim}

\begin{sphinxadmonition}{note}{Note:}
\sphinxAtStartPar
Vault KV Secrets Engine uses different read and write secrets paths. To insert data to vault, specify the secret path without the \sphinxcode{\sphinxupquote{data/}} prefix.
\end{sphinxadmonition}

\item {} 
\sphinxAtStartPar
Edit the configuration file to provision the HashiCorp Vault configuration options instead of the key file encryption options.

\begin{sphinxVerbatim}[commandchars=\\\{\}]
\PYG{n+nt}{security}\PYG{p}{:}
\PYG{+w}{   }\PYG{n+nt}{enableEncryption}\PYG{p}{:}\PYG{+w}{ }\PYG{l+lScalar+lScalarPlain}{true}
\PYG{+w}{   }\PYG{n+nt}{vault}\PYG{p}{:}
\PYG{+w}{      }\PYG{n+nt}{serverName}\PYG{p}{:}\PYG{+w}{ }\PYG{l+lScalar+lScalarPlain}{10.0.2.15}
\PYG{+w}{      }\PYG{n+nt}{port}\PYG{p}{:}\PYG{+w}{ }\PYG{l+lScalar+lScalarPlain}{8200}
\PYG{+w}{      }\PYG{n+nt}{secret}\PYG{p}{:}\PYG{+w}{ }\PYG{l+lScalar+lScalarPlain}{secret/data/dc/psmongodb1}
\PYG{+w}{      }\PYG{n+nt}{tokenFile}\PYG{p}{:}\PYG{+w}{ }\PYG{l+lScalar+lScalarPlain}{/etc/mongodb/token}
\PYG{+w}{      }\PYG{n+nt}{serverCAFile}\PYG{p}{:}\PYG{+w}{ }\PYG{l+lScalar+lScalarPlain}{/etc/mongodb/vault.crt}
\end{sphinxVerbatim}

\item {} 
\sphinxAtStartPar
Start the \sphinxcode{\sphinxupquote{mongod}} service

\begin{sphinxVerbatim}[commandchars=\\\{\}]
\PYGZdl{} sudo systemctl start mongod
\end{sphinxVerbatim}

\end{enumerate}


\section{Using the Key Management Interoperability Protocol (KMIP)}
\label{\detokenize{kmip:using-the-key-management-interoperability-protocol-kmip}}\label{\detokenize{kmip:kmip}}\label{\detokenize{kmip::doc}}
\sphinxAtStartPar
\sphinxstyleemphasis{Percona Server for MongoDB} adds support for secure transfer of keys using the \sphinxhref{https://docs.oasis-open.org/kmip/kmip-spec/v2.0/os/kmip-spec-v2.0-os.html}{OASIS Key Management Interoperability Protocol (KMIP)}. The KMIP implementation was tested with the \sphinxhref{https://pykmip.readthedocs.io/en/latest/server.html}{PyKMIP server} and the \sphinxhref{https://www.vaultproject.io/docs/secrets/kmip}{HashiCorp Vault Enterprise KMIP Secrets Engine}.

\sphinxAtStartPar
KMIP enables the communication between key management systems and the database server. KMIP provides the following benefits:
\begin{itemize}
\item {} 
\sphinxAtStartPar
Streamlines encryption key management

\item {} 
\sphinxAtStartPar
Eliminates redundant key management processes

\end{itemize}

\sphinxAtStartPar
Starting with version 4.4.15\sphinxhyphen{}15, you can specify multiple KMIP servers for failover. On startup, \sphinxstyleemphasis{Percona Server for MongoDB} connects to the servers in the order listed and selects the one with which the connection is successful.

\sphinxAtStartPar
Starting with version 4.4.16\sphinxhyphen{}16, the \sphinxcode{\sphinxupquote{kmipKeyIdentifier}} option is no longer mandatory. When left blank, the database server creates a key on the KMIP server and uses that for encryption. When you specify the identifier, the key with such an ID must exist on the keystore.

\sphinxAtStartPar
\sphinxstyleemphasis{Percona Server for MongoDB} cannot encrypt existing data. If there is data in place, see the steps how to \sphinxhref{https://www.mongodb.com/docs/v4.4/tutorial/configure-encryption/\#std-label-encrypt-existing-data}{encrypt existing data}.
\subsubsection*{KMIP parameters}


\begin{savenotes}\sphinxattablestart
\centering
\begin{tabulary}{\linewidth}[t]{|T|T|T|}
\hline
\sphinxstyletheadfamily 
\sphinxAtStartPar
Option
&\sphinxstyletheadfamily 
\sphinxAtStartPar
Type
&\sphinxstyletheadfamily 
\sphinxAtStartPar
Description
\\
\hline
\sphinxAtStartPar
\sphinxcode{\sphinxupquote{\sphinxhyphen{}\sphinxhyphen{}kmipServerName}}
&
\sphinxAtStartPar
string
&
\sphinxAtStartPar
The hostname or IP address of the KMIP server. As of version 4.4.15\sphinxhyphen{}15, multiple KMIP servers are supported as the comma\sphinxhyphen{}separated list, e.g. \sphinxcode{\sphinxupquote{kmip1.example.com,kmip2.example.com}}
\\
\hline
\sphinxAtStartPar
\sphinxcode{\sphinxupquote{\sphinxhyphen{}\sphinxhyphen{}kmipPort}}
&
\sphinxAtStartPar
number
&
\sphinxAtStartPar
The port used to communicate with the KMIP server. When undefined, the default port 5696 is used.
\\
\hline
\sphinxAtStartPar
\sphinxcode{\sphinxupquote{\sphinxhyphen{}\sphinxhyphen{}kmipServerCAFile}}
&
\sphinxAtStartPar
string
&
\sphinxAtStartPar
The path to the CA certificate file. CA file is used to validate secure client connection to the KMIP server.
\\
\hline
\sphinxAtStartPar
\sphinxcode{\sphinxupquote{\sphinxhyphen{}\sphinxhyphen{}kmipClientCertificateFile}}
&
\sphinxAtStartPar
string
&
\sphinxAtStartPar
The path to the PEM file with the KMIP client private key and the certificate chain. The database server uses this PEM file to authenticate the KMIP server.
\\
\hline
\sphinxAtStartPar
\sphinxcode{\sphinxupquote{\sphinxhyphen{}\sphinxhyphen{}kmipKeyIdentifier}}
&
\sphinxAtStartPar
string
&
\sphinxAtStartPar
Optional starting with version 4.4.16\sphinxhyphen{}16. The identifier of the KMIP key. If the key does not exist, the database server creates a key on the KMIP server with the specified identifier. When you specify the identifier, the key with such an ID must exist on the key storage. You can only use this setting for the first time you enable encryption.
\\
\hline
\sphinxAtStartPar
\sphinxcode{\sphinxupquote{kmipRotateMasterKey}}
&
\sphinxAtStartPar
boolean
&
\sphinxAtStartPar
Controls master keys rotation. When enabled, generates the new master key version and re\sphinxhyphen{}encrypts the keystore. Available as of version 4.4.14\sphinxhyphen{}14. Requires the unique \sphinxcode{\sphinxupquote{\sphinxhyphen{}\sphinxhyphen{}kmipKeyIdentifier}} for every \sphinxcode{\sphinxupquote{mongod}} node.
\\
\hline
\sphinxAtStartPar
\sphinxcode{\sphinxupquote{\sphinxhyphen{}\sphinxhyphen{}kmipClientCertificatePassword}}
&
\sphinxAtStartPar
string
&
\sphinxAtStartPar
The password for the KMIP client private key or certificate. Use this parameter only if the KMIP client private key or certificate is encrypted. Available starting with version 4.4.15\sphinxhyphen{}15.
\\
\hline
\end{tabulary}
\par
\sphinxattableend\end{savenotes}


\section{Key rotation}
\label{\detokenize{kmip:key-rotation}}
\sphinxAtStartPar
Starting with release 4.4.14\sphinxhyphen{}14, the support for \sphinxhref{https://www.mongodb.com/docs/manual/tutorial/rotate-encryption-key/\#kmip-master-key-rotation}{master key rotation} is added. This enables users to comply with data security regulations when using KMIP.


\section{Configuration}
\label{\detokenize{kmip:configuration}}\subsubsection*{Considerations}

\sphinxAtStartPar
Make sure you have obtained the root certificate, and the keypair for the KMIP server and the \sphinxcode{\sphinxupquote{mongod}} client. For testing purposes you can use the \sphinxhref{https://www.openssl.org/}{OpenSSL} to issue self\sphinxhyphen{}signed certificates. For production use we recommend you use the valid certificates issued by the key management appliance.

\sphinxAtStartPar
To enable data\sphinxhyphen{}at\sphinxhyphen{}rest encryption in \sphinxstyleemphasis{Percona Server for MongoDB} using KMIP, edit the \sphinxcode{\sphinxupquote{/etc/mongod.conf}} configuration file as follows:

\begin{sphinxVerbatim}[commandchars=\\\{\}]
\PYG{n+nt}{security}\PYG{p}{:}
\PYG{+w}{  }\PYG{n+nt}{enableEncryption}\PYG{p}{:}\PYG{+w}{ }\PYG{l+lScalar+lScalarPlain}{true}
\PYG{+w}{  }\PYG{n+nt}{kmip}\PYG{p}{:}
\PYG{+w}{    }\PYG{n+nt}{serverName}\PYG{p}{:}\PYG{+w}{ }\PYG{l+lScalar+lScalarPlain}{\PYGZlt{}kmip\PYGZus{}server\PYGZus{}name\PYGZgt{}}
\PYG{+w}{    }\PYG{n+nt}{port}\PYG{p}{:}\PYG{+w}{ }\PYG{l+lScalar+lScalarPlain}{5696}
\PYG{+w}{    }\PYG{n+nt}{clientCertificateFile}\PYG{p}{:}\PYG{+w}{ }\PYG{l+lScalar+lScalarPlain}{\PYGZlt{}/path/client\PYGZus{}certificate.pem\PYGZgt{}}
\PYG{+w}{    }\PYG{n+nt}{serverCAFile}\PYG{p}{:}\PYG{+w}{ }\PYG{l+lScalar+lScalarPlain}{\PYGZlt{}/path/ca.pem\PYGZgt{}}
\PYG{+w}{    }\PYG{n+nt}{keyIdentifier}\PYG{p}{:}\PYG{+w}{ }\PYG{l+lScalar+lScalarPlain}{\PYGZlt{}key\PYGZus{}name\PYGZgt{}}
\end{sphinxVerbatim}

\sphinxAtStartPar
Alternatively, you can start \sphinxstyleemphasis{Percona Server for MongoDB} using the command line as follows:

\begin{sphinxVerbatim}[commandchars=\\\{\}]
\PYGZdl{} mongod \PYGZhy{}\PYGZhy{}enableEncryption \PYG{l+s+se}{\PYGZbs{}}
  \PYGZhy{}\PYGZhy{}kmipServerName \PYGZlt{}kmip\PYGZus{}servername\PYGZgt{} \PYG{l+s+se}{\PYGZbs{}}
  \PYGZhy{}\PYGZhy{}kmipPort \PYG{l+m}{5696} \PYG{l+s+se}{\PYGZbs{}}
  \PYGZhy{}\PYGZhy{}kmipServerCAFile \PYGZlt{}path\PYGZus{}to\PYGZus{}ca\PYGZus{}file\PYGZgt{} \PYG{l+s+se}{\PYGZbs{}}
  \PYGZhy{}\PYGZhy{}kmipClientCertificateFile \PYGZlt{}path\PYGZus{}to\PYGZus{}client\PYGZus{}certificate\PYGZgt{} \PYG{l+s+se}{\PYGZbs{}}
  \PYGZhy{}\PYGZhy{}kmipKeyIdentifier \PYGZlt{}kmip\PYGZus{}identifier\PYGZgt{}
\end{sphinxVerbatim}
\subsubsection*{Important Configuration Options}

\sphinxAtStartPar
\sphinxstyleemphasis{Percona Server for MongoDB} supports the \sphinxcode{\sphinxupquote{encryptionCipherMode}} option where you choose one of the following cipher modes:
\begin{itemize}
\item {} 
\sphinxAtStartPar
AES256\sphinxhyphen{}CBC

\item {} 
\sphinxAtStartPar
AES256\sphinxhyphen{}GCM

\end{itemize}

\sphinxAtStartPar
By default, the AES256\sphinxhyphen{}CBC cipher mode is applied. The following example
demonstrates how to apply the AES256\sphinxhyphen{}GCM cipher mode when starting the
\sphinxstyleliteralstrong{\sphinxupquote{mongod}} service:

\begin{sphinxVerbatim}[commandchars=\\\{\}]
\PYGZdl{} mongod ... \PYGZhy{}\PYGZhy{}encryptionCipherMode AES256\PYGZhy{}GCM
\end{sphinxVerbatim}


\sphinxstrong{See also:}
\nopagebreak

\begin{description}
\item[{\sphinxstyleemphasis{MongoDB} Documentation: encryptionCipherMode Option}] \leavevmode
\sphinxAtStartPar
\sphinxurl{https://docs.mongodb.com/manual/reference/program/mongod/\#cmdoption-mongod-encryptionciphermode}

\end{description}




\section{Encrypting Rollback Files}
\label{\detokenize{data-at-rest-encryption:encrypting-rollback-files}}
\sphinxAtStartPar
Starting from version 3.6, \sphinxstyleemphasis{Percona Server for MongoDB} also encrypts rollback files when data at
rest encryption is enabled. To inspect the contents of these files, use
\sphinxstyleliteralstrong{\sphinxupquote{perconadecrypt}}. This is a tool that you run from the command line as follows:

\begin{sphinxVerbatim}[commandchars=\\\{\}]
\PYGZdl{} perconadecrypt \PYGZhy{}\PYGZhy{}encryptionKeyFile FILE  \PYGZhy{}\PYGZhy{}inputPath FILE \PYGZhy{}\PYGZhy{}outputPath FILE \PYG{o}{[}\PYGZhy{}\PYGZhy{}encryptionCipherMode MODE\PYG{o}{]}
\end{sphinxVerbatim}

\sphinxAtStartPar
When decrypting, the cipher mode must match the cipher mode which was used for
the encryption. By default, the \sphinxcode{\sphinxupquote{\sphinxhyphen{}\sphinxhyphen{}encryptionCipherMode}} option uses the
AES256\sphinxhyphen{}CBC mode.

\begin{sphinxadmonition}{note}{Parameters of \sphinxstyleliteralstrong{\sphinxupquote{perconadecrypt}}}


\begin{savenotes}\sphinxattablestart
\centering
\begin{tabulary}{\linewidth}[t]{|T|T|}
\hline
\sphinxstyletheadfamily 
\sphinxAtStartPar
Option
&\sphinxstyletheadfamily 
\sphinxAtStartPar
Purpose
\\
\hline
\sphinxAtStartPar
\textendash{}encryptionKeyFile
&
\sphinxAtStartPar
The path to the encryption key file
\\
\hline
\sphinxAtStartPar
\textendash{}encryptionCipherMode
&
\sphinxAtStartPar
The cipher mode for decryption. The supported values are AES256\sphinxhyphen{}CBC or AES256\sphinxhyphen{}GCM
\\
\hline
\sphinxAtStartPar
\textendash{}inputPath
&
\sphinxAtStartPar
The path to the encrypted rollback file
\\
\hline
\sphinxAtStartPar
\textendash{}outputPath
&
\sphinxAtStartPar
The path to save the decrypted rollback file
\\
\hline
\end{tabulary}
\par
\sphinxattableend\end{savenotes}
\end{sphinxadmonition}


\chapter{Additional text search algorithm \sphinxhyphen{} \sphinxstyleemphasis{ngram}}
\label{\detokenize{ngram-full-text-search:additional-text-search-algorithm-ngram}}\label{\detokenize{ngram-full-text-search::doc}}
\sphinxAtStartPar
The \sphinxhref{https://en.wikipedia.org/wiki/N-gram}{ngram} text search algorithm is useful for searching text for a specific string
of characters in a field of a collection. This feature can be used to find exact sub\sphinxhyphen{}string matches, which provides an alternative to parsing text from languages other than the list of European languages already supported by MongoDB Community’s full text search engine. It
may also turn out to be more convenient when working with the text where symbols
like dash(‘\sphinxhyphen{}‘), underscore(‘\_’), or slash(“/”) are not token delimiters.

\sphinxAtStartPar
Unlike MongoDB full text search engine, \sphinxstyleemphasis{ngram} search algorithm uses only the following token delimiter
characters that do not count as word characters in human languages:
\begin{itemize}
\item {} 
\sphinxAtStartPar
Horizontal tab

\item {} 
\sphinxAtStartPar
Vertical tab

\item {} 
\sphinxAtStartPar
Line feed

\item {} 
\sphinxAtStartPar
Carriage return

\item {} 
\sphinxAtStartPar
Space

\end{itemize}

\sphinxAtStartPar
The \sphinxstyleemphasis{ngram} text search is slower than MongoDB full text search.


\section{Usage}
\label{\detokenize{ngram-full-text-search:usage}}
\sphinxAtStartPar
To use \sphinxstyleemphasis{ngram}, create a text index on
a collection setting the \sphinxcode{\sphinxupquote{default\_language}} parameter to \sphinxstylestrong{ngram}:

\begin{sphinxVerbatim}[commandchars=\\\{\}]
mongo \PYGZgt{} db.collection.createIndex\PYG{o}{(}\PYG{o}{\PYGZob{}}name:\PYG{l+s+s2}{\PYGZdq{}text\PYGZdq{}}\PYG{o}{\PYGZcb{}}, \PYG{o}{\PYGZob{}}default\PYGZus{}language: \PYG{l+s+s2}{\PYGZdq{}ngram\PYGZdq{}}\PYG{o}{\PYGZcb{}}\PYG{o}{)}
\end{sphinxVerbatim}

\sphinxAtStartPar
\sphinxstyleemphasis{ngram} search algorithm treats special characters like individual terms. Therefore, you don’t have to enclose the search string in escaped double quotes (\sphinxcode{\sphinxupquote{\textbackslash{}"}}) to query the text index. For example, to search for documents that contain the date \sphinxcode{\sphinxupquote{2021\sphinxhyphen{}02\sphinxhyphen{}12}}, specify the following:

\begin{sphinxVerbatim}[commandchars=\\\{\}]
mongo \PYGZgt{} db.collection.find\PYG{o}{(}\PYG{o}{\PYGZob{}} \PYG{n+nv}{\PYGZdl{}text}: \PYG{o}{\PYGZob{}} \PYG{n+nv}{\PYGZdl{}search}: \PYG{l+s+s2}{\PYGZdq{}2021\PYGZhy{}02\PYGZhy{}12\PYGZdq{}} \PYG{o}{\PYGZcb{}} \PYG{o}{\PYGZcb{}}\PYG{o}{)}
\end{sphinxVerbatim}

\sphinxAtStartPar
However, both \sphinxstyleemphasis{ngram} and MongoDB full text search engine treat words with the hyphen\sphinxhyphen{}minus \sphinxcode{\sphinxupquote{\sphinxhyphen{}}} sign  in front of them as negated (e.g. “\sphinxhyphen{}coffee”)  and exclude such words from the search results.


\sphinxstrong{See also:}
\nopagebreak

\begin{description}
\item[{\sphinxstyleemphasis{MongoDB} documentation:}] \leavevmode\begin{itemize}
\item {} 
\sphinxAtStartPar
\sphinxhref{https://docs.mongodb.com/manual/text-search/}{Text search}

\item {} 
\sphinxAtStartPar
\sphinxhref{https://docs.mongodb.com/manual/core/index-text/\#index-feature-text}{Text indexes}

\item {} 
\sphinxAtStartPar
\sphinxhref{https://docs.mongodb.com/manual/reference/operator/query/text/\#op.\_S\_text}{\$text operator}

\end{itemize}

\item[{More information about ngram implementation:}] \leavevmode\begin{itemize}
\item {} 
\sphinxAtStartPar
\sphinxurl{https://github.com/percona/percona-server-mongodb/blob/v4.4/src/mongo/db/fts/ngram-tokenizer.md}

\end{itemize}

\end{description}




\chapter{\$backupCursor and \$backupCursorExtend aggregation stages}
\label{\detokenize{backup-cursor:backupcursor-and-backupcursorextend-aggregation-stages}}\label{\detokenize{backup-cursor:backup-cursor}}\label{\detokenize{backup-cursor::doc}}
\sphinxAtStartPar
\sphinxcode{\sphinxupquote{\$backupCursor}} and \sphinxcode{\sphinxupquote{\$backupCursorExtend}} aggregation stages expose the WiredTiger API which allows making consistent backups. Running these stages allows listing and freezing the files so you can copy them without the files being deleted or necessary parts within them being overwritten.
\begin{itemize}
\item {} 
\sphinxAtStartPar
\sphinxcode{\sphinxupquote{\$backupCursor}} outputs the list of files and their size to copy.

\item {} 
\sphinxAtStartPar
\sphinxcode{\sphinxupquote{\$backupCursorExtend}} outputs the list of WiredTiger transaction log files that have been updated or newly added since the \sphinxcode{\sphinxupquote{\$backupCursor}} was first run. Saving these files enables restoring the database to any arbitrary time between the \sphinxcode{\sphinxupquote{\$backupCursor}} and \sphinxcode{\sphinxupquote{\$backupCursorExtend}} execution times.

\end{itemize}

\sphinxAtStartPar
They are available in \sphinxstyleemphasis{Percona Server for MongoDB} starting with version 4.4.6\sphinxhyphen{}8.

\sphinxAtStartPar
Percona provides \sphinxhref{https://www.percona.com/doc/percona-backup-mongodb/index.html}{Percona Backup for MongoDB (PBM)} \textendash{} a light\sphinxhyphen{}weight open source solution for consistent backups and restores across sharded clusters. PBM relies on these aggregation stages for physical backups and restores. However, if you wish to develop your own backup application, this document describes the \sphinxcode{\sphinxupquote{\$backupCursor}} and \sphinxcode{\sphinxupquote{\$backupCursorExtend}} aggregation stages.


\section{Usage}
\label{\detokenize{backup-cursor:usage}}
\sphinxAtStartPar
You can run these stages in any type of MongoDB deployment. If you need to back up a single node in a replica set, first run the \sphinxcode{\sphinxupquote{\$backupCursor}}, then the \sphinxcode{\sphinxupquote{\$backupCursorExtend}} and save the output files to the backup storage.

\sphinxAtStartPar
To make a consistent backup of a sharded cluster, run both aggregation stages on one node from each shard and the config server replica set. It can be either the primary or the secondary node. Note that since the secondary node may lag in syncing the data from the primary one, you will have to wait for the exact same time before running the \sphinxcode{\sphinxupquote{\$backupCursorExtend}}.

\sphinxAtStartPar
Note that for standalone MongoDB node with disabled oplogs, you can only run the \sphinxcode{\sphinxupquote{\$backupCursor}} aggregation stage.


\subsection{Get a list of all files to copy with \$backupCursor}
\label{\detokenize{backup-cursor:get-a-list-of-all-files-to-copy-with-backupcursor}}
\begin{sphinxVerbatim}[commandchars=\\\{\}]
\PYG{k+kd}{var}\PYG{+w}{ }\PYG{n+nx}{bkCsr}\PYG{+w}{ }\PYG{o}{=}\PYG{+w}{ }\PYG{n+nx}{db}\PYG{p}{.}\PYG{n+nx}{getSiblingDB}\PYG{p}{(}\PYG{l+s+s2}{\PYGZdq{}admin\PYGZdq{}}\PYG{p}{)}\PYG{p}{.}\PYG{n+nx}{aggregate}\PYG{p}{(}\PYG{p}{[}\PYG{p}{\PYGZob{}}\PYG{n+nx}{\PYGZdl{}backupCursor}\PYG{o}{:}\PYG{+w}{ }\PYG{p}{\PYGZob{}}\PYG{p}{\PYGZcb{}}\PYG{p}{\PYGZcb{}}\PYG{p}{]}\PYG{p}{)}
\PYG{n+nx}{bkCsrMetadata}\PYG{+w}{ }\PYG{o}{=}\PYG{+w}{ }\PYG{n+nx}{bkCsr}\PYG{p}{.}\PYG{n+nx}{next}\PYG{p}{(}\PYG{p}{)}\PYG{p}{.}\PYG{n+nx}{metadata}
\end{sphinxVerbatim}

\sphinxAtStartPar
Sample output:

\begin{sphinxVerbatim}[commandchars=\\\{\}]
\PYG{p}{[}
  \PYG{p}{\PYGZob{}}
    \PYG{n}{metadata}\PYG{p}{:} \PYG{p}{\PYGZob{}}
      \PYG{n}{backupId}\PYG{p}{:} \PYG{n}{UUID}\PYG{p}{(}\PYG{l+s+s2}{\PYGZdq{}}\PYG{l+s+s2}{35c34101\PYGZhy{}0107\PYGZhy{}44cf\PYGZhy{}bdec\PYGZhy{}fad285e07534}\PYG{l+s+s2}{\PYGZdq{}}\PYG{p}{)}\PYG{p}{,}
      \PYG{n}{dbpath}\PYG{p}{:} \PYG{l+s+s1}{\PYGZsq{}}\PYG{l+s+s1}{/var/lib/mongodb}\PYG{l+s+s1}{\PYGZsq{}}\PYG{p}{,}
      \PYG{n}{oplogStart}\PYG{p}{:} \PYG{p}{\PYGZob{}} \PYG{n}{ts}\PYG{p}{:} \PYG{n}{Timestamp}\PYG{p}{(}\PYG{p}{\PYGZob{}} \PYG{n}{t}\PYG{p}{:} \PYG{l+m+mi}{1666631297}\PYG{p}{,} \PYG{n}{i}\PYG{p}{:} \PYG{l+m+mi}{1} \PYG{p}{\PYGZcb{}}\PYG{p}{)}\PYG{p}{,} \PYG{n}{t}\PYG{p}{:} \PYG{n}{Long}\PYG{p}{(}\PYG{l+s+s2}{\PYGZdq{}}\PYG{l+s+s2}{\PYGZhy{}1}\PYG{l+s+s2}{\PYGZdq{}}\PYG{p}{)} \PYG{p}{\PYGZcb{}}\PYG{p}{,}
      \PYG{n}{oplogEnd}\PYG{p}{:} \PYG{p}{\PYGZob{}} \PYG{n}{ts}\PYG{p}{:} \PYG{n}{Timestamp}\PYG{p}{(}\PYG{p}{\PYGZob{}} \PYG{n}{t}\PYG{p}{:} \PYG{l+m+mi}{1666631408}\PYG{p}{,} \PYG{n}{i}\PYG{p}{:} \PYG{l+m+mi}{1} \PYG{p}{\PYGZcb{}}\PYG{p}{)}\PYG{p}{,} \PYG{n}{t}\PYG{p}{:} \PYG{n}{Long}\PYG{p}{(}\PYG{l+s+s2}{\PYGZdq{}}\PYG{l+s+s2}{1}\PYG{l+s+s2}{\PYGZdq{}}\PYG{p}{)} \PYG{p}{\PYGZcb{}}\PYG{p}{,}
      \PYG{n}{checkpointTimestamp}\PYG{p}{:} \PYG{n}{Timestamp}\PYG{p}{(}\PYG{p}{\PYGZob{}} \PYG{n}{t}\PYG{p}{:} \PYG{l+m+mi}{1666631348}\PYG{p}{,} \PYG{n}{i}\PYG{p}{:} \PYG{l+m+mi}{1} \PYG{p}{\PYGZcb{}}\PYG{p}{)}
    \PYG{p}{\PYGZcb{}}
  \PYG{p}{\PYGZcb{}}\PYG{p}{,}
\end{sphinxVerbatim}

\sphinxAtStartPar
Store the \sphinxcode{\sphinxupquote{metadata}} document somewhere, because you need to pass the \sphinxcode{\sphinxupquote{backupId}} parameter from this document as the input parameter for the \sphinxcode{\sphinxupquote{\$backupCursorExtend}} stage. Also you need the \sphinxcode{\sphinxupquote{oplogEnd}} timestamp.
Make sure that the \sphinxcode{\sphinxupquote{\$backupCursor}} is complete on all shards in your cluster.

\begin{sphinxadmonition}{note}{Note:}
\sphinxAtStartPar
Note that when running \sphinxcode{\sphinxupquote{\$backupCursor}} in a standalone node deployment, \sphinxcode{\sphinxupquote{oplogStart}}, \sphinxcode{\sphinxupquote{oplogEnd}}, \sphinxcode{\sphinxupquote{checkpointTimesatmp}} values may be absent. This is because standalone node deployments don’t have oplogs.
\end{sphinxadmonition}


\subsection{Run \sphinxstyleliteralintitle{\sphinxupquote{\$backupCursorExtend}} to retrieve the WiredTiger transaction logs}
\label{\detokenize{backup-cursor:run-backupcursorextend-to-retrieve-the-wiredtiger-transaction-logs}}
\sphinxAtStartPar
Pass the \sphinxcode{\sphinxupquote{backupId}} from the metadata document as the first parameter. For the \sphinxcode{\sphinxupquote{timestamp}} parameter, use the maximum (latest) value among the \sphinxcode{\sphinxupquote{oplogEnd}} timestamps from all shards and config server replica set. This will be the target time to restore.

\begin{sphinxVerbatim}[commandchars=\\\{\}]
\PYG{k+kd}{var}\PYG{+w}{ }\PYG{n+nx}{bkExtCsr}\PYG{+w}{ }\PYG{o}{=}\PYG{+w}{ }\PYG{n+nx}{db}\PYG{p}{.}\PYG{n+nx}{aggregate}\PYG{p}{(}\PYG{p}{[}\PYG{p}{\PYGZob{}}\PYG{n+nx}{\PYGZdl{}backupCursorExtend}\PYG{o}{:}\PYG{+w}{ }\PYG{p}{\PYGZob{}}\PYG{n+nx}{backupId}\PYG{o}{:}\PYG{+w}{ }\PYG{n+nx}{bkCsrMetadata}\PYG{p}{.}\PYG{n+nx}{backupId}\PYG{p}{,}\PYG{+w}{ }\PYG{n+nx}{timestamp}\PYG{o}{:}\PYG{+w}{ }\PYG{o+ow}{new}\PYG{+w}{ }\PYG{n+nx}{Timestamp}\PYG{p}{(}\PYG{l+m+mf}{1666631418}\PYG{p}{,}\PYG{+w}{ }\PYG{l+m+mf}{1}\PYG{p}{)}\PYG{p}{\PYGZcb{}}\PYG{p}{\PYGZcb{}}\PYG{p}{]}\PYG{p}{)}
\end{sphinxVerbatim}

\sphinxAtStartPar
Sample output:

\begin{sphinxVerbatim}[commandchars=\\\{\}]
\PYG{p}{\PYGZob{}} \PYG{l+s+s2}{\PYGZdq{}}\PYG{l+s+s2}{filename}\PYG{l+s+s2}{\PYGZdq{}} \PYG{p}{:} \PYG{l+s+s2}{\PYGZdq{}}\PYG{l+s+s2}{/data/plain\PYGZus{}rs/n1/data/journal/WiredTigerLog.0000000042}\PYG{l+s+s2}{\PYGZdq{}} \PYG{p}{\PYGZcb{}}
\PYG{p}{\PYGZob{}} \PYG{l+s+s2}{\PYGZdq{}}\PYG{l+s+s2}{filename}\PYG{l+s+s2}{\PYGZdq{}} \PYG{p}{:} \PYG{l+s+s2}{\PYGZdq{}}\PYG{l+s+s2}{/data/plain\PYGZus{}rs/n1/data/journal/WiredTigerLog.0000000043}\PYG{l+s+s2}{\PYGZdq{}} \PYG{p}{\PYGZcb{}}
\PYG{p}{\PYGZob{}} \PYG{l+s+s2}{\PYGZdq{}}\PYG{l+s+s2}{filename}\PYG{l+s+s2}{\PYGZdq{}} \PYG{p}{:} \PYG{l+s+s2}{\PYGZdq{}}\PYG{l+s+s2}{/data/plain\PYGZus{}rs/n1/data/journal/WiredTigerLog.0000000044}\PYG{l+s+s2}{\PYGZdq{}} \PYG{p}{\PYGZcb{}}
\end{sphinxVerbatim}


\subsection{Loop the \sphinxstyleliteralintitle{\sphinxupquote{\$backupCursor}}}
\label{\detokenize{backup-cursor:loop-the-backupcursor}}
\sphinxAtStartPar
Prevent the backup cursor from closing on timeout (default \textendash{} 10 minutes). This is crucial since it prevents overwriting backup snapshot file blocks with new ones if the files take longer than 10 minutes to copy.  Use the \sphinxhref{https://www.mongodb.com/docs/v6.0/reference/command/getMore/\#getmore}{getMore} command for this purpose.


\subsection{Copy the files to the storage}
\label{\detokenize{backup-cursor:copy-the-files-to-the-storage}}
\sphinxAtStartPar
Now you can copy the output of both aggregation stages to your backup storage.

\sphinxAtStartPar
After the backup is copied to the storage, terminate the \sphinxhref{https://www.mongodb.com/docs/v6.0/reference/command/getMore/\#getmore}{getMore} command and close the cursor.

\begin{sphinxadmonition}{note}{Note:}
\sphinxAtStartPar
Save the timestamp that you passed for the \$backupCursorExtend stage somewhere since you will need it for the restore.
\end{sphinxadmonition}

\begin{sphinxadmonition}{note}{Based on the material from Percona blog}

\sphinxAtStartPar
Based on the blog post: \sphinxhref{https://www.percona.com/blog/2021/06/07/experimental-feature-backupcursorextend-in-percona-server-for-mongodb/}{Experimental Feature: \$backupCursorExtend in Percona Server for MongoDB} by Akira Kurogane
\end{sphinxadmonition}


\part{How to}
\label{\detokenize{index:how-to}}

\chapter{Enabling Authentication}
\label{\detokenize{enable-auth:enabling-authentication}}\label{\detokenize{enable-auth:enable-auth}}\label{\detokenize{enable-auth::doc}}
\sphinxAtStartPar
By default, \sphinxstyleemphasis{Percona Server for MongoDB} does not restrict access to data and configuration.

\sphinxAtStartPar
Enabling authentication enforces users to identify themselves when accessing the database. This documents describes how to enable built\sphinxhyphen{}in authentication mechanism. \sphinxstyleemphasis{Percona Server for MongoDB} also supports the number of external authentication mechanisms. To learn more, refer to {\hyperref[\detokenize{authentication:ext-auth}]{\sphinxcrossref{\DUrole{std,std-ref}{Authentication}}}}.

\sphinxAtStartPar
You can enable authentication either automatically or manually.


\section{Automatic setup}
\label{\detokenize{enable-auth:automatic-setup}}
\sphinxAtStartPar
To enable authentication and automatically set it up,
run the \sphinxcode{\sphinxupquote{/usr/bin/percona\sphinxhyphen{}server\sphinxhyphen{}mongodb\sphinxhyphen{}enable\sphinxhyphen{}auth.sh}} script
as root or using \sphinxcode{\sphinxupquote{sudo}}.

\sphinxAtStartPar
This script creates the \sphinxcode{\sphinxupquote{dba}} user with the \sphinxcode{\sphinxupquote{root}} role.
The password is randomly generated and printed out in the output.
Then the script restarts \sphinxstyleemphasis{Percona Server for MongoDB} with access control enabled.
The \sphinxcode{\sphinxupquote{dba}} user has full superuser privileges on the server.
You can add other users with various roles depending on your needs.

\sphinxAtStartPar
For usage information, run the script with the \sphinxcode{\sphinxupquote{\sphinxhyphen{}h}} option.


\section{Manual setup}
\label{\detokenize{enable-auth:manual-setup}}
\sphinxAtStartPar
To enable access control manually:
\begin{enumerate}
\sphinxsetlistlabels{\arabic}{enumi}{enumii}{}{.}%
\item {} 
\sphinxAtStartPar
Add the following lines to the configuration file:

\begin{sphinxVerbatim}[commandchars=\\\{\}]
\PYG{n}{security}\PYG{p}{:}
  \PYG{n}{authorization}\PYG{p}{:} \PYG{n}{enabled}
\end{sphinxVerbatim}

\item {} 
\sphinxAtStartPar
Run the following command on the \sphinxcode{\sphinxupquote{admin}} database:

\begin{sphinxVerbatim}[commandchars=\\\{\}]
\PYG{o}{\PYGZgt{}} \PYG{n}{db}\PYG{o}{.}\PYG{n}{createUser}\PYG{p}{(}\PYG{p}{\PYGZob{}}\PYG{n}{user}\PYG{p}{:} \PYG{l+s+s1}{\PYGZsq{}}\PYG{l+s+s1}{USER}\PYG{l+s+s1}{\PYGZsq{}}\PYG{p}{,} \PYG{n}{pwd}\PYG{p}{:} \PYG{l+s+s1}{\PYGZsq{}}\PYG{l+s+s1}{PASSWORD}\PYG{l+s+s1}{\PYGZsq{}}\PYG{p}{,} \PYG{n}{roles}\PYG{p}{:} \PYG{p}{[}\PYG{l+s+s1}{\PYGZsq{}}\PYG{l+s+s1}{root}\PYG{l+s+s1}{\PYGZsq{}}\PYG{p}{]} \PYG{p}{\PYGZcb{}}\PYG{p}{)}\PYG{p}{;}
\end{sphinxVerbatim}

\item {} 
\sphinxAtStartPar
Restart the \sphinxcode{\sphinxupquote{mongod}} service:

\begin{sphinxVerbatim}[commandchars=\\\{\}]
\PYGZdl{} systemctl restart mongod
\end{sphinxVerbatim}

\item {} 
\sphinxAtStartPar
Connect to the database as the newly created user:

\begin{sphinxVerbatim}[commandchars=\\\{\}]
\PYGZdl{} mongo \PYGZhy{}\PYGZhy{}port 27017 \PYGZhy{}u \PYGZsq{}USER\PYGZsq{} \PYGZhy{}p \PYGZsq{}PASSWORD\PYGZsq{}  \PYGZhy{}\PYGZhy{}authenticationDatabase \PYGZdq{}admin\PYGZdq{}
\end{sphinxVerbatim}

\end{enumerate}


\sphinxstrong{See also:}
\nopagebreak

\begin{description}
\item[{\sphinxstyleemphasis{MongoDB} Documentation: Enable Access Control}] \leavevmode
\sphinxAtStartPar
\sphinxurl{https://docs.mongodb.com/v4.4/tutorial/enable-authentication/}

\end{description}




\chapter{Setting up LDAP authentication with SASL}
\label{\detokenize{sasl-auth:setting-up-ldap-authentication-with-sasl}}\label{\detokenize{sasl-auth:sasl}}\label{\detokenize{sasl-auth::doc}}
\sphinxAtStartPar
This document describes an example configuration
suitable only to test out the external authentication functionality
in a non\sphinxhyphen{}production environment.
Use common sense to adapt these guidelines to your production environment.

\sphinxAtStartPar
To learn more about how the authentication works, see {\hyperref[\detokenize{authentication:ldap-authentication-sasl}]{\sphinxcrossref{\DUrole{std,std-ref}{LDAP authentication with SASL}}}}.


\section{Environment setup and configuration}
\label{\detokenize{sasl-auth:environment-setup-and-configuration}}
\sphinxAtStartPar
The following components are required:
\begin{itemize}
\item {} 
\sphinxAtStartPar
\sphinxcode{\sphinxupquote{slapd}}: OpenLDAP server.

\item {} 
\sphinxAtStartPar
\sphinxcode{\sphinxupquote{libsasl2}} version 2.1.25 or later.

\item {} 
\sphinxAtStartPar
\sphinxcode{\sphinxupquote{saslauthd}}: \sphinxstyleabbreviation{SASL} Authentication Daemon (distinct from \sphinxcode{\sphinxupquote{libsasl2}}).

\end{itemize}

\sphinxAtStartPar
The following steps will help you configure your environment:

\begin{sphinxShadowBox}
\begin{itemize}
\item {} 
\sphinxAtStartPar
\phantomsection\label{\detokenize{sasl-auth:id1}}{\hyperref[\detokenize{sasl-auth:configuring-saslauthd}]{\sphinxcrossref{Configuring \sphinxcode{\sphinxupquote{saslauthd}}}}}
\begin{itemize}
\item {} 
\sphinxAtStartPar
\phantomsection\label{\detokenize{sasl-auth:id2}}{\hyperref[\detokenize{sasl-auth:openldap-server}]{\sphinxcrossref{OpenLDAP server}}}

\item {} 
\sphinxAtStartPar
\phantomsection\label{\detokenize{sasl-auth:id3}}{\hyperref[\detokenize{sasl-auth:microsoft-windows-active-directory}]{\sphinxcrossref{Microsoft Windows Active Directory}}}

\end{itemize}

\item {} 
\sphinxAtStartPar
\phantomsection\label{\detokenize{sasl-auth:id4}}{\hyperref[\detokenize{sasl-auth:sanity-check}]{\sphinxcrossref{Sanity check}}}

\item {} 
\sphinxAtStartPar
\phantomsection\label{\detokenize{sasl-auth:id5}}{\hyperref[\detokenize{sasl-auth:configuring-libsasl2}]{\sphinxcrossref{Configuring libsasl2}}}

\item {} 
\sphinxAtStartPar
\phantomsection\label{\detokenize{sasl-auth:id6}}{\hyperref[\detokenize{sasl-auth:configuring-mongod-server}]{\sphinxcrossref{Configuring \sphinxcode{\sphinxupquote{mongod}} Server}}}
\begin{itemize}
\item {} 
\sphinxAtStartPar
\phantomsection\label{\detokenize{sasl-auth:id7}}{\hyperref[\detokenize{sasl-auth:create-a-root-user}]{\sphinxcrossref{Create a root user}}}

\item {} 
\sphinxAtStartPar
\phantomsection\label{\detokenize{sasl-auth:id8}}{\hyperref[\detokenize{sasl-auth:enable-external-authentication}]{\sphinxcrossref{Enable external authentication}}}

\item {} 
\sphinxAtStartPar
\phantomsection\label{\detokenize{sasl-auth:id9}}{\hyperref[\detokenize{sasl-auth:add-an-external-user-to-psmdb}]{\sphinxcrossref{Add an external user to \sphinxstyleemphasis{Percona Server for MongoDB}}}}

\end{itemize}

\end{itemize}
\end{sphinxShadowBox}
\subsubsection*{Assumptions}

\sphinxAtStartPar
Before we move on to the configuration steps, we assume the following:
\begin{enumerate}
\sphinxsetlistlabels{\arabic}{enumi}{enumii}{}{.}%
\item {} 
\sphinxAtStartPar
You have the LDAP server up and running and have configured users on it. The LDAP server is accessible to the server with \sphinxstyleemphasis{Percona Server for MongoDB} installed. This document focuses on OpenLDAP server. If you use Microsoft Windows Active Directory, refer to the {\hyperref[\detokenize{sasl-auth:windows-ad}]{\sphinxcrossref{\DUrole{std,std-ref}{Microsoft Windows Active Directory}}}} section for \sphinxcode{\sphinxupquote{saslauthd}} configuration.

\item {} 
\sphinxAtStartPar
You must place these two servers behind a firewall as the communications between them will be in plain text. This is because the SASL mechanism of PLAIN can only be used when authenticating and credentials will be sent in plain text.

\item {} 
\sphinxAtStartPar
You have \sphinxcode{\sphinxupquote{sudo}} privilege to the server with the \sphinxstyleemphasis{Percona Server for MongoDB} installed.

\end{enumerate}


\subsection{Configuring \sphinxstyleliteralintitle{\sphinxupquote{saslauthd}}}
\label{\detokenize{sasl-auth:configuring-saslauthd}}\begin{enumerate}
\sphinxsetlistlabels{\arabic}{enumi}{enumii}{}{.}%
\item {} 
\sphinxAtStartPar
Install the SASL packages. Depending on your OS, use the following command:
\begin{quote}

\begin{sphinxadmonition}{note}{RedHat and CentOS}

\begin{sphinxVerbatim}[commandchars=\\\{\}]
\PYGZdl{} sudo yum install \PYGZhy{}y cyrus\PYGZhy{}sasl
\end{sphinxVerbatim}

\begin{sphinxadmonition}{note}{Note:}
\sphinxAtStartPar
For \sphinxstyleemphasis{Percona Server for MongoDB} versions earlier than 4.0.26\sphinxhyphen{}21,  4.4.8\sphinxhyphen{}9, 4.2.16\sphinxhyphen{}17, also install the \sphinxcode{\sphinxupquote{cyrus\sphinxhyphen{}sasl\sphinxhyphen{}plain}} package.
\end{sphinxadmonition}
\end{sphinxadmonition}

\begin{sphinxadmonition}{note}{Debian and Ubuntu}

\begin{sphinxVerbatim}[commandchars=\\\{\}]
\PYGZdl{} sudo apt install \PYGZhy{}y sasl2\PYGZhy{}bin
\end{sphinxVerbatim}
\end{sphinxadmonition}
\end{quote}

\item {} 
\sphinxAtStartPar
Configure SASL to use \sphinxcode{\sphinxupquote{ldap}} as the  authentication mechanism.

\begin{sphinxadmonition}{note}{Note:}
\sphinxAtStartPar
Back up the original configuration file before making changes.

\begin{sphinxadmonition}{note}{RedHat and CentOS}

\sphinxAtStartPar
Specify the \sphinxcode{\sphinxupquote{ldap}} value for the \sphinxcode{\sphinxupquote{\sphinxhyphen{}\sphinxhyphen{}MECH}} option using the following command:

\begin{sphinxVerbatim}[commandchars=\\\{\}]
\PYGZdl{} sudo sed \PYGZhy{}i \PYGZhy{}e s/\PYGZca{}MECH\PYG{o}{=}pam/MECH\PYG{o}{=}ldap/g /etc/sysconfig/saslauthd
\end{sphinxVerbatim}

\sphinxAtStartPar
Alternatively, you can edit the \sphinxcode{\sphinxupquote{/etc/sysconfig/saslauthd}} configuration file:

\begin{sphinxVerbatim}[commandchars=\\\{\}]
\PYG{l+lScalar+lScalarPlain}{MECH=ldap}
\end{sphinxVerbatim}
\end{sphinxadmonition}

\begin{sphinxadmonition}{note}{Debian and Ubuntu}

\sphinxAtStartPar
Use the following commands to enable the \sphinxcode{\sphinxupquote{saslauthd}} to auto\sphinxhyphen{}run on startup and to set the \sphinxcode{\sphinxupquote{ldap}} value for the \sphinxcode{\sphinxupquote{\sphinxhyphen{}\sphinxhyphen{}MECHANISMS}} option:

\begin{sphinxVerbatim}[commandchars=\\\{\}]
\PYGZdl{} sudo sed \PYGZhy{}i \PYGZhy{}e s/\PYGZca{}MECH\PYG{o}{=}pam/MECH\PYG{o}{=}ldap/g /etc/sysconfig/saslauthdsudo sed \PYGZhy{}i \PYGZhy{}e s/\PYGZca{}MECHANISMS\PYG{o}{=}\PYG{l+s+s2}{\PYGZdq{}pam\PYGZdq{}}/MECHANISMS\PYG{o}{=}\PYG{l+s+s2}{\PYGZdq{}ldap\PYGZdq{}}/g /etc/default/saslauthd
\PYGZdl{} sudo sed \PYGZhy{}i \PYGZhy{}e s/\PYGZca{}START\PYG{o}{=}no/START\PYG{o}{=}yes/g /etc/default/saslauthd
\end{sphinxVerbatim}

\sphinxAtStartPar
Alternatively, you can edit the \sphinxcode{\sphinxupquote{/etc/default/sysconfig/saslauthd}} configuration file:

\begin{sphinxVerbatim}[commandchars=\\\{\}]
\PYG{l+lScalar+lScalarPlain}{START=yes}
\PYG{l+lScalar+lScalarPlain}{MECHANISMS=\PYGZdq{}ldap\PYGZdq{}}
\end{sphinxVerbatim}
\end{sphinxadmonition}
\end{sphinxadmonition}

\item {} 
\sphinxAtStartPar
Create the \sphinxcode{\sphinxupquote{/etc/saslauthd.conf}} configuration file and specify the settings that \sphinxcode{\sphinxupquote{saslauthd}} requires to connect to a local LDAP service:

\end{enumerate}


\subsubsection{OpenLDAP server}
\label{\detokenize{sasl-auth:openldap-server}}
\sphinxAtStartPar
The following is the example configuration file. Note that the server address \sphinxstylestrong{MUST} match the OpenLDAP installation:

\begin{sphinxVerbatim}[commandchars=\\\{\}]
ldap\PYGZus{}servers: ldap://localhost
ldap\PYGZus{}mech: PLAIN
ldap\PYGZus{}search\PYGZus{}base: dc=example,dc=com
ldap\PYGZus{}filter: (cn=\PYGZpc{}u)
ldap\PYGZus{}bind\PYGZus{}dn: cn=admin,dc=example,dc=com
ldap\PYGZus{}password: secret
\end{sphinxVerbatim}

\sphinxAtStartPar
Note the LDAP password (\sphinxcode{\sphinxupquote{ldap\_password}}) and bind domain name (\sphinxcode{\sphinxupquote{ldap\_bind\_dn}}).
This allows the \sphinxcode{\sphinxupquote{saslauthd}} service to connect to the LDAP service as admin.
In production, this would not be the case; users should not store administrative passwords in unencrypted files.


\subsubsection{Microsoft Windows Active Directory}
\label{\detokenize{sasl-auth:microsoft-windows-active-directory}}\label{\detokenize{sasl-auth:windows-ad}}
\sphinxAtStartPar
In order for LDAP operations to be performed
against a Windows Active Directory server,
a user record must be created to perform the lookups.

\sphinxAtStartPar
The following example shows configuration parameters for \sphinxcode{\sphinxupquote{saslauthd}}
to communicate with an Active Directory server:

\begin{sphinxVerbatim}[commandchars=\\\{\}]
ldap\PYGZus{}servers: ldap://localhost
ldap\PYGZus{}mech: PLAIN
ldap\PYGZus{}search\PYGZus{}base: CN=Users,DC=example,DC=com
ldap\PYGZus{}filter: (sAMAccountName=\PYGZpc{}u)
ldap\PYGZus{}bind\PYGZus{}dn: CN=ldapmgr,CN=Users,DC=\PYGZlt{}AD Domain\PYGZgt{},DC=\PYGZlt{}AD TLD\PYGZgt{}
ldap\PYGZus{}password: ld@pmgr\PYGZus{}Pa55word
\end{sphinxVerbatim}

\sphinxAtStartPar
In order to determine and test the correct search base
and filter for your Active Directory installation,
the Microsoft \sphinxhref{https://technet.microsoft.com/en-us/library/Cc772839\%28v=WS.10\%29.aspx}{LDP GUI Tool}
can be used to bind and search the LDAP\sphinxhyphen{}compatible directory.
\begin{enumerate}
\sphinxsetlistlabels{\arabic}{enumi}{enumii}{}{.}%
\setcounter{enumi}{3}
\item {} 
\sphinxAtStartPar
Start the \sphinxcode{\sphinxupquote{saslauthd}} process and set it to run at restart:

\begin{sphinxVerbatim}[commandchars=\\\{\}]
\PYGZdl{} sudo systemctl start saslauthd
\PYGZdl{} sudo systemctl \PYG{n+nb}{enable} saslauthd
\end{sphinxVerbatim}

\item {} 
\sphinxAtStartPar
Give write permissions to the \sphinxcode{\sphinxupquote{/run/saslauthd}} folder for the \sphinxcode{\sphinxupquote{mongod}}. Either change permissions to the  \sphinxcode{\sphinxupquote{/run/saslauthd}} folder:

\begin{sphinxVerbatim}[commandchars=\\\{\}]
\PYGZdl{} sudo chmod \PYG{l+m}{755} /run/saslauthd
\end{sphinxVerbatim}

\sphinxAtStartPar
Or add the \sphinxcode{\sphinxupquote{mongod}} user to the \sphinxcode{\sphinxupquote{sasl}} group:

\begin{sphinxVerbatim}[commandchars=\\\{\}]
\PYGZdl{} sudo usermod \PYGZhy{}a \PYGZhy{}G sasl mongod
\end{sphinxVerbatim}

\end{enumerate}


\subsection{Sanity check}
\label{\detokenize{sasl-auth:sanity-check}}
\sphinxAtStartPar
Verify that the \sphinxcode{\sphinxupquote{saslauthd}} service can authenticate
against the users created in the LDAP service:

\begin{sphinxVerbatim}[commandchars=\\\{\}]
\PYGZdl{} testsaslauthd \PYGZhy{}u christian \PYGZhy{}p secret  \PYGZhy{}f /var/run/saslauthd/mux
\end{sphinxVerbatim}

\sphinxAtStartPar
This should return \sphinxcode{\sphinxupquote{0:OK "Success"}}.
If it doesn’t, then either the user name and password
are not in the LDAP service, or \sphinxcode{\sphinxupquote{sasaluthd}} is not configured properly.


\subsection{Configuring libsasl2}
\label{\detokenize{sasl-auth:configuring-libsasl2}}
\sphinxAtStartPar
The \sphinxcode{\sphinxupquote{mongod}} also uses the SASL library for communications. To configure the SASL library, create a configuration file.

\sphinxAtStartPar
The configuration file \sphinxstylestrong{must} be named \sphinxcode{\sphinxupquote{mongodb.conf}} and placed in a directory
where \sphinxcode{\sphinxupquote{libsasl2}} can find and read it.
\sphinxcode{\sphinxupquote{libsasl2}} is hard\sphinxhyphen{}coded to look in certain directories at build time.
This location may be different depending on the installation method.

\sphinxAtStartPar
In the configuration file, specify the following:

\begin{sphinxVerbatim}[commandchars=\\\{\}]
pwcheck\PYGZus{}method: saslauthd
saslauthd\PYGZus{}path: /var/run/saslauthd/mux
log\PYGZus{}level: 5
mech\PYGZus{}list: plain
\end{sphinxVerbatim}

\sphinxAtStartPar
The first two entries (\sphinxcode{\sphinxupquote{pwcheck\_method}} and \sphinxcode{\sphinxupquote{saslauthd\_path}})
are required for \sphinxcode{\sphinxupquote{mongod}} to successfully use the \sphinxcode{\sphinxupquote{saslauthd}} service.
The \sphinxcode{\sphinxupquote{log\_level}} is optional but may help determine configuration errors.


\sphinxstrong{See also:}
\nopagebreak


\sphinxAtStartPar
\sphinxhref{https://www.cyrusimap.org/sasl/index.html}{SASL documentation:}




\subsection{Configuring \sphinxstyleliteralintitle{\sphinxupquote{mongod}} Server}
\label{\detokenize{sasl-auth:configuring-mongod-server}}
\sphinxAtStartPar
The configuration consists of the following steps:
\begin{itemize}
\item {} 
\sphinxAtStartPar
Creating a user with the \sphinxstylestrong{root} privileges. This user is required to log in to \sphinxstyleemphasis{Percona Server for MongoDB} after the external authentication is enabled.

\item {} 
\sphinxAtStartPar
Editing the configuration file to enable the external authentication

\end{itemize}


\subsubsection{Create a root user}
\label{\detokenize{sasl-auth:create-a-root-user}}\label{\detokenize{sasl-auth:root-user}}
\sphinxAtStartPar
Create a user with the \sphinxstylestrong{root} privileges in the \sphinxcode{\sphinxupquote{admin}} database. If you have already created this user, skip this step. Otherwise, run the following command to create the admin user:

\begin{sphinxVerbatim}[commandchars=\\\{\}]
\PYG{o}{\PYGZgt{}}\PYG{+w}{ }\PYG{n+nx}{use}\PYG{+w}{ }\PYG{n+nx}{admin}
\PYG{n+nx}{switched}\PYG{+w}{ }\PYG{n+nx}{to}\PYG{+w}{ }\PYG{n+nx}{db}\PYG{+w}{ }\PYG{n+nx}{admin}
\PYG{o}{\PYGZgt{}}\PYG{+w}{ }\PYG{n+nx}{db}\PYG{p}{.}\PYG{n+nx}{createUser}\PYG{p}{(}\PYG{p}{\PYGZob{}}\PYG{l+s+s2}{\PYGZdq{}user\PYGZdq{}}\PYG{o}{:}\PYG{+w}{ }\PYG{l+s+s2}{\PYGZdq{}admin\PYGZdq{}}\PYG{p}{,}\PYG{+w}{ }\PYG{l+s+s2}{\PYGZdq{}pwd\PYGZdq{}}\PYG{o}{:}\PYG{+w}{ }\PYG{l+s+s2}{\PYGZdq{}\PYGZdl{}3cr3tP4ssw0rd\PYGZdq{}}\PYG{p}{,}\PYG{+w}{ }\PYG{l+s+s2}{\PYGZdq{}roles\PYGZdq{}}\PYG{o}{:}\PYG{+w}{ }\PYG{p}{[}\PYG{l+s+s2}{\PYGZdq{}root\PYGZdq{}}\PYG{p}{]}\PYG{p}{\PYGZcb{}}\PYG{p}{)}
\PYG{n+nx}{Successfully}\PYG{+w}{ }\PYG{n+nx}{added}\PYG{+w}{ }\PYG{n+nx}{user}\PYG{o}{:}\PYG{+w}{ }\PYG{p}{\PYGZob{}}\PYG{+w}{ }\PYG{l+s+s2}{\PYGZdq{}user\PYGZdq{}}\PYG{+w}{ }\PYG{o}{:}\PYG{+w}{ }\PYG{l+s+s2}{\PYGZdq{}admin\PYGZdq{}}\PYG{p}{,}\PYG{+w}{ }\PYG{l+s+s2}{\PYGZdq{}roles\PYGZdq{}}\PYG{+w}{ }\PYG{o}{:}\PYG{+w}{ }\PYG{p}{[}\PYG{+w}{ }\PYG{l+s+s2}{\PYGZdq{}root\PYGZdq{}}\PYG{+w}{ }\PYG{p}{]}\PYG{+w}{ }\PYG{p}{\PYGZcb{}}
\end{sphinxVerbatim}


\subsubsection{Enable external authentication}
\label{\detokenize{sasl-auth:enable-external-authentication}}
\sphinxAtStartPar
Edit the \sphinxcode{\sphinxupquote{etc/mongod.conf}} configuration file to enable the external authentication:

\begin{sphinxVerbatim}[commandchars=\\\{\}]
\PYG{n+nt}{security}\PYG{p}{:}
\PYG{+w}{  }\PYG{n+nt}{authorization}\PYG{p}{:}\PYG{+w}{ }\PYG{l+lScalar+lScalarPlain}{enabled}

\PYG{n+nt}{setParameter}\PYG{p}{:}
\PYG{+w}{  }\PYG{n+nt}{authenticationMechanisms}\PYG{p}{:}\PYG{+w}{ }\PYG{l+lScalar+lScalarPlain}{PLAIN,SCRAM\PYGZhy{}SHA\PYGZhy{}1}
\end{sphinxVerbatim}

\sphinxAtStartPar
Restart the \sphinxcode{\sphinxupquote{mongod}} service:

\begin{sphinxVerbatim}[commandchars=\\\{\}]
\PYGZdl{} sudo systemctl restart mongod
\end{sphinxVerbatim}


\subsubsection{Add an external user to \sphinxstyleemphasis{Percona Server for MongoDB}}
\label{\detokenize{sasl-auth:add-an-external-user-to-psmdb}}
\sphinxAtStartPar
User authentication is done by mapping a user object on the LDAP server against a user created in the \sphinxcode{\sphinxupquote{\$external}} database. Thus, at this step, you create the user in the \sphinxcode{\sphinxupquote{\$external}} database and they inherit the roles and privileges. Note that the username must exactly match the name of the user object on the LDAP server.

\sphinxAtStartPar
Connect to \sphinxstyleemphasis{Percona Server for MongoDB} and authenticate as {\hyperref[\detokenize{sasl-auth:root-user}]{\sphinxcrossref{\DUrole{std,std-ref}{the root user}}}}.

\begin{sphinxVerbatim}[commandchars=\\\{\}]
\PYGZdl{} mongo \PYGZhy{}\PYGZhy{}host localhost \PYGZhy{}\PYGZhy{}port \PYG{l+m}{27017} \PYGZhy{}u admin \PYGZhy{}p \PYG{l+s+s1}{\PYGZsq{}\PYGZdl{}3cr3tP4ssw0rd\PYGZsq{}} \PYGZhy{}\PYGZhy{}authenticationDatabase \PYG{l+s+s1}{\PYGZsq{}admin\PYGZsq{}}
\end{sphinxVerbatim}

\sphinxAtStartPar
Use the following command to add an external user to \sphinxstyleemphasis{Percona Server for MongoDB}:

\begin{sphinxVerbatim}[commandchars=\\\{\}]
\PYG{o}{\PYGZgt{}}\PYG{+w}{ }\PYG{n+nx}{db}\PYG{p}{.}\PYG{n+nx}{getSiblingDB}\PYG{p}{(}\PYG{l+s+s2}{\PYGZdq{}\PYGZdl{}external\PYGZdq{}}\PYG{p}{)}\PYG{p}{.}\PYG{n+nx}{createUser}\PYG{p}{(}\PYG{+w}{ }\PYG{p}{\PYGZob{}}\PYG{n+nx}{user}\PYG{+w}{ }\PYG{o}{:}\PYG{+w}{ }\PYG{l+s+s2}{\PYGZdq{}christian\PYGZdq{}}\PYG{p}{,}\PYG{+w}{ }\PYG{n+nx}{roles}\PYG{o}{:}\PYG{+w}{ }\PYG{p}{[}\PYG{+w}{ }\PYG{p}{\PYGZob{}}\PYG{n+nx}{role}\PYG{o}{:}\PYG{+w}{ }\PYG{l+s+s2}{\PYGZdq{}read\PYGZdq{}}\PYG{p}{,}\PYG{+w}{ }\PYG{n+nx}{db}\PYG{o}{:}\PYG{+w}{ }\PYG{l+s+s2}{\PYGZdq{}test\PYGZdq{}}\PYG{p}{\PYGZcb{}}\PYG{+w}{ }\PYG{p}{]}\PYG{p}{\PYGZcb{}}\PYG{+w}{ }\PYG{p}{)}\PYG{p}{;}
\end{sphinxVerbatim}


\section{Authenticate as an external user in \sphinxstyleemphasis{Percona Server for MongoDB}}
\label{\detokenize{sasl-auth:authenticate-as-an-external-user-in-psmdb}}
\sphinxAtStartPar
When running the \sphinxcode{\sphinxupquote{mongo}} client, a user can authenticate
against a given database using the following command:

\begin{sphinxVerbatim}[commandchars=\\\{\}]
\PYG{o}{\PYGZgt{}}\PYG{+w}{ }\PYG{n+nx}{db}\PYG{p}{.}\PYG{n+nx}{getSiblingDB}\PYG{p}{(}\PYG{l+s+s2}{\PYGZdq{}\PYGZdl{}external\PYGZdq{}}\PYG{p}{)}\PYG{p}{.}\PYG{n+nx}{auth}\PYG{p}{(}\PYG{p}{\PYGZob{}}\PYG{+w}{ }\PYG{n+nx}{mechanism}\PYG{o}{:}\PYG{l+s+s2}{\PYGZdq{}PLAIN\PYGZdq{}}\PYG{p}{,}\PYG{+w}{ }\PYG{n+nx}{user}\PYG{o}{:}\PYG{l+s+s2}{\PYGZdq{}christian\PYGZdq{}}\PYG{p}{,}\PYG{+w}{ }\PYG{n+nx}{pwd}\PYG{o}{:}\PYG{l+s+s2}{\PYGZdq{}secret\PYGZdq{}}\PYG{p}{,}\PYG{+w}{ }\PYG{n+nx}{digestPassword}\PYG{o}{:}\PYG{k+kc}{false}\PYG{p}{\PYGZcb{}}\PYG{p}{)}
\end{sphinxVerbatim}

\sphinxAtStartPar
Alternatively, a user can authenticate while connecting to \sphinxstyleemphasis{Percona Server for MongoDB}:

\begin{sphinxVerbatim}[commandchars=\\\{\}]
\PYGZdl{} mongo \PYGZhy{}\PYGZhy{}host localhost \PYGZhy{}\PYGZhy{}port \PYG{l+m}{27017} \PYGZhy{}\PYGZhy{}authenticationMechanism PLAIN \PYGZhy{}\PYGZhy{}authenticationDatabase \PYG{l+s+se}{\PYGZbs{}\PYGZdl{}}external \PYGZhy{}u christian \PYGZhy{}p
\end{sphinxVerbatim}

\begin{sphinxadmonition}{note}{Based on the material from \sphinxstylestrong{Percona Database Performance Blog}}

\sphinxAtStartPar
This section is based on the blog post \sphinxhref{https://www.percona.com/blog/2018/12/21/percona-server-for-mongodb-authentication-using-active-directory/}{Percona Server for MongoDB Authentication Using Active Directory} by \sphinxstyleemphasis{Doug Duncan}:
\end{sphinxadmonition}


\chapter{Setting up LDAP authentication and authorization using NativeLDAP}
\label{\detokenize{ldap-setup:setting-up-ldap-authentication-and-authorization-using-nativeldap}}\label{\detokenize{ldap-setup:ldap-setup}}\label{\detokenize{ldap-setup::doc}}
\sphinxAtStartPar
This document describes an example configuration of LDAP authentication and authorization using direct binding to an LDAP server (Native LDAP). We recommend testing this setup in a non\sphinxhyphen{}production environment first, before applying it in production.


\section{Assumptions}
\label{\detokenize{ldap-setup:assumptions}}\begin{enumerate}
\sphinxsetlistlabels{\arabic}{enumi}{enumii}{}{.}%
\item {} 
\sphinxAtStartPar
The setup of an LDAP server is out of scope of this document. We assume that you are familiar with the LDAP server schema.

\item {} 
\sphinxAtStartPar
You have the LDAP server up and running and it is accessible to the servers with Percona Server for MongoDB installed.

\item {} 
\sphinxAtStartPar
This document primarily focuses on OpenLDAP used as the LDAP server and the examples are given based on the OpenLDAP format. If you are using Active Directory, refer to the {\hyperref[\detokenize{ldap-setup:active-directory}]{\sphinxcrossref{\DUrole{std,std-ref}{Active Directory configuration}}}} section.

\item {} 
\sphinxAtStartPar
You have the \sphinxcode{\sphinxupquote{sudo}} privilege to the server with the Percona Server for MongoDB installed.

\end{enumerate}


\section{Prerequisites}
\label{\detokenize{ldap-setup:prerequisites}}\begin{itemize}
\item {} 
\sphinxAtStartPar
In this setup we use anonymous binds to the LDAP server. If your LDAP server disallows anonymous binds, create the user that \sphinxstyleemphasis{Percona Server for MongoDB} will use to connect to and query the LDAP server. Define this user’s credentials for the \sphinxcode{\sphinxupquote{security.ldap.bind.queryUser}} and \sphinxcode{\sphinxupquote{security.ldap.bind.queryPassword}}  parameters in the \sphinxcode{\sphinxupquote{mongod.conf}} configuration file.

\item {} 
\sphinxAtStartPar
In this setup, we use the following OpenLDAP groups:

\begin{sphinxVerbatim}[commandchars=\\\{\}]
dn: cn=testusers,dc=percona,dc=com
objectClass: groupOfNames
cn: testusers
member: cn=alice,dc=percona,dc=com

dn: cn=otherusers,dc=percona,dc=com
objectClass: groupOfNames
cn: otherusers
member: cn=bob,dc=percona,dc=com
\end{sphinxVerbatim}

\end{itemize}


\section{Setup procedure}
\label{\detokenize{ldap-setup:setup-procedure}}

\subsection{Configure TLS/SSL connection for \sphinxstyleemphasis{Percona Server for MongoDB}}
\label{\detokenize{ldap-setup:configure-tls-ssl-connection-for-psmdb}}
\sphinxAtStartPar
By default, \sphinxstyleemphasis{Percona Server for MongoDB} establishes the TLS connection when binding to the LDAP server and thus, it requires access to the LDAP \sphinxstyleabbreviation{CA} certificates. To make \sphinxstyleemphasis{Percona Server for MongoDB} aware of the certificates, do the following:
\begin{enumerate}
\sphinxsetlistlabels{\arabic}{enumi}{enumii}{}{.}%
\item {} 
\sphinxAtStartPar
Place the certificate in the \sphinxcode{\sphinxupquote{certs}} directory. The path to the \sphinxcode{\sphinxupquote{certs}} directory is:
\begin{itemize}
\item {} 
\sphinxAtStartPar
On Debian / Ubuntu: \sphinxcode{\sphinxupquote{/etc/ssl/certs/}}

\item {} 
\sphinxAtStartPar
On RHEL / CentOS: \sphinxcode{\sphinxupquote{/etc/openldap/certs/}}

\end{itemize}

\item {} 
\sphinxAtStartPar
Specify the path to the certificates in the \sphinxcode{\sphinxupquote{ldap.conf}} file:
\begin{quote}

\begin{sphinxadmonition}{note}{Debian / Ubuntu}

\begin{sphinxVerbatim}[commandchars=\\\{\}]
tee \PYGZhy{}a /etc/openldap/ldap.conf \PYG{l+s}{\PYGZlt{}\PYGZlt{}EOF}
\PYG{l+s}{TLS\PYGZus{}CACERT /etc/ssl/certs/my\PYGZus{}CA.crt}
\PYG{l+s}{EOF}
\end{sphinxVerbatim}
\end{sphinxadmonition}

\begin{sphinxadmonition}{note}{RHEL / CentOS}

\begin{sphinxVerbatim}[commandchars=\\\{\}]
tee \PYGZhy{}a /etc/openldap/ldap.conf \PYG{l+s}{\PYGZlt{}\PYGZlt{}EOF}
\PYG{l+s}{TLS\PYGZus{}CACERT /etc/openldap/certs/my\PYGZus{}CA.crt}
\PYG{l+s}{EOF}
\end{sphinxVerbatim}
\end{sphinxadmonition}
\end{quote}

\end{enumerate}


\subsection{Create roles for LDAP groups in \sphinxstyleemphasis{Percona Server for MongoDB}}
\label{\detokenize{ldap-setup:create-roles-for-ldap-groups-in-psmdb}}
\sphinxAtStartPar
\sphinxstyleemphasis{Percona Server for MongoDB} authorizes users based on LDAP group membership. For every group, you must create the role in the \sphinxcode{\sphinxupquote{admin}} database with the name that exactly matches the \sphinxstyleabbreviation{DN} of the LDAP group.

\sphinxAtStartPar
\sphinxstyleemphasis{Percona Server for MongoDB} maps the user’s LDAP group to the roles and determines what role is assigned to the user. \sphinxstyleemphasis{Percona Server for MongoDB} then grants privileges defined by this role.

\sphinxAtStartPar
To create the roles, use the following command:

\begin{sphinxVerbatim}[commandchars=\\\{\}]
\PYG{k+kd}{var}\PYG{+w}{ }\PYG{n+nx}{admin}\PYG{+w}{ }\PYG{o}{=}\PYG{+w}{ }\PYG{n+nx}{db}\PYG{p}{.}\PYG{n+nx}{getSiblingDB}\PYG{p}{(}\PYG{l+s+s2}{\PYGZdq{}admin\PYGZdq{}}\PYG{p}{)}
\PYG{n+nx}{db}\PYG{p}{.}\PYG{n+nx}{createRole}\PYG{p}{(}
\PYG{+w}{   }\PYG{p}{\PYGZob{}}
\PYG{+w}{     }\PYG{n+nx}{role}\PYG{o}{:}\PYG{+w}{ }\PYG{l+s+s2}{\PYGZdq{}cn=testusers,dc=percona,dc=com\PYGZdq{}}\PYG{p}{,}
\PYG{+w}{     }\PYG{n+nx}{privileges}\PYG{o}{:}\PYG{+w}{ }\PYG{p}{[}\PYG{p}{]}\PYG{p}{,}
\PYG{+w}{     }\PYG{n+nx}{roles}\PYG{o}{:}\PYG{+w}{ }\PYG{p}{[}\PYG{+w}{ }\PYG{l+s+s2}{\PYGZdq{}readWrite\PYGZdq{}}\PYG{p}{]}
\PYG{+w}{   }\PYG{p}{\PYGZcb{}}
\PYG{p}{)}

\PYG{n+nx}{db}\PYG{p}{.}\PYG{n+nx}{createRole}\PYG{p}{(}
\PYG{+w}{   }\PYG{p}{\PYGZob{}}
\PYG{+w}{     }\PYG{n+nx}{role}\PYG{o}{:}\PYG{+w}{ }\PYG{l+s+s2}{\PYGZdq{}cn=otherusers,dc=percona,dc=com\PYGZdq{}}\PYG{p}{,}
\PYG{+w}{     }\PYG{n+nx}{privileges}\PYG{o}{:}\PYG{+w}{ }\PYG{p}{[}\PYG{p}{]}\PYG{p}{,}
\PYG{+w}{     }\PYG{n+nx}{roles}\PYG{o}{:}\PYG{+w}{ }\PYG{p}{[}\PYG{+w}{ }\PYG{l+s+s2}{\PYGZdq{}read\PYGZdq{}}\PYG{p}{]}
\PYG{+w}{   }\PYG{p}{\PYGZcb{}}
\PYG{p}{)}
\end{sphinxVerbatim}


\subsection{\sphinxstyleemphasis{Percona Server for MongoDB} configuration}
\label{\detokenize{ldap-setup:psmdb-configuration}}

\subsubsection{Access without username transformation}
\label{\detokenize{ldap-setup:access-without-username-transformation}}
\sphinxAtStartPar
This section assumes that users connect to \sphinxstyleemphasis{Percona Server for MongoDB} by providing their LDAP \sphinxstyleabbreviation{DN} as the username.
\begin{enumerate}
\sphinxsetlistlabels{\arabic}{enumi}{enumii}{}{.}%
\item {} 
\sphinxAtStartPar
Edit the \sphinxstyleemphasis{Percona Server for MongoDB} configuration file (by default, \sphinxcode{\sphinxupquote{/etc/mongod.conf}}) and specify the following configuration:

\begin{sphinxVerbatim}[commandchars=\\\{\}]
\PYG{n+nt}{security}\PYG{p}{:}
\PYG{+w}{  }\PYG{n+nt}{authorization}\PYG{p}{:}\PYG{+w}{ }\PYG{l+s}{\PYGZdq{}}\PYG{l+s}{enabled}\PYG{l+s}{\PYGZdq{}}
\PYG{+w}{  }\PYG{n+nt}{ldap}\PYG{p}{:}
\PYG{+w}{    }\PYG{n+nt}{servers}\PYG{p}{:}\PYG{+w}{ }\PYG{l+s}{\PYGZdq{}}\PYG{l+s}{ldap.example.com}\PYG{l+s}{\PYGZdq{}}
\PYG{+w}{    }\PYG{n+nt}{transportSecurity}\PYG{p}{:}\PYG{+w}{ }\PYG{l+lScalar+lScalarPlain}{tls}
\PYG{+w}{    }\PYG{n+nt}{authz}\PYG{p}{:}
\PYG{+w}{       }\PYG{n+nt}{queryTemplate}\PYG{p}{:}\PYG{+w}{ }\PYG{l+s}{\PYGZdq{}}\PYG{l+s}{dc=percona,dc=com??sub?(\PYGZam{}(objectClass=groupOfNames)(member=\PYGZob{}PROVIDED\PYGZus{}USER\PYGZcb{}))}\PYG{l+s}{\PYGZdq{}}

\PYG{n+nt}{setParameter}\PYG{p}{:}
\PYG{+w}{  }\PYG{n+nt}{authenticationMechanisms}\PYG{p}{:}\PYG{+w}{ }\PYG{l+s}{\PYGZdq{}}\PYG{l+s}{PLAIN}\PYG{l+s}{\PYGZdq{}}
\end{sphinxVerbatim}

\sphinxAtStartPar
The \{PROVIDED\_USER\} variable substitutes the provided username before authentication or username transformation takes place.

\sphinxAtStartPar
Replace \sphinxcode{\sphinxupquote{ldap.example.com}} with the hostname of your LDAP server. In the LDAP query template, replace the domain controllers \sphinxcode{\sphinxupquote{percona}} and \sphinxcode{\sphinxupquote{com}} with those relevant to your organization.

\item {} 
\sphinxAtStartPar
Restart the \sphinxcode{\sphinxupquote{mongod}} service:

\begin{sphinxVerbatim}[commandchars=\\\{\}]
\PYGZdl{} sudo systemctl restart mongod
\end{sphinxVerbatim}

\item {} 
\sphinxAtStartPar
Test the access to \sphinxstyleemphasis{Percona Server for MongoDB}:

\begin{sphinxVerbatim}[commandchars=\\\{\}]
mongo \PYGZhy{}u \PYG{l+s+s2}{\PYGZdq{}cn=alice,dc=percona,dc=com\PYGZdq{}} \PYGZhy{}p \PYG{l+s+s2}{\PYGZdq{}secretpwd\PYGZdq{}} \PYGZhy{}\PYGZhy{}authenticationDatabase \PYG{l+s+s1}{\PYGZsq{}\PYGZdl{}external\PYGZsq{}} \PYGZhy{}\PYGZhy{}authenticationMechanism \PYG{l+s+s1}{\PYGZsq{}PLAIN\PYGZsq{}}
\end{sphinxVerbatim}

\end{enumerate}


\subsubsection{Access with username transformation}
\label{\detokenize{ldap-setup:access-with-username-transformation}}
\sphinxAtStartPar
If users connect to \sphinxstyleemphasis{Percona Server for MongoDB} with usernames that are not LDAP \sphinxstyleabbreviation{DN}, you need to transform these usernames to be accepted by the LDAP server.

\sphinxAtStartPar
Using the \sphinxcode{\sphinxupquote{\sphinxhyphen{}\sphinxhyphen{}ldapUserToDNMapping}} configuration parameter allows you to do this. You specify the match pattern as a regexp to capture a username. Next, specify how to transform it \sphinxhyphen{} either to use a substitution value or to query the LDAP server for a username.

\sphinxAtStartPar
If you don’t know what the substitution or LDAP query string should be, please consult with the LDAP administrators to figure this out.

\sphinxAtStartPar
Note that you can use only the \sphinxcode{\sphinxupquote{query}} or the \sphinxcode{\sphinxupquote{substitution}} stage, the combination of two is not allowed.
\begin{quote}

\begin{sphinxadmonition}{note}{Substitution}
\begin{enumerate}
\sphinxsetlistlabels{\arabic}{enumi}{enumii}{}{.}%
\item {} 
\sphinxAtStartPar
Edit the \sphinxstyleemphasis{Percona Server for MongoDB} configuration file (by default, \sphinxcode{\sphinxupquote{/etc/mongod.conf}}) and specify the \sphinxcode{\sphinxupquote{userToDNMapping}} parameter:

\begin{sphinxVerbatim}[commandchars=\\\{\}]
\PYG{n+nt}{security}\PYG{p}{:}
\PYG{+w}{  }\PYG{n+nt}{authorization}\PYG{p}{:}\PYG{+w}{ }\PYG{l+s}{\PYGZdq{}}\PYG{l+s}{enabled}\PYG{l+s}{\PYGZdq{}}
\PYG{+w}{  }\PYG{n+nt}{ldap}\PYG{p}{:}
\PYG{+w}{    }\PYG{n+nt}{servers}\PYG{p}{:}\PYG{+w}{ }\PYG{l+s}{\PYGZdq{}}\PYG{l+s}{ldap.example.com}\PYG{l+s}{\PYGZdq{}}
\PYG{+w}{    }\PYG{n+nt}{transportSecurity}\PYG{p}{:}\PYG{+w}{ }\PYG{l+lScalar+lScalarPlain}{tls}
\PYG{+w}{    }\PYG{n+nt}{authz}\PYG{p}{:}
\PYG{+w}{       }\PYG{n+nt}{queryTemplate}\PYG{p}{:}\PYG{+w}{ }\PYG{l+s}{\PYGZdq{}}\PYG{l+s}{dc=percona,dc=com??sub?(\PYGZam{}(objectClass=groupOfNames)(member=\PYGZob{}USER\PYGZcb{}))}\PYG{l+s}{\PYGZdq{}}
\PYG{+w}{    }\PYG{n+nt}{userToDNMapping}\PYG{p}{:}\PYG{+w}{ }\PYG{p+pIndicator}{\PYGZgt{}}\PYG{p+pIndicator}{\PYGZhy{}}
\PYG{+w}{          }\PYG{n+no}{[}
\PYG{+w}{            }\PYG{n+no}{\PYGZob{}}
\PYG{+w}{              }\PYG{n+no}{match: \PYGZdq{}([\PYGZca{}@]+)@percona\PYGZbs{}\PYGZbs{}.com\PYGZdq{},}
\PYG{+w}{              }\PYG{n+no}{substitution: \PYGZdq{}CN=\PYGZob{}0\PYGZcb{},DC=percona,DC=com\PYGZdq{}}
\PYG{+w}{            }\PYG{n+no}{\PYGZcb{}}
\PYG{+w}{          }\PYG{n+no}{]}

\PYG{n+nt}{setParameter}\PYG{p}{:}
\PYG{+w}{  }\PYG{n+nt}{authenticationMechanisms}\PYG{p}{:}\PYG{+w}{ }\PYG{l+s}{\PYGZdq{}}\PYG{l+s}{PLAIN}\PYG{l+s}{\PYGZdq{}}
\end{sphinxVerbatim}

\sphinxAtStartPar
The \sphinxcode{\sphinxupquote{\{USER\}}} variable substitutes the username transformed during the \sphinxcode{\sphinxupquote{userToDNMapping}} stage.

\sphinxAtStartPar
Modify the given example configuration to match your deployment.

\item {} 
\sphinxAtStartPar
Restart the \sphinxcode{\sphinxupquote{mongod}} service:

\begin{sphinxVerbatim}[commandchars=\\\{\}]
\PYGZdl{} sudo systemctl restart mongod
\end{sphinxVerbatim}

\item {} 
\sphinxAtStartPar
Test the access to \sphinxstyleemphasis{Percona Server for MongoDB}:

\begin{sphinxVerbatim}[commandchars=\\\{\}]
mongo \PYGZhy{}u \PYG{l+s+s2}{\PYGZdq{}alice@percona.com\PYGZdq{}} \PYGZhy{}p \PYG{l+s+s2}{\PYGZdq{}secretpwd\PYGZdq{}} \PYGZhy{}\PYGZhy{}authenticationDatabase \PYG{l+s+s1}{\PYGZsq{}\PYGZdl{}external\PYGZsq{}} \PYGZhy{}\PYGZhy{}authenticationMechanism \PYG{l+s+s1}{\PYGZsq{}PLAIN\PYGZsq{}}
\end{sphinxVerbatim}

\end{enumerate}
\end{sphinxadmonition}

\begin{sphinxadmonition}{note}{LDAP query}
\begin{enumerate}
\sphinxsetlistlabels{\arabic}{enumi}{enumii}{}{.}%
\item {} 
\sphinxAtStartPar
Edit the \sphinxstyleemphasis{Percona Server for MongoDB} configuration file (by default, \sphinxcode{\sphinxupquote{/etc/mongod.conf}}) and specify \sphinxcode{\sphinxupquote{userToDNMapping}} parameter:

\begin{sphinxVerbatim}[commandchars=\\\{\}]
\PYG{n+nt}{security}\PYG{p}{:}
\PYG{+w}{  }\PYG{n+nt}{authorization}\PYG{p}{:}\PYG{+w}{ }\PYG{l+s}{\PYGZdq{}}\PYG{l+s}{enabled}\PYG{l+s}{\PYGZdq{}}
\PYG{+w}{  }\PYG{n+nt}{ldap}\PYG{p}{:}
\PYG{+w}{    }\PYG{n+nt}{servers}\PYG{p}{:}\PYG{+w}{ }\PYG{l+s}{\PYGZdq{}}\PYG{l+s}{ldap.example.com}\PYG{l+s}{\PYGZdq{}}
\PYG{+w}{    }\PYG{n+nt}{transportSecurity}\PYG{p}{:}\PYG{+w}{ }\PYG{l+lScalar+lScalarPlain}{tls}
\PYG{+w}{    }\PYG{n+nt}{authz}\PYG{p}{:}
\PYG{+w}{       }\PYG{n+nt}{queryTemplate}\PYG{p}{:}\PYG{+w}{ }\PYG{l+s}{\PYGZdq{}}\PYG{l+s}{dc=percona,dc=com??sub?(\PYGZam{}(objectClass=groupOfNames)(member=\PYGZob{}USER\PYGZcb{}))}\PYG{l+s}{\PYGZdq{}}
\PYG{+w}{    }\PYG{n+nt}{userToDNMapping}\PYG{p}{:}\PYG{+w}{ }\PYG{p+pIndicator}{\PYGZgt{}}\PYG{p+pIndicator}{\PYGZhy{}}
\PYG{+w}{          }\PYG{n+no}{[}
\PYG{+w}{            }\PYG{n+no}{\PYGZob{}}
\PYG{+w}{              }\PYG{n+no}{match: \PYGZdq{}([\PYGZca{}@]+)@percona\PYGZbs{}\PYGZbs{}.com\PYGZdq{},}
\PYG{+w}{              }\PYG{n+no}{ldapQuery: \PYGZdq{}dc=percona,dc=com??sub?(\PYGZam{}(objectClass=organizationalPerson)(cn=\PYGZob{}0\PYGZcb{}))\PYGZdq{}}
\PYG{+w}{            }\PYG{n+no}{\PYGZcb{}}
\PYG{+w}{          }\PYG{n+no}{]}

\PYG{n+nt}{setParameter}\PYG{p}{:}
\PYG{+w}{  }\PYG{n+nt}{authenticationMechanisms}\PYG{p}{:}\PYG{+w}{ }\PYG{l+s}{\PYGZdq{}}\PYG{l+s}{PLAIN}\PYG{l+s}{\PYGZdq{}}
\end{sphinxVerbatim}

\sphinxAtStartPar
The \{USER\} variable substitutes the username transformed during the userToDNMapping stage.

\sphinxAtStartPar
Modify the given example configuration to match your deployment, For example, replace \sphinxcode{\sphinxupquote{ldap.example.com}} with the hostname of your LDAP server. Replace the domain controllers (DC) \sphinxcode{\sphinxupquote{percona}} and \sphinxcode{\sphinxupquote{com}} with those relevant to your organization. Depending on your LDAP schema, further modifications of the LDAP query may be required.

\item {} 
\sphinxAtStartPar
Restart the \sphinxcode{\sphinxupquote{mongod}} service:

\begin{sphinxVerbatim}[commandchars=\\\{\}]
\PYGZdl{} sudo systemctl restart mongod
\end{sphinxVerbatim}

\item {} 
\sphinxAtStartPar
Test the access to \sphinxstyleemphasis{Percona Server for MongoDB}:

\begin{sphinxVerbatim}[commandchars=\\\{\}]
mongo \PYGZhy{}u \PYG{l+s+s2}{\PYGZdq{}alice\PYGZdq{}} \PYGZhy{}p \PYG{l+s+s2}{\PYGZdq{}secretpwd\PYGZdq{}} \PYGZhy{}\PYGZhy{}authenticationDatabase \PYG{l+s+s1}{\PYGZsq{}\PYGZdl{}external\PYGZsq{}} \PYGZhy{}\PYGZhy{}authenticationMechanism \PYG{l+s+s1}{\PYGZsq{}PLAIN\PYGZsq{}}
\end{sphinxVerbatim}

\end{enumerate}
\end{sphinxadmonition}
\end{quote}


\subsection{Active Directory configuration}
\label{\detokenize{ldap-setup:active-directory-configuration}}\label{\detokenize{ldap-setup:active-directory}}
\sphinxAtStartPar
Microsoft Active Directory uses a different schema for user and group definition. To illustrate \sphinxstyleemphasis{Percona Server for MongoDB} configuration, we will use the following \sphinxstyleabbreviation{AD} (Active Directory) users:

\begin{sphinxVerbatim}[commandchars=\\\{\}]
dn:CN=alice,CN=Users,DC=testusers,DC=percona,DC=com
userPrincipalName: alice@testusers.percona.com
memberOf: CN=testusers,CN=Users,DC=percona,DC=com

dn:CN=bob,CN=Users,DC=otherusers,DC=percona,DC=com
userPrincipalName: bob@otherusers.percona.com
memberOf: CN=otherusers,CN=Users,DC=percona,DC=com
\end{sphinxVerbatim}

\sphinxAtStartPar
The following are respective \sphinxstyleabbreviation{AD} groups:

\begin{sphinxVerbatim}[commandchars=\\\{\}]
dn:CN=testusers,CN=Users,DC=percona,DC=com
member:CN=alice,CN=Users,DC=testusers,DC=example,DC=com

dn:CN=otherusers,CN=Users,DC=percona,DC=com
member:CN=bob,CN=Users,DC=otherusers,DC=example,DC=com
\end{sphinxVerbatim}

\sphinxAtStartPar
Use one of the given \sphinxstyleemphasis{Percona Server for MongoDB} configurations for user authentication and authorization in Active Directory:
\begin{quote}

\begin{sphinxadmonition}{note}{No username transformation}
\begin{enumerate}
\sphinxsetlistlabels{\arabic}{enumi}{enumii}{}{.}%
\item {} 
\sphinxAtStartPar
Edit the \sphinxcode{\sphinxupquote{mongod}} configuration file:

\begin{sphinxVerbatim}[commandchars=\\\{\}]
\PYG{n+nt}{ldap}\PYG{p}{:}
\PYG{+w}{  }\PYG{n+nt}{servers}\PYG{p}{:}\PYG{+w}{ }\PYG{l+s}{\PYGZdq{}}\PYG{l+s}{ldap.example.com}\PYG{l+s}{\PYGZdq{}}
\PYG{+w}{  }\PYG{n+nt}{authz}\PYG{p}{:}
\PYG{+w}{    }\PYG{n+nt}{queryTemplate}\PYG{p}{:}\PYG{+w}{ }\PYG{l+s}{\PYGZdq{}}\PYG{l+s}{DC=percona,DC=com??sub?(\PYGZam{}(objectClass=group)(member:1.2.840.113556.1.4.1941:=\PYGZob{}PROVIDED\PYGZus{}USER\PYGZcb{}))}\PYG{l+s}{\PYGZdq{}}

\PYG{+w}{  }\PYG{n+nt}{setParameter}\PYG{p}{:}
\PYG{+w}{    }\PYG{n+nt}{authenticationMechanisms}\PYG{p}{:}\PYG{+w}{ }\PYG{l+s}{\PYGZdq{}}\PYG{l+s}{PLAIN}\PYG{l+s}{\PYGZdq{}}
\end{sphinxVerbatim}

\item {} 
\sphinxAtStartPar
Restart the \sphinxcode{\sphinxupquote{mongod}} service:

\begin{sphinxVerbatim}[commandchars=\\\{\}]
\PYGZdl{} sudo systemctl restart mongod
\end{sphinxVerbatim}

\item {} 
\sphinxAtStartPar
Test the access to \sphinxstyleemphasis{Percona Server for MongoDB}:

\begin{sphinxVerbatim}[commandchars=\\\{\}]
mongo \PYGZhy{}u \PYG{l+s+s2}{\PYGZdq{}CN=alice,CN=Users,DC=testusers,DC=percona,DC=com\PYGZdq{}} \PYGZhy{}p \PYG{l+s+s2}{\PYGZdq{}secretpwd\PYGZdq{}} \PYGZhy{}\PYGZhy{}authenticationDatabase \PYG{l+s+s1}{\PYGZsq{}\PYGZdl{}external\PYGZsq{}} \PYGZhy{}\PYGZhy{}authenticationMechanism \PYG{l+s+s1}{\PYGZsq{}PLAIN\PYGZsq{}}
\end{sphinxVerbatim}

\end{enumerate}
\end{sphinxadmonition}

\begin{sphinxadmonition}{note}{Username substitution}
\begin{enumerate}
\sphinxsetlistlabels{\arabic}{enumi}{enumii}{}{.}%
\item {} 
\sphinxAtStartPar
Edit the \sphinxcode{\sphinxupquote{mongod}} configuration file:

\begin{sphinxVerbatim}[commandchars=\\\{\}]
\PYG{n+nt}{ldap}\PYG{p}{:}
\PYG{+w}{  }\PYG{n+nt}{servers}\PYG{p}{:}\PYG{+w}{ }\PYG{l+s}{\PYGZdq{}}\PYG{l+s}{ldap.example.com}\PYG{l+s}{\PYGZdq{}}
\PYG{+w}{  }\PYG{n+nt}{authz}\PYG{p}{:}
\PYG{+w}{    }\PYG{n+nt}{queryTemplate}\PYG{p}{:}\PYG{+w}{ }\PYG{l+s}{\PYGZdq{}}\PYG{l+s}{DC=percona,DC=com??sub?(\PYGZam{}(objectClass=group)(member:1.2.840.113556.1.4.1941:=\PYGZob{}USER\PYGZcb{}))}\PYG{l+s}{\PYGZdq{}}
\PYG{+w}{  }\PYG{n+nt}{userToDNMapping}\PYG{p}{:}\PYG{+w}{ }\PYG{p+pIndicator}{\PYGZgt{}}\PYG{p+pIndicator}{\PYGZhy{}}
\PYG{+w}{        }\PYG{n+no}{[}
\PYG{+w}{          }\PYG{n+no}{\PYGZob{}}
\PYG{+w}{            }\PYG{n+no}{match: \PYGZdq{}([\PYGZca{}@]+)@([\PYGZca{}\PYGZbs{}\PYGZbs{}.]+)\PYGZbs{}\PYGZbs{}.percona\PYGZbs{}\PYGZbs{}.com\PYGZdq{},}
\PYG{+w}{            }\PYG{n+no}{substitution: \PYGZdq{}CN=\PYGZob{}0\PYGZcb{},CN=Users,DC=\PYGZob{}1\PYGZcb{},DC=percona,DC=com\PYGZdq{}}
\PYG{+w}{          }\PYG{n+no}{\PYGZcb{}}
\PYG{+w}{        }\PYG{n+no}{]}

\PYG{+w}{  }\PYG{n+nt}{setParameter}\PYG{p}{:}
\PYG{+w}{    }\PYG{n+nt}{authenticationMechanisms}\PYG{p}{:}\PYG{+w}{ }\PYG{l+s}{\PYGZdq{}}\PYG{l+s}{PLAIN}\PYG{l+s}{\PYGZdq{}}
\end{sphinxVerbatim}

\item {} 
\sphinxAtStartPar
Restart the \sphinxcode{\sphinxupquote{mongod}} service:

\begin{sphinxVerbatim}[commandchars=\\\{\}]
\PYGZdl{} sudo systemctl restart mongod
\end{sphinxVerbatim}

\item {} 
\sphinxAtStartPar
Test the access to \sphinxstyleemphasis{Percona Server for MongoDB}:

\begin{sphinxVerbatim}[commandchars=\\\{\}]
mongo \PYGZhy{}u \PYG{l+s+s2}{\PYGZdq{}alice@percona.com\PYGZdq{}} \PYGZhy{}p \PYG{l+s+s2}{\PYGZdq{}secretpwd\PYGZdq{}} \PYGZhy{}\PYGZhy{}authenticationDatabase \PYG{l+s+s1}{\PYGZsq{}\PYGZdl{}external\PYGZsq{}} \PYGZhy{}\PYGZhy{}authenticationMechanism \PYG{l+s+s1}{\PYGZsq{}PLAIN\PYGZsq{}}
\end{sphinxVerbatim}

\end{enumerate}
\end{sphinxadmonition}

\begin{sphinxadmonition}{note}{LDAP query}
\begin{enumerate}
\sphinxsetlistlabels{\arabic}{enumi}{enumii}{}{.}%
\item {} 
\sphinxAtStartPar
Edit the \sphinxcode{\sphinxupquote{mongod}} configuration file:

\begin{sphinxVerbatim}[commandchars=\\\{\}]
\PYG{n+nt}{ldap}\PYG{p}{:}
\PYG{+w}{  }\PYG{n+nt}{servers}\PYG{p}{:}\PYG{+w}{ }\PYG{l+s}{\PYGZdq{}}\PYG{l+s}{ldap.example.com}\PYG{l+s}{\PYGZdq{}}
\PYG{+w}{  }\PYG{n+nt}{authz}\PYG{p}{:}
\PYG{+w}{    }\PYG{n+nt}{queryTemplate}\PYG{p}{:}\PYG{+w}{ }\PYG{l+s}{\PYGZdq{}}\PYG{l+s}{DC=percona,DC=com??sub?(\PYGZam{}(objectClass=group)(member:1.2.840.113556.1.4.1941:=\PYGZob{}USER\PYGZcb{}))}\PYG{l+s}{\PYGZdq{}}
\PYG{+w}{  }\PYG{n+nt}{userToDNMapping}\PYG{p}{:}\PYG{+w}{ }\PYG{p+pIndicator}{\PYGZgt{}}\PYG{p+pIndicator}{\PYGZhy{}}
\PYG{+w}{        }\PYG{n+no}{[}
\PYG{+w}{          }\PYG{n+no}{\PYGZob{}}
\PYG{+w}{            }\PYG{n+no}{match: \PYGZdq{}(.+)\PYGZdq{},}
\PYG{+w}{            }\PYG{n+no}{ldapQuery: \PYGZdq{}dc=example,dc=com??sub?(\PYGZam{}(objectClass=organizationalPerson)(userPrincipalName=\PYGZob{}0\PYGZcb{}))\PYGZdq{}}
\PYG{+w}{          }\PYG{n+no}{\PYGZcb{}}
\PYG{+w}{        }\PYG{n+no}{]}

\PYG{+w}{  }\PYG{n+nt}{setParameter}\PYG{p}{:}
\PYG{+w}{    }\PYG{n+nt}{authenticationMechanisms}\PYG{p}{:}\PYG{+w}{ }\PYG{l+s}{\PYGZdq{}}\PYG{l+s}{PLAIN}\PYG{l+s}{\PYGZdq{}}
\end{sphinxVerbatim}

\item {} 
\sphinxAtStartPar
Restart the \sphinxcode{\sphinxupquote{mongod}} service:

\begin{sphinxVerbatim}[commandchars=\\\{\}]
\PYGZdl{} sudo systemctl restart mongod
\end{sphinxVerbatim}

\item {} 
\sphinxAtStartPar
Test the access to \sphinxstyleemphasis{Percona Server for MongoDB}:

\begin{sphinxVerbatim}[commandchars=\\\{\}]
mongo \PYGZhy{}u \PYG{l+s+s2}{\PYGZdq{}alice\PYGZdq{}} \PYGZhy{}p \PYG{l+s+s2}{\PYGZdq{}secretpwd\PYGZdq{}} \PYGZhy{}\PYGZhy{}authenticationDatabase \PYG{l+s+s1}{\PYGZsq{}\PYGZdl{}external\PYGZsq{}} \PYGZhy{}\PYGZhy{}authenticationMechanism \PYG{l+s+s1}{\PYGZsq{}PLAIN\PYGZsq{}}
\end{sphinxVerbatim}

\end{enumerate}
\end{sphinxadmonition}
\end{quote}

\sphinxAtStartPar
Modify one of this example configuration to match your deployment.

\begin{sphinxadmonition}{note}{Based on the material from \sphinxstylestrong{Percona Database Performance Blog}}

\sphinxAtStartPar
This document is based on the following blog posts:
\begin{itemize}
\item {} 
\sphinxAtStartPar
\sphinxhref{https://www.percona.com/blog/2020/04/24/percona-server-for-mongodb-ldap-enhancements-user-to-dn-mapping/}{Percona Server for MongoDB LDAP Enhancements: User\sphinxhyphen{}to\sphinxhyphen{}DN Mapping} by Igor Solodovnikov

\item {} 
\sphinxAtStartPar
\sphinxhref{https://www.percona.com/blog/2021/07/08/authenticate-percona-server-for-mongodb-users-via-native-ldap/}{Authenticate Percona Server for MongoDB Users via Native LDAP} by Ivan Groenewold

\end{itemize}
\end{sphinxadmonition}


\chapter{Set up x.509 authentication and LDAP authorization}
\label{\detokenize{x509-ldap:set-up-x-509-authentication-and-ldap-authorization}}\label{\detokenize{x509-ldap:ldap-x509}}\label{\detokenize{x509-ldap::doc}}
\sphinxAtStartPar
{\hyperref[\detokenize{authentication:x509}]{\sphinxcrossref{\DUrole{std,std-ref}{x.509 certificate authentication}}}} is one of the supported authentication mechanisms in \sphinxstyleemphasis{Percona Server for MongoDB}. It is compatible with {\hyperref[\detokenize{authorization:ldap-authorization}]{\sphinxcrossref{\DUrole{std,std-ref}{LDAP authorization}}}} to enable you to control user access and operations in your database environment.

\sphinxAtStartPar
This document provides the steps on how to configure and use x.509 certificates for authentication in \sphinxstyleemphasis{Percona Server for MongoDB} and authorize users in the LDAP server.


\section{Considerations}
\label{\detokenize{x509-ldap:considerations}}\begin{enumerate}
\sphinxsetlistlabels{\arabic}{enumi}{enumii}{}{.}%
\item {} 
\sphinxAtStartPar
For testing purposes, in this tutorial we use \sphinxhref{https://www.openssl.org/}{OpenSSL} to issue self\sphinxhyphen{}signed certificates. For production use, we recommend using certificates issued and signed by the \sphinxstyleabbreviation{CA} in \sphinxstyleemphasis{Percona Server for MongoDB}. Client certificates must meet the \sphinxhref{https://docs.mongodb.com/manual/core/security-x.509/\#client-certificate-requirements}{client certificate requirements}.

\item {} 
\sphinxAtStartPar
The setup of the LDAP server and the configuration of the LDAP schema is out of scope of this document. We assume that you have the LDAP server up and running and accessible to \sphinxstyleemphasis{Percona Server for MongoDB}.

\end{enumerate}


\section{Setup procedure}
\label{\detokenize{x509-ldap:setup-procedure}}

\subsection{Issue certificates}
\label{\detokenize{x509-ldap:issue-certificates}}\begin{enumerate}
\sphinxsetlistlabels{\arabic}{enumi}{enumii}{}{.}%
\item {} 
\sphinxAtStartPar
Create a directory to store the certificates. For example, \sphinxcode{\sphinxupquote{/var/lib/mongocerts}}.

\begin{sphinxVerbatim}[commandchars=\\\{\}]
\PYGZdl{} sudo mkdir \PYGZhy{}p /var/lib/mongocerts
\end{sphinxVerbatim}

\item {} 
\sphinxAtStartPar
Grant access to the \sphinxcode{\sphinxupquote{mongod}} user to this directory:

\begin{sphinxVerbatim}[commandchars=\\\{\}]
\PYGZdl{} sudo chown mongod:mongod /var/lib/mongocerts
\end{sphinxVerbatim}

\end{enumerate}


\subsubsection{Generate the root Certificate Authority certificate}
\label{\detokenize{x509-ldap:generate-the-root-certificate-authority-certificate}}
\sphinxAtStartPar
The root Certificate Authority certificate will be used to sign the SSL certificates.

\sphinxAtStartPar
Run the following command and in the \sphinxcode{\sphinxupquote{\sphinxhyphen{}subj}} flag, provide the details about your organization:
\begin{itemize}
\item {} 
\sphinxAtStartPar
C \sphinxhyphen{} Country Name (2 letter code);

\item {} 
\sphinxAtStartPar
ST \sphinxhyphen{} State or Province Name (full name);

\item {} 
\sphinxAtStartPar
L \sphinxhyphen{} Locality Name (city);

\item {} 
\sphinxAtStartPar
O \sphinxhyphen{} Organization Name (company);

\item {} 
\sphinxAtStartPar
CN \sphinxhyphen{} Common Name (your name or your server’s hostname) .

\end{itemize}

\begin{sphinxVerbatim}[commandchars=\\\{\}]
\PYGZdl{} \PYG{n+nb}{cd} /var/lib/mongocerts
\PYGZdl{} sudo openssl req \PYGZhy{}nodes \PYGZhy{}x509 \PYGZhy{}newkey rsa:4096 \PYGZhy{}keyout ca.key \PYGZhy{}out ca.crt \PYGZhy{}subj \PYG{l+s+s2}{\PYGZdq{}/C=US/ST=California/L=SanFrancisco/O=Percona/OU=root/CN=localhost\PYGZdq{}}
\end{sphinxVerbatim}


\subsubsection{Generate server certificate}
\label{\detokenize{x509-ldap:generate-server-certificate}}\begin{enumerate}
\sphinxsetlistlabels{\arabic}{enumi}{enumii}{}{.}%
\item {} 
\sphinxAtStartPar
Create the server certificate request and key. In the \sphinxcode{\sphinxupquote{\sphinxhyphen{}subj}} flag, provide the details about your organization:
\begin{itemize}
\item {} 
\sphinxAtStartPar
C \sphinxhyphen{} Country Name (2 letter code);

\item {} 
\sphinxAtStartPar
ST \sphinxhyphen{} State or Province Name (full name);

\item {} 
\sphinxAtStartPar
L \sphinxhyphen{} Locality Name (city);

\item {} 
\sphinxAtStartPar
O \sphinxhyphen{} Organization Name (company);

\item {} 
\sphinxAtStartPar
CN \sphinxhyphen{} Common Name (your name or your server’s hostname) .

\end{itemize}

\begin{sphinxVerbatim}[commandchars=\\\{\}]
\PYGZdl{} sudo openssl req \PYGZhy{}nodes \PYGZhy{}newkey rsa:4096 \PYGZhy{}keyout server.key \PYGZhy{}out server.csr \PYGZhy{}subj \PYG{l+s+s2}{\PYGZdq{}/C=US/ST=California/L=SanFrancisco/O=Percona/OU=server/CN=localhost\PYGZdq{}}
\end{sphinxVerbatim}

\item {} 
\sphinxAtStartPar
Sign the server certificate request with the root CA certificate:

\begin{sphinxVerbatim}[commandchars=\\\{\}]
\PYGZdl{} sudo openssl x509 \PYGZhy{}req \PYGZhy{}in server.csr \PYGZhy{}CA ca.crt \PYGZhy{}CAkey ca.key \PYGZhy{}set\PYGZus{}serial \PYG{l+m}{01} \PYGZhy{}out server.crt
\end{sphinxVerbatim}

\item {} 
\sphinxAtStartPar
Combine the server certificate and key to create a certificate key file. Run this command as the \sphinxcode{\sphinxupquote{root}} user:

\begin{sphinxVerbatim}[commandchars=\\\{\}]
\PYGZdl{} cat server.key server.crt \PYGZgt{} server.pem
\end{sphinxVerbatim}

\end{enumerate}


\subsubsection{Generate client certificates}
\label{\detokenize{x509-ldap:generate-client-certificates}}\begin{enumerate}
\sphinxsetlistlabels{\arabic}{enumi}{enumii}{}{.}%
\item {} 
\sphinxAtStartPar
Generate client certificate request and key. In the \sphinxcode{\sphinxupquote{\sphinxhyphen{}subj}} flag, specify the information about clients in the \sphinxstyleabbreviation{DN} format.

\begin{sphinxVerbatim}[commandchars=\\\{\}]
\PYGZdl{} openssl req \PYGZhy{}nodes \PYGZhy{}newkey rsa:4096 \PYGZhy{}keyout client.key \PYGZhy{}out client.csr \PYGZhy{}subj \PYG{l+s+s2}{\PYGZdq{}/DC=com/DC=percona/CN=John Doe\PYGZdq{}}
\end{sphinxVerbatim}

\item {} 
\sphinxAtStartPar
Sign the client certificate request with the root CA certificate.

\begin{sphinxVerbatim}[commandchars=\\\{\}]
\PYGZdl{} openssl x509 \PYGZhy{}req \PYGZhy{}in client.csr \PYGZhy{}CA ca.crt \PYGZhy{}CAkey ca.key \PYGZhy{}set\PYGZus{}serial \PYG{l+m}{02} \PYGZhy{}out client.crt
\end{sphinxVerbatim}

\item {} 
\sphinxAtStartPar
Combine the client certificate and key to create a certificate key file.

\begin{sphinxVerbatim}[commandchars=\\\{\}]
\PYGZdl{} cat client.key client.crt \PYGZgt{} client.pem
\end{sphinxVerbatim}

\end{enumerate}


\subsection{Set up the LDAP server}
\label{\detokenize{x509-ldap:set-up-the-ldap-server}}
\sphinxAtStartPar
The setup of the LDAP server is out of scope of this document. Please work with your LDAP administrators to set up the LDAP server and configure the LDAP schema.


\subsection{Configure \sphinxstyleliteralintitle{\sphinxupquote{mongod}} server}
\label{\detokenize{x509-ldap:configure-mongod-server}}
\sphinxAtStartPar
The configuration consists of the following steps:
\begin{itemize}
\item {} 
\sphinxAtStartPar
Creating a role that matches the user group on the LDAP server

\item {} 
\sphinxAtStartPar
Editing the configuration file to enable the x.509 authentication

\end{itemize}

\begin{sphinxadmonition}{note}{Note:}
\sphinxAtStartPar
When you use x.509 authentication with LDAP authorization, you don’t need to create users in the \sphinxcode{\sphinxupquote{\$external}} database.  User management is done on the LDAP server so when a client connects to the database, they are authenticated and authorized through the LDAP server.
\end{sphinxadmonition}


\subsubsection{Create roles}
\label{\detokenize{x509-ldap:create-roles}}
\sphinxAtStartPar
At this step, create the roles in the \sphinxcode{\sphinxupquote{admin}} database with the names that exactly match the names of the user groups on the LDAP server. These roles are used for user {\hyperref[\detokenize{authorization:ldap-authorization}]{\sphinxcrossref{\DUrole{std,std-ref}{LDAP authorization}}}} in \sphinxstyleemphasis{Percona Server for MongoDB}.

\sphinxAtStartPar
In our example, we create the role \sphinxtitleref{cn=otherusers,dc=percona,dc=com} that has the corresponding LDAP group.

\begin{sphinxVerbatim}[commandchars=\\\{\}]
\PYG{k+kd}{var}\PYG{+w}{ }\PYG{n+nx}{admin}\PYG{+w}{ }\PYG{o}{=}\PYG{+w}{ }\PYG{n+nx}{db}\PYG{p}{.}\PYG{n+nx}{getSiblingDB}\PYG{p}{(}\PYG{l+s+s2}{\PYGZdq{}admin\PYGZdq{}}\PYG{p}{)}
\PYG{n+nx}{db}\PYG{p}{.}\PYG{n+nx}{createRole}\PYG{p}{(}
\PYG{+w}{   }\PYG{p}{\PYGZob{}}
\PYG{+w}{     }\PYG{n+nx}{role}\PYG{o}{:}\PYG{+w}{ }\PYG{l+s+s2}{\PYGZdq{}cn=otherusers,dc=percona,dc=com\PYGZdq{}}\PYG{p}{,}
\PYG{+w}{     }\PYG{n+nx}{privileges}\PYG{o}{:}\PYG{+w}{ }\PYG{p}{[}\PYG{p}{]}\PYG{p}{,}
\PYG{+w}{     }\PYG{n+nx}{roles}\PYG{o}{:}\PYG{+w}{ }\PYG{p}{[}
\PYG{+w}{           }\PYG{l+s+s2}{\PYGZdq{}userAdminAnyDatabase\PYGZdq{}}\PYG{p}{,}
\PYG{+w}{           }\PYG{l+s+s2}{\PYGZdq{}clusterMonitor\PYGZdq{}}\PYG{p}{,}
\PYG{+w}{           }\PYG{l+s+s2}{\PYGZdq{}clusterManager\PYGZdq{}}\PYG{p}{,}
\PYG{+w}{           }\PYG{l+s+s2}{\PYGZdq{}clusterAdmin\PYGZdq{}}
\PYG{+w}{            }\PYG{p}{]}
\PYG{+w}{   }\PYG{p}{\PYGZcb{}}
\PYG{p}{)}
\end{sphinxVerbatim}

\sphinxAtStartPar
Output:

\begin{sphinxVerbatim}[commandchars=\\\{\}]
\PYG{p}{\PYGZob{}}
\PYG{+w}{     }\PYG{n+nt}{\PYGZdq{}role\PYGZdq{}}\PYG{+w}{ }\PYG{p}{:}\PYG{+w}{ }\PYG{l+s+s2}{\PYGZdq{}cn=otherusers,dc=percona,dc=com\PYGZdq{}}\PYG{p}{,}
\PYG{+w}{     }\PYG{n+nt}{\PYGZdq{}privileges\PYGZdq{}}\PYG{+w}{ }\PYG{p}{:}\PYG{+w}{ }\PYG{p}{[}\PYG{+w}{ }\PYG{p}{],}
\PYG{+w}{     }\PYG{n+nt}{\PYGZdq{}roles\PYGZdq{}}\PYG{+w}{ }\PYG{p}{:}\PYG{+w}{ }\PYG{p}{[}
\PYG{+w}{             }\PYG{l+s+s2}{\PYGZdq{}userAdminAnyDatabase\PYGZdq{}}\PYG{p}{,}
\PYG{+w}{             }\PYG{l+s+s2}{\PYGZdq{}clusterMonitor\PYGZdq{}}\PYG{p}{,}
\PYG{+w}{             }\PYG{l+s+s2}{\PYGZdq{}clusterManager\PYGZdq{}}\PYG{p}{,}
\PYG{+w}{             }\PYG{l+s+s2}{\PYGZdq{}clusterAdmin\PYGZdq{}}
\PYG{+w}{     }\PYG{p}{]}
\PYG{p}{\PYGZcb{}}
\end{sphinxVerbatim}


\subsubsection{Enable x.509 authentication}
\label{\detokenize{x509-ldap:enable-x-509-authentication}}\begin{enumerate}
\sphinxsetlistlabels{\arabic}{enumi}{enumii}{}{.}%
\item {} 
\sphinxAtStartPar
Stop the \sphinxcode{\sphinxupquote{mongod}} service

\begin{sphinxVerbatim}[commandchars=\\\{\}]
\PYGZdl{} sudo systemctl stop mongod
\end{sphinxVerbatim}

\item {} 
\sphinxAtStartPar
Edit the \sphinxcode{\sphinxupquote{/etc/mongod.conf}} configuration file.

\begin{sphinxVerbatim}[commandchars=\\\{\}]
\PYG{n+nt}{net}\PYG{p}{:}
\PYG{+w}{  }\PYG{n+nt}{port}\PYG{p}{:}\PYG{+w}{ }\PYG{l+lScalar+lScalarPlain}{27017}
\PYG{+w}{  }\PYG{n+nt}{bindIp}\PYG{p}{:}\PYG{+w}{ }\PYG{l+lScalar+lScalarPlain}{127.0.0.1}
\PYG{+w}{  }\PYG{n+nt}{tls}\PYG{p}{:}
\PYG{+w}{    }\PYG{n+nt}{mode}\PYG{p}{:}\PYG{+w}{ }\PYG{l+lScalar+lScalarPlain}{requireTLS}
\PYG{+w}{    }\PYG{n+nt}{certificateKeyFile}\PYG{p}{:}\PYG{+w}{ }\PYG{l+lScalar+lScalarPlain}{/var/lib/mongocerts/server.pem}
\PYG{+w}{    }\PYG{n+nt}{CAFile}\PYG{p}{:}\PYG{+w}{ }\PYG{l+lScalar+lScalarPlain}{/var/lib/mongocerts/ca.crt}

\PYG{n+nt}{security}\PYG{p}{:}
\PYG{+w}{  }\PYG{n+nt}{authorization}\PYG{p}{:}\PYG{+w}{ }\PYG{l+lScalar+lScalarPlain}{enabled}
\PYG{+w}{  }\PYG{n+nt}{ldap}\PYG{p}{:}
\PYG{+w}{    }\PYG{n+nt}{servers}\PYG{p}{:}\PYG{+w}{ }\PYG{l+s}{\PYGZdq{}}\PYG{l+s}{ldap.example.com}\PYG{l+s}{\PYGZdq{}}
\PYG{+w}{    }\PYG{n+nt}{transportSecurity}\PYG{p}{:}\PYG{+w}{ }\PYG{l+lScalar+lScalarPlain}{none}
\PYG{+w}{    }\PYG{n+nt}{authz}\PYG{p}{:}
\PYG{+w}{      }\PYG{n+nt}{queryTemplate}\PYG{p}{:}\PYG{+w}{ }\PYG{l+s}{\PYGZdq{}}\PYG{l+s}{dc=percona,dc=com??sub?(\PYGZam{}(objectClass=groupOfNames)(member=\PYGZob{}USER\PYGZcb{}))}\PYG{l+s}{\PYGZdq{}}

\PYG{n+nt}{setParameter}\PYG{p}{:}
\PYG{+w}{  }\PYG{n+nt}{authenticationMechanisms}\PYG{p}{:}\PYG{+w}{ }\PYG{l+lScalar+lScalarPlain}{PLAIN,MONGODB\PYGZhy{}X509}
\end{sphinxVerbatim}

\sphinxAtStartPar
Replace \sphinxcode{\sphinxupquote{ldap.example.com}} with the hostname of your LDAP server. In the LDAP query template, replace the domain controllers \sphinxcode{\sphinxupquote{percona}} and \sphinxcode{\sphinxupquote{com}} with those relevant to your organization.

\item {} 
\sphinxAtStartPar
Start the \sphinxcode{\sphinxupquote{mongod}} service

\begin{sphinxVerbatim}[commandchars=\\\{\}]
\PYGZdl{} sudo systemctl start mongod
\end{sphinxVerbatim}

\end{enumerate}


\subsection{Authenticate with the x.509 certificate}
\label{\detokenize{x509-ldap:authenticate-with-the-x-509-certificate}}
\sphinxAtStartPar
To test the authentication, connect to \sphinxstyleemphasis{Percona Server for MongoDB} using the following command:

\begin{sphinxVerbatim}[commandchars=\\\{\}]
\PYGZdl{} mongo \PYGZhy{}\PYGZhy{}host localhost \PYGZhy{}\PYGZhy{}tls \PYGZhy{}\PYGZhy{}tlsCAFile /var/lib/mongocerts/ca.crt \PYGZhy{}\PYGZhy{}tlsCertificateKeyFile \PYGZlt{}path\PYGZus{}to\PYGZus{}client\PYGZus{}certificate\PYGZgt{}/client.pem  \PYGZhy{}\PYGZhy{}authenticationMechanism MONGODB\PYGZhy{}X509 \PYGZhy{}\PYGZhy{}authenticationDatabase\PYG{o}{=}\PYG{l+s+s1}{\PYGZsq{}\PYGZdl{}external\PYGZsq{}}
\end{sphinxVerbatim}

\sphinxAtStartPar
The result should look like the following:

\begin{sphinxVerbatim}[commandchars=\\\{\}]
\PYG{o}{\PYGZgt{}}\PYG{+w}{ }\PYG{n+nx}{db}\PYG{p}{.}\PYG{n+nx}{runCommand}\PYG{p}{(}\PYG{p}{\PYGZob{}}\PYG{n+nx}{connectionStatus}\PYG{+w}{ }\PYG{o}{:}\PYG{+w}{ }\PYG{l+m+mf}{1}\PYG{p}{\PYGZcb{}}\PYG{p}{)}
\PYG{p}{\PYGZob{}}
\PYG{+w}{     }\PYG{l+s+s2}{\PYGZdq{}authInfo\PYGZdq{}}\PYG{+w}{ }\PYG{o}{:}\PYG{+w}{ }\PYG{p}{\PYGZob{}}
\PYG{+w}{             }\PYG{l+s+s2}{\PYGZdq{}authenticatedUsers\PYGZdq{}}\PYG{+w}{ }\PYG{o}{:}\PYG{+w}{ }\PYG{p}{[}
\PYG{+w}{                     }\PYG{p}{\PYGZob{}}
\PYG{+w}{                             }\PYG{l+s+s2}{\PYGZdq{}user\PYGZdq{}}\PYG{+w}{ }\PYG{o}{:}\PYG{+w}{ }\PYG{l+s+s2}{\PYGZdq{}CN=John Doe,DC=percona,DC=com\PYGZdq{}}\PYG{p}{,}
\PYG{+w}{                             }\PYG{l+s+s2}{\PYGZdq{}db\PYGZdq{}}\PYG{+w}{ }\PYG{o}{:}\PYG{+w}{ }\PYG{l+s+s2}{\PYGZdq{}\PYGZdl{}external\PYGZdq{}}
\PYG{+w}{                     }\PYG{p}{\PYGZcb{}}
\PYG{+w}{             }\PYG{p}{]}\PYG{p}{,}
\PYG{+w}{             }\PYG{l+s+s2}{\PYGZdq{}authenticatedUserRoles\PYGZdq{}}\PYG{+w}{ }\PYG{o}{:}\PYG{+w}{ }\PYG{p}{[}
\PYG{+w}{                     }\PYG{p}{\PYGZob{}}
\PYG{+w}{                             }\PYG{l+s+s2}{\PYGZdq{}role\PYGZdq{}}\PYG{+w}{ }\PYG{o}{:}\PYG{+w}{ }\PYG{l+s+s2}{\PYGZdq{}cn=otherreaders,dc=percona,dc=com\PYGZdq{}}\PYG{p}{,}
\PYG{+w}{                             }\PYG{l+s+s2}{\PYGZdq{}db\PYGZdq{}}\PYG{+w}{ }\PYG{o}{:}\PYG{+w}{ }\PYG{l+s+s2}{\PYGZdq{}admin\PYGZdq{}}
\PYG{+w}{                     }\PYG{p}{\PYGZcb{}}\PYG{p}{,}
\PYG{+w}{                     }\PYG{p}{\PYGZob{}}
\PYG{+w}{                             }\PYG{l+s+s2}{\PYGZdq{}role\PYGZdq{}}\PYG{+w}{ }\PYG{o}{:}\PYG{+w}{ }\PYG{l+s+s2}{\PYGZdq{}clusterAdmin\PYGZdq{}}\PYG{p}{,}
\PYG{+w}{                             }\PYG{l+s+s2}{\PYGZdq{}db\PYGZdq{}}\PYG{+w}{ }\PYG{o}{:}\PYG{+w}{ }\PYG{l+s+s2}{\PYGZdq{}admin\PYGZdq{}}
\PYG{+w}{                     }\PYG{p}{\PYGZcb{}}\PYG{p}{,}
\PYG{+w}{                     }\PYG{p}{\PYGZob{}}
\PYG{+w}{                             }\PYG{l+s+s2}{\PYGZdq{}role\PYGZdq{}}\PYG{+w}{ }\PYG{o}{:}\PYG{+w}{ }\PYG{l+s+s2}{\PYGZdq{}userAdminAnyDatabase\PYGZdq{}}\PYG{p}{,}
\PYG{+w}{                             }\PYG{l+s+s2}{\PYGZdq{}db\PYGZdq{}}\PYG{+w}{ }\PYG{o}{:}\PYG{+w}{ }\PYG{l+s+s2}{\PYGZdq{}admin\PYGZdq{}}
\PYG{+w}{                     }\PYG{p}{\PYGZcb{}}\PYG{p}{,}
\PYG{+w}{                     }\PYG{p}{\PYGZob{}}
\PYG{+w}{                             }\PYG{l+s+s2}{\PYGZdq{}role\PYGZdq{}}\PYG{+w}{ }\PYG{o}{:}\PYG{+w}{ }\PYG{l+s+s2}{\PYGZdq{}clusterManager\PYGZdq{}}\PYG{p}{,}
\PYG{+w}{                             }\PYG{l+s+s2}{\PYGZdq{}db\PYGZdq{}}\PYG{+w}{ }\PYG{o}{:}\PYG{+w}{ }\PYG{l+s+s2}{\PYGZdq{}admin\PYGZdq{}}
\PYG{+w}{                     }\PYG{p}{\PYGZcb{}}\PYG{p}{,}
\PYG{+w}{                     }\PYG{p}{\PYGZob{}}
\PYG{+w}{                             }\PYG{l+s+s2}{\PYGZdq{}role\PYGZdq{}}\PYG{+w}{ }\PYG{o}{:}\PYG{+w}{ }\PYG{l+s+s2}{\PYGZdq{}clusterMonitor\PYGZdq{}}\PYG{p}{,}
\PYG{+w}{                             }\PYG{l+s+s2}{\PYGZdq{}db\PYGZdq{}}\PYG{+w}{ }\PYG{o}{:}\PYG{+w}{ }\PYG{l+s+s2}{\PYGZdq{}admin\PYGZdq{}}
\PYG{+w}{                     }\PYG{p}{\PYGZcb{}}
\PYG{+w}{             }\PYG{p}{]}
\PYG{+w}{     }\PYG{p}{\PYGZcb{}}\PYG{p}{,}
\PYG{+w}{     }\PYG{l+s+s2}{\PYGZdq{}ok\PYGZdq{}}\PYG{+w}{ }\PYG{o}{:}\PYG{+w}{ }\PYG{l+m+mf}{1}
\PYG{p}{\PYGZcb{}}
\end{sphinxVerbatim}


\chapter{Setting up Kerberos authentication}
\label{\detokenize{kerberos:setting-up-kerberos-authentication}}\label{\detokenize{kerberos:kerberos}}\label{\detokenize{kerberos::doc}}
\sphinxAtStartPar
This document provides configuration steps for setting up {\hyperref[\detokenize{authentication:kerberos-authentication}]{\sphinxcrossref{\DUrole{std,std-ref}{Kerberos Authentication}}}} in \sphinxstyleemphasis{Percona Server for MongoDB}.


\section{Assumptions}
\label{\detokenize{kerberos:assumptions}}
\sphinxAtStartPar
The setup of the Kerberos server itself is out of scope of this document. Please refer to the \sphinxhref{https://web.mit.edu/kerberos/krb5-latest/doc/admin/install\_kdc.html}{Kerberos documentation} for the installation and configuration steps relevant to your operation system.

\sphinxAtStartPar
We assume that you have successfully completed the following steps:
\begin{itemize}
\item {} 
\sphinxAtStartPar
Installed and configured the Kerberos server

\item {} 
\sphinxAtStartPar
Added necessary \sphinxhref{https://web.mit.edu/kerberos/krb5-1.12/doc/admin/realm\_config.html}{realms}

\item {} 
\sphinxAtStartPar
Added service, admin and user \sphinxhref{https://web.mit.edu/kerberos/krb5-1.5/krb5-1.5.4/doc/krb5-user/What-is-a-Kerberos-Principal\_003f.html\#What-is-a-Kerberos-Principal\_003f}{principals}

\item {} 
\sphinxAtStartPar
Configured the \sphinxcode{\sphinxupquote{A}} and \sphinxcode{\sphinxupquote{PTR}} DNS records for every host running \sphinxtitleref{mongod} instance to resolve the hostnames onto Kerberos realm.

\end{itemize}


\sphinxstrong{See also:}
\nopagebreak


\sphinxAtStartPar
MongoDB Documentation: \sphinxhref{https://www.mongodb.com/docs/manual/core/kerberos/}{Kerberos Authentication}




\section{Add user principals to \sphinxstyleemphasis{Percona Server for MongoDB}}
\label{\detokenize{kerberos:add-user-principals-to-psmdb}}
\sphinxAtStartPar
To get authenticated, users must exist both in the Kerberos and \sphinxstyleemphasis{Percona Server for MongoDB} servers with exactly matching names.

\sphinxAtStartPar
After you defined the user principals in the Kerberos server, add them to the \sphinxcode{\sphinxupquote{\$external}} database in \sphinxstyleemphasis{Percona Server for MongoDB} and assigned required roles:

\begin{sphinxVerbatim}[commandchars=\\\{\}]
\PYG{n+nx}{use}\PYG{+w}{ }\PYG{n+nx}{\PYGZdl{}external}
\PYG{n+nx}{db}\PYG{p}{.}\PYG{n+nx}{createUser}\PYG{p}{(}\PYG{p}{\PYGZob{}}\PYG{n+nx}{user}\PYG{o}{:}\PYG{+w}{ }\PYG{l+s+s2}{\PYGZdq{}demo@PERCONATEST.COM\PYGZdq{}}\PYG{p}{,}\PYG{n+nx}{roles}\PYG{o}{:}\PYG{+w}{ }\PYG{p}{[}\PYG{p}{\PYGZob{}}\PYG{n+nx}{role}\PYG{o}{:}\PYG{+w}{ }\PYG{l+s+s2}{\PYGZdq{}read\PYGZdq{}}\PYG{p}{,}\PYG{+w}{ }\PYG{n+nx}{db}\PYG{o}{:}\PYG{+w}{ }\PYG{l+s+s2}{\PYGZdq{}admin\PYGZdq{}}\PYG{p}{\PYGZcb{}}\PYG{p}{]}\PYG{p}{\PYGZcb{}}\PYG{p}{)}
\end{sphinxVerbatim}

\sphinxAtStartPar
Replace \sphinxcode{\sphinxupquote{demo@PERCONATEST.COM}} with your username and Kerberos realm.


\section{Configure Kerberos keytab files}
\label{\detokenize{kerberos:configure-kerberos-keytab-files}}
\sphinxAtStartPar
A keytab file stores the authentication keys for a service principal representing a \sphinxcode{\sphinxupquote{mongod}} instance to access the Kerberos admin server.

\sphinxAtStartPar
After you have added the service principal to the Kerberos admin server, the entry for this principal is added to the \sphinxcode{\sphinxupquote{/etc/krb5.keytab}} keytab file.

\sphinxAtStartPar
The \sphinxcode{\sphinxupquote{mongod}} server must have access to the keytab file to authenticate. To keep the keytab file secure, restrict the access to it only for the user running the \sphinxcode{\sphinxupquote{mongod}} process.
\begin{enumerate}
\sphinxsetlistlabels{\arabic}{enumi}{enumii}{}{.}%
\item {} 
\sphinxAtStartPar
Stop the \sphinxcode{\sphinxupquote{mongod}} service

\begin{sphinxVerbatim}[commandchars=\\\{\}]
\PYGZdl{} sudo systemctl stop mongod
\end{sphinxVerbatim}

\item {} 
\sphinxAtStartPar
\sphinxhref{https://web.mit.edu/kerberos/krb5-1.5/krb5-1.5.4/doc/krb5-install/The-Keytab-File.html}{Generate the keytab file} or get a copy of it if you generated the keytab file on another host. Save the keyfile under a separate path (e.g. /etc/mongodb.keytab)

\begin{sphinxVerbatim}[commandchars=\\\{\}]
\PYGZdl{} cp /etc/krb5.keytab /etc/mongodb.keytab
\end{sphinxVerbatim}

\item {} 
\sphinxAtStartPar
Change the ownership to the keytab file

\begin{sphinxVerbatim}[commandchars=\\\{\}]
\PYGZdl{} sudo chown mongod:mongod /etc/mongodb.keytab
\end{sphinxVerbatim}

\item {} 
\sphinxAtStartPar
Set the \sphinxcode{\sphinxupquote{KRB5\_KTNAME}} variable in the environment file for the \sphinxcode{\sphinxupquote{mongod}} process.
\begin{quote}

\begin{sphinxadmonition}{note}{On Debian and Ubuntu}

\sphinxAtStartPar
Edit the environment file at the path \sphinxcode{\sphinxupquote{/etc/default/mongod}} and specify the \sphinxcode{\sphinxupquote{KRB5\_KTNAME}} variable:

\begin{sphinxVerbatim}[commandchars=\\\{\}]
KRB5\PYGZus{}KTNAME=/etc/mongodb.keytab
\end{sphinxVerbatim}

\sphinxAtStartPar
If you have a different path to the keytab file, specify it accordingly.
\end{sphinxadmonition}

\begin{sphinxadmonition}{note}{On RHEL and derivatives}

\sphinxAtStartPar
Edit the environment file at the path \sphinxcode{\sphinxupquote{/etc/sysconfig/mongod}} and specify the \sphinxcode{\sphinxupquote{KRB5\_KTNAME}} variable:

\begin{sphinxVerbatim}[commandchars=\\\{\}]
KRB5\PYGZus{}KTNAME=/etc/mongodb.keytab
\end{sphinxVerbatim}

\sphinxAtStartPar
If you have a different path to the keytab file, specify it accordingly.
\end{sphinxadmonition}
\end{quote}

\item {} 
\sphinxAtStartPar
Restart the \sphinxcode{\sphinxupquote{mongod}} service

\begin{sphinxVerbatim}[commandchars=\\\{\}]
\PYGZdl{} sudo systemctl start mongod
\end{sphinxVerbatim}

\end{enumerate}


\section{\sphinxstyleemphasis{Percona Server for MongoDB} configuration}
\label{\detokenize{kerberos:psmdb-configuration}}
\sphinxAtStartPar
Enable external authentication in \sphinxstyleemphasis{Percona Server for MongoDB} configuration. Edit the \sphinxcode{\sphinxupquote{etc/mongod.conf}} configuration file and specify the following configuration:

\begin{sphinxVerbatim}[commandchars=\\\{\}]
\PYG{n+nt}{security}\PYG{p}{:}
\PYG{+w}{  }\PYG{n+nt}{authorization}\PYG{p}{:}\PYG{+w}{ }\PYG{l+s}{\PYGZdq{}}\PYG{l+s}{enabled}\PYG{l+s}{\PYGZdq{}}

\PYG{n+nt}{setParameter}\PYG{p}{:}
\PYG{+w}{  }\PYG{n+nt}{authenticationMechanisms}\PYG{p}{:}\PYG{+w}{ }\PYG{l+lScalar+lScalarPlain}{GSSAPI}
\end{sphinxVerbatim}

\sphinxAtStartPar
Restart the \sphinxcode{\sphinxupquote{mongod}} service to apply the configuration:

\begin{sphinxVerbatim}[commandchars=\\\{\}]
\PYGZdl{} sudo systemctl start mongod
\end{sphinxVerbatim}


\section{Test the access to \sphinxstyleemphasis{Percona Server for MongoDB}}
\label{\detokenize{kerberos:test-the-access-to-psmdb}}\begin{enumerate}
\sphinxsetlistlabels{\arabic}{enumi}{enumii}{}{.}%
\item {} 
\sphinxAtStartPar
Obtain the Kerberos ticket for the user using the \sphinxcode{\sphinxupquote{kinit}} command and specify the user password:

\begin{sphinxVerbatim}[commandchars=\\\{\}]
\PYGZdl{} kinit demo
Password \PYG{k}{for} demo@PERCONATEST.COM:
\end{sphinxVerbatim}

\item {} 
\sphinxAtStartPar
Check the user ticket:

\begin{sphinxVerbatim}[commandchars=\\\{\}]
\PYGZdl{} klist \PYGZhy{}l
Principal name                 Cache name
\PYGZhy{}\PYGZhy{}\PYGZhy{}\PYGZhy{}\PYGZhy{}\PYGZhy{}\PYGZhy{}\PYGZhy{}\PYGZhy{}\PYGZhy{}\PYGZhy{}\PYGZhy{}\PYGZhy{}\PYGZhy{}                 \PYGZhy{}\PYGZhy{}\PYGZhy{}\PYGZhy{}\PYGZhy{}\PYGZhy{}\PYGZhy{}\PYGZhy{}\PYGZhy{}\PYGZhy{}
demo@PERCONATEST.COM           FILE:/tmp/\PYGZlt{}ticket\PYGZgt{}
\end{sphinxVerbatim}

\item {} 
\sphinxAtStartPar
Connect to \sphinxstyleemphasis{Percona Server for MongoDB}:

\begin{sphinxVerbatim}[commandchars=\\\{\}]
\PYGZdl{} mongo \PYGZhy{}\PYGZhy{}host \PYGZlt{}hostname\PYGZgt{} \PYGZhy{}\PYGZhy{}authenticationMechanism\PYG{o}{=}GSSAPI \PYGZhy{}\PYGZhy{}authenticationDatabase\PYG{o}{=}\PYG{l+s+s1}{\PYGZsq{}\PYGZdl{}external\PYGZsq{}} \PYGZhy{}\PYGZhy{}username demo@PERCONATEST.COM
\end{sphinxVerbatim}

\end{enumerate}

\sphinxAtStartPar
The result should look like the following:

\begin{sphinxVerbatim}[commandchars=\\\{\}]
\PYG{o}{\PYGZgt{}}\PYG{+w}{ }\PYG{n+nx}{db}\PYG{p}{.}\PYG{n+nx}{runCommand}\PYG{p}{(}\PYG{p}{\PYGZob{}}\PYG{n+nx}{connectionStatus}\PYG{+w}{ }\PYG{o}{:}\PYG{+w}{ }\PYG{l+m+mf}{1}\PYG{p}{\PYGZcb{}}\PYG{p}{)}
\PYG{p}{\PYGZob{}}
\PYG{+w}{     }\PYG{l+s+s2}{\PYGZdq{}authInfo\PYGZdq{}}\PYG{+w}{ }\PYG{o}{:}\PYG{+w}{ }\PYG{p}{\PYGZob{}}
\PYG{+w}{             }\PYG{l+s+s2}{\PYGZdq{}authenticatedUsers\PYGZdq{}}\PYG{+w}{ }\PYG{o}{:}\PYG{+w}{ }\PYG{p}{[}
\PYG{+w}{                     }\PYG{p}{\PYGZob{}}
\PYG{+w}{                             }\PYG{l+s+s2}{\PYGZdq{}user\PYGZdq{}}\PYG{+w}{ }\PYG{o}{:}\PYG{+w}{ }\PYG{l+s+s2}{\PYGZdq{}demo@PERCONATEST.COM\PYGZdq{}}\PYG{p}{,}
\PYG{+w}{                             }\PYG{l+s+s2}{\PYGZdq{}db\PYGZdq{}}\PYG{+w}{ }\PYG{o}{:}\PYG{+w}{ }\PYG{l+s+s2}{\PYGZdq{}\PYGZdl{}external\PYGZdq{}}
\PYG{+w}{                     }\PYG{p}{\PYGZcb{}}
\PYG{+w}{             }\PYG{p}{]}\PYG{p}{,}
\PYG{+w}{             }\PYG{l+s+s2}{\PYGZdq{}authenticatedUserRoles\PYGZdq{}}\PYG{+w}{ }\PYG{o}{:}\PYG{+w}{ }\PYG{p}{[}
\PYG{+w}{                     }\PYG{p}{\PYGZob{}}
\PYG{+w}{                             }\PYG{l+s+s2}{\PYGZdq{}role\PYGZdq{}}\PYG{+w}{ }\PYG{o}{:}\PYG{+w}{ }\PYG{l+s+s2}{\PYGZdq{}read\PYGZdq{}}\PYG{p}{,}
\PYG{+w}{                             }\PYG{l+s+s2}{\PYGZdq{}db\PYGZdq{}}\PYG{+w}{ }\PYG{o}{:}\PYG{+w}{ }\PYG{l+s+s2}{\PYGZdq{}admin\PYGZdq{}}
\PYG{+w}{                     }\PYG{p}{\PYGZcb{}}
\PYG{+w}{             }\PYG{p}{]}
\PYG{+w}{     }\PYG{p}{\PYGZcb{}}\PYG{p}{,}
\PYG{+w}{     }\PYG{l+s+s2}{\PYGZdq{}ok\PYGZdq{}}\PYG{+w}{ }\PYG{o}{:}\PYG{+w}{ }\PYG{l+m+mf}{1}
\PYG{p}{\PYGZcb{}}
\end{sphinxVerbatim}


\chapter{\sphinxstyleemphasis{Percona Server for MongoDB} Parameter Tuning Guide}
\label{\detokenize{set-parameter:psmdb-parameter-tuning-guide}}\label{\detokenize{set-parameter:setparameter}}\label{\detokenize{set-parameter::doc}}
\sphinxAtStartPar
\sphinxstyleemphasis{Percona Server for MongoDB} includes several parameters that can be changed
in one of the following ways:
\begin{itemize}
\item {} 
\sphinxAtStartPar
The \sphinxcode{\sphinxupquote{setParameter}} admonitions in the configuration file
for persistent changes in production:

\begin{sphinxVerbatim}[commandchars=\\\{\}]
\PYG{n}{setParameter}\PYG{p}{:}
  \PYG{n}{cursorTimeoutMillis}\PYG{p}{:} \PYG{o}{\PYGZlt{}}\PYG{n+nb}{int}\PYG{o}{\PYGZgt{}}
  \PYG{n}{failIndexKeyTooLong}\PYG{p}{:} \PYG{o}{\PYGZlt{}}\PYG{n}{boolean}\PYG{o}{\PYGZgt{}}
  \PYG{n}{internalQueryPlannerEnableIndexIntersection}\PYG{p}{:} \PYG{o}{\PYGZlt{}}\PYG{n}{boolean}\PYG{o}{\PYGZgt{}}
  \PYG{n}{ttlMonitorEnabled}\PYG{p}{:} \PYG{o}{\PYGZlt{}}\PYG{n}{boolean}\PYG{o}{\PYGZgt{}}
  \PYG{n}{ttlMonitorSleepSecs}\PYG{p}{:} \PYG{o}{\PYGZlt{}}\PYG{n+nb}{int}\PYG{o}{\PYGZgt{}}
\end{sphinxVerbatim}

\item {} 
\sphinxAtStartPar
The \sphinxcode{\sphinxupquote{\sphinxhyphen{}\sphinxhyphen{}setParameter}} option arguments when running the \sphinxcode{\sphinxupquote{mongod}} process
for development or testing purposes:

\begin{sphinxVerbatim}[commandchars=\\\{\}]
\PYGZdl{} mongod \PYG{l+s+se}{\PYGZbs{}}
  \PYGZhy{}\PYGZhy{}setParameter \PYG{n+nv}{cursorTimeoutMillis}\PYG{o}{=}\PYGZlt{}int\PYGZgt{} \PYG{l+s+se}{\PYGZbs{}}
  \PYGZhy{}\PYGZhy{}setParameter \PYG{n+nv}{failIndexKeyTooLong}\PYG{o}{=}\PYGZlt{}boolean\PYGZgt{} \PYG{l+s+se}{\PYGZbs{}}
  \PYGZhy{}\PYGZhy{}setParameter \PYG{n+nv}{internalQueryPlannerEnableIndexIntersection}\PYG{o}{=}\PYGZlt{}boolean\PYGZgt{} \PYG{l+s+se}{\PYGZbs{}}
  \PYGZhy{}\PYGZhy{}setParameter \PYG{n+nv}{ttlMonitorEnabled}\PYG{o}{=}\PYGZlt{}boolean\PYGZgt{} \PYG{l+s+se}{\PYGZbs{}}
  \PYGZhy{}\PYGZhy{}setParameter \PYG{n+nv}{ttlMonitorSleepSecs}\PYG{o}{=}\PYGZlt{}int\PYGZgt{}
\end{sphinxVerbatim}

\item {} 
\sphinxAtStartPar
The \sphinxcode{\sphinxupquote{setParameter}} command on the \sphinxcode{\sphinxupquote{admin}} database
to make changes at runtime:

\begin{sphinxVerbatim}[commandchars=\\\{\}]
\PYGZgt{} db = db.getSiblingDB(\PYGZsq{}admin\PYGZsq{})
\PYGZgt{} db.runCommand( \PYGZob{} setParameter: 1, cursorTimeoutMillis: \PYGZlt{}int\PYGZgt{} \PYGZcb{} )
\PYGZgt{} db.runCommand( \PYGZob{} setParameter: 1, failIndexKeyTooLong: \PYGZlt{}boolean\PYGZgt{} \PYGZcb{} )
\PYGZgt{} db.runCommand( \PYGZob{} setParameter: 1, internalQueryPlannerEnableIndexIntersection: \PYGZlt{}boolean\PYGZgt{} \PYGZcb{} )
\PYGZgt{} db.runCommand( \PYGZob{} setParameter: 1, ttlMonitorEnabled: \PYGZlt{}int\PYGZgt{} \PYGZcb{} )
\PYGZgt{} db.runCommand( \PYGZob{} setParameter: 1, ttlMonitorSleepSecs: \PYGZlt{}int\PYGZgt{} \PYGZcb{} )
\end{sphinxVerbatim}

\end{itemize}


\section{Parameters}
\label{\detokenize{set-parameter:parameters}}

\subsection{cursorTimeoutMillis}
\label{\detokenize{set-parameter:cursortimeoutmillis}}\begin{quote}\begin{description}
\item[{Value Type}] \leavevmode
\sphinxAtStartPar
\sphinxstyleemphasis{integer}

\item[{Default}] \leavevmode
\sphinxAtStartPar
\sphinxcode{\sphinxupquote{600000}} (ten minutes)

\end{description}\end{quote}

\sphinxAtStartPar
Sets the duration of time after which idle query cursors
are removed from memory.


\subsection{failIndexKeyTooLong}
\label{\detokenize{set-parameter:failindexkeytoolong}}\begin{quote}\begin{description}
\item[{Value Type}] \leavevmode
\sphinxAtStartPar
\sphinxstyleemphasis{boolean}

\item[{Default}] \leavevmode
\sphinxAtStartPar
\sphinxcode{\sphinxupquote{true}}

\end{description}\end{quote}

\sphinxAtStartPar
Versions of MongoDB prior to 2.6 would insert and update documents
even if an index key was too long.
The documents would not be included in the index.
Newer versions of MongoDB ignore documents with long index key.
By setting this value to \sphinxcode{\sphinxupquote{false}}, the old behavior is enabled.


\subsection{internalQueryPlannerEnableIndexIntersection}
\label{\detokenize{set-parameter:internalqueryplannerenableindexintersection}}\begin{quote}\begin{description}
\item[{Value Type}] \leavevmode
\sphinxAtStartPar
\sphinxstyleemphasis{boolean}

\item[{Default}] \leavevmode
\sphinxAtStartPar
\sphinxcode{\sphinxupquote{true}}

\end{description}\end{quote}

\sphinxAtStartPar
Due to changes introduced in MongoDB 2.6.4,
some queries that reference multiple indexed fields,
where one field matches no documents,
may choose a non\sphinxhyphen{}optimal single\sphinxhyphen{}index plan.
Setting this value to \sphinxcode{\sphinxupquote{false}} will enable the old behavior
and select the index intersection plan.


\subsection{ttlMonitorEnabled}
\label{\detokenize{set-parameter:ttlmonitorenabled}}\begin{quote}\begin{description}
\item[{Value Type}] \leavevmode
\sphinxAtStartPar
\sphinxstyleemphasis{boolean}

\item[{Default}] \leavevmode
\sphinxAtStartPar
\sphinxcode{\sphinxupquote{true}}

\end{description}\end{quote}

\sphinxAtStartPar
If this option is set to \sphinxcode{\sphinxupquote{false}},
the worker thread that monitors TTL Indexes and removes old documents
will be disabled.


\subsection{ttlMonitorSleepSecs}
\label{\detokenize{set-parameter:ttlmonitorsleepsecs}}\begin{quote}\begin{description}
\item[{Value Type}] \leavevmode
\sphinxAtStartPar
\sphinxstyleemphasis{integer}

\item[{Default}] \leavevmode
\sphinxAtStartPar
\sphinxcode{\sphinxupquote{60}} (one minute)

\end{description}\end{quote}

\sphinxAtStartPar
Defines the number of seconds to wait
between checking TTL Indexes for old documents and removing them.


\chapter{Upgrading \sphinxstyleemphasis{Percona Server for MongoDB}}
\label{\detokenize{install/upgrade-from-mongodb:upgrading-psmdb}}\label{\detokenize{install/upgrade-from-mongodb:upgrade-psmdb}}\label{\detokenize{install/upgrade-from-mongodb::doc}}
\sphinxAtStartPar
An in\sphinxhyphen{}place upgrade is done by keeping the existing data in the server. It involves changing out the MongoDB binaries. Generally speaking, the upgrade steps include:
\begin{enumerate}
\sphinxsetlistlabels{\arabic}{enumi}{enumii}{}{.}%
\item {} 
\sphinxAtStartPar
Stopping the \sphinxcode{\sphinxupquote{mongod}} service

\item {} 
\sphinxAtStartPar
Removing the old binaries

\item {} 
\sphinxAtStartPar
Installing the new server version binaries

\item {} 
\sphinxAtStartPar
Restarting the \sphinxcode{\sphinxupquote{mongod}} service with the same \sphinxcode{\sphinxupquote{dbpath}} data directory.

\end{enumerate}

\sphinxAtStartPar
An in\sphinxhyphen{}place upgrade is suitable for most environments except the ones that use ephemeral storage and/or host addresses.

\sphinxAtStartPar
This document provides upgrade instructions for the following use cases:
\begin{itemize}
\item {} 
\sphinxAtStartPar
{\hyperref[\detokenize{install/upgrade-from-mongodb:upgrade-from-mongodb}]{\sphinxcrossref{\DUrole{std,std-ref}{Upgrading from MongoDB 4.4 Community Edition}}}};

\item {} 
\sphinxAtStartPar
{\hyperref[\detokenize{install/upgrade-from-mongodb:minor-upgrade}]{\sphinxcrossref{\DUrole{std,std-ref}{Minor upgrade of Percona Server for MongoDB}}}}

\end{itemize}


\section{Upgrading from MongoDB 4.4 Community Edition}
\label{\detokenize{install/upgrade-from-mongodb:upgrading-from-mongodb-ce}}\label{\detokenize{install/upgrade-from-mongodb:upgrade-from-mongodb}}
\begin{sphinxadmonition}{note}{Note:}
\sphinxAtStartPar
MongoDB creates a user that belongs to two groups, which is a potential
security risk.  This is fixed in \sphinxstyleemphasis{Percona Server for MongoDB}: the user is included only in the
\sphinxcode{\sphinxupquote{mongod}} group.  To avoid problems with current MongoDB setups, existing
user group membership is not changed when you migrate to \sphinxstyleemphasis{Percona Server for MongoDB}.  Instead, a
new \sphinxcode{\sphinxupquote{mongod}} user is created during installation, and it belongs to the
\sphinxcode{\sphinxupquote{mongod}} group.
\end{sphinxadmonition}

\sphinxAtStartPar
This section describes an in\sphinxhyphen{}place upgrade of a \sphinxcode{\sphinxupquote{mongod}} instance. If you are using data at rest encryption, refer to the {\hyperref[\detokenize{install/upgrade-from-mongodb:upgrade-encryption}]{\sphinxcrossref{\DUrole{std,std-ref}{Upgrading to Percona Server for MongoDB with data at rest encryption enabled}}}} section.

\begin{sphinxShadowBox}
\begin{itemize}
\item {} 
\sphinxAtStartPar
\phantomsection\label{\detokenize{install/upgrade-from-mongodb:id1}}{\hyperref[\detokenize{install/upgrade-from-mongodb:prerequisites}]{\sphinxcrossref{Prerequisites}}}

\end{itemize}
\end{sphinxShadowBox}


\subsection{Prerequisites}
\label{\detokenize{install/upgrade-from-mongodb:prerequisites}}
\sphinxAtStartPar
Before you start the upgrade, update the MongoDB configuration file
(\sphinxcode{\sphinxupquote{/etc/mongod.conf}}) to contain the following settings.

\begin{sphinxVerbatim}[commandchars=\\\{\}]
\PYG{n+nt}{processManagement}\PYG{p}{:}
\PYG{+w}{   }\PYG{n+nt}{fork}\PYG{p}{:}\PYG{+w}{ }\PYG{l+lScalar+lScalarPlain}{true}
\PYG{+w}{   }\PYG{n+nt}{pidFilePath}\PYG{p}{:}\PYG{+w}{ }\PYG{l+lScalar+lScalarPlain}{/var/run/mongod.pid}
\end{sphinxVerbatim}

\sphinxAtStartPar
Troubleshooting tip: The \sphinxcode{\sphinxupquote{pidFilePath}} setting in \sphinxcode{\sphinxupquote{mongod.conf}} must  match the \sphinxcode{\sphinxupquote{PIDFile}} option in the \sphinxcode{\sphinxupquote{systemd mongod}} service unit. Otherwise, the service will kill the \sphinxcode{\sphinxupquote{mongod}} process after a timeout.

\begin{sphinxadmonition}{warning}{Warning:}
\sphinxAtStartPar
Before starting the upgrade, we recommend to perform a full
backup of your data.

\begin{sphinxadmonition}{note}{Upgrading on Debian or Ubuntu}
\begin{enumerate}
\sphinxsetlistlabels{\arabic}{enumi}{enumii}{}{.}%
\item {} 
\sphinxAtStartPar
Stop the \sphinxcode{\sphinxupquote{mongod}} service:

\begin{sphinxVerbatim}[commandchars=\\\{\}]
\PYGZdl{} sudo systemctl stop mongod
\end{sphinxVerbatim}

\item {} 
\sphinxAtStartPar
Check for installed packages:

\begin{sphinxVerbatim}[commandchars=\\\{\}]
\PYGZdl{} sudo dpkg \PYGZhy{}l \PYG{p}{|} grep mongod
\end{sphinxVerbatim}

\begin{sphinxadmonition}{note}{Output}

\begin{sphinxVerbatim}[commandchars=\\\{\}]
ii  mongodb\PYGZhy{}org            4.4.0    amd64      MongoDB document\PYGZhy{}oriented database system (metapackage)
ii  mongodb\PYGZhy{}org\PYGZhy{}mongos     4.4.0    amd64      MongoDB sharded cluster query router
ii  mongodb\PYGZhy{}org\PYGZhy{}server     4.4.0    amd64      MongoDB database server
ii  mongodb\PYGZhy{}org\PYGZhy{}shell      4.4.0    amd64      MongoDB shell client
ii  mongodb\PYGZhy{}org\PYGZhy{}tools      4.4.0    amd64      MongoDB tools
\end{sphinxVerbatim}
\end{sphinxadmonition}

\item {} 
\sphinxAtStartPar
Remove the installed packages:

\begin{sphinxVerbatim}[commandchars=\\\{\}]
\PYGZdl{} apt remove mongodb\PYGZhy{}org mongodb\PYGZhy{}org\PYGZhy{}mongos mongodb\PYGZhy{}org\PYGZhy{}server \PYG{l+s+se}{\PYGZbs{}}
\PYGZdl{} mongodb\PYGZhy{}org\PYGZhy{}shell mongodb\PYGZhy{}org\PYGZhy{}tools
\end{sphinxVerbatim}

\item {} 
\sphinxAtStartPar
Remove log files:

\begin{sphinxVerbatim}[commandchars=\\\{\}]
\PYGZdl{} sudo rm \PYGZhy{}r /var/log/mongodb
\end{sphinxVerbatim}

\item {} 
\sphinxAtStartPar
Install \sphinxstyleemphasis{Percona Server for MongoDB} {\hyperref[\detokenize{install/apt:apt}]{\sphinxcrossref{\DUrole{std,std-ref}{using apt}}}}.

\item {} 
\sphinxAtStartPar
Verify that the configuration file includes the correct options. For example, \sphinxstyleemphasis{Percona Server for MongoDB} stores data files in \sphinxcode{\sphinxupquote{/var/lib/mongodb}} by default. If you used another \sphinxcode{\sphinxupquote{dbPath}} data directory, edit the configuration file accordingly

\item {} 
\sphinxAtStartPar
Start the \sphinxcode{\sphinxupquote{mongod}} service:

\begin{sphinxVerbatim}[commandchars=\\\{\}]
\PYGZdl{} sudo systemctl mongod start
\end{sphinxVerbatim}

\end{enumerate}
\end{sphinxadmonition}

\begin{sphinxadmonition}{note}{Upgrading on Red Hat Enterprise Linux or CentOS}
\begin{enumerate}
\sphinxsetlistlabels{\arabic}{enumi}{enumii}{}{.}%
\item {} 
\sphinxAtStartPar
Stop the \sphinxcode{\sphinxupquote{mongod}} service:

\begin{sphinxVerbatim}[commandchars=\\\{\}]
\PYGZdl{} sudo systemctl stop mongod
\end{sphinxVerbatim}

\item {} 
\sphinxAtStartPar
Check for installed packages:

\begin{sphinxVerbatim}[commandchars=\\\{\}]
\PYGZdl{} sudo rpm \PYGZhy{}qa \PYG{p}{|} grep mongo
\end{sphinxVerbatim}

\begin{sphinxadmonition}{note}{Output}

\begin{sphinxVerbatim}[commandchars=\\\{\}]
mongodb\PYGZhy{}org\PYGZhy{}mongos\PYGZhy{}4.4.0\PYGZhy{}1.el6.x86\PYGZus{}64
mongodb\PYGZhy{}org\PYGZhy{}shell\PYGZhy{}4.4.0\PYGZhy{}1.el6.x86\PYGZus{}64
mongodb\PYGZhy{}org\PYGZhy{}server\PYGZhy{}4.4.0\PYGZhy{}1.el6.x86\PYGZus{}64
mongodb\PYGZhy{}org\PYGZhy{}tools\PYGZhy{}4.4.0\PYGZhy{}1.el6.x86\PYGZus{}64
mongodb\PYGZhy{}org\PYGZhy{}4.4.0\PYGZhy{}1.el6.x86\PYGZus{}64
\end{sphinxVerbatim}
\end{sphinxadmonition}

\item {} 
\sphinxAtStartPar
Remove the installed packages:

\begin{sphinxVerbatim}[commandchars=\\\{\}]
\PYGZdl{} yum remove \PYG{l+s+se}{\PYGZbs{}}
mongodb\PYGZhy{}org\PYGZhy{}mongos\PYGZhy{}4.4.0\PYGZhy{}1.el6.x86\PYGZus{}64 \PYG{l+s+se}{\PYGZbs{}}
mongodb\PYGZhy{}org\PYGZhy{}shell\PYGZhy{}4.4.0\PYGZhy{}1.el6.x86\PYGZus{}64 \PYG{l+s+se}{\PYGZbs{}}
mongodb\PYGZhy{}org\PYGZhy{}server\PYGZhy{}4.4.0\PYGZhy{}1.el6.x86\PYGZus{}64 \PYG{l+s+se}{\PYGZbs{}}
mongodb\PYGZhy{}org\PYGZhy{}tools\PYGZhy{}4.4.0\PYGZhy{}1.el6.x86\PYGZus{}64 \PYG{l+s+se}{\PYGZbs{}}
mongodb\PYGZhy{}org\PYGZhy{}4.4.0\PYGZhy{}1.el6.x86\PYGZus{}64
\end{sphinxVerbatim}

\item {} 
\sphinxAtStartPar
Remove log files:

\begin{sphinxVerbatim}[commandchars=\\\{\}]
\PYGZdl{} sudo rm \PYGZhy{}r /var/log/mongodb
\end{sphinxVerbatim}

\item {} 
\sphinxAtStartPar
Install Percona Server for MongoDB {\hyperref[\detokenize{install/yum:yum}]{\sphinxcrossref{\DUrole{std,std-ref}{using yum}}}}.

\end{enumerate}

\begin{sphinxadmonition}{note}{Note:}
\sphinxAtStartPar
When you remove old packages, your existing configuration file is saved as
\sphinxcode{\sphinxupquote{/etc/mongod.conf.rpmsave}}.  If you want to use this configuration with
the new version, replace the default \sphinxcode{\sphinxupquote{/etc/mongod.conf}} file.  For
example, existing data may not be compatible with the default WiredTiger
storage engine.
\end{sphinxadmonition}

\sphinxAtStartPar
\# Start the \sphinxcode{\sphinxupquote{mongod}} service:
\begin{quote}

\begin{sphinxVerbatim}[commandchars=\\\{\}]
\PYGZdl{} sudo systemctl start mongod
\end{sphinxVerbatim}
\end{quote}
\end{sphinxadmonition}
\end{sphinxadmonition}

\sphinxAtStartPar
To upgrade a replica set or a sharded cluster, use the {\hyperref[\detokenize{glossary:term-Rolling-restart}]{\sphinxtermref{\DUrole{xref,std,std-term}{rolling restart}}}} method. It allows you to perform the upgrade with minimum downtime. You upgrade the nodes one by one, while the whole cluster / replica set remains operational.


\sphinxstrong{See also:}
\nopagebreak

\begin{description}
\item[{MongoDB Documentation:}] \leavevmode\begin{itemize}
\item {} 
\sphinxAtStartPar
\sphinxhref{https://docs.mongodb.com/manual/release-notes/4.4-upgrade-replica-set/}{Upgrade a Replica Set}

\item {} 
\sphinxAtStartPar
\sphinxhref{https://docs.mongodb.com/manual/release-notes/4.4-upgrade-sharded-cluster/}{Upgrade a Sharded Cluster}

\end{itemize}

\end{description}




\section{Minor upgrade of \sphinxstyleemphasis{Percona Server for MongoDB}}
\label{\detokenize{install/upgrade-from-mongodb:minor-upgrade-of-psmdb}}\label{\detokenize{install/upgrade-from-mongodb:minor-upgrade}}
\sphinxAtStartPar
To upgrade \sphinxstyleemphasis{Percona Server for MongoDB} to the latest version, follow these steps:
\begin{enumerate}
\sphinxsetlistlabels{\arabic}{enumi}{enumii}{}{.}%
\item {} 
\sphinxAtStartPar
Stop the \sphinxtitleref{mongod} service:

\begin{sphinxVerbatim}[commandchars=\\\{\}]
\PYGZdl{} sudo systemctl stop mongod
\end{sphinxVerbatim}

\item {} 
\sphinxAtStartPar
Install the latest version packages. Use the command relevant to your operating system.
\begin{quote}

\begin{sphinxadmonition}{note}{On Debian and Ubuntu:}

\begin{sphinxVerbatim}[commandchars=\\\{\}]
\PYGZdl{} sudo apt install percona\PYGZhy{}server\PYGZhy{}mongodb
\end{sphinxVerbatim}
\end{sphinxadmonition}

\begin{sphinxadmonition}{note}{On Red Hat Enterprise Linux or CentOS:}

\begin{sphinxVerbatim}[commandchars=\\\{\}]
\PYGZdl{} sudo yum install percona\PYGZhy{}server\PYGZhy{}mongodb
\end{sphinxVerbatim}
\end{sphinxadmonition}
\end{quote}

\item {} 
\sphinxAtStartPar
Start the \sphinxtitleref{mongod} service:

\begin{sphinxVerbatim}[commandchars=\\\{\}]
\PYGZdl{} sudo systemctl start mongod
\end{sphinxVerbatim}

\end{enumerate}

\sphinxAtStartPar
To upgrade a replica set or a sharded cluster, use the {\hyperref[\detokenize{glossary:term-Rolling-restart}]{\sphinxtermref{\DUrole{xref,std,std-term}{rolling restart}}}} method. It allows you to perform the upgrade with minimum downtime. You upgrade the nodes one by one, while the whole cluster / replica set remains operational.


\section{Upgrading to \sphinxstyleemphasis{Percona Server for MongoDB} with data at rest encryption enabled}
\label{\detokenize{install/upgrade-from-mongodb:upgrading-to-psmdb-with-data-at-rest-encryption-enabled}}\label{\detokenize{install/upgrade-from-mongodb:upgrade-encryption}}
\sphinxAtStartPar
Steps to upgrade from MongoDB 4.4 Community Edition with data encryption enabled to \sphinxstyleemphasis{Percona Server for MongoDB} are different. \sphinxcode{\sphinxupquote{mongod}} requires an empty \sphinxcode{\sphinxupquote{dbPath}} data directory because it cannot encrypt data files in place. It must receive data from other replica set members during the initial sync. Please refer to the {\hyperref[\detokenize{inmemory:switch-storage-engines}]{\sphinxcrossref{\DUrole{std,std-ref}{Switching storage engines}}}} for more information on migration of encrypted data. \sphinxhref{https://www.percona.com/about-percona/contact\#us}{Contact us} for working at the detailed migration steps, if further assistance is needed.


\chapter{Upgrading from \sphinxstyleemphasis{Percona Server for MongoDB} 4.2 to 4.4}
\label{\detokenize{install/upgrade-from-42:upgrading-from-psmdb-prev-version-to-version}}\label{\detokenize{install/upgrade-from-42:upgrade-from-42}}\label{\detokenize{install/upgrade-from-42::doc}}
\sphinxAtStartPar
To upgrade \sphinxstyleemphasis{Percona Server for MongoDB} to version 4.4, you must be running version
4.2. Upgrades from earlier versions are not supported.

\sphinxAtStartPar
Before upgrading your production \sphinxstyleemphasis{Percona Server for MongoDB} deployments, test all your applications
in a testing environment to make sure they are compatible with the new version.
For more information, see \sphinxhref{https://docs.mongodb.com/manual/release-notes/4.4-compatibility/}{Compatibility Changes in MongoDB 4.4}

\sphinxAtStartPar
The general procedure for performing an in\sphinxhyphen{}place upgrade (where your existing
data and configuration files are preserved) includes the following steps:
\begin{enumerate}
\sphinxsetlistlabels{\arabic}{enumi}{enumii}{}{.}%
\item {} 
\sphinxAtStartPar
Stop the \sphinxcode{\sphinxupquote{mongod}} instance

\item {} 
\sphinxAtStartPar
Enable Percona repository for \sphinxstyleemphasis{Percona Server for MongoDB} 4.4

\item {} 
\sphinxAtStartPar
Install new packages. Old packages are considered obsolete and automatically removed

\item {} 
\sphinxAtStartPar
Start the \sphinxcode{\sphinxupquote{mongod}} instance

\end{enumerate}

\sphinxAtStartPar
It is recommended to upgrade \sphinxstyleemphasis{Percona Server for MongoDB} from official Percona repositories using
the corresponding package manager for your system.  For more information, see
{\hyperref[\detokenize{install/index:install}]{\sphinxcrossref{\DUrole{std,std-ref}{Installing Percona Server for MongoDB}}}}.

\begin{sphinxadmonition}{warning}{Warning:}
\sphinxAtStartPar
Perform a full backup of your data and configuration files before upgrading.

\begin{sphinxadmonition}{note}{Upgrading on Debian or Ubuntu}
\begin{enumerate}
\sphinxsetlistlabels{\arabic}{enumi}{enumii}{}{.}%
\item {} 
\sphinxAtStartPar
Stop the \sphinxcode{\sphinxupquote{mongod}} instance:

\begin{sphinxVerbatim}[commandchars=\\\{\}]
\PYGZdl{} sudo systemctl stop mongod
\end{sphinxVerbatim}

\item {} 
\sphinxAtStartPar
Enable Percona repository for \sphinxstyleemphasis{Percona Server for MongoDB} 4.4:

\begin{sphinxVerbatim}[commandchars=\\\{\}]
\PYGZdl{} sudo percona\PYGZhy{}release \PYG{n+nb}{enable} psmdb\PYGZhy{}44
\end{sphinxVerbatim}

\item {} 
\sphinxAtStartPar
Update the local cache:

\begin{sphinxVerbatim}[commandchars=\\\{\}]
\PYGZdl{} sudo apt update
\end{sphinxVerbatim}

\item {} 
\sphinxAtStartPar
Install \sphinxstyleemphasis{Percona Server for MongoDB} 4.4 packages:

\begin{sphinxVerbatim}[commandchars=\\\{\}]
\PYGZdl{} sudo apt install percona\PYGZhy{}server\PYGZhy{}mongodb
\end{sphinxVerbatim}

\item {} 
\sphinxAtStartPar
Start the \sphinxcode{\sphinxupquote{mongod}} instance:

\begin{sphinxVerbatim}[commandchars=\\\{\}]
\PYGZdl{} sudo systemctl start mongod
\end{sphinxVerbatim}

\end{enumerate}

\sphinxAtStartPar
For more information, see {\hyperref[\detokenize{install/apt:apt}]{\sphinxcrossref{\DUrole{std,std-ref}{Installing Percona Server for MongoDB on Debian and Ubuntu}}}}.
\end{sphinxadmonition}

\begin{sphinxadmonition}{note}{Upgrading on RHEL and CentOS}
\begin{enumerate}
\sphinxsetlistlabels{\arabic}{enumi}{enumii}{}{.}%
\item {} 
\sphinxAtStartPar
Stop the \sphinxcode{\sphinxupquote{mongod}} instance:

\begin{sphinxVerbatim}[commandchars=\\\{\}]
\PYGZdl{} sudo systemctl stop mongod
\end{sphinxVerbatim}

\item {} 
\sphinxAtStartPar
Enable Percona repository for \sphinxstyleemphasis{Percona Server for MongoDB} 4.4:

\begin{sphinxVerbatim}[commandchars=\\\{\}]
\PYGZdl{} sudo percona\PYGZhy{}release \PYG{n+nb}{enable} psmdb\PYGZhy{}44
\end{sphinxVerbatim}

\item {} 
\sphinxAtStartPar
Install \sphinxstyleemphasis{Percona Server for MongoDB} 4.4 packages:

\item {} 
\sphinxAtStartPar
Start the \sphinxcode{\sphinxupquote{mongod}} instance:

\begin{sphinxVerbatim}[commandchars=\\\{\}]
\PYGZdl{} sudo systemctl start mongod
\end{sphinxVerbatim}

\end{enumerate}

\sphinxAtStartPar
For more information, see {\hyperref[\detokenize{install/yum:yum}]{\sphinxcrossref{\DUrole{std,std-ref}{Installing Percona Server for MongoDB on Red Hat Enterprise Linux and CentOS}}}}.
\end{sphinxadmonition}
\end{sphinxadmonition}

\sphinxAtStartPar
After the upgrade, \sphinxstyleemphasis{Percona Server for MongoDB} is started with the feature set of 4.2 version. Assuming that your applications are compatible with the new version, enable 4.4 version features. Run the following command against the \sphinxcode{\sphinxupquote{admin}} database:

\begin{sphinxVerbatim}[commandchars=\\\{\}]
db.adminCommand( \PYGZob{} setFeatureCompatibilityVersion: \PYGZdq{}4.4\PYGZdq{} \PYGZcb{} )
\end{sphinxVerbatim}


\sphinxstrong{See also:}
\nopagebreak

\begin{description}
\item[{MongoDB Documentation:}] \leavevmode\begin{itemize}
\item {} 
\sphinxAtStartPar
\sphinxhref{https://docs.mongodb.com/manual/release-notes/4.4-upgrade-standalone/}{Upgrade a Standalone}

\item {} 
\sphinxAtStartPar
\sphinxhref{https://docs.mongodb.com/manual/release-notes/4.4-upgrade-replica-set/}{Upgrade a Replica Set}

\item {} 
\sphinxAtStartPar
\sphinxhref{https://docs.mongodb.com/manual/release-notes/4.4-upgrade-sharded-cluster/}{Upgrade a Sharded Cluster}

\end{itemize}

\end{description}




\chapter{Uninstalling \sphinxstyleemphasis{Percona Server for MongoDB}}
\label{\detokenize{install/uninstall:uninstalling-psmdb}}\label{\detokenize{install/uninstall:uninstall}}\label{\detokenize{install/uninstall::doc}}
\sphinxAtStartPar
To completely remove \sphinxstyleemphasis{Percona Server for MongoDB} you need to remove all the installed packages, data and configuration files. If you need the data, consider making a backup before uninstalling Percona Server for MongoDB.

\sphinxAtStartPar
Follow the instructions, relevant to your operating system:

\phantomsection\label{\detokenize{install/uninstall:apt-uninstall}}\begin{quote}

\begin{sphinxadmonition}{note}{Uninstall on Debian and Ubuntu}

\sphinxAtStartPar
You can remove \sphinxstyleemphasis{Percona Server for MongoDB} packages with one of the following commands:
\begin{itemize}
\item {} 
\sphinxAtStartPar
\sphinxstyleliteralstrong{\sphinxupquote{apt remove}} will only remove the packages and leave the configuration and data files.

\item {} 
\sphinxAtStartPar
\sphinxstyleliteralstrong{\sphinxupquote{apt purge}} will remove all the packages with configuration files and data.

\end{itemize}

\sphinxAtStartPar
Choose which command better suits you depending on your needs.
\begin{enumerate}
\sphinxsetlistlabels{\arabic}{enumi}{enumii}{}{.}%
\item {} 
\sphinxAtStartPar
Stop the \sphinxstyleliteralstrong{\sphinxupquote{mongod}} server:

\begin{sphinxVerbatim}[commandchars=\\\{\}]
\PYGZdl{} sudo systemctl stop mongod
\end{sphinxVerbatim}

\item {} 
\sphinxAtStartPar
Remove the packages. There are two options.
\begin{itemize}
\item {} 
\sphinxAtStartPar
To keep the configuration and data files, run:

\begin{sphinxVerbatim}[commandchars=\\\{\}]
\PYGZdl{} sudo apt remove percona\PYGZhy{}server\PYGZhy{}mongodb*
\end{sphinxVerbatim}

\item {} 
\sphinxAtStartPar
To delete both the configuration and data files and the packages, run:

\begin{sphinxVerbatim}[commandchars=\\\{\}]
\PYGZdl{} sudo apt purge percona\PYGZhy{}server\PYGZhy{}mongodb*
\end{sphinxVerbatim}

\end{itemize}

\end{enumerate}
\end{sphinxadmonition}

\begin{sphinxadmonition}{note}{Uninstall on Red Hat Enterprise Linux and CentOS}
\begin{enumerate}
\sphinxsetlistlabels{\arabic}{enumi}{enumii}{}{.}%
\item {} 
\sphinxAtStartPar
Stop the Percona Server for MongoDB service:

\begin{sphinxVerbatim}[commandchars=\\\{\}]
\PYGZdl{} sudo systemctl stop mongod
\end{sphinxVerbatim}

\item {} 
\sphinxAtStartPar
Remove the packages:

\begin{sphinxVerbatim}[commandchars=\\\{\}]
\PYGZdl{} sudo yum remove percona\PYGZhy{}server\PYGZhy{}mongodb*
\end{sphinxVerbatim}

\item {} 
\sphinxAtStartPar
Remove the data and configuration files:

\begin{sphinxVerbatim}[commandchars=\\\{\}]
\PYGZdl{} sudo rm \PYGZhy{}rf /var/lib/mongodb
\PYGZdl{} sudo rm \PYGZhy{}f /etc/mongod.conf
\end{sphinxVerbatim}

\end{enumerate}

\begin{sphinxadmonition}{warning}{Warning:}
\sphinxAtStartPar
This will remove all the packages and delete all the data files (databases, tables, logs, etc.).  You might want to back up your data before doing this in case you need the data later.
\end{sphinxadmonition}
\end{sphinxadmonition}
\end{quote}


\part{Release Notes}
\label{\detokenize{index:release-notes}}

\chapter{Percona Server for MongoDB 4.4 Release Notes}
\label{\detokenize{release_notes/index:percona-server-for-mongodb-version-release-notes}}\label{\detokenize{release_notes/index:rel-notes}}\label{\detokenize{release_notes/index::doc}}

\section{\sphinxstyleemphasis{Percona Server for MongoDB} 4.4.17\sphinxhyphen{}17 (2022\sphinxhyphen{}11\sphinxhyphen{}10)}
\label{\detokenize{release_notes/4.4.17-17:percona-server-for-mongodb-4-4-17-17-2022-11-10}}\label{\detokenize{release_notes/4.4.17-17:psmdb-4-4-17-17}}\label{\detokenize{release_notes/4.4.17-17::doc}}\begin{quote}\begin{description}
\item[{Date}] \leavevmode
\sphinxAtStartPar
November 10, 2022

\item[{Installation}] \leavevmode
\sphinxAtStartPar
\sphinxhref{https://www.percona.com/doc/percona-server-for-mongodb/4.4/install/index.html}{Installing Percona Server for MongoDB}

\end{description}\end{quote}

\sphinxAtStartPar
Percona Server for MongoDB 4.4.17\sphinxhyphen{}17 is an enhanced, source available, and highly\sphinxhyphen{}scalable database that is a
fully\sphinxhyphen{}compatible, drop\sphinxhyphen{}in replacement for MongoDB 4.4.17 Community Edition.
It supports MongoDB 4.4.17 protocols and drivers.


\subsection{Release Highlights}
\label{\detokenize{release_notes/4.4.17-17:release-highlights}}\begin{itemize}
\item {} 
\sphinxAtStartPar
{\hyperref[\detokenize{kmip:kmip}]{\sphinxcrossref{\DUrole{std,std-ref}{Data\sphinxhyphen{}at\sphinxhyphen{}rest encryption using the Key Management Interoperability Protocol (KMIP)}}}} is generally available enabling you to use it in your production environment

\item {} 
\sphinxAtStartPar
{\hyperref[\detokenize{backup-cursor:backup-cursor}]{\sphinxcrossref{\DUrole{std,std-ref}{\$backupCursor and \$backupCursorExtend aggregation stages}}}} functionality is generally available, enabling your application developers to use it for building custom backup solutions.

\begin{sphinxadmonition}{note}{Note:}
\sphinxAtStartPar
Percona provides \sphinxhref{https://docs.percona.com/percona-backup-mongodb/index.html}{Percona Backup for MongoDB} \sphinxhyphen{} the open source tool for consistent backups and restores in MongoDB sharded clusters.
\end{sphinxadmonition}

\item {} 
\sphinxAtStartPar
Fixed security vulnerability \sphinxhref{https://cve.mitre.org/cgi-bin/cvename.cgi?name=CVE-2022-3602}{CVE\sphinxhyphen{}2022\sphinxhyphen{}3602} for \sphinxstyleemphasis{Percona Server for MongoDB} 4.4.15\sphinxhyphen{}15 and higher installed from tarballs on Ubuntu 22.04.

\end{itemize}

\sphinxAtStartPar
The bug fixes, provided by MongoDB and included in Percona Server for MongoDB, are the following:
\begin{itemize}
\item {} 
\sphinxAtStartPar
\sphinxhref{https://jira.mongodb.org/browse/SERVER-68925}{SERVER\sphinxhyphen{}68925} \sphinxhyphen{} Detect and resolve table logging inconsistencies for WiredTiger tables at startup

\item {} 
\sphinxAtStartPar
\sphinxhref{https://jira.mongodb.org/browse/SERVER-56127}{SERVER\sphinxhyphen{}56127} \sphinxhyphen{} Fixed retryable writes on update and delete commands to not execute more than once if chunk is migrated and shard key pattern uses nested fields

\item {} 
\sphinxAtStartPar
\sphinxhref{https://jira.mongodb.org/browse/SERVER-64142}{SERVER\sphinxhyphen{}64142} \sphinxhyphen{} Verify that any unique indexes are prefixed by the new shard key pattern

\item {} 
\sphinxAtStartPar
\sphinxhref{https://jira.mongodb.org/browse/SERVER-65382}{SERVER\sphinxhyphen{}65382} \sphinxhyphen{} Prevent the use of \sphinxcode{\sphinxupquote{clientReadable}} function in \sphinxcode{\sphinxupquote{AutoSplitVector}} when reordering shard key fields

\item {} 
\sphinxAtStartPar
\sphinxhref{https://jira.mongodb.org/browse/WT-9870}{WT\sphinxhyphen{}9870} \sphinxhyphen{} Fixed the global time window state before performing the rollback to stable operation by updating the pinned timestamp as part of the transaction setup.

\end{itemize}

\sphinxAtStartPar
Find the full list of changes in the \sphinxhref{https://www.mongodb.com/docs/v4.4/release-notes/4.4/\#4.4.17---sep-28--2022}{MongoDB 4.4.17 Community Edition release notes}


\subsection{New Features}
\label{\detokenize{release_notes/4.4.17-17:new-features}}\begin{itemize}
\item {} 
\sphinxAtStartPar
\sphinxhref{https://jira.percona.com/browse/PSMDB-776}{PSMDB\sphinxhyphen{}776}: Align Docker container with upstream by adding missing \sphinxcode{\sphinxupquote{mongodb\sphinxhyphen{}tools}} utilities (Thanks to Denys Holius for reporting this issue)

\end{itemize}


\subsection{Improvements}
\label{\detokenize{release_notes/4.4.17-17:improvements}}\begin{itemize}
\item {} 
\sphinxAtStartPar
\sphinxhref{https://jira.percona.com/browse/PSMDB-1116}{PSMDB\sphinxhyphen{}1116}: Use proper exit code and logging severity for successful master key rotation

\end{itemize}


\subsection{Bugs Fixed}
\label{\detokenize{release_notes/4.4.17-17:bugs-fixed}}\begin{itemize}
\item {} 
\sphinxAtStartPar
\sphinxhref{https://jira.percona.com/browse/PSMDB-1172}{PSMDB\sphinxhyphen{}1172}: Fixed CVE\sphinxhyphen{}2022\sphinxhyphen{}3602 by updating libssl for Ubuntu 22.04 tarball

\item {} 
\sphinxAtStartPar
\sphinxhref{https://jira.percona.com/browse/PSMDB-1134}{PSMDB\sphinxhyphen{}1134}: Prevent the server crash by ensuring the backup cursor is closed before the server shutdown

\item {} 
\sphinxAtStartPar
\sphinxhref{https://jira.percona.com/browse/PSMDB-1130}{PSMDB\sphinxhyphen{}1130}: Improve handling of the missing encryption key during KMIP key rotation

\item {} 
\sphinxAtStartPar
\sphinxhref{https://jira.percona.com/browse/PSMDB-1129}{PSMDB\sphinxhyphen{}1129}: Prevent \sphinxstyleemphasis{Percona Server for MongoDB} from starting if the configured encryption key doesn’t match the one used for data encryption

\item {} 
\sphinxAtStartPar
\sphinxhref{https://jira.percona.com/browse/PSMDB-1082}{PSMDB\sphinxhyphen{}1082}: Improve error handling for \sphinxstyleemphasis{Percona Server for MongoDB} when the wrong encryption key is used

\end{itemize}


\section{\sphinxstyleemphasis{Percona Server for MongoDB} 4.4.16\sphinxhyphen{}16 (2022\sphinxhyphen{}08\sphinxhyphen{}30)}
\label{\detokenize{release_notes/4.4.16-16:percona-server-for-mongodb-4-4-16-16-2022-08-30}}\label{\detokenize{release_notes/4.4.16-16:psmdb-4-4-16-16}}\label{\detokenize{release_notes/4.4.16-16::doc}}\begin{quote}\begin{description}
\item[{Date}] \leavevmode
\sphinxAtStartPar
August 30, 2022

\item[{Installation}] \leavevmode
\sphinxAtStartPar
\sphinxhref{https://www.percona.com/doc/percona-server-for-mongodb/4.4/install/index.html}{Installing Percona Server for MongoDB}

\end{description}\end{quote}

\sphinxAtStartPar
Percona Server for MongoDB 4.4.16\sphinxhyphen{}16 is an enhanced, source\sphinxhyphen{}available, and highly\sphinxhyphen{}scalable database that is a
fully\sphinxhyphen{}compatible, drop\sphinxhyphen{}in replacement for MongoDB 4.4.16 Community Edition.
It supports MongoDB 4.4.16 protocols and drivers.

\begin{sphinxadmonition}{warning}{Warning:}
\sphinxAtStartPar
Take caution when upgrading from earlier versions of v4.4.x to later versions of 4.4 or on to v5.0. See \sphinxhref{https://jira.mongodb.org/browse/SERVER-68511}{SERVER\sphinxhyphen{}68511} for more details.
\end{sphinxadmonition}


\subsection{Release Highlights}
\label{\detokenize{release_notes/4.4.16-16:release-highlights}}\begin{itemize}
\item {} 
\sphinxAtStartPar
\sphinxhref{https://jira.mongodb.org/browse/SERVER-67302}{SERVER\sphinxhyphen{}67302} \sphinxhyphen{} Fixed the server crash with the CLOCK\_REALTIME set to forward by making the linearizable reads robust to primary catch\sphinxhyphen{}up and simultaneous stepdown.

\item {} 
\sphinxAtStartPar
\sphinxhref{https://jira.mongodb.org/browse/SERVER-61321}{SERVER\sphinxhyphen{}61321}, \sphinxhref{https://jira.mongodb.org/browse/SERVER-60607}{SERVER\sphinxhyphen{}60607} \sphinxhyphen{} Improved handling of large/NaN (Not a Number) values for text index and geo index version.

\item {} 
\sphinxAtStartPar
\sphinxhref{https://jira.mongodb.org/browse/SERVER-66418}{SERVER\sphinxhyphen{}66418} \sphinxhyphen{}  Fixed the issue with bad projection created during dependency analysis due to string order assumption. It resulted in the \sphinxcode{\sphinxupquote{PathCollision}} error. The issue is fixed by improving dependency analysis for projections by folding dependencies into ancestor dependencies where possible.

\item {} 
\sphinxAtStartPar
\sphinxhref{https://jira.mongodb.org/browse/WT-9096}{WT\sphinxhyphen{}9096} \sphinxhyphen{} Fixed the wrong key/value returning during search near when the key doesn’t exist.

\item {} 
\sphinxAtStartPar
\sphinxhref{https://jira.mongodb.org/browse/SERVER-63243}{SERVER\sphinxhyphen{}63243} \sphinxhyphen{} This bug fix adjusts the functioning of the range\sphinxhyphen{}deleter to prevent the balancer from getting blocked, hung, or ranges being scheduled behind other ranges.

\item {} 
\sphinxAtStartPar
\sphinxhref{https://jira.mongodb.org/browse/SERVER-67492}{SERVER\sphinxhyphen{}67492} \sphinxhyphen{} Failed chunk migrations can lead to recipient shard having different config.transactions records between primaries and secondaries \sphinxhyphen{} inconsistent data.

\item {} 
\sphinxAtStartPar
\sphinxhref{https://jira.mongodb.org/browse/SERVER-60958}{SERVER\sphinxhyphen{}60958} \sphinxhyphen{} This fix avoids server hang in chunk migration when a step\sphinxhyphen{}down occurs.

\end{itemize}

\sphinxAtStartPar
Find the full list of changes in the \sphinxhref{https://www.mongodb.com/docs/v4.4/release-notes/4.4/\#4.4.16---aug-19--2022}{MongoDB 4.4.16 Community Edition release notes}.


\subsection{Improvements}
\label{\detokenize{release_notes/4.4.16-16:improvements}}\begin{itemize}
\item {} 
\sphinxAtStartPar
\sphinxhref{https://jira.percona.com/browse/PSMDB-1046}{PSMDB\sphinxhyphen{}1046}: Make the \sphinxcode{\sphinxupquote{kmipKeyIdentifier}} option not mandatory

\end{itemize}

\begin{sphinxadmonition}{note}{Note:}
\sphinxAtStartPar
If you have configured data at rest encryption using the KMIP server and wish to upgrade \sphinxstyleemphasis{Percona Server for MongoDB}, go through the \sphinxhref{https://www.mongodb.com/docs/v4.4/tutorial/configure-encryption/\#std-label-encrypt-existing-data}{encrypting existing data steps} during the upgrade as follows:
\begin{enumerate}
\sphinxsetlistlabels{\arabic}{enumi}{enumii}{}{.}%
\item {} 
\sphinxAtStartPar
Prepare the server

\item {} 
\sphinxAtStartPar
{\hyperref[\detokenize{install/upgrade-from-mongodb:minor-upgrade}]{\sphinxcrossref{\DUrole{std,std-ref}{Upgrade Percona Server for MongoDB}}}}

\item {} 
\sphinxAtStartPar
Enable encryption and initiate the data synchronization.

\end{enumerate}
\end{sphinxadmonition}


\subsection{Bugs Fixed}
\label{\detokenize{release_notes/4.4.16-16:bugs-fixed}}\begin{itemize}
\item {} 
\sphinxAtStartPar
\sphinxhref{https://jira.percona.com/browse/PSMDB-1119}{PSMDB\sphinxhyphen{}1119}: Fixed the issue with backup cursor not opening if data\sphinxhyphen{}at\sphinxhyphen{}rest encryption is enabled

\end{itemize}


\subsection{Packaging Notes}
\label{\detokenize{release_notes/4.4.16-16:packaging-notes}}
\sphinxAtStartPar
Debian 9 (“Stretch”) is no longer supported.


\sphinxstrong{See also:}
\nopagebreak


\sphinxAtStartPar
Percona Blog: \sphinxhref{https://www.percona.com/blog/os-platform-end-of-life-eol-announcement-for-debian-linux-9/}{OS Platform End of Life (EOL) Announcement for Debian Linux 9}




\section{Percona Server for MongoDB 4.4.15\sphinxhyphen{}15 (2022\sphinxhyphen{}07\sphinxhyphen{}19)}
\label{\detokenize{release_notes/4.4.15-15:percona-server-for-mongodb-4-4-15-15-2022-07-19}}\label{\detokenize{release_notes/4.4.15-15:psmdb-4-4-15-15}}\label{\detokenize{release_notes/4.4.15-15::doc}}\begin{quote}\begin{description}
\item[{Date}] \leavevmode
\sphinxAtStartPar
July 19, 2022

\item[{Installation}] \leavevmode
\sphinxAtStartPar
{\hyperref[\detokenize{install/index:install}]{\sphinxcrossref{\DUrole{std,std-ref}{Installing Percona Server for MongoDB}}}}

\end{description}\end{quote}

\sphinxAtStartPar
Percona Server for MongoDB 4.4.15\sphinxhyphen{}15 is an enhanced, source available, and highly\sphinxhyphen{}scalable database that is a
fully\sphinxhyphen{}compatible, drop\sphinxhyphen{}in replacement for MongoDB 4.4.15 Community Edition.
It supports MongoDB 4.4.15 protocols and drivers.


\subsection{Release Highlights}
\label{\detokenize{release_notes/4.4.15-15:release-highlights}}\begin{itemize}
\item {} 
\sphinxAtStartPar
Support of {\hyperref[\detokenize{kmip:kmip}]{\sphinxcrossref{\DUrole{std,std-ref}{multiple KMIP servers}}}} adds failover to your data\sphinxhyphen{}at\sphinxhyphen{}rest encryption setup.

\item {} 
\sphinxAtStartPar
Allow users to set KMIP client certificate password through a flag to simplify the migration from MongoDB Enterprise to \sphinxstyleemphasis{Percona Server for MongoDB}.

\end{itemize}

\sphinxAtStartPar
Other improvements and bug fixes introduced by MongoDB and included in Percona Server for MongoDB are the following:
\begin{itemize}
\item {} 
\sphinxAtStartPar
\sphinxhref{https://jira.mongodb.org/browse/SERVER-66433}{SERVER\sphinxhyphen{}66433} \sphinxhyphen{} Backported the check for user errors in case deadline on the migration destination manager is hit while waiting for a range to be cleared up. This prevents the balancer from getting blocked.

\item {} 
\sphinxAtStartPar
\sphinxhref{https://jira.mongodb.org/browse/SERVER-65821}{SERVER\sphinxhyphen{}65821} \sphinxhyphen{} Fixed the deadlock situation in cross shard transactions that could occur when the FCV (Feature Compatibility Version) was set after the “prepared” state of the transactions. That ended up with both the the \sphinxhref{https://www.mongodb.com/docs/manual/reference/command/setFeatureCompatibilityVersion/}{setFCV} thread and the \sphinxtitleref{TransactionCoordinator} hung.

\item {} 
\sphinxAtStartPar
\sphinxhref{https://jira.mongodb.org/browse/SERVER-65131}{SERVER\sphinxhyphen{}65131} \sphinxhyphen{} This is a v6.0 backport fix to v4.4 that disables opportunistic read targeting (except for specified hedged reads) in order to prevent possible performance problems associated with uneven read distribution across the secondaries.

\item {} 
\sphinxAtStartPar
\sphinxhref{https://jira.mongodb.org/browse/SERVER-54900}{SERVER\sphinxhyphen{}54900} \sphinxhyphen{} Fixed an issue where competing/blocking network calls to the sync source could prevent selecting a new sync\sphinxhyphen{}source. This is resolved by canceling the ASIO session when SSL handshake times out.

\end{itemize}

\sphinxAtStartPar
Find the full list of changes in the \sphinxhref{https://www.mongodb.com/docs/manual/release-notes/4.4/\#4.4.15---jun-21--2022}{MongoDB 4.4.15 Community Edition Release notes}.


\subsection{Supported versions}
\label{\detokenize{release_notes/4.4.15-15:supported-versions}}
\sphinxAtStartPar
Percona Server for MongoDB is now available on Ubuntu 22.04 (Jammy Jellyfish).


\subsection{Improvements}
\label{\detokenize{release_notes/4.4.15-15:improvements}}\begin{itemize}
\item {} 
\sphinxAtStartPar
\sphinxhref{https://jira.percona.com/browse/PSMDB-1045}{PSMDB\sphinxhyphen{}1045}: Add support for several KMIP servers

\item {} 
\sphinxAtStartPar
\sphinxhref{https://jira.percona.com/browse/PSMDB-1054}{PSMDB\sphinxhyphen{}1054}: Add the ability to specify the password for the KMIP client keys and certificates to simplify migration from MongoDB Enterprise.

\end{itemize}


\section{\sphinxstyleemphasis{Percona Server for MongoDB} 4.4.14\sphinxhyphen{}14 (2022\sphinxhyphen{}05\sphinxhyphen{}25)}
\label{\detokenize{release_notes/4.4.14-14:percona-server-for-mongodb-4-4-14-14-2022-05-25}}\label{\detokenize{release_notes/4.4.14-14:psmdb-4-4-14-14}}\label{\detokenize{release_notes/4.4.14-14::doc}}\begin{quote}\begin{description}
\item[{Date}] \leavevmode
\sphinxAtStartPar
May 25, 2022

\item[{Installation}] \leavevmode
\sphinxAtStartPar
\sphinxhref{https://www.percona.com/doc/percona-server-for-mongodb/4.4/install/index.html}{Installing Percona Server for MongoDB}

\end{description}\end{quote}

\sphinxAtStartPar
Percona Server for MongoDB 4.4.14\sphinxhyphen{}14 is an enhanced, source\sphinxhyphen{}available, and highly\sphinxhyphen{}scalable database that is a
fully\sphinxhyphen{}compatible, drop\sphinxhyphen{}in replacement for MongoDB 4.4.14 Community Edition.
It supports MongoDB 4.4.14 protocols and drivers.


\subsection{Release Highlights}
\label{\detokenize{release_notes/4.4.14-14:release-highlights}}
\sphinxAtStartPar
\sphinxstyleemphasis{Percona Server for MongoDB} now supports the master key rotation for data encrypted using the  {\hyperref[\detokenize{kmip:kmip}]{\sphinxcrossref{\DUrole{std,std-ref}{Keys Management Interoperability Protocol (KMIP)}}}} protocol (tech preview feature). This improvement allows users to comply with regulatory standards for data security.

\sphinxAtStartPar
Other improvements and bug fixes introduced by MongoDB and included in \sphinxstyleemphasis{Percona Server for MongoDB} are the following:
\begin{itemize}
\item {} 
\sphinxAtStartPar
\sphinxhref{https://jira.mongodb.org/browse/WT-8924}{WT\sphinxhyphen{}8924} \sphinxhyphen{} Fixed a bug that causes replication to stall on secondary replica set members in a sharded cluster handling cross\sphinxhyphen{}shard transactions. It is caused by WiredTger to erroneously return a write conflict when deciding if an update to a record is allowed. If MongoDB decides to retry the operation that caused the conflict in WiredTiger, it will enter an indefinite retry loop, and oplog application will stall on secondary nodes.

\sphinxAtStartPar
If this bug is hit, the secondary nodes will experience indefinite growth in replication lag. Restart the secondary nodes to resume replication.

\sphinxAtStartPar
This bug affects MongoDB 4.4.10 through 4.4.13 and 5.0.4 to 5.0.7.

\sphinxAtStartPar
Update to the latest version to avoid the secondary replication stall and lag issues.

\item {} 
\sphinxAtStartPar
\sphinxhref{https://jira.mongodb.org/browse/SERVER-60412}{SERVER\sphinxhyphen{}60412} \sphinxhyphen{} Check if the host has \sphinxcode{\sphinxupquote{cgroups}} v2 enabled and read the memory limits according to that.

\item {} 
\sphinxAtStartPar
\sphinxhref{https://jira.mongodb.org/browse/SERVER-62229}{SERVER\sphinxhyphen{}62229} \sphinxhyphen{} Fix invariant by allowing applying index build abort entry when in \sphinxcode{\sphinxupquote{recoverFromOplogAsStandalone}} mode

\item {} 
\sphinxAtStartPar
\sphinxhref{https://jira.mongodb.org/browse/SERVER-55429}{SERVER\sphinxhyphen{}55429} \sphinxhyphen{} Fixed the issue with blocked migrations by adding a timeout to migrations when waiting for range deletions on intersecting ranges

\end{itemize}

\sphinxAtStartPar
Find the full list of changes in the \sphinxhref{https://www.mongodb.com/docs/manual/release-notes/4.4/\#4.4.14---may-9--2022}{MongoDB 4.4.14 Community Edition Release notes}.


\subsection{Improvements}
\label{\detokenize{release_notes/4.4.14-14:improvements}}\begin{itemize}
\item {} 
\sphinxAtStartPar
\sphinxhref{https://jira.percona.com/browse/PSMDB-1011}{PSMDB\sphinxhyphen{}1011}: Add KMIP master key rotation

\item {} 
\sphinxAtStartPar
\sphinxhref{https://jira.percona.com/browse/PSMDB-1043}{PSMDB\sphinxhyphen{}1043}: The \sphinxcode{\sphinxupquote{kmipClientCertificateFile}} option now includes both the client private key and public certificate

\item {} 
\sphinxAtStartPar
\sphinxhref{https://jira.percona.com/browse/PSMDB-1044}{PSMDB\sphinxhyphen{}1044}: Make the \sphinxcode{\sphinxupquote{kmipPort}} option not mandatory and assign the default value

\end{itemize}


\subsection{Bugs Fixed}
\label{\detokenize{release_notes/4.4.14-14:bugs-fixed}}\begin{itemize}
\item {} 
\sphinxAtStartPar
\sphinxhref{https://jira.percona.com/browse/PSMDB-979}{PSMDB\sphinxhyphen{}979}: Rotate audit logs in the mode as defined in the configuration.

\item {} 
\sphinxAtStartPar
\sphinxhref{https://jira.percona.com/browse/PSMDB-1030}{PSMDB\sphinxhyphen{}1030}: Fix descriptions and mutual dependencies of KMIP related options for \sphinxcode{\sphinxupquote{mongod}} and \sphinxcode{\sphinxupquote{perconadecrypt}}

\end{itemize}


\section{\sphinxstyleemphasis{Percona Server for MongoDB} 4.4.13\sphinxhyphen{}13}
\label{\detokenize{release_notes/4.4.13-13:percona-server-for-mongodb-4-4-13-13}}\label{\detokenize{release_notes/4.4.13-13:psmdb-4-4-13-13}}\label{\detokenize{release_notes/4.4.13-13::doc}}\begin{quote}\begin{description}
\item[{Date}] \leavevmode
\sphinxAtStartPar
March 23, 2022

\item[{Installation}] \leavevmode
\sphinxAtStartPar
\sphinxhref{https://www.percona.com/doc/percona-server-for-mongodb/4.4/install/index.html}{Installing Percona Server for MongoDB}

\end{description}\end{quote}

\sphinxAtStartPar
Percona Server for MongoDB 4.4.13\sphinxhyphen{}13 is an enhanced, source\sphinxhyphen{}available, and highly\sphinxhyphen{}scalable database that is a
fully\sphinxhyphen{}compatible, drop\sphinxhyphen{}in replacement for MongoDB 4.4.13 Community Edition.
It supports MongoDB 4.4.13 protocols and drivers.

\begin{sphinxadmonition}{warning}{Warning:}
\sphinxAtStartPar
Inconsistent data is observed after the upgrade from MongoDB 4.4.3 and 4.4.4 to versions 4.4.8+ and 5.0.2+.
This issue is fixed upstream in versions 4.4.11+ and 5.0.6+. \sphinxstyleemphasis{Percona Server for MongoDB} also includes the fix starting from versions 4.4.12\sphinxhyphen{}12 and 5.0.6\sphinxhyphen{}5.

\sphinxAtStartPar
See the upgrade recommendations below:
\begin{itemize}
\item {} 
\sphinxAtStartPar
Clusters on versions 4.4.0 and 4.4.1 are safe to upgrade to 4.4.8+ or 5.0.2+ but should upgrade to recommended versions 4.4.11+ or 5.0.5+.

\item {} 
\sphinxAtStartPar
Clusters on versions 4.4.2, 4.4.3, or 4.4.4 should \sphinxstylestrong{downgrade} to 4.4.1 and then upgrade to versions  4.4.11+ or 5.0.5+.

\item {} 
\sphinxAtStartPar
Clusters running versions 4.4.5 \sphinxhyphen{} 4.4.7 can and should upgrade to versions 4.4.11+ or 5.0.5+.

\end{itemize}

\sphinxAtStartPar
Note that clusters running versions 4.4.2 \sphinxhyphen{} 4.4.8 are affected by the bug \sphinxhref{https://jira.mongodb.org/browse/WT-7995}{WT\sphinxhyphen{}7995}. See \sphinxhref{https://jira.mongodb.org/browse/WT-7995}{WT\sphinxhyphen{}7995} for specific explanation and instructions on running the \sphinxhref{https://docs.mongodb.com/manual/reference/command/validate/}{validate} command to check for data inconsistencies. These data inconsistencies can lead to data loss if not identified and repaired at this point between versions 4.4.8 and 4.4.9.

\sphinxAtStartPar
If the \sphinxhref{https://docs.mongodb.com/manual/reference/command/validate/}{validate}  command output reports any failures, resync the impacted node from an unaffected node.   \sphinxstylestrong{The validate command must be run against all collections in the database. This process can be resource intensive and can negatively impact performance.}
\end{sphinxadmonition}


\subsection{Release Highlights}
\label{\detokenize{release_notes/4.4.13-13:release-highlights}}
\sphinxAtStartPar
\sphinxstyleemphasis{Percona Server for MongoDB} now supports {\hyperref[\detokenize{kmip:kmip}]{\sphinxcrossref{\DUrole{std,std-ref}{Keys Management Interoperability Protocol (KMIP)}}}} so that users can store encryption keys in their favorite KMIP\sphinxhyphen{}compatible key manager to set up encryption at rest. This is a tech preview feature.

\sphinxAtStartPar
The list of bug fixes introduced by MongoDB and included in \sphinxstyleemphasis{Percona Server for MongoDB} is the following:
\begin{itemize}
\item {} 
\sphinxAtStartPar
\sphinxhref{https://jira.mongodb.org/browse/SERVER-63203}{SERVER\sphinxhyphen{}63203} \sphinxhyphen{} Fixed the issue where having a large number of split points causes the chunk splitter to not function correctly and huge chunks would not be split without manual intervention. This can be caused when having small shard key ranges and a very high number of documents and where more than 8192 split points would be needed.

\item {} 
\sphinxAtStartPar
\sphinxhref{https://jira.mongodb.org/browse/SERVER-62065}{SERVER\sphinxhyphen{}62065} \sphinxhyphen{} Added the \sphinxcode{\sphinxupquote{repairShardedCollectionChunksHistory}} command to restore history fields for some chunks. This aims to fix broken snapshot reads and distributed transactions.

\item {} 
\sphinxAtStartPar
\sphinxhref{https://jira.mongodb.org/browse/SERVER-59754}{SERVER\sphinxhyphen{}59754} \sphinxhyphen{} Fixed incorrect logging of queryHash/planCacheKey for operations that share the same \sphinxcode{\sphinxupquote{\$lookup}} shape

\item {} 
\sphinxAtStartPar
\sphinxhref{https://jira.mongodb.org/browse/SERVER-55483}{SERVER\sphinxhyphen{}55483} \sphinxhyphen{} Added a new startup parameter that skips verifying the table logging settings on restarting as a replica set node from the standalone mode during the restore. This speeds up the restore process.

\end{itemize}

\sphinxAtStartPar
Find the full list of changes in the \sphinxhref{https://docs.mongodb.com/manual/release-notes/4.4/\#4.4.13---mar-7--2022}{MongoDB 4.4.13 Community Edition Release notes}.


\subsection{New Features}
\label{\detokenize{release_notes/4.4.13-13:new-features}}\begin{itemize}
\item {} 
\sphinxAtStartPar
\sphinxhref{https://jira.percona.com/browse/PSMDB-971}{PSMDB\sphinxhyphen{}971}: Added support for \sphinxstyleabbreviation{KMIP} (Keys Management Interoperability Protocol) encryption. Now users can store encryption keys in their favorite KMIP\sphinxhyphen{}compatible key manager to set up encryption at rest.

\end{itemize}


\subsection{Bug Fixes}
\label{\detokenize{release_notes/4.4.13-13:bug-fixes}}\begin{itemize}
\item {} 
\sphinxAtStartPar
\sphinxhref{https://jira.percona.com/browse/PSMDB-1010}{PSMDB\sphinxhyphen{}1010}: Fixed the parameters order in the \sphinxcode{\sphinxupquote{LOGV2\_DEBUG}} statement for LDAP logging.

\item {} 
\sphinxAtStartPar
\sphinxhref{https://jira.percona.com/browse/PSMDB-957}{PSMDB\sphinxhyphen{}957}: Fixed server crash caused by LDAP misconfiguration. Now the server logs an error message and exits.

\end{itemize}


\section{\sphinxstyleemphasis{Percona Server for MongoDB} 4.4.12\sphinxhyphen{}12}
\label{\detokenize{release_notes/4.4.12-12:percona-server-for-mongodb-4-4-12-12}}\label{\detokenize{release_notes/4.4.12-12:psmdb-4-4-12-12}}\label{\detokenize{release_notes/4.4.12-12::doc}}\begin{quote}\begin{description}
\item[{Date}] \leavevmode
\sphinxAtStartPar
February 7, 2022

\item[{Installation}] \leavevmode
\sphinxAtStartPar
\sphinxhref{https://www.percona.com/doc/percona-server-for-mongodb/4.4/install/index.html}{Installing Percona Server for MongoDB}

\end{description}\end{quote}

\sphinxAtStartPar
Percona Server for MongoDB 4.4.12\sphinxhyphen{}12 is an enhanced, source\sphinxhyphen{}available and highly\sphinxhyphen{}scalable database that is a
fully\sphinxhyphen{}compatible, drop\sphinxhyphen{}in replacement for MongoDB Community Edition.
It is based on MongoDB 4.4.11 and 4.4.12 and supports MongoDB 4.4.11 \sphinxhyphen{} 4.4.12 protocols and drivers.

\begin{sphinxadmonition}{warning}{Warning:}
\sphinxAtStartPar
Inconsistent data is observed after the upgrade from MongoDB 4.4.3 and 4.4.4 to versions 4.4.8+ and 5.0.2+.
This issue is fixed upstream in versions 4.4.11 and 5.0.6. \sphinxstyleemphasis{Percona Server for MongoDB} also includes the fix in versions 4.4.12\sphinxhyphen{}12 and 5.0.6\sphinxhyphen{}5

\sphinxAtStartPar
See the upgrade recommendations below:
\begin{itemize}
\item {} 
\sphinxAtStartPar
Clusters on versions 4.4.0 and 4.4.1 are safe to upgrade to 4.4.8+ or 5.0.2+ but should upgrade to recommended versions 4.4.11+ or 5.0.5+

\item {} 
\sphinxAtStartPar
Clusters on versions 4.4.2, 4.4.3, or 4.4.4 should \sphinxstylestrong{downgrade} to 4.4.1 and then upgrade to versions  4.4.11+ or 5.0.5+.

\item {} 
\sphinxAtStartPar
Clusters running versions 4.4.5 \sphinxhyphen{} 4.4.7 can and should upgrade to versions 4.4.11+ or 5.0.5+.

\end{itemize}

\sphinxAtStartPar
Note that clusters running versions 4.4.2 \sphinxhyphen{} 4.4.8 are affected by the bug \sphinxhref{https://jira.mongodb.org/browse/WT-7995}{WT\sphinxhyphen{}7995}. See \sphinxhref{https://jira.mongodb.org/browse/WT-7995}{WT\sphinxhyphen{}7995} for specific explanation and instructions on running the \sphinxhref{https://docs.mongodb.com/manual/reference/command/validate/}{validate} command to check for data inconsistencies. These data inconsistencies can lead to data loss if not identified and repaired at this point between versions 4.4.8 and 4.4.9.

\sphinxAtStartPar
If the \sphinxhref{https://docs.mongodb.com/manual/reference/command/validate/}{validate}  command output reports any failures, resync the impacted node from an unaffected node.   \sphinxstylestrong{The validate command must be run against all collections in the database. This process can be resource intensive and can negatively impact performance.}
\end{sphinxadmonition}


\subsection{Release Highlights}
\label{\detokenize{release_notes/4.4.12-12:release-highlights}}
\sphinxAtStartPar
The bug fixes and improvements, provided by MongoDB and included in Percona Server for MongoDB, are the following:
\begin{itemize}
\item {} 
\sphinxAtStartPar
\sphinxhref{https://jira.mongodb.org/browse/WT-8395}{WT\sphinxhyphen{}8395} \sphinxhyphen{} Fixed an issue with inconsistent data observed during the direct upgrade from from 4.4.3 and 4.4.4 to 4.4.8+ and 5.0.2+. Data inconsistency was caused by the incorrect checkpoint metadata to sometimes be recorded by MongoDB versions 4.4.3 and 4.4.4. WiredTiger used this metadata during node startup that could lead to data corruption and could cause the DuplicateKey error. The fix requires the upgrade to versions 4.4.11+ or 5.0.5+.

\item {} 
\sphinxAtStartPar
\sphinxhref{https://jira.mongodb.org/browse/SERVER-61930}{SERVER\sphinxhyphen{}61930} \sphinxhyphen{} Defined a timeout for a health check process and throw an error when the process fails to complete within a timeout. This prevents health check to hang.

\item {} 
\sphinxAtStartPar
\sphinxhref{https://jira.mongodb.org/browse/SERVER-61637}{SERVER\sphinxhyphen{}61637} \sphinxhyphen{} Changed the \sphinxcode{\sphinxupquote{rangeDeleterBatchSize}} value to unlimited to avoid the balancer starvation during batched deletes.

\item {} 
\sphinxAtStartPar
\sphinxhref{https://jira.mongodb.org/browse/SERVER-59362}{SERVER\sphinxhyphen{}59362} \sphinxhyphen{} Added the ability to transition through the valid states of the fault manager, and the interface to observer and log its state transitions.

\item {} 
\sphinxAtStartPar
\sphinxhref{https://jira.mongodb.org/browse/SERVER-62147}{SERVER\sphinxhyphen{}62147} \sphinxhyphen{} Fixed broken OP\_QUERY exhaust cursor implementation

\item {} 
\sphinxAtStartPar
\sphinxhref{https://jira.mongodb.org/browse/SERVER-62065}{SERVER\sphinxhyphen{}62065} \sphinxhyphen{} Added the \sphinxcode{\sphinxupquote{repairShardedCollectionChunksHistory}} command to restore \sphinxcode{\sphinxupquote{history}} fields for some chunks. This aims to fix broken snapshot reads and distributed transactions.

\end{itemize}

\sphinxAtStartPar
Find the full list of changes in the \sphinxhref{https://docs.mongodb.com/manual/release-notes/4.4/\#4.4.11---dec-30--2021}{MongoDB 4.4.11 Community Edition} and \sphinxhref{https://docs.mongodb.com/manual/release-notes/4.4/\#4.4.12---jan-21--2022}{MongoDB 4.4.12 Community Edition release notes}


\subsection{Bugs Fixed}
\label{\detokenize{release_notes/4.4.12-12:bugs-fixed}}\begin{itemize}
\item {} 
\sphinxAtStartPar
\sphinxhref{https://jira.percona.com/browse/PSMDB-756}{PSMDB\sphinxhyphen{}756}: Fixed an issue with unmet dependencies for installing MongoDB on Debian (Thanks to Stefan Schlesi for reporting this issue)

\item {} 
\sphinxAtStartPar
\sphinxhref{https://jira.percona.com/browse/PSMDB-950}{PSMDB\sphinxhyphen{}950}: Fixed LDAP authentication using mongo CLI for \sphinxstyleemphasis{Percona Server for MongoDB} installed from a tarball.

\end{itemize}


\section{\sphinxstyleemphasis{Percona Server for MongoDB} 4.4.10\sphinxhyphen{}11}
\label{\detokenize{release_notes/4.4.10-11:percona-server-for-mongodb-4-4-10-11}}\label{\detokenize{release_notes/4.4.10-11:psmdb-4-4-10-11}}\label{\detokenize{release_notes/4.4.10-11::doc}}\begin{quote}\begin{description}
\item[{Date}] \leavevmode
\sphinxAtStartPar
November 10, 2021

\item[{Installation}] \leavevmode
\sphinxAtStartPar
\sphinxhref{https://docs.percona.com/percona-server-for-mongodb/4.4/install/index.html}{Installing Percona Server for MongoDB}

\end{description}\end{quote}

\sphinxAtStartPar
Percona Server for MongoDB 4.4.10\sphinxhyphen{}11 is an enhanced, source\sphinxhyphen{}available, and highly\sphinxhyphen{}scalable database that is a
fully\sphinxhyphen{}compatible, drop\sphinxhyphen{}in replacement for MongoDB 4.4.10 Community Edition.

\sphinxAtStartPar
It is rebased on \sphinxhref{https://docs.mongodb.com/v4.4/release-notes/4.4/\#4.4.10---oct-15--2021}{MongoDB 4.4.10 Community Edition} and supports MongoDB 4.4.10 protocols and drivers.


\subsection{Release Highlights}
\label{\detokenize{release_notes/4.4.10-11:release-highlights}}
\sphinxAtStartPar
The changes and bug fixes introduced by MongoDB include the following:
\begin{itemize}
\item {} 
\sphinxAtStartPar
Fixed delays in establishing egress connections on \sphinxcode{\sphinxupquote{mongos}} due to delayed responses from \sphinxcode{\sphinxupquote{libcrypto.so}}

\end{itemize}

\sphinxAtStartPar
Find the full list of changes in the \sphinxhref{https://docs.mongodb.com/v4.4/release-notes/4.4/\#4.4.10---oct-15--2021}{MongoDB 4.4.10 Community Edition release notes}.


\section{\sphinxstyleemphasis{Percona Server for MongoDB} 4.4.9\sphinxhyphen{}10}
\label{\detokenize{release_notes/4.4.9-10:percona-server-for-mongodb-4-4-9-10}}\label{\detokenize{release_notes/4.4.9-10:psmdb-4-4-9-10}}\label{\detokenize{release_notes/4.4.9-10::doc}}\begin{quote}\begin{description}
\item[{Date}] \leavevmode
\sphinxAtStartPar
October 7, 2021

\item[{Installation}] \leavevmode
\sphinxAtStartPar
\sphinxhref{https://www.percona.com/doc/percona-server-for-mongodb/4.4/install/index.html}{Installing Percona Server for MongoDB}

\end{description}\end{quote}

\sphinxAtStartPar
Percona Server for MongoDB 4.4.9\sphinxhyphen{}10 is an enhanced, source available, and highly\sphinxhyphen{}scalable database that is a
fully\sphinxhyphen{}compatible, drop\sphinxhyphen{}in replacement for MongoDB 4.4.9 Community Edition.
It supports MongoDB 4.4.9 protocols and drivers.

\begin{sphinxadmonition}{warning}{Warning:}
\sphinxAtStartPar
Beginning with MongoDB 4.4.2, several data impacting or corrupting bugs were introduced. Details are listed below.

\sphinxAtStartPar
These bugs are fixed in MongoDB 4.4.9. Percona Server for MongoDB 4.4.9\sphinxhyphen{}10 includes the upstream fixes of these bugs.

\sphinxAtStartPar
Please upgrade to MongoDB 4.4.9 or Percona Server for MongoDB 4.4.9\sphinxhyphen{}10  as soon as possible.

\begin{sphinxadmonition}{note}{Additional upgrade planning note}

\sphinxAtStartPar
If you are running MongoDB version 4.4.4 / Percona Server for MongoDB 4.4.4\sphinxhyphen{}5 or below, please skip upgrading to version 4.4.8 as that has been found to trigger the Duplicate Key error most consistently. Two WiredTiger bugs (\sphinxhref{https://jira.mongodb.org/browse/WT-7984}{WT\sphinxhyphen{}7984} and \sphinxhref{https://jira.mongodb.org/browse/WT-7995}{WT\sphinxhyphen{}7995}) are found in all versions of MongoDB 4.4.x except 4.4.1. Ideally, upgrade directly from version 4.4.1 to 4.4.9 to attempt to avoid those data impacting or corrupting bugs.
\end{sphinxadmonition}
\end{sphinxadmonition}


\subsection{Release Highlights}
\label{\detokenize{release_notes/4.4.9-10:release-highlights}}
\sphinxAtStartPar
The bug fixes, provided by MongoDB and included in Percona Server for MongoDB, are the following:
\begin{itemize}
\item {} 
\sphinxAtStartPar
\sphinxhref{https://jira.mongodb.org/browse/WT-7426}{WT\sphinxhyphen{}7426} \sphinxhyphen{} After upgrade to v4.4.5, startups or restarts can trigger WiredTigers RTS bug which can corrupt page metadata causing documents on affected pages to become invisible to MongoDB. This can lead to temporary query incorrectness, or more likely a fatal error and inability to restart. Affects only MongoDB 4.4.5 and Percona Server for MongoDB 4.4.5\sphinxhyphen{}7.

\item {} 
\sphinxAtStartPar
\sphinxhref{https://jira.mongodb.org/browse/SERVER-58936}{SERVER\sphinxhyphen{}58936} \sphinxhyphen{} Unique Index constraint violations possible \sphinxhyphen{} can cause duplicate data \sphinxhyphen{} fixed in version 4.4.8.

\item {} 
\sphinxAtStartPar
\sphinxhref{https://jira.mongodb.org/browse/WT-7995}{WT\sphinxhyphen{}7995} \sphinxhyphen{} Checkpoint thread can read and persist inconsistent version of data to disk. Can cause Duplicate Key error on startup and prevent the node from starting. Unclean shutdowns can cause data inconsistency within documents, deleted documents to still exist, incomplete query results due to lost or inaccurate index entries, and/or missing documents. Affects MongoDB versions 4.4.2 through 4.4.8 and Percona Server for MongoDB 4.4.2\sphinxhyphen{}4 \sphinxhyphen{} 4.4.8\sphinxhyphen{}9 as well as MongoDB 5.0.0 through 5.0.2. Upgrade to fixed version of MongoDB 4.4.9 / Percona Server for MongoDB 4.4.9\sphinxhyphen{}10 as soon as possible.

\item {} 
\sphinxAtStartPar
WT\sphinxhyphen{}7984 and associated Server Bug SERVER\sphinxhyphen{}60371.

\item {} 
\sphinxAtStartPar
\sphinxhref{https://jira.mongodb.org/browse/WT-7984}{WT\sphinxhyphen{}7984} \sphinxhyphen{} Bug that could cause Checkpoint thread to omit a page of data. If the server experiences an unclean shutdown, an inconsistent checkpoint is used for recovery and causes data corruption. Fixed in version 4.4.9.
Requires the \sphinxhref{https://docs.mongodb.com/manual/reference/command/validate/}{validate}  command to be run and possible data remediation via complete initial sync.

\item {} 
\sphinxAtStartPar
\sphinxhref{https://jira.mongodb.org/browse/SERVER-60371}{SERVER\sphinxhyphen{}60371} \sphinxhyphen{} If previously upgraded to version 4.4.8 then upgrade to 4.4.9, could still experience Duplicate Key error and Fatal assertion. Related to the two previous WiredTiger bugs.

\sphinxAtStartPar
Requires the \sphinxhref{https://docs.mongodb.com/manual/reference/command/validate/}{validate}  command to be run and possible data remediation via complete initial sync. Currently under review.

\end{itemize}

\sphinxAtStartPar
Find the full list of changes in the \sphinxhref{https://docs.mongodb.com/manual/release-notes/4.4/\#4.4.9---sep-21--2021}{MongoDB 4.4.9 Community Edition release notes}.


\subsection{Improvements}
\label{\detokenize{release_notes/4.4.9-10:improvements}}\begin{itemize}
\item {} 
\sphinxAtStartPar
\sphinxhref{https://jira.percona.com/browse/PSMDB-918}{PSMDB\sphinxhyphen{}918}: Disable the ability to delete the \sphinxtitleref{mongod} user in RPM packages \sphinxhyphen{} This preserves the permissions to the MongoDB data directory for the \sphinxcode{\sphinxupquote{mongod}} user as its user ID and group ID remain unchanged.

\end{itemize}


\section{\sphinxstyleemphasis{Percona Server for MongoDB} 4.4.8\sphinxhyphen{}9}
\label{\detokenize{release_notes/4.4.8-9:percona-server-for-mongodb-4-4-8-9}}\label{\detokenize{release_notes/4.4.8-9:psmdb-4-4-8-9}}\label{\detokenize{release_notes/4.4.8-9::doc}}\begin{quote}\begin{description}
\item[{Date}] \leavevmode
\sphinxAtStartPar
August 16, 2021

\item[{Installation}] \leavevmode
\sphinxAtStartPar
\sphinxhref{https://www.percona.com/doc/percona-server-for-mongodb/4.4/install/index.html}{Installing Percona Server for MongoDB}

\end{description}\end{quote}

\begin{sphinxadmonition}{warning}{Warning:}
\sphinxAtStartPar
This version is not recommended for production use due to the following critical issues: \sphinxhref{https://jira.mongodb.org/browse/WT-7984}{WT\sphinxhyphen{}7984} and \sphinxhref{https://jira.mongodb.org/browse/WT-7995}{WT\sphinxhyphen{}7995}. They are fixed in \sphinxhref{https://docs.mongodb.com/manual/release-notes/4.4/\#4.4.9---sep-21--2021}{MongoDB 4.4.9 Community Edition} and {\hyperref[\detokenize{release_notes/4.4.9-10:psmdb-4-4-9-10}]{\sphinxcrossref{\DUrole{std,std-ref}{Percona Server for MongoDB 4.4.9\sphinxhyphen{}10}}}}.

\sphinxAtStartPar
We recommend you to upgrade to Percona Server for MongoDB 4.4.9\sphinxhyphen{}10 and run the \sphinxhref{https://docs.mongodb.com/manual/reference/command/validate/}{validate} command on every collection on every replica set node.

\sphinxAtStartPar
Read more about the  post\sphinxhyphen{}upgrade steps in \sphinxhref{https://jira.mongodb.org/browse/WT-7984}{WT\sphinxhyphen{}7984} and \sphinxhref{https://jira.mongodb.org/browse/WT-7995}{WT\sphinxhyphen{}7995}.
\end{sphinxadmonition}

\sphinxAtStartPar
Percona Server for MongoDB 4.4.8\sphinxhyphen{}9 is an enhanced, source available, and highly\sphinxhyphen{}scalable database that is a
fully\sphinxhyphen{}compatible, drop\sphinxhyphen{}in replacement for \sphinxhref{https://docs.mongodb.com/manual/release-notes/4.4/\#4.4.8---aug-4--2021}{MongoDB 4.4.8 Community Edition}.
It supports MongoDB 4.4.8 protocols and drivers.


\subsection{Improvements}
\label{\detokenize{release_notes/4.4.8-9:improvements}}\begin{itemize}
\item {} 
\sphinxAtStartPar
\sphinxhref{https://jira.percona.com/browse/PSMDB-211}{PSMDB\sphinxhyphen{}211}: Add the ability to view the status of {\hyperref[\detokenize{hot-backup:hot-backup}]{\sphinxcrossref{\DUrole{std,std-ref}{hot backup}}}} using the \sphinxcode{\sphinxupquote{mongo}} shell

\item {} 
\sphinxAtStartPar
\sphinxhref{https://jira.percona.com/browse/PSMDB-824}{PSMDB\sphinxhyphen{}824}, \sphinxhref{https://jira.percona.com/browse/PSMDB-892}{PSMDB\sphinxhyphen{}892}, \sphinxhref{https://jira.percona.com/browse/PSMDB-807}{PSMDB\sphinxhyphen{}807}: Remove excessive log messages and improve error messages for various backup cases

\end{itemize}


\subsection{Bugs Fixed}
\label{\detokenize{release_notes/4.4.8-9:bugs-fixed}}\begin{itemize}
\item {} 
\sphinxAtStartPar
\sphinxhref{https://jira.percona.com/browse/PSMDB-210}{PSMDB\sphinxhyphen{}210}: Add the ability to stop a running hot backup with the  \sphinxcode{\sphinxupquote{killOp()}} method

\item {} 
\sphinxAtStartPar
\sphinxhref{https://jira.percona.com/browse/PSMDB-902}{PSMDB\sphinxhyphen{}902}: The \sphinxcode{\sphinxupquote{rateLimit}} field can now be used to filter audit logs

\item {} 
\sphinxAtStartPar
\sphinxhref{https://jira.percona.com/browse/PSMDB-876}{PSMDB\sphinxhyphen{}876}: Fix limiting of the {\hyperref[\detokenize{rate-limit:rate-limit}]{\sphinxcrossref{\DUrole{std,std-ref}{database Profiler}}}} with the \sphinxcode{\sphinxupquote{rateLimit}} option

\item {} 
\sphinxAtStartPar
\sphinxhref{https://jira.percona.com/browse/PSMDB-873}{PSMDB\sphinxhyphen{}873}: Add missing runtime dependencies for LDAP authentication against active directory in RPM packages

\end{itemize}


\section{\sphinxstyleemphasis{Percona Server for MongoDB} 4.4.6\sphinxhyphen{}8}
\label{\detokenize{release_notes/4.4.6-8:percona-server-for-mongodb-4-4-6-8}}\label{\detokenize{release_notes/4.4.6-8:psmdb-4-4-6-8}}\label{\detokenize{release_notes/4.4.6-8::doc}}\begin{quote}\begin{description}
\item[{Date}] \leavevmode
\sphinxAtStartPar
June 2, 2021

\item[{Installation}] \leavevmode
\sphinxAtStartPar
\sphinxhref{https://www.percona.com/doc/percona-server-for-mongodb/4.4/install/index.html}{Installing Percona Server for MongoDB}

\end{description}\end{quote}

\begin{sphinxadmonition}{warning}{Warning:}
\sphinxAtStartPar
This version is not recommended for production use due to the following critical issues: \sphinxhref{https://jira.mongodb.org/browse/WT-7984}{WT\sphinxhyphen{}7984} and \sphinxhref{https://jira.mongodb.org/browse/WT-7995}{WT\sphinxhyphen{}7995}. They are fixed in \sphinxhref{https://docs.mongodb.com/manual/release-notes/4.4/\#4.4.9---sep-21--2021}{MongoDB 4.4.9 Community Edition} and {\hyperref[\detokenize{release_notes/4.4.9-10:psmdb-4-4-9-10}]{\sphinxcrossref{\DUrole{std,std-ref}{Percona Server for MongoDB 4.4.9\sphinxhyphen{}10}}}}.

\sphinxAtStartPar
We recommend you to upgrade to Percona Server for MongoDB 4.4.9\sphinxhyphen{}10 as soon as possible and run the \sphinxhref{https://docs.mongodb.com/manual/reference/command/validate/}{validate} command on every collection on every replica set node.

\sphinxAtStartPar
Read more about the issues and post\sphinxhyphen{}upgrade steps in \sphinxhref{https://jira.mongodb.org/browse/WT-7984}{WT\sphinxhyphen{}7984} and \sphinxhref{https://jira.mongodb.org/browse/WT-7995}{WT\sphinxhyphen{}7995}.
\end{sphinxadmonition}

\sphinxAtStartPar
Percona Server for MongoDB 4.4.6\sphinxhyphen{}8 is an enhanced, source available, and highly\sphinxhyphen{}scalable database that is a
fully\sphinxhyphen{}compatible, drop\sphinxhyphen{}in replacement for \sphinxhref{https://docs.mongodb.com/manual/release-notes/4.4/\#4.4.6---may-10--2021}{MongoDB 4.4.6 Community Edition}.
It supports MongoDB 4.4.6 protocols and drivers.


\subsection{New Features}
\label{\detokenize{release_notes/4.4.6-8:new-features}}\begin{itemize}
\item {} 
\sphinxAtStartPar
\sphinxhref{https://jira.percona.com/browse/PSMDB-802}{PSMDB\sphinxhyphen{}802}: Create \$backupCursor and \$backupCursorExtend aggregation stages. (Tech Preview Feature %
\begin{footnote}[1]\sphinxAtStartFootnote
Tech Preview Features are not yet ready for enterprise use and are not included in support via SLA (Service License Agreement). They are included in this release so that users can provide feedback prior to the full release of the feature in a future release (or removal of the feature if it is deemed not useful). This functionality can change (APIs, CLIs, etc.) from tech preview to GA.
%
\end{footnote})

\end{itemize}


\section{\sphinxstyleemphasis{Percona Server for MongoDB} 4.4.5\sphinxhyphen{}7}
\label{\detokenize{release_notes/4.4.5-7:percona-server-for-mongodb-4-4-5-7}}\label{\detokenize{release_notes/4.4.5-7:psmdb-4-4-5-7}}\label{\detokenize{release_notes/4.4.5-7::doc}}\begin{quote}\begin{description}
\item[{Date}] \leavevmode
\sphinxAtStartPar
April 19, 2021

\item[{Installation}] \leavevmode
\sphinxAtStartPar
\sphinxhref{https://www.percona.com/doc/percona-server-for-mongodb/4.4/install/index.html}{Installing Percona Server for MongoDB}

\end{description}\end{quote}

\begin{sphinxadmonition}{warning}{Warning:}
\sphinxAtStartPar
This version is not recommended for production use due to the following critical issues:
\begin{itemize}
\item {} 
\sphinxAtStartPar
\sphinxhref{https://jira.mongodb.org/browse/WT-7426}{WT\sphinxhyphen{}7426},

\item {} 
\sphinxAtStartPar
\sphinxhref{https://jira.mongodb.org/browse/WT-7984}{WT\sphinxhyphen{}7984} and

\item {} 
\sphinxAtStartPar
\sphinxhref{https://jira.mongodb.org/browse/WT-7995}{WT\sphinxhyphen{}7995}.

\end{itemize}

\sphinxAtStartPar
The issue \sphinxhref{https://jira.mongodb.org/browse/WT-7426}{WT\sphinxhyphen{}7426} is fixed in \sphinxhref{https://docs.mongodb.com/v5.0/release-notes/4.4/\#4.4.6---may-10--2021}{MongoDB 4.4.6 Community Edition} and {\hyperref[\detokenize{release_notes/4.4.6-8:psmdb-4-4-6-8}]{\sphinxcrossref{\DUrole{std,std-ref}{Percona Server for MongoDB 4.4.6\sphinxhyphen{}8}}}}.

\sphinxAtStartPar
\sphinxhref{https://jira.mongodb.org/browse/WT-7984}{WT\sphinxhyphen{}7984} and \sphinxhref{https://jira.mongodb.org/browse/WT-7995}{WT\sphinxhyphen{}7995} are fixed in \sphinxhref{https://docs.mongodb.com/manual/release-notes/4.4/\#4.4.9---sep-21--2021}{MongoDB 4.4.9 Community Edition} and {\hyperref[\detokenize{release_notes/4.4.9-10:psmdb-4-4-9-10}]{\sphinxcrossref{\DUrole{std,std-ref}{Percona Server for MongoDB 4.4.9\sphinxhyphen{}10}}}}.

\sphinxAtStartPar
We recommend you to upgrade to Percona Server for MongoDB 4.4.9\sphinxhyphen{}10 as soon as possible and run the \sphinxhref{https://docs.mongodb.com/manual/reference/command/validate/}{validate} command on every collection on every replica set node.

\sphinxAtStartPar
Read more about the issues and post\sphinxhyphen{}upgrade steps in \sphinxhref{https://jira.mongodb.org/browse/WT-7984}{WT\sphinxhyphen{}7984} and \sphinxhref{https://jira.mongodb.org/browse/WT-7995}{WT\sphinxhyphen{}7995}.
\end{sphinxadmonition}

\sphinxAtStartPar
Percona Server for MongoDB 4.4.5\sphinxhyphen{}7 is based on \sphinxhref{https://docs.mongodb.com/manual/release-notes/4.4/\#4.4.5---apr-8--2021}{MongoDB 4.4.5 Community Edition}
and does not include any additional changes.


\section{\sphinxstyleemphasis{Percona Server for MongoDB} 4.4.4\sphinxhyphen{}6}
\label{\detokenize{release_notes/4.4.4-6:percona-server-for-mongodb-4-4-4-6}}\label{\detokenize{release_notes/4.4.4-6:psmdb-4-4-4-6}}\label{\detokenize{release_notes/4.4.4-6::doc}}\begin{quote}\begin{description}
\item[{Date}] \leavevmode
\sphinxAtStartPar
February 25, 2021

\item[{Installation}] \leavevmode
\sphinxAtStartPar
\sphinxhref{https://www.percona.com/doc/percona-server-for-mongodb/4.4/install/index.html}{Installing Percona Server for MongoDB}

\end{description}\end{quote}

\begin{sphinxadmonition}{warning}{Warning:}
\sphinxAtStartPar
This version is not recommended for production use due to the following critical issues: \sphinxhref{https://jira.mongodb.org/browse/WT-7984}{WT\sphinxhyphen{}7984} and \sphinxhref{https://jira.mongodb.org/browse/WT-7995}{WT\sphinxhyphen{}7995}. They are fixed in \sphinxhref{https://docs.mongodb.com/manual/release-notes/4.4/\#4.4.9---sep-21--2021}{MongoDB 4.4.9 Community Edition} and {\hyperref[\detokenize{release_notes/4.4.9-10:psmdb-4-4-9-10}]{\sphinxcrossref{\DUrole{std,std-ref}{Percona Server for MongoDB 4.4.9\sphinxhyphen{}10}}}}.

\sphinxAtStartPar
We recommend you to upgrade to Percona Server for MongoDB 4.4.9\sphinxhyphen{}10 as soon as possible and run the \sphinxhref{https://docs.mongodb.com/manual/reference/command/validate/}{validate} command on every collection on every replica set node.

\sphinxAtStartPar
Read more about the issues and post\sphinxhyphen{}upgrade steps in \sphinxhref{https://jira.mongodb.org/browse/WT-7984}{WT\sphinxhyphen{}7984} and \sphinxhref{https://jira.mongodb.org/browse/WT-7995}{WT\sphinxhyphen{}7995}.
\end{sphinxadmonition}

\sphinxAtStartPar
Percona Server for MongoDB 4.4.4\sphinxhyphen{}6 is an enhanced, source available, and highly\sphinxhyphen{}scalable database that is a
fully\sphinxhyphen{}compatible, drop\sphinxhyphen{}in replacement for \sphinxhref{https://docs.mongodb.com/manual/release-notes/4.4/\#feb-16-2021}{MongoDB 4.4.4 Community Edition}.
It supports MongoDB 4.4.4 protocols and drivers.


\subsection{Bugs Fixed}
\label{\detokenize{release_notes/4.4.4-6:bugs-fixed}}\begin{itemize}
\item {} 
\sphinxAtStartPar
\sphinxhref{https://jira.percona.com/browse/PSMDB-817}{PSMDB\sphinxhyphen{}817}: LDAP ConnectionPoller always uses up CPU of one core (Thanks to user cleiton.domazak for reporting this issue)

\end{itemize}


\section{\sphinxstyleemphasis{Percona Server for MongoDB} 4.4.3\sphinxhyphen{}5}
\label{\detokenize{release_notes/4.4.3-5:percona-server-for-mongodb-4-4-3-5}}\label{\detokenize{release_notes/4.4.3-5:psmdb-4-4-3-5}}\label{\detokenize{release_notes/4.4.3-5::doc}}\begin{quote}\begin{description}
\item[{Date}] \leavevmode
\sphinxAtStartPar
January 27, 2021

\item[{Installation}] \leavevmode
\sphinxAtStartPar
\sphinxhref{https://www.percona.com/doc/percona-server-for-mongodb/4.4/install/index.html}{Installing Percona Server for MongoDB}

\end{description}\end{quote}

\begin{sphinxadmonition}{warning}{Warning:}
\sphinxAtStartPar
This version is not recommended for production use due to the following critical issues: \sphinxhref{https://jira.mongodb.org/browse/WT-7984}{WT\sphinxhyphen{}7984} and \sphinxhref{https://jira.mongodb.org/browse/WT-7995}{WT\sphinxhyphen{}7995}. They are fixed in \sphinxhref{https://docs.mongodb.com/manual/release-notes/4.4/\#4.4.9---sep-21--2021}{MongoDB 4.4.9 Community Edition} and {\hyperref[\detokenize{release_notes/4.4.9-10:psmdb-4-4-9-10}]{\sphinxcrossref{\DUrole{std,std-ref}{Percona Server for MongoDB 4.4.9\sphinxhyphen{}10}}}}.

\sphinxAtStartPar
We recommend you to upgrade to Percona Server for MongoDB 4.4.9\sphinxhyphen{}10 as soon as possible and run the \sphinxhref{https://docs.mongodb.com/manual/reference/command/validate/}{validate} command on every collection on every replica set node.

\sphinxAtStartPar
Read more about the issues and post\sphinxhyphen{}upgrade steps in \sphinxhref{https://jira.mongodb.org/browse/WT-7984}{WT\sphinxhyphen{}7984} and \sphinxhref{https://jira.mongodb.org/browse/WT-7995}{WT\sphinxhyphen{}7995}.
\end{sphinxadmonition}

\sphinxAtStartPar
Percona Server for MongoDB 4.4.3\sphinxhyphen{}5 is an enhanced, source available, and highly\sphinxhyphen{}scalable database that is a
fully\sphinxhyphen{}compatible, drop\sphinxhyphen{}in replacement for \sphinxhref{https://docs.mongodb.com/manual/release-notes/4.4/\#jan-4-2021}{MongoDB 4.4.3 Community Edition}.
It supports MongoDB 4.4.3 protocols and drivers.


\subsection{Improvements}
\label{\detokenize{release_notes/4.4.3-5:improvements}}\begin{itemize}
\item {} 
\sphinxAtStartPar
\sphinxhref{https://jira.percona.com/browse/PSMDB-745}{PSMDB\sphinxhyphen{}745}: Add support for multiple LDAP servers for authentication

\item {} 
\sphinxAtStartPar
\sphinxhref{https://jira.percona.com/browse/PSMDB-761}{PSMDB\sphinxhyphen{}761}: Add \sphinxcode{\sphinxupquote{validateLDAPServerConfig}} config option

\end{itemize}


\subsection{Bugs Fixed}
\label{\detokenize{release_notes/4.4.3-5:bugs-fixed}}\begin{itemize}
\item {} 
\sphinxAtStartPar
\sphinxhref{https://jira.percona.com/browse/PSMDB-788}{PSMDB\sphinxhyphen{}788}: Fix LDAP rebind procedure to allow LDAP referrals to work with ldapBindMethod==sasl

\end{itemize}


\section{\sphinxstyleemphasis{Percona Server for MongoDB} 4.4.2\sphinxhyphen{}4}
\label{\detokenize{release_notes/4.4.2-4:percona-server-for-mongodb-4-4-2-4}}\label{\detokenize{release_notes/4.4.2-4:psmdb-4-4-2-4}}\label{\detokenize{release_notes/4.4.2-4::doc}}\begin{quote}\begin{description}
\item[{Date}] \leavevmode
\sphinxAtStartPar
December 1, 2020

\item[{Installation}] \leavevmode
\sphinxAtStartPar
\sphinxhref{https://www.percona.com/doc/percona-server-for-mongodb/4.4/install/index.html}{Installing Percona Server for MongoDB}

\end{description}\end{quote}

\begin{sphinxadmonition}{warning}{Warning:}
\sphinxAtStartPar
This version is not recommended for production use due to the following critical issues: \sphinxhref{https://jira.mongodb.org/browse/WT-7984}{WT\sphinxhyphen{}7984} and \sphinxhref{https://jira.mongodb.org/browse/WT-7995}{WT\sphinxhyphen{}7995}. They are fixed in \sphinxhref{https://docs.mongodb.com/manual/release-notes/4.4/\#4.4.9---sep-21--2021}{MongoDB 4.4.9 Community Edition} and {\hyperref[\detokenize{release_notes/4.4.9-10:psmdb-4-4-9-10}]{\sphinxcrossref{\DUrole{std,std-ref}{Percona Server for MongoDB 4.4.9\sphinxhyphen{}10}}}}.

\sphinxAtStartPar
We recommend you to upgrade to Percona Server for MongoDB 4.4.9\sphinxhyphen{}10 as soon as possible and run the \sphinxhref{https://docs.mongodb.com/manual/reference/command/validate/}{validate} command on every collection on every replica set node.

\sphinxAtStartPar
Read more about the issues and post\sphinxhyphen{}upgrade steps in \sphinxhref{https://jira.mongodb.org/browse/WT-7984}{WT\sphinxhyphen{}7984} and \sphinxhref{https://jira.mongodb.org/browse/WT-7995}{WT\sphinxhyphen{}7995}.
\end{sphinxadmonition}

\sphinxAtStartPar
Percona Server for MongoDB 4.4.2\sphinxhyphen{}4 is an enhanced, source available, and highly\sphinxhyphen{}scalable database that is a
fully\sphinxhyphen{}compatible, drop\sphinxhyphen{}in replacement for \sphinxhref{https://docs.mongodb.com/manual/release-notes/4.4/\#nov-18-2020}{MongoDB 4.4.2 Community Edition}.
It supports MongoDB 4.4.2 protocols and drivers.


\subsection{Improvements}
\label{\detokenize{release_notes/4.4.2-4:improvements}}\begin{itemize}
\item {} 
\sphinxAtStartPar
\sphinxhref{https://jira.percona.com/browse/PSMDB-758}{PSMDB\sphinxhyphen{}758}: Add mongobridge as a part of percona\sphinxhyphen{}server\sphinxhyphen{}mongodb\sphinxhyphen{}server package

\item {} 
\sphinxAtStartPar
\sphinxhref{https://jira.percona.com/browse/PSMDB-755}{PSMDB\sphinxhyphen{}755}: Add ldapDebug, ldapFollowReferrals and ldapConnectionPoolSizePerHost server parameters

\item {} 
\sphinxAtStartPar
\sphinxhref{https://jira.percona.com/browse/PSMDB-711}{PSMDB\sphinxhyphen{}711}: Improve audit log performance

\end{itemize}


\subsection{Bugs Fixed}
\label{\detokenize{release_notes/4.4.2-4:bugs-fixed}}\begin{itemize}
\item {} 
\sphinxAtStartPar
\sphinxhref{https://jira.percona.com/browse/PSMDB-718}{PSMDB\sphinxhyphen{}718}: Audit log messages include datatype output

\item {} 
\sphinxAtStartPar
\sphinxhref{https://jira.percona.com/browse/PSMDB-712}{PSMDB\sphinxhyphen{}712}: User can’t be authorized via LDAP due to ‘LDAP search failed with error: Referral’

\item {} 
\sphinxAtStartPar
\sphinxhref{https://jira.percona.com/browse/PSMDB-766}{PSMDB\sphinxhyphen{}766}: Redirect openldap debug messages to mongodb log

\item {} 
\sphinxAtStartPar
\sphinxhref{https://jira.percona.com/browse/PSMDB-715}{PSMDB\sphinxhyphen{}715}: createBackup using AWS remote location fails with “EntityTooLarge”

\item {} 
\sphinxAtStartPar
\sphinxhref{https://jira.percona.com/browse/PSMDB-544}{PSMDB\sphinxhyphen{}544}: Binaries perconadecrypt and mongobridge do not have a version

\end{itemize}


\section{\sphinxstyleemphasis{Percona Server for MongoDB} 4.4.1\sphinxhyphen{}3}
\label{\detokenize{release_notes/4.4.1-3:percona-server-for-mongodb-4-4-1-3}}\label{\detokenize{release_notes/4.4.1-3:psmdb-4-4-1-3}}\label{\detokenize{release_notes/4.4.1-3::doc}}\begin{quote}\begin{description}
\item[{Date}] \leavevmode
\sphinxAtStartPar
October 9, 2020

\item[{Installation}] \leavevmode
\sphinxAtStartPar
\sphinxhref{https://www.percona.com/doc/percona-server-for-mongodb/4.4/install/index.html}{Installing Percona Server for MongoDB}

\end{description}\end{quote}

\sphinxAtStartPar
Percona Server for MongoDB 4.4.1\sphinxhyphen{}3 is an enhanced, source\sphinxhyphen{}available, and highly\sphinxhyphen{}scalable database that is a
fully\sphinxhyphen{}compatible, drop\sphinxhyphen{}in replacement for \sphinxhref{https://docs.mongodb.com/manual/release-notes/4.4/\#sep-9-2020}{MongoDB 4.4.1 Community Edition}.
It supports MongoDB 4.4.1 protocols and drivers.

\sphinxAtStartPar
This release fixes security vulnerability \sphinxhref{https://cve.mitre.org/cgi-bin/cvename.cgi?name=CVE-2020-26542}{CVE\sphinxhyphen{}2020\sphinxhyphen{}26542}.


\section{\sphinxstyleemphasis{Percona Server for MongoDB} 4.4.1\sphinxhyphen{}2}
\label{\detokenize{release_notes/4.4.1-2:percona-server-for-mongodb-4-4-1-2}}\label{\detokenize{release_notes/4.4.1-2:psmdb-4-4-1-2}}\label{\detokenize{release_notes/4.4.1-2::doc}}\begin{quote}\begin{description}
\item[{Date}] \leavevmode
\sphinxAtStartPar
September 24, 2020

\item[{Installation}] \leavevmode
\sphinxAtStartPar
\sphinxhref{https://www.percona.com/doc/percona-server-for-mongodb/4.4/install/index.html}{Installing Percona Server for MongoDB}

\end{description}\end{quote}

\sphinxAtStartPar
Percona Server for MongoDB 4.4.1\sphinxhyphen{}2 is an enhanced, source available, and highly\sphinxhyphen{}scalable database that is a
fully\sphinxhyphen{}compatible, drop\sphinxhyphen{}in replacement for \sphinxhref{https://docs.mongodb.com/manual/release-notes/4.4/\#sep-9-2020}{MongoDB 4.4.1 Community Edition}.
It supports MongoDB 4.4.1 protocols and drivers.


\subsection{Bugs Fixed}
\label{\detokenize{release_notes/4.4.1-2:bugs-fixed}}\begin{itemize}
\item {} 
\sphinxAtStartPar
\sphinxhref{https://jira.percona.com/browse/PSMDB-707}{PSMDB\sphinxhyphen{}707}: LDAP authentication randomly fails with the “Bad parameter to an ldap routine” message in the log

\item {} 
\sphinxAtStartPar
\sphinxhref{https://jira.percona.com/browse/PSMDB-677}{PSMDB\sphinxhyphen{}677}: \sphinxcode{\sphinxupquote{mongosh}} cannot authenticate LDAP user

\item {} 
\sphinxAtStartPar
\sphinxhref{https://jira.percona.com/browse/PSMDB-674}{PSMDB\sphinxhyphen{}674}: Provide binary tarball with shared libs and glibc suffix

\end{itemize}


\section{\sphinxstyleemphasis{Percona Server for MongoDB} 4.4.0\sphinxhyphen{}1}
\label{\detokenize{release_notes/4.4.0-1:percona-server-for-mongodb-4-4-0-1}}\label{\detokenize{release_notes/4.4.0-1:psmdb-4-4-0-1}}\label{\detokenize{release_notes/4.4.0-1::doc}}\begin{quote}\begin{description}
\item[{Date}] \leavevmode
\sphinxAtStartPar
August 26, 2020

\item[{Installation}] \leavevmode
\sphinxAtStartPar
\sphinxhref{https://www.percona.com/doc/percona-server-for-mongodb/4.4/install/index.html}{Installing Percona Server for MongoDB}

\end{description}\end{quote}

\sphinxAtStartPar
Percona Server for MongoDB 4.4.0\sphinxhyphen{}1 is an enhanced, source\sphinxhyphen{}available, and highly\sphinxhyphen{}scalable database that is a
fully\sphinxhyphen{}compatible, drop\sphinxhyphen{}in replacement for MongoDB 4.4.0 Community Edition.
It supports MongoDB 4.4.0 protocols and drivers.

\sphinxAtStartPar
This release includes \sphinxhref{https://docs.mongodb.com/manual/release-notes/4.4/\#mongodb-4-4-released-july-30-2020}{all features of MongoDB 4.4.0 Community Edition} and provides Enterprise\sphinxhyphen{}level \sphinxhref{https://www.percona.com/software/mongodb/feature-comparison}{enhancements} for free.


\sphinxstrong{See also:}
\nopagebreak

\begin{description}
\item[{Percona Blog Post}] \leavevmode\begin{itemize}
\item {} 
\sphinxAtStartPar
\sphinxhref{https://www.percona.com/blog/2020/06/11/mongodb-4-4-coming-out-soon-what-does-the-code-tell-us/}{MongoDB 4.4 Is Coming Out Soon \sphinxhyphen{} What Does the Code Tell Us?}

\end{itemize}

\end{description}



\sphinxAtStartPar
\sphinxstyleemphasis{Percona Server for MongoDB} requires no changes to MongoDB applications or code and is available for download from \sphinxhref{https://www.percona.com/downloads/percona-server-mongodb-4.4/}{Percona website}.


\part{Reference}
\label{\detokenize{index:reference}}

\chapter{Contacting, Contributing, Reporting Bugs}
\label{\detokenize{contact:contacting-contributing-reporting-bugs}}\label{\detokenize{contact:contact}}\label{\detokenize{contact::doc}}
\sphinxAtStartPar
If you want to contact developers,
use the \sphinxhref{https://www.percona.com/forums/questions-discussions/percona-server-for-mongodb}{community forum}.

\begin{sphinxadmonition}{note}{Note:}
\sphinxAtStartPar
Please search the forum for similar questions and discussions
before openning a new thread.

\sphinxAtStartPar
You will need to sign up for an account to post in the forum.
\end{sphinxadmonition}

\sphinxAtStartPar
If you want to explore source code and contribute to the project
use the \sphinxhref{https://github.com/percona/percona-server-mongodb}{GitHub repo} and the \sphinxhref{https://github.com/percona/percona-server-mongodb/blob/v4.4/CONTRIBUTING.rst}{Contributing guide}.

\begin{sphinxadmonition}{note}{Note:}
\sphinxAtStartPar
Search existing pull requests and recent commits
for code that may fix what you are planning to suggest.

\sphinxAtStartPar
You will need a public GitHub account
and request contributor access to the repo.
\end{sphinxadmonition}

\sphinxAtStartPar
If you want to report a bug or feature request,
use the \sphinxhref{https://jira.percona.com/projects/PSMDB/summary}{PSMDB project in JIRA}.

\begin{sphinxadmonition}{note}{Note:}
\sphinxAtStartPar
Search JIRA for existing tickets
before submitting a bug or feature request.

\sphinxAtStartPar
You will need a JIRA account to \sphinxhref{https://jira.percona.com/secure/CreateIssueDetails!init.jspa?pid=11601\&issuetype=1}{report bugs}.
\end{sphinxadmonition}


\chapter{Glossary}
\label{\detokenize{glossary:glossary}}\label{\detokenize{glossary::doc}}\begin{description}
\item[{ACID\index{ACID@\spxentry{ACID}|spxpagem}\phantomsection\label{\detokenize{glossary:term-ACID}}}] \leavevmode
\sphinxAtStartPar
Set of properties that guarantee database transactions are
processed reliably. Stands for {\hyperref[\detokenize{glossary:term-Atomicity}]{\sphinxtermref{\DUrole{xref,std,std-term}{Atomicity}}}},
{\hyperref[\detokenize{glossary:term-Consistency}]{\sphinxtermref{\DUrole{xref,std,std-term}{Consistency}}}}, {\hyperref[\detokenize{glossary:term-Isolation}]{\sphinxtermref{\DUrole{xref,std,std-term}{Isolation}}}}, {\hyperref[\detokenize{glossary:term-Durability}]{\sphinxtermref{\DUrole{xref,std,std-term}{Durability}}}}.

\item[{Atomicity\index{Atomicity@\spxentry{Atomicity}|spxpagem}\phantomsection\label{\detokenize{glossary:term-Atomicity}}}] \leavevmode
\sphinxAtStartPar
Atomicity means that database operations are applied following a
“all or nothing” rule. A transaction is either fully applied or not
at all.

\item[{Consistency\index{Consistency@\spxentry{Consistency}|spxpagem}\phantomsection\label{\detokenize{glossary:term-Consistency}}}] \leavevmode
\sphinxAtStartPar
Consistency means that each transaction that modifies the database
takes it from one consistent state to another.

\item[{Durability\index{Durability@\spxentry{Durability}|spxpagem}\phantomsection\label{\detokenize{glossary:term-Durability}}}] \leavevmode
\sphinxAtStartPar
Once a transaction is committed, it will remain so.

\item[{Foreign Key\index{Foreign Key@\spxentry{Foreign Key}|spxpagem}\phantomsection\label{\detokenize{glossary:term-Foreign-Key}}}] \leavevmode
\sphinxAtStartPar
A referential constraint between two tables. Example: A purchase
order in the purchase\_orders table must have been made by a customer
that exists in the customers table.

\item[{Isolation\index{Isolation@\spxentry{Isolation}|spxpagem}\phantomsection\label{\detokenize{glossary:term-Isolation}}}] \leavevmode
\sphinxAtStartPar
The Isolation requirement means that no transaction can interfere
with another.

\item[{Jenkins\index{Jenkins@\spxentry{Jenkins}|spxpagem}\phantomsection\label{\detokenize{glossary:term-Jenkins}}}] \leavevmode
\sphinxAtStartPar
\sphinxhref{http://www.jenkins-ci.org}{Jenkins} is a continuous integration
system that we use to help ensure the continued quality of the
software we produce. It helps us achieve the aims of:
\begin{itemize}
\item {} 
\sphinxAtStartPar
no failed tests in trunk on any platform,

\item {} 
\sphinxAtStartPar
aid developers in ensuring merge requests build and test on all platforms,

\item {} 
\sphinxAtStartPar
no known performance regressions (without a damn good explanation).

\end{itemize}

\item[{Kerberos\index{Kerberos@\spxentry{Kerberos}|spxpagem}\phantomsection\label{\detokenize{glossary:term-Kerberos}}}] \leavevmode
\sphinxAtStartPar
Kerberos is an authentication protocol for client/server authentication without sending the passwords over an insecure network. Kerberos uses symmetric encryption in the form of tickets \sphinxhyphen{} small pieces of encrypted data used for authentication. A ticket is issued for the client and validated by the server.

\item[{Rolling restart\index{Rolling restart@\spxentry{Rolling restart}|spxpagem}\phantomsection\label{\detokenize{glossary:term-Rolling-restart}}}] \leavevmode
\sphinxAtStartPar
A rolling restart (rolling upgrade) is shutting down and upgrading nodes one by one. The whole cluster remains operational. There is no interruption to clients assuming the elections are short and all writes directed to the old primary use the retryWrite mechanism.

\end{description}


\chapter{Copyright and Licensing Information}
\label{\detokenize{copyright:copyright-and-licensing-information}}\label{\detokenize{copyright::doc}}

\section{Documentation Licensing}
\label{\detokenize{copyright:documentation-licensing}}
\sphinxAtStartPar
This software documentation is (C)2016\sphinxhyphen{}2022 Percona LLC and/or its affiliates
and is distributed under the \sphinxhref{https://creativecommons.org/licenses/by/4.0/}{Creative Commons Attribution 4.0 International License}.


\section{Software License}
\label{\detokenize{copyright:software-license}}
\sphinxAtStartPar
\sphinxstyleemphasis{Percona Server for MongoDB} is \sphinxhref{https://en.wikipedia.org/wiki/Source-available\_software}{source\sphinxhyphen{}available software}.


\chapter{Trademark Policy}
\label{\detokenize{trademark-policy:trademark-policy}}\label{\detokenize{trademark-policy::doc}}
\sphinxAtStartPar
This Trademark Policy is to ensure that users of Percona\sphinxhyphen{}branded products or
services know that what they receive has really been developed, approved,
tested and maintained by Percona. Trademarks help to prevent confusion in the
marketplace, by distinguishing one company’s or person’s products and services
from another’s.

\sphinxAtStartPar
Percona owns a number of marks, including but not limited to Percona, XtraDB,
Percona XtraDB, XtraBackup, Percona XtraBackup, Percona Server, and Percona
Live, plus the distinctive visual icons and logos associated with these marks.
Both the unregistered and registered marks of Percona are protected.

\sphinxAtStartPar
Use of any Percona trademark in the name, URL, or other identifying
characteristic of any product, service, website, or other use is not permitted
without Percona’s written permission with the following three limited
exceptions.

\sphinxAtStartPar
\sphinxstyleemphasis{First}, you may use the appropriate Percona mark when making a nominative fair
use reference to a bona fide Percona product.

\sphinxAtStartPar
\sphinxstyleemphasis{Second}, when Percona has released a product under a version of the GNU
General Public License (“GPL”), you may use the appropriate Percona mark when
distributing a verbatim copy of that product in accordance with the terms and
conditions of the GPL.

\sphinxAtStartPar
\sphinxstyleemphasis{Third}, you may use the appropriate Percona mark to refer to a distribution of
GPL\sphinxhyphen{}released Percona software that has been modified with minor changes for
the sole purpose of allowing the software to operate on an operating system or
hardware platform for which Percona has not yet released the software, provided
that those third party changes do not affect the behavior, functionality,
features, design or performance of the software. Users who acquire this
Percona\sphinxhyphen{}branded software receive substantially exact implementations of the
Percona software.

\sphinxAtStartPar
Percona reserves the right to revoke this authorization at any time in its sole
discretion. For example, if Percona believes that your modification is beyond
the scope of the limited license granted in this Policy or that your use of the
Percona mark is detrimental to Percona, Percona will revoke this authorization.
Upon revocation, you must immediately cease using the applicable Percona mark.
If you do not immediately cease using the Percona mark upon revocation, Percona
may take action to protect its rights and interests in the Percona mark.
Percona does not grant any license to use any Percona mark for any other
modified versions of Percona software; such use will require our prior written
permission.

\sphinxAtStartPar
Neither trademark law nor any of the exceptions set forth in this Trademark
Policy permit you to truncate, modify or otherwise use any Percona mark as part
of your own brand. For example, if XYZ creates a modified version of the
Percona Server, XYZ may not brand that modification as “XYZ Percona Server” or
“Percona XYZ Server”, even if that modification otherwise complies with the
third exception noted above.

\sphinxAtStartPar
In all cases, you must comply with applicable law, the underlying license, and
this Trademark Policy, as amended from time to time. For instance, any mention
of Percona trademarks should include the full trademarked name, with proper
spelling and capitalization, along with attribution of ownership to Percona
Inc. For example, the full proper name for XtraBackup is Percona XtraBackup.
However, it is acceptable to omit the word “Percona” for brevity on the second
and subsequent uses, where such omission does not cause confusion.

\sphinxAtStartPar
In the event of doubt as to any of the conditions or exceptions outlined in
this Trademark Policy, please contact \sphinxhref{mailto:trademarks@percona.com}{trademarks@percona.com} for assistance and
we will do our very best to be helpful.



\renewcommand{\indexname}{Index}
\printindex
\end{document}